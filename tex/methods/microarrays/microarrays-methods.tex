\section{Microarray technology}
\label{sec:microarray-methods}

Microarray technology relies on the non-covalent, sequence-specific interaction
between complementary strands of nucleic acids into a single
complex,\cite{crick_protein_1958} to detect and quantify specific populations of
m\smallcaps{rna} on a solution.  A microarray chip consists of a universe of
m\smallcaps{rna}-specific complementary oligonucleotide probes attached to a
substrate (Figure~\ref{fig:microarray-economist}).

\begin{figure}%
  \begin{center}
    \includegraphics[width=9cm]{microarrays-economist.jpg}
    \caption[Schematic representation of how microarrays work]{A schematic
      representation of how microarrays work. Source: \smallcaps{The
        Economist}.}
    \label{fig:microarray-economist}
  \end{center}
\end{figure}



%%% Local Variables:
%%% mode: latex
%%% TeX-master: "../../thesis"
%%% End:
