\section{Microarray technology}
\label{sec:microarray-methods}

Microarray technology relies on the non-covalent, sequence-specific interaction
between complementary strands of nucleic acids\cite{crick_protein_1958} to
detect and quantify specific populations of m\smallcaps{rna} on a solution.  A
microarray chip consists of a universe of oligonucleotide probes attached to a
substrate through covalent bonds.  Each such probe is synthesized to
specifically match a unique messenger \smallcaps{rna} molecule.  When the chip
is exposed to a solution of fluorecently labeled m\smallcaps{rna}s, only those
that hybridize with their respective probes will be retained upon washing off
non-specific bonding sequences.  This allows for the quantification of the
fluorescent signals emitted when the chip is scanned with a laser beam of a
specific wavelength.  The measured signals relay the relative quantity of each
m\smallcaps{rna} molecule assayed by the microarray, as each spot has a known
position on the chip (Figure~\ref{fig:microarray-economist}).

% Oligonucleotide microarrays often carry control probes designed to hybridize
% with RNA spike-ins. The degree of hybridization between the spike-ins and the
% control probes is used to normalize the hybridization measurements for the
% target probes.

\begin{marginfigure}%
  \begin{center}
    % \includegraphics[width=9cm]{microarrays-economist.jpg}
    \includegraphics{microarrays-economist-17Feb2015.pdf}
    \caption[Schematic representation of how microarrays work]{A schematic
      representation of how microarrays work.  \textbf{1.}~Microarrays rely on a
      fundamental property of nucleic acids, the monomeric units that polymerize
      into \smallcaps{dna} or \smallcaps{rna} strands.  Adenine (\smallcaps{A})
      are complementary to thymine (\smallcaps{t}), and cytosine (\smallcaps{c})
      are complementary to guanine (\smallcaps{g}).  Just one incorrect base can
      prevent two strands from binding.  \textbf{2.}~A microarray typically
      contains thousands of squares, or spots.  Each spot anchors many copies of
      a particular sequence of single-stranded \smallcaps{dna}, corresponding to
      a particular gene.  \textbf{3.}~Messenger \smallcaps{rna} fragments
      extracted from a tissue and labeled with different fluorescent dyes are
      washed over the microarray and hybridize with \smallcaps{dna} strands with
      the complementary sequence.  \textbf{4.}~The dyes are illuminated using
      fluorescent light.  It is then possible to show which \smallcaps{rna}
      fragments were retained in which spots---and hence which genes were being
      expressed in the tissue from which the \smallcaps{rna} was extracted.
      Source: \emph{The Economist; Affymetrix}.}
    \label{fig:microarray-economist}
  \end{center}
\end{marginfigure}

%%% Local Variables:
%%% TeX-engine: xetex
%%% mode: latex
%%% TeX-master: "../../thesis"
%%% End:
