\section{Microarray technology}
\label{sec:microarray-methods}

Microarray technology relies on the non-covalent, sequence-specific interaction
between complementary strands of nucleic acids\cite{watson_molecular_1953} to
detect and quantify specific populations of m\smallcaps{rna} in a solution.  A
microarray chip consists of a universe of oligonucleotide probes attached to a
substrate through covalent bonds.  Each such probe is synthesized to
specifically match a unique messenger \smallcaps{rna} molecule.  When the chip
is exposed to a solution of fluorecently labeled m\smallcaps{rna}s, only those
that hybridize with their respective probes will be retained upon washing off
non-specific bonding sequences.  This allows for the quantification of the
fluorescent signals emitted when the chip is scanned with a laser beam of a
specific wavelength.  The measured signals relay the relative quantity of each
m\smallcaps{rna} molecule assayed by the microarray, as each spot has a known
position on the chip (Figure~\ref{fig:microarray-economist}).

% Oligonucleotide microarrays often carry control probes designed to hybridize
% with RNA spike-ins. The degree of hybridization between the spike-ins and the
% control probes is used to normalize the hybridization measurements for the
% target probes.

\begin{marginfigure}%
  \begin{center}
    % \includegraphics[width=9cm]{microarrays-economist.jpg}
    \includegraphics{microarrays-economist-17Feb2015.pdf}
    \caption[Schematic representation of how microarrays work]{A schematic
      representation of how microarrays work.  \textbf{1.}~Microarrays rely on a
      fundamental property of nucleic acids, the monomeric units that polymerize
      into \smallcaps{dna} or \smallcaps{rna} strands.  Adenine (\smallcaps{A})
      are complementary to thymine (\smallcaps{t}), and cytosine (\smallcaps{c})
      are complementary to guanine (\smallcaps{g}).  Just one incorrect base can
      prevent two strands from binding.  \textbf{2.}~A microarray typically
      contains thousands of squares, or spots.  Each spot anchors many copies of
      a particular sequence of single-stranded \smallcaps{dna}, corresponding to
      a particular gene.  \textbf{3.}~Messenger \smallcaps{rna} fragments
      extracted from a tissue and labeled with different fluorescent dyes are
      washed over the microarray and hybridize with \smallcaps{dna} strands with
      the complementary sequence.  \textbf{4.}~The dyes are illuminated using
      fluorescent light.  It is then possible to show which \smallcaps{rna}
      fragments were retained in which spots---and hence which genes were being
      expressed in the tissue from which the \smallcaps{rna} was extracted.
      Source: \emph{The Economist;
        Affymetrix}.}\label{fig:microarray-economist}%
  \end{center}
\end{marginfigure}

\section{Microarray data preprocessing}
\label{sec:microarray-methods-data-preprocessing}

By virtue of their design, microarrays allow for the monitoring of expression
levels for thousands of gene transcription products simultaneously.  Microarray
expression data are thus characterized by high dimensionality and noisiness.
This prompts the need for preprocessing methods aiming at removing systematic
biases in expression measurements, introduced during
experimentation.\cite{shakya_comparison_2010}

The goal of microarray data preprocessing is to convert raw imaging data into
meaningful biological data and to enable comparison of results obtained from
different arrays.  It comprises three steps: (\emph{a})~the transformation of
image data into intensity values; (\emph{b})~the assessment of array quality;
and (\emph{c})~the removing of technical biases (through background adjustment,
normalization and feature filtering and summarization).

The digital imaging of fluorescence signals is typically performed by
proprietary software designed by the microarray manufacturer.  These software
packages assign coordinates to each spot in the array, quantify signal intensity
and uniformity of each spot, and compare their signal intensity relative to
background.

Quality control is a critical step in the preprocessing of microarray data.  In
spite of the many efforts to provide standards for the technology, such as the
External \smallcaps{rna} Control Consortium\cite{baker_external_2005} and the
MicroArray Quality Control\cite{consortium_microarray_2010} initiatives, there
remains a lack of consensus in both defining and measuring microarray
quality.\cite{mccall_assessing_2011} Computational strategies to tackle quality
assessment in single-channel and double-channel arrays have been implemented in
the \textsf{Bioconductor} package
\textsf{arrayQualityMetrics}.\cite{kauffmann_arrayqualitymetrics--bioconductor_2009,kauffmann_microarray_2010}

% Quality control is performed by visual inspection of imaging data from the
% scanner or the platform software, with special attention given to washing
% artifacts, odd or missing spots, and array uniformity.  Aberrant chips may have
% to be discarded at this stage.

Chips meeting quality standards then undergo background adjustment and
normalization.  Normalization methods aim to compensate for procedural biases
that are independent from biological signal.  Early approaches for microarray
normalization were based on the assumption that most genes, and in particular
so-called housekeeping genes,\footnote{\emph{Housekeeping} genes are genes
  defined as participating in basic, thus universal, cellular processes.}
should have similar expression levels across samples.  Housekeeping genes have
since been shown to vary in expression by 30\% or more across healthy samples,
and even more in tumour samples.\cite{lee_control_2002,eisenberg_human_2003}
Data-driven normalization approaches were then developed, such as median
correction,\cite{cho_genome-wide_1998,selinger_rna_2000} variance stabilizing
transformation,\cite{durbin_variance-stabilizing_2002} locally weighted linear
regression\cite{yang_normalization_2002} and spline based
methods.\cite{workman_new_2002} % Data-driven presupposes that most of the
% observed variation in expression values is due to technical biases rather than
% to biological sources.\cite{hicks_when_2014}

Normalization strategies for double-channel microarrays (spotted oligonucleotide
or c\smallcaps{dna} arrays\cite{schena_quantitative_1995}) are different from
those for single-channel microarrays (\emph{in situ} synthesized high density
oligonucleotide arrays,\cite{lockhart_expression_1996} such as \emph{Affymetrix
  GeneChip}).  \emph{Affymetrix GeneChip} arrays use multiple probes per gene
and a single-colour detection system, as one sample is hybridized per chip.
Spotted oligonucleotide or c\smallcaps{dna} arrays use one probe per gene and a
two-colour scheme, where two different samples are hybridized on the same array.
Consequently, single-channel arrays measure the overall abundance of a probe
sequence in a target sample, whereas c\smallcaps{dna} arrays measure the
relative abundance of a probe sequence in two target samples.
% In other words, the expression measures for single-channel arrays are absolute
% (log) intensities, whereas they represent (log) ratios of intensities for
% c\smallcaps{dna} arrays.\footnote{In many cases, one of the samples in a
% c\smallcaps{DNA} array hybridization is a common reference used across
% multiple slides and whose sole purpose is to provide a baseline for direct
% comparison of expression measures between arrays.}  As a result, normalization
% of single-channel microarrays is performed at the ``between-array'' level,
% whereas normalization of double-channel microarrays is conducted at the
% ``within-array'' level.\cite{do_normalization_2006}

For double-channel arrays chips, normalization methods commonly seek to remove
biases within each array with local regression algorithms.  Terry Speed's lab,
at Berkeley, identified an intensity-dependent dye bias concerning
c\smallcaps{dna} arrays.  In these arrays, the $\log_2$ of the dye intensity
ratios shows a systematic dependence on intensity, characterized by a deviation
from zero for low-intensity spots.  Frequently, under-expressed genes appear
up-regulated in the red channel ($R$), and moderately expressed genes appear
up-regulated in the green channel ($G$).
% This artifact is the result of a ``quenching'' effect, whereby dye molecules
% in close proximity tend to re-absorb light from each other, hence diminishing
% the signal.  The amount of re-absorption is a function of the template
% concentration and differs for the two dyes.
This effect can be visualized by plotting the measured
$\log_2(\frac{R_{i}}{G_{i}})$ for each feature in the array as a function of the
$\log_2(R_{i}G_{i})$ product intensities.  This ratio-intensity plot is termed
\smallcaps{ma} plot.\footnote[][-5.5cm]{The name of the plot comes from
  ``minus'' and ``add'', respectively the ratio and product in the logarithmic
  scale.}  This technical bias may be corrected by fitting a locally weighted
regression, known as \emph{lowess} smoothing
(Figure~\ref{fig:ma-plot}).\cite[-4.2cm]{yang_normalization_2001} More specific
sources of technical bias, including spatially-dependent bias resulting from the
print tips used in the manufacturing process of the array, may also be addressed
by a \emph{lowess}-based, within group normalization.

\begin{marginfigure}[-4.2cm]%
  \begin{center}
    \includegraphics{lowess-normalization-23Feb2013.pdf}
    \caption[\emph{Lowess} normalization]{Example of \emph{lowess}
      normalization.  \textbf{A:}~\smallcaps{ma} plot showing colour dye
      dependent bias.  \textbf{B:}~\smallcaps{ma} plot after correction with
      \emph{lowess} normalization
      (\citealp{yang_normalization_2002}).}\label{fig:ma-plot}%
  \end{center}
\end{marginfigure}

\emph{Affymetrix GeneChip} are the reference arrays in the single-channel class
and consist of several tens of thousands probe-sets.  A probe-set is a
collection of probe pairs designed to interrogate a specific sequence and
contains \numrange{11}{20} probe pairs of 25-mer oligonucleotides each.  Each
probe pair consists of a perfect match probe (\smallcaps{pm}) and a mismatch
probe (\smallcaps{mm}).  The \smallcaps{mm} probe differs from the
\smallcaps{pm} probe by a single substitution at the center base position,
conceived to disturb the binding of the target gene transcript.  This design
allows for the quantification of background and nonspecific hybridization
effects.

Different between-array normalization methods have been proposed in the
literature.  The \smallcaps{mas5} algorithm\cite{hubbell_robust_2002} normalizes
each array independently and sequentially and uses the value of \smallcaps{mm}
probes to compute summarized averages with linear scaling.  The \smallcaps{rma}
(robust multi-array average) algorithm\cite{irizarry_exploration_2003}
normalizes the value of each probe by quantile normalization\footnote{Quantile
  normalization is a global adjustment method that assumes the statistical
  distribution of expression values of each sample is the same.  Normalization
  is achieved by imposing the same distribution to all samples, using an average
  distribution as reference.  The average distribution is estimated from the
  average of each quantile across samples.} in multiple arrays, neglecting
information from \smallcaps{mm} probes.  The \smallcaps{gcrma}
algorithm\cite{wu_model-based_2004} applies the same normalization and
summarization methods as \smallcaps{rma}, but uses probe sequence information to
estimate and correct for probe affinity to non-specific binding.  More recently,
the f\smallcaps{rma} (frozen \smallcaps{rma}) algorithm\cite{mccall_frozen_2010}
leverages pre-computed estimates of probe-specific effects and variance from
public microarray databases to, in concert with the information from new arrays,
normalize and summarize the data.

%%%%%%%%%%%%%%%%%%%%%%%%%%%%%%%%%%%%%%%%%%%%%%%%%%%%%%%%%%%%%%%%%%%%%%%%%%%%%%%%
%% http://www.people.vcu.edu/~mreimers/OGMDA/normalize.expression.html
%%%%%%%%%%%%%%%%%%%%%%%%%%%%%%%%%%%%%%%%%%%%%%%%%%%%%%%%%%%%%%%%%%%%%%%%%%%%%%%%

% The first ``\mbox{across-array}'' normalization methods relied on the
% assumption that most genes, and in particular ``housekeeping''
% genes\footnote{\emph{Housekeeping} genes are genes recognized as participating
% in basic, thus universal, cellular processes.}, should not be expected to be
% differentially expressed across samples.  However, housekeeping genes have
% since been shown to vary in expression by 30\% or more across healthy samples,
% and even more in tumour samples.\cite{lee_control_2002} This lead to new
% approaches to identify stable patterns of expression around which to pin
% global levels of expression.  Yet, these approaches have been shown to suffer
% from significant levels of cross-hybridization between \mbox{25-mers} in
% \emph{Affymetrix} chips.

% Terry Speed's lab, at Berkeley, identified an important intensity-dependent
% dye bias regarding double channel microarrays, and introduced a popular method
% for adjusting it.  In these arrays, the $log_{2} (ratio)$ shows a systematic
% dependence on intensity, characterized by a deviation from zero for
% low-intensity spots.  Typically, under-expressed genes appear up-regulated in
% the red channel ($R$) and moderately expressed genes appear up-regulated in
% the green channel ($G$).  This artifact is the result of a ``quenching''
% effect, whereby dye molecules in close proximity tend to re-absorb light from
% each other, hence diminishing the signal.  The amount of re-absorption is a
% function of the template concentration and differs for the two dyes.  This
% effect can be visualized by plotting the measured
% $log_{2}(\frac{R_{i}}{G_{i}})$ for each feature in the array as a function of
% the $log_{2}(R_{i}G_{i})$ product intensities.  This ratio-intensity plot is
% termed \smallcaps{ma} plot.\footnote{The name of the plot comes from ``minus''
% and ``add'', or ratio and product in the logarithmic scale.}

%%%%%%%%%%%%%%%%%%%%%%%%%%%%%%%%%%%%%%%%%%%%%%%%%%%%%%%%%%%%%%%%%%%%%%%%%%%%%%%%

% Among these are methods designed to normalize single chips, known as
% ``within-arrays'' methods, such as quantile normalization, or \smallcaps{mas5}
% for \emph{Affymetrix} chips.

Several reviews on these and other normalization methods were produced in the
specialized
literature,\cite{ploner_correlation_2005,bolstad_comparison_2003,harr_comparison_2006}
and a more detailed technical exposition of the computational implementation of
these algorithms can be found in Gentleman et
al.,\cite{gentleman_bioinformatics_2006} in the context of the
\href{http://www.bioconductor.org/}{\textsf{Bioconductor}} project.\footnote{The
  Bioconductor project is an open source and open development software project
  for the analysis and comprehension of genomic data
  (\citealp{gentleman_bioconductor:_2004}).  It is rooted in the open source
  statistical computing environment \textsf{R}.}

\medskip

The normalized microarray data for $p$ genes and $n$ biological samples are
denoted by $X_{n \times p}$ such that $x_{ij}$ represents the expression of gene
$j$ of sample $i$.

\section{Microarray data analysis}
\label{sec:microarray-methods-data-analysis}

%%%%%%%%%%%%%%%%%%%%%%%%%%%%%%%%%%%%%%%%%%%%%%%%%%%%%%%%%%%%%%%%%%%%%%%%%%%%%%%%
%% http://discover.nci.nih.gov/microarrayAnalysis/Exploratory.Analysis.jsp
%%%%%%%%%%%%%%%%%%%%%%%%%%%%%%%%%%%%%%%%%%%%%%%%%%%%%%%%%%%%%%%%%%%%%%%%%%%%%%%%

In cancer research, analysis of genomic experiments performed with
\smallcaps{dna} microarray data may address a variety of tasks, including
finding gene associations with particular phenotypes, tumour class discovery or
tumour class prediction.  The work contributed in this dissertation concerns:
(\emph{a}) the class prediction of tumour types based on gene expression
profiles; and (\emph{b}) the modeling of association of gene expression profiles
with the time until a particular outcome is observed (e.g., death of a patient
or relapse of disease), using survival analysis.  Both these problems are
examples of application of supervised learning techniques to the analysis of
microarray data.

In machine learning,\footnote{Machine learning is a branch of computational
  science whose purpose is to implement algorithms capable to infer models from
  training data.  Models derived from machine learning methods are then expected
  to make predictions or to inform decisions on testing data.} supervised
learning methods seek to infer a function from labeled training data.
Conversely, unsupervised learning methods aim to find intrinsic structure from
unlabeled data.\cite{webb_statistical_2003} Unsupervised learning is thus suited
to uncover coherent genomic signals from microarray data, whereas supervised
learning can be used to find associations between genomic signals and phenotypic
classes of samples.

Due to the high dimensionality of expression profiles, methods for
dimensionality reduction are required for microarray data analysis.  These
include \emph{feature transformation} methods and \emph{feature selection}
methods.\cite{haibe-kains_identification_2009}

Feature transformation is an unsupervised approach that consists in reducing the
feature space of a microarray gene expression matrix, such that new features
retain biological pertinence, maximum information and generalizability to
similar experiments:
\begin{equation}
  \label{eq:feature-transformation}
  X_{n \times p} \to X'_{n \times p'} : p \gg p'.
\end{equation}
Examples of microarray feature transformation techniques include principal
component analysis and clustering methods.

Feature selection techniques, on the other hand, are supervised approaches to
reduce data dimensionality, in order to produce simplified and interpretable
models.  % In this dissertation, particular attention will be
% given to gene expression signatures, which configure an example of microarray
% feature selection.

Gene expression signatures, or metagenes, are collections of genes sharing a
combined expression pattern associated with a given phenotype.  The range of
biological phenotypes confining the characterization of expression signatures
include the modulation of signaling pathways,\cite{itadani_can_2008} the
specification of tumour classes,\cite{ramaswamy_multiclass_2001} or the
definition of distinct clinical outcomes.\cite{vant_veer_gene_2002} In the last
two decades, a wealth of microarray studies on perturbed \emph{in vitro}
biological systems have generated an extensive number of gene signatures related
to various cellular mechanisms.\cite{chibon_cancer_2013} Public repositories,
like the Gene Ontology consortium (\smallcaps{go}),\cite{ashburner_gene_2000}
the Kyoto Encyclopedia of Genes and Genomes
(\smallcaps{kegg}),\cite{kanehisa_kegg:_2000}
\mbox{GeneSigDB},\cite{culhane_genesigdb:_2012} or the Molecular Signatures
Database (\mbox{MSigDB}),\cite{subramanian_gene_2005} have sought to curate and
articulate this volume of information in order to guide the interpretation of
genome-wide expression profiles.  Because biologically motivated gene signatures
can act as surrogate markers for the molecular processes they capture, they
provide entry points for hypothesis testing in public microarray data.

The use of a common vocabulary to refer to microarray features is thus essential
to this goal.  Given the wide range of microarray platforms on the market,
preprocessing routines often produce genomic expression data with feature
annotations that are disjoint, inconsistent, or conflicting.  Computational
strategies to interface curated annotation databases in order to update and
standardize feature nomenclatures are generally poorly discussed in the
literature.  A solid foundation for \emph{in silico} solutions to bridge the gap
between the knowledge of transcript sequence and the knowledge of transcript
function is provided in Gentleman et al.\cite{gentleman_bioinformatics_2006}

Accordingly, for all datasets used in this dissertation (described in the
\hyperref[sec:methods-datasets]{\textsf{Microarray datasets}} section),
\href{http://www.bioconductor.org/}{\textsf{Bioconductor}} resources were used
to update feature annotations, and referents for \smallcaps{hugo} Gene
Nomenclature Committee (\smallcaps{hgnc}) gene
symbols\footnote{\href{http://www.genenames.org/}{http://www.genenames.org/}}
were universally retained as feature descriptors.  Microarray gene annotation
may also be used to perform dimension reduction of the expression feature space.
Hence, expression matrices where multiple features addressed the expression of
the same gene product were collapsed using a \texttt{maxMean}
routine.\cite{miller_strategies_2011} This approach consists of selecting, among
all the features measuring the expression of the same gene, the probeset with
highest mean expression.

All analyses described in the \hyperref[chap:results]{\textsf{Results}} chapter
were conducted in the \href{http://www.r-project.org/}{\textsf{R}} environment
for statistical computing,\cite{r_core_team_r:_2014} with extensive use of
computational resources from the
\href{http://www.bioconductor.org/}{\textsf{Bioconductor}}
project.\cite{gentleman_bioconductor:_2004}

% metagenes. how are they derived. correlations with gene expression. they can
% behave as surrogate markers of biological processes, states or responses and
% be used as classifiers.  Metagenes can also be also non-biologically
% motivated, and thus be defined randomly.

% a word about meta-data resources and bioconductor tools.

%%%%%%%%%%%%%%%%%%%%%%%%%%%%%%%%%%%%%%%%%%%%%%%%%%%%%%%%%%%%%%%%%%%%%%%%%%%%%%%%
%% abstract of chapter seven of the book
%% Bioinformatics and Computational Biology Solutions Using R and Bioconductor
%% Meta-data Resources and Tools in Bioconductor
%%%%%%%%%%%%%%%%%%%%%%%%%%%%%%%%%%%%%%%%%%%%%%%%%%%%%%%%%%%%%%%%%%%%%%%%%%%%%%%%

% Closing the gap between knowledge of sequence and knowledge of function
% requires aggressive, integrative use of biological research databases of many
% different types.  For greatest effectiveness, analysis processes and
% interpretation of analytic results must be guided using relevant knowledge
% about the systems under investigation.  However, this knowledge is often
% widely scattered and encoded in a variety of formats.  In this section, we
% consider some of the different sources of biological information as well as
% the software tools that can be used to access these data and to integrate them
% into an analysis.  Bioconductor provides tools for creating, distributing, and
% accessing annotation resources in ways that have been found effective in work-
% flows for statistical analysis of microarray and other high-throughput assays.

%%%%%%%%%%%%%%%%%%%%%%%%%%%%%%%%%%%%%%%%%%%%%%%%%%%%%%%%%%%%%%%%%%%%%%%%%%%%%%%%
%%%%%%%%%%%%%%%%%%%%%%%%%%%%%%%%%%%%%%%%%%%%%%%%%%%%%%%%%%%%%%%%%%%%%%%%%%%%%%%%

\medskip

This section will proceed with a brief overview of tools for visualization of
genomic data, namely heatmaps and multidimensional scaling.  It will then
discuss the unsupervised learning tools used in this dissertation, including
principal component analysis % , cluster analysis
and a summary on machine learning algorithms.  Next, it will cover
\smallcaps{roc} curves, a tool to evaluate the performance of classifiers.
% , including \smallcaps{roc} curves (for binary classifiers) and
% \smallcaps{gsea} (an algorithm that uses microarray data to determine whether
% a gene expression classifier is statistically associated with any of two
% phenotypic classes of biological samples).
Finally, it will detail survival analysis, the branch of supervised learning
that seeks to explain the relationship between a number of measured features
(gene expression data) and the time duration until the occurrence of a
particular event (survival outcome).

\subsection{Visualization techniques}
\label{sec:methods-visualization}
% 02Mar2015
\begin{marginfigure}%
    \includegraphics{heatmap.png}
    \caption[Example of a heatmap]{Example of a heatmap generated from an
      expression matrix, $X_{n \times p}$, issued from a \smallcaps{dna}
      microarray experiment.  The expression of a selection of features
      (columns) is shown for all the profiled samples in the experiment (rows).
      The traditional colour coding scheme ranges from bright green to bright
      red, for features highly expressed or repressed between conditions (or
      regarding a control sample, in double channel arrays), respectively.
      Features coded in darker shades are not differentially expressed between
      conditions.  Features and samples are hierarchically clustered in
      dendrograms, to reflect gene co-expression motifs and related expression
      patterns between samples.}
    \label{fig:heatmap}%
\end{marginfigure}

The heatmap is the most recognizable visual representation of microarray
expression matrices (Figure~\ref{fig:heatmap}).  It translates the $\log_2$
values of expression into a colour-coding scheme that seeks to convey the
phenotype-specific transcription patterns of the profiled samples.  Because of
the high dimensionality of microarray data, and of the nuanced nature of
transcription profiles, methods of feature selection are routinely used to
narrow down the choice of genes represented in heatmaps.  The result is a direct
visual representation of the expression matrix, highlighting a subset of genes
that maximize the contrast between sampled phenotypes or conditions.

% Heatmaps typically only show the subset of the feature space (narrowed down by
% feature selection) that maximizes the contrast between sampled phenotypes or
% conditions.

\begin{marginfigure}%
    \includegraphics{mds-15Mar2015.pdf}
    \caption[Example of a multidimesional scaling]{Example of a multidimensional
      scaling (\smallcaps{mds}) based on a distance matrix of road distances, in
      km, between 16 European cities.  In this representation, the data
      transformation involves the projection of uni-dimensional variables into a
      two-dimensional space.  In microarray data analysis, \smallcaps{mds}
      transformation requires optimally solving the projection of the high
      dimensional expression space into a plane, so that the between-sample
      distances are best exposed (see text for details).}
    \label{fig:mds}%
\end{marginfigure}

More sophisticated techniques of visualization may require a degree of data
transformation.  Such is the case of multidimensional scaling (\smallcaps{mds}),
a visualization technique that seeks to project the $n$-dimensional feature
space of a collection of samples into a two-dimensional space, in such a way
that the similarities between samples are preserved as best as possible
(Figure~\ref{fig:mds}).  The implementation of the \smallcaps{mds} algorithm
used in this dissertation, known as non-metric multidimensional scaling,
requires finding the optimal coordinate matrix whose configuration minimizes a
loss function called \emph{stress}, based on a dissimilarity matrix derived from
the gene expression space.\cite{borg_modern_2005}

\subsection{Principal component analysis}
\label{sec:methods-pc}
% 03Mar2015
Principal component analysis, or \smallcaps{pca}, is an unsupervised technique
of feature transformation, whose purpose is to reduce the dimensionality of a
dataset to a few, interpretable, and linear combinations of
variables.\cite{pearson_liii._1901} It proceeds by identifying orthogonal
directions, termed principal components, along which the variation of the data
is maximal.  Principal components are uncorrelated meta-variables, each
explaining a decreasing fraction of the partitioned variation within the
original data.  Similarities and differences between samples are thus best
visualized when projected in this new feature space.

Due to the intricate patterns of correlations in their expression space, microarray
data are particularly suited for \smallcaps{pca}-driven dimensionality
reduction.\cite{ringner_what_2008} One appealing aspect of \smallcaps{pca}
analysis of expression profiles is the potential biological interpretability of
principal components.  For instance, the first principal component of a
collection of expression profiles of breast cancers has been shown to explain
the separation of samples according to estrogen receptor status (a major
predictor of disease progression).\cite{saal_poor_2007}

In this dissertation, \smallcaps{pca} is used to derive eigengenes from the
feature expression space of cancer biospecimens.  An eigengene is a
meta-variable seizing the modulation of a gene signature in a collection of
samples.  It can be defined, for instance, by the first principal component of
the expression of the metagene in those samples.

% defined by the first principal component of a gene expression
% signature, that can be used to capture the modulation of the underlying
% expression patterns in the sampled expression profiles.

%  This is done by computing the first principal component of gene expression
% signatures define new expression vectors that best represent the expression

% can be used to project samples in a new feature space that . In this dissertation, \smallcaps{pca} is used to find
% meta-variables in the expression space of expression profiles that best
% represent the

% This is achieved through a series of orthogonal transformations to cast a set of
% observations of potentially correlated variables into new meta-variables,
% represented by a series of linearly uncorrelated vectors, each capturing
% decreasing fractions of variation among the original data.

% Because of their high dimensionality, microarray data are
% characterized by a high number of correlations in their feature space.

%  has a long record of applications
% in the analysis of microarray data

% \subsection{Cluster analysis}
% \label{sec:methods-clustering}
% % 04Mar2015
% Cluster analysis is an unsupervised technique for exploratory data mining that
% seeks to group objects according to their degree of similarity.  In microarray
% data analysis, clustering methods are mainly employed to address class discovery
% tasks, based on gene expression profiles of biospecimens.  Several clustering
% techniques have been used to this goal.

% Hierarchical clustering is a collection of algorithms that aim to build a binary
% tree of the data that successively merges similar groups of samples.  It
% comprises agglomerative (or bottom-up) methods, which proceed by sequentially
% combining $n$ objects into groups; and divisive (or top-down) methods, which
% recursively separate $n$ objects into finer groupings.  The results of
% hierarchical clustering are presented in a dendrogram.

% The \emph{k}-means algorithm works by iteratively assigning each sample to one
% of $k$ clusters, each being defined as the set of samples showing the lowest
% dispersion from the respective centroid (the multidimensional version of the
% mean).  Because the number $k$ of clusters has to be defined \emph{a~priori},
% \emph{k}-means are an example of supervised clustering.  Whenever the mean of
% objects can not be computed, cost functions can be used to define a
% \emph{representative object} (the medoid, or the multidimensional representation
% of the median), as is the case in the partitioning around medoids algorithm, or
% \smallcaps{pam}.

% Self-organization maps, or \smallcaps{som}, are a group of algorithms inspired
% by neural networks of the brain.  Their aim is to produce a two-dimensional,
% discretized representation of a high dimension input space, called a map.  This
% is achieved by associating a weight vector with the same dimension as the
% feature space to each of a set of spatially distributed components, called nodes
% or neurons.  The algorithm then trains a model that optimizes the projection of
% the feature space onto the map, by iteratively finding the node whose weight
% vector shows the minimal distance to each feature space vector.  Clusters are
% then defined as groups of objects from the input space that are projected onto
% the same node.

% However supervised or unsupervised, learning techniques require the selection of
% a measure of distance between, or similarity among, the objects to be classified
% or clustered.  Distances are functions that satisfy at least three properties:
% (\emph{a})~non-negativity, $d(x,y)\ge 0$; (\emph{b})~symmetry, $d(x,y)=d(y,x)$;
% and (\emph{c})~identification mark, $d(x,x)=0$.  Examples of metrics used in the
% computation of distances between the vectors of gene expression between $m$
% expression profiles include the Euclidean distance:
% \begin{equation}
%   \label{eq:euclidean-distance}
%   d_{euc}=\sqrt{\sum_{i=1}^{m}(x_i-y_i)^2},
% \end{equation}
% and the Manhattan distance:
% \begin{equation}
%   \label{eq:manhattan-distance}
%   d_{man}=\sum_{i=1}^{m}\left|x_i-y_i\right|.
% \end{equation}
% % Many options are available in selection of a distance for machine learning
% % tasks.
% Correlation-based distance measures have been widely used in the microarray
% literature.\cite{eisen_cluster_1998} They include one minus the standard Pearson
% correlation coefficient, Spearman’s rank correlation, and Kendall’s~$\tau$.

% In microarray
% data analysis, it can be used as a feature transformation methodology, through
% the selection of

% in microarray data analysis,
% and can also be used to perform feature transformation.

\subsection{Machine learning analysis}
\label{sec:methods-machine-learning}
% 05Mar2015
Machine learning refers to computational and statistical inference processes
employed to create, on the basis of observational data, reusable algorithms for
prediction.\cite{gentleman_bioinformatics_2006} An essential property of a
learning algorithm is generalization.  In the context of machine learning,
generalization refers to the ability of the trained classifier to perform
accurately on new samples.  In microarray data analysis, this means identifying
expression features related to the underlying biology of the phenotypic classes
under analysis.

Several classification and feature selection methods have been co-opted for the
identification of differentially expressed genes in microarray
data.\cite{pirooznia_comparative_2008} In this dissertation, we made use of
\href{http://www.bioconductor.org/}{\textsf{Bioconductor}} implementations of
two supervised machine learning algorithms in order to optimize expression
feature selection towards two clinically challenging diagnostic problems in
thyroid cancer.

Support vector machines, or \smallcaps{svm}, are classifiers that operate by
mapping input vectors into a high dimensional feature space in a non-linear
fashion.  Linear decision surfaces, or hyperplanes, are then constructed, which
can be used to separate classes of data.  The optimal hyperplane is the linear
decision function with maximal margin between the vectors of the considered
classes.  Margins can be functionally described by a subset of the training
data, the so-called \emph{support vectors}.\cite{cortes_support-vector_1995}
% are an example of a non-probabilistic binary linear classifier.  They seek to
% find the function that maximizes the distance between training samples of
% distinct categories when projected on a hyperplane
Random forests, or \smallcaps{rf}, are classifiers consisting of a collection of
decision trees grown from the features that are most discerning in the training
set.\cite{breiman_random_2001} They operate under the assumption that, in the
feature expression space of microarray data, while each individual classifier is
a weak learner, the ensemble of all classifiers taken together produce a strong
learner.  Both of these algorithms were employed with a feature-selection step.
Feature selections and algorithm parameters were nested in an inner
cross-validation loop to preclude any possibility of parameters and
feature-selection biases.\cite{johannes_mcrestimate:_2010}

Classifiers were thusly trained on \sfrac{9}{10} of our samples and used to
predict the remaining \sfrac{1}{10}.  They were tuned on \sfrac{9}{10} of the
training samples, and optimized by comparing the prediction on the remaining
\sfrac{1}{10} of the training samples.  For each sample, our implementations of
the algorithms returns, for \smallcaps{rf}, a percentage of decision trees
votes, and for \smallcaps{svm}, a real score between $-1$ and $1$.  This loop
was repeated 10 times and the results were averaged.  Those scores were then
used as a diagnostic test to compute receiver characteristic operating curves.

\begin{marginfigure}%
    \includegraphics{roc-curve.pdf}
    \caption[Example of a receiver operating characteristic (\smallcaps{roc}
    curve)]{Receiver operating characteristic curves (\smallcaps{roc}) are
      visual representations of the trade-off between sensitivity (the
      proportion of actual positives which are correctly identified) and
      specificity (the proportion of true negatives correctly identified) of a
      diagnostic test, or classifier.  % The area under the \smallcaps{roc} curve
      % (\smallcaps{auc}) is equal to the probability that a classifier will
      % classify a sample correctly.
      The ability to superimpose different \smallcaps{auc}s on the same plot
      allows for direct comparison of different classifiers.}
    \label{fig:roc}%
\end{marginfigure}

\subsection{Receiver operating characteristic curves}
\label{sec:methods-roc}
% 06Mar2015

Receiver operating characteristic curves, or \smallcaps{roc} curves, are a tool
for diagnostic test evaluation that allows for the creation of a sensitivity and
specificity report.\cite{fawcett_introduction_2006} A typical \smallcaps{roc}
curve plots the true positive rate (sensitivity), as a function of the false
positive rate ($1-\text{specificity}$), for different cut-off points of a
parameter produced by a classifier model.  Each point on the
\smallcaps{roc} curve represents a sensitivity/specificity pair, corresponding
to a particular decision threshold (Figure~\ref{fig:roc}).

The area under the curve, or \smallcaps{auc} (0 < \smallcaps{auc} < 1), can be
interpreted as the probability that the predictor of choice will rank a randomly
chosen positive instance higher than a random negative
one.\cite{hanley_meaning_1982}
% represents the overall probability that the phenotype being investigated of a
% randomly chosen subject is correctly identified by the test
The \smallcaps{auc} metric allows for direct comparison of classifier
performance.  A \smallcaps{auc} of 50\% represents a chance of correct
classification no better than a random classifier.

% \subsection{Gene set enrichment analysis}
% \label{sec:methods-gsea}
% 07Mar2015

\subsection{Survival analysis}
\label{sec:methods-survival-analysis}

Survival analysis is a collection of statistical procedures for data analysis
for which the outcome variable of interest is time until the event
occurs.\cite{kleinbaum_survival_1996} In follow-up studies of cancer patients,
survival analysis is used to model association of the expression of genomic
markers in cancer biospecimens with the time until a given clinical outcome is
observed.

Clinical outcomes may include death of the patient (overall survival, or
\smallcaps{os}), death of the patient caused by the cancer (disease-specific
survival, or \smallcaps{dss}), the finding of new metastases in the patient
(distant metastasis free survival, or \smallcaps{dmfs}) or recurrence of the
cancer (disease-free survival, or \smallcaps{dfs}).  \emph{Survival time} refers
to the lapse of time since the beginning of the study up to the moment when an
event is observed, regardless of the clinical outcome considered.  Whenever the
information about an individual's survival time is incomplete, that observation
is said to be censored (Figure~\ref{fig:censorship}).  This may be due because
the event was not observed by the end of the study (in which case the follow-up
time considered for that patient is the entire duration of the study) or because
the patient quit or withdrew from the study before its end (in which case the
follow-up time considered for that patient is the time up to dropping out).

\begin{marginfigure}[-5cm]%
  \includegraphics{censorship-28Feb2015.pdf}
  \caption[Right-censored survival data]{A schematic representation of
    right-censored survival data.  Survival time is said to be
    \emph{right-}censored when the information regarding the right side of the
    follow-up period is incomplete.  Observed events are denoted by (\CIRCLE).
    Censored observations are denoted by (\Circle).  Notice that patient
    \smallcaps{b} is also censored, as no event had been observed by the end
    of the study (see text for details).}\label{fig:censorship}%
\end{marginfigure}

Follow-up studies are not amenable to ordinary regression models because the
time to event is typically not normally distributed and these models cannot
incorporate censoring data.  Instead, survival analysis uses two functions to
estimate survival time, the \emph{survival} function and the \emph{hazard}
function.

The survival function, $S(t)$, describes the probability of an event occurring
later than some specified time $t$:
\begin{equation}
  \label{eq:survival-function}
  S(t) = P"r(T > t),
\end{equation}
where $T$ is a random variable for a patient's survival time.

The hazard function, $h(t)$, describes the instantaneous potential per time unit
for the event to occur, given the patient has survived up to time $t$:
\begin{equation}
  \label{eq:hazard-function}
  h(t) = \lim_{\Delta t \to 0}\frac{P"r\{t \leqslant T \less t + \Delta t
    \mid T \geqslant t \}}{\Delta t}.
\end{equation}

% While the survival function is non-increasing i.e., $S(t)$ heads downwards as
% $t$ increases, (Figure~\ref{fig:survival-function}), the hazard function
% describes a failure rate conditional to the interval of time considered, and
% can therefore be modeled by any number of distributions.

Both functions are related to each other.  However, while the survival function
is non-increasing (Figure~\ref{fig:survival-function}), the hazard function may
be modeled by any number of distributions, as it describes a failure rate
conditional to the interval of time considered.  Building on these functions,
survival analysis stipulates a suite of parametric, non-parametric and
semi-parametric methods to make inferences over survival time.  These methods
can then be used to ascertain the relationship between a variable of interest
and the time to a clinical outcome.

\begin{marginfigure}[-5cm]%
  \includegraphics{survival-function-28Feb2015.pdf}
  % \caption[Survival function]{The survival function, \emph{S}\,(\emph{t}),
  \caption[Survival function]{The survival function, $S(t)$ describes the
    likelihood that a patient will have a lifetime exceeding time $t$.
    \textbf{A:}~The theoretical distribution is non-increasing, and
    % characterized by \mbox{\emph{S}\,(0) = 1} and \mbox{\emph{S}\,($\infty$) =
    %   0}.
    characterized by $S(0)=1$ and $S(\infty)=0$.  \textbf{B:}~In practice, the
    % estimated survival function, \emph{\^{S}}\,(\emph{t}), often takes a shape
    estimated survival function, $\hat{S}(t)$, often takes the shape of a step
    function.  Because study periods are never infinite and there may be
    competing risks for failure, it is likely that not all patients will
    experience a clinical outcome by the end of the
    study.}\label{fig:survival-function}%
\end{marginfigure}

In the biomedical literature, the most recognizable non-parametric method to
estimate the survival function is the Kaplan-Meier
estimator.\cite{kaplan_nonparametric_1958} The Kaplan-Meier estimator is defined
as the probability of surviving in a given length of time while considering time
in many small intervals.\footnote{\citealp[pp. 365--93]{altman_practical_1990}}
It estimates the probability of occurrence of an event at time $t$ by
cumulatively multiplying prior probabilities of survival at preceding $t_i$
intervals.  The Kaplan-Meier, or product limit estimator, is thus formulated as:
\begin{equation}
  \label{eq:kaplan-meier}
\hat{S}(t)=\prod_{t_i<t}\frac{n_i-d_i}{n_i},
\end{equation}
where $n_i$ and $d_i$ are, respectively and for each prior time interval $t_i$,
the number of patients at risk (right-censored observations removed), and the
number of patients experiencing an event.  The Kaplan-Meier estimator can be
used to obtain univariate descriptive statistics for survival data or to compare
the survival time for two or more groups of subjects.

The logrank test is a non-parametric hypothesis test to compare the survival
distribution of two
samples.\cite{mantel_evaluation_1966,peto_asymptotically_1972} It challenges the
null hypothesis that there is no difference between the survival functions
underlying each observed population. % the populations in the
% probability of an event at any time point.
To do so, it compares the estimates
of the hazard functions of the two groups at each observed event time.  The
logrank test is based on the same assumptions as the Kaplan-Meier survival
curve---namely, that censoring is unrelated to prognosis, the survival
probabilities are the same for subjects recruited early and late in the study,
and the events happened at the times specified.\cite{bland_logrank_2004}
Importantly, both the Kaplan-Meier estimator and the logrank test are able to
incorporate right-censored survival data.

When modeling the presence of covariates or explanatory variables to explain
survival time, fully parametric and semi-parametric approaches are available.
Parametric approaches, like the accelerated life class of models, assume that
the effect of covariates is proportional with respect to survival
time.\cite{kalbfleisch_statistical_2011} Under this model,
\begin{equation}
  \label{eq:aft}
  S_1(t) = S_0(t/\gamma).
\end{equation}
Effectively, this means that the probability that a member of group one will be
alive at time $t$ is exactly the same as the probability that a member of group
zero will be alive at time $t/\gamma$.  In parametric models, the estimation of
the covariates is conditional to the prior definition of the hazard
distributions in both groups.

% \footnote{For instance, for $\lambda = 2$, this would be half the age, so the
% probability that a member of group one would be alive at age 40 (or 60) would
% be the same as the probability that a member of group zero would be alive at
% age 20 (or 30). Thus, $\lambda$ can be interpreted as affecting the passage of
% time.  In this example, people in group zero age ``twice as fast''.}

Alternatively, the Cox proportional hazards family of models assumes that the
effects of the covariates is proportional with respect to the
hazard.\cite{cox_regression_1972} This assumption obviates the need of
specifying the underlying hazard functions---the model only seeks to fit the
regression parameters, hence being referred to has semi-parametric.  In its
simplest form, it can be formulated in such a way that the hazard at time $t$
for an individual with covariates $X$ is assumed to be:
\begin{equation}
  \label{eq:proportional-hazards}
  h(t \mid X)=h_0(t) \exp(\beta_1 X_{1}+\ldots+\beta_pX_{p}).
\end{equation}
This model formulates the hazard of individual $i$ at time $t$ as the product of
a baseline hazard function, $h_0(t)$, and of a linear function of a set of $p$
covariates, which is exponentiated.  Along with the specification of the model,
Sir David Cox developed a maximum partial likelihood method for estimation
of its covariates.\cite{cox_regression_1972} A logrank statistic can be
derived as the score test for the Cox proportional hazards model comparing two
groups.  The term Cox regression refers to the combination of both the model and
the estimation procedure.  It has become the tool of choice to estimate the
association of the expression of genomic metagenes with differential survival
times in cancer follow-up studies.

% An important advantage The modeling of the survival time by the Kaplan-Meier
% can also incorporate right-censored data

% Parametric methods assume that the underlying distribution of survival times
% follow certain known probability distributions.  For instance, the hazard
% function can be assumed to:
% \begin{itemize}
% \item be constant over time, e.g., when considering the likelihood of an healthy
%   individual to become ill;
% \item follow an increasing Weibull distribution, e.g., when considering the
%   likelihood of death for cancer patients not responding to treatment---as it is
%   expected to increase with time;
% \item follow an decreasing Weibull distribution, e.g., when considering the
%   likelihood of death for post-surgical cancer patients with fair prognosis---as
%   it is expected to decrease with time;
% \item follow a lognormal distribution, e.g., when considering the likelihood of
%   death for tuberculosis patients---as it first increases upon diagnosis, only
%   to decrease later.
% \end{itemize}
% Once the parametric function of choice is defined, model parameters
% are usually estimated using an appropriate modification of a maximum likelihood
% procedure.

% be constant over time (e.g., when considering the likelihood of an healthy
% individual of becoming ill); follow an increasing Weibull distribution (e.g.,
% when considering the likelihood of death for cancer patients not responding to
% treatment, as it is expected to increase with time); follow a decreasing
% Weibull distribution (e.g., when considering the likelihood of death for
% post-surgical cancer patients, as it is expected to decrease with time), or
% follow a lognormal distribution (e.g., when considering the likelihood of
% death for tuberculosis patients, as it first increases after diagnosis, then
% decreases).

% In follow-up studies of cancer patients, it is generally of interest to
% describe the relationship of the expression of a biologically motivated
% metagene to the time to event---possibly in the presence of clinical
% covariates, such as age, treatment, or tumour class.  This calls for models
% that describe the relationship between the predictor variable and survival
% time.  Survival analysis offers a range of parametric, non-parametric and
% semi-parametric approaches to model this relationship.

\medskip

When considering the association of genomic markers with survival time, the
dependent variable is composed of two parts: one consisting of the time to the
event (or lack thereof), encoded as a non-negative number; the other registering
the event status, encoded as a binary variable (routinely 1 if the event is
observed, and 0 if not).

Typically, the predictor is also a binary class, the result of discretizing the
patients in groups of good and bad prognosis as a function of the metagene
classifier.  This requires a method to stratify the cohort with an unsupervised
classification procedure.  Venet et al.\cite{venet_most_2011}\,investigated
three methods to this goal: (\emph{a})~the standard spliting of the cohort along
the two main clusters defined by hierarchical clustering of the metagene's
expression data;
% \footnote{Computed with the \texttt{hcluster} function of the \textsf{amap}
% \textsf{R} package, and implemented with average linkage and (1
% -- \emph{correlation}) between pairs of observations as the dissimilarity
% measure.}
(\emph{b})~spliting the cohort along the two main clusters defined
by applying the \emph{k}-means clustering algorithm to the metagene's expression
data;
% \footnote{Computed with the \textsf{R} function \texttt{kmeans}.}
and (\emph{c})~spliting the cohort along the median of the metagene's first
principal component.
% \footnote{Computed with the \textsf{R} function \texttt{prcomp}.}
They confirmed that the methods based on the first principal
component % (henceforth referred to as \textsf{PC1} method)
yielded significantly higher hazard ratios and smaller \emph{p}-values.

In this dissertation, we chose to model survival time as a function of a single
\emph{continuous} predictor, defined by the first principal component of the
metagene's expression (or by a single vector of gene expression) throughout the
entire cohort.  This departure from the standard approach has the disadvantage
of making it impossible to calculate a Kaplan-Meier estimator for the
model---therefore lacking a visual representation and making it harder to assess
the validity of the proportional hazards assumption.  However, it has the
advantage of using a predictor variable that faithfully mirrors the patterns of
expression captured by the metagene and of not requiring the artificial
stratification of the cohort in (balanced or not) groups of differential
prognosis.

Whenever applicable, association of metagene expression with outcome was thusly
computed by fitting a proportional hazards regression model between predictor
and dependent variables with the \texttt{coxph} function of the
\textsf{Bioconductor} \textsf{survival} package.

% We have shown this approach to be more sensitive (data not shown).

%%% Local Variables:
%%% TeX-engine: xetex
%%% mode: latex
%%% TeX-master: "../../thesis"
%%% End:
