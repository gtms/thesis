This chapter is comprised of three sections.  The first consists of a published
article, entitled ``A general method to derive robust organ-specific gene
expression-based differentiation indices: application to thyroid cancer
diagnostic.''\cite{tomas_general_2012} The results exposed in this manuscript
are developed in the
\hyperref[discussion-differentiation-microarrays]{\textsf{Discussion}} chapter,
under the section \emph{Differentiation and proliferation signatures in cancer
  diagnostic}.

The second section, entitled ``The extent of prognostic signals in the cancer
transcriptomes,'' details the results of a systematic analysis of 114 microarray
datasets of cancers affecting 22 different tissue types, aiming at quantifying
the range and the nature of the signals linked with differential survival in
neoplastic expression profiles.  These results are elaborated upon in the
corresponding section of the
\hyperref[sec:discussion-prognostic-microarrays]{\textsf{Discussion}} chapter.

The remaining section reports other publications to which the author contributed
during this dissertation.

% The results of this analysis are currently under revision.

% \resetcitations

\section{Differentiation and proliferation signatures in cancer diagnostic}
\label{sec:results-differentiation-proliferation}

\emph{See next page.}

% \includepdf[pages=1-9,fitpaper=true]{../pdf/include/onc-tomas-2012.pdf}
\includepdf[pages=1-9,trim=5cm 3cm 5cm 3cm]{../pdf/include/onc-tomas-2012.pdf}
% \includegraphics{../pdf/include/onc-tomas-2012.pdf}

\section{The extent of prognostic signals in the cancer transcriptomes}
\label{sec:results-prognostic-survival}

The vast majority of mechanistic cancer studies are performed in animals and/or
\emph{in vitro} models for ethical reasons.  Proving that a biological mechanism
established in these experimental systems contributes to cancer progression in
human requires clinical trials, which are costly and take years to perform.
Therefore, researchers have relied on weaker correlative evidence to back the
relevance of their findings to human diseases.  One such evidence is the
statistically significant association of a molecular maker of the biological
mechanism under investigation with disease outcome in human.  Such association
can be established with non-interventional clinical studies and has been
extensively used in the past decades.  Moreover, this approach has recently
gained popularity with the availability of web
servers\cite{gyorffy_online_2010,ringner_gobo:_2011,gyorffy_online_2013} that
provide, for free, association statistics between any single- or multi-gene
markers and cancer outcomes using publicly available cancer transcriptome
databases.

To retain a biological bearing, the reported association must however be
specific of the biomarker under analysis and not reflect a global property of
the transcriptome.  Most studies estimate the association with a log-rank test
based on a Cox Proportional Hazards model and use a Kaplan-Meier curve to
visualize it.  None of these tools control for the possibility that a large
fraction of transcriptome could be associated with outcome and that, therefore,
the prognostic signal is non-specific.

Disturbingly, this possibility has been proven true in
bladder\cite{lauss_prediction_2010} and
breast\cite{ein-dor_outcome_2005,mosley_cell_2008,venet_most_2011} cancers.  In
particular, our group has shown\cite{venet_most_2011} that more than half of the
transcriptome is correlated with proliferation in two breast cancer cohorts and
that one out of four single genes and that nine out of ten random signatures
comprised of more than 100 genes were associated with overall survival at
$p<0.05$.  As a consequence, most published signatures were found to be
significantly prognostic, but no more prognostic than random sets of genes.

To investigate whether the pervasive association of the transcriptome with
outcome extends to other cohorts and other types of cancers we compiled 114
published gene-expression datasets with patient follow-up from the
\smallcaps{geo}\cite{edgar_gene_2002}, InSilicoDB\cite{coletta_insilico_2012}
and the \smallcaps{tcga} Research Network
databases.\footnote{\href{http://cancergenome.nih.gov/}{http://cancergenome.nih.gov/}}
These included human cancers from 19 organ systems, and a representative range
of microarray platforms (Table~\ref{tab:datasets}).

\begin{figure}%
  \includegraphics{fraction-significant-tests-base-R-graphics-15Jan2015.pdf}
  \caption[Prognostic content in human cancers]{Prognostic content in human
    cancers.  The fraction of markers significantly associated with outcome at
    logrank $p<0.05$ is shown for single-gene markers, multi-gene markers from
    the \mbox{MSigDB \smallcaps{c2}} database, and randomly generated
    multi-genes markers devoid of biological meaning.  The latter was used to
    order the datasets. Fractions were computed from death-related end-points
    when available (\smallcaps{os} or \smallcaps{dss}), or relapse otherwise
    (\smallcaps{dfs} or \smallcaps{dmfs}). The dotted red line marks the 0.05
    threshold.}
  \label{fig:prognostic-fraction}%
\end{figure}

For each dataset, we computed the fraction of genes associated with outcome at
log-rank $p<0.05$ (Figure~\ref{fig:prognostic-fraction}).  This quantity may be
viewed as the probability of observing a significantly prognostic single gene
marker by chance alone.  We thus refer to these estimates as the baseline
prognostic content\footnote{Prognostic content will henceforth refer to the
  fraction of tested markers found associated with outcome in a given dataset.}
of cancer transcriptomes.  Overall, the median prognostic content across all
studies for single-gene markers was 12\%.  In 100 of the 114 datasets
analyzed (88\%), more than five percent of single-gene markers showed a
significant association with outcome.  % It was higher than 5\% and random %
% multi-gene markers was, respectively, 12\% and 16\%.                     %
% The median fraction was 9%. It was higher than 5% for 95 (83%) datasets.
To take an extreme example, an investigator measuring the association of a gene
with outcome in the \smallcaps{kirc} datasets (kidney cancer) has a 50\% chance
to obtain a positive result---a value far above the canonical 5\% significance
threshold.

To investigate recent multi-gene approaches, we ran a similar calculation for
each of the 4722 curated gene sets of the \mbox{MSigDB \smallcaps{c2}}
database\cite{liberzon_molecular_2011} (Figure~\ref{fig:prognostic-fraction}).
% The prognostic fraction was larger for signatures than for single genes, with
% a median of 12%.
The prognostic content was larger for multi-gene markers than for single-gene
markers, with a median of 19\%.  The prognostic content for multi-gene markers
was larger than 5\% in 76\% of the datasets (87 out of 114), larger than 20\% in
48\% (55/114), and larger than 50\% in 19\% (22/114) of the datasets.  To
control for possible biases of \mbox{MSigDB \smallcaps{c2}} towards
oncology-related signatures, we reran the same computation, but replacing each
signature by a similarly sized set of randomly selected genes.  The overall
qualitative result is unchanged (Figure~\ref{fig:prognostic-fraction}).

Intriguingly, single-gene prognostic content was found to be largely
heterogeneous across datasets related to the same organ system.  For example, it
ranged from 4\% to 59\% among the 33 breast cancer datasets analyzed.  To
investigate the contributions of potential biological and demographic
dataset-specific factors to this effect, we quantified single-marker prognostic
fraction for re-samplings of the \num{1972} transcriptomes of the
\smallcaps{metabric} breast cancer cohort, regarding modulations of four
variables: sample size, duration of follow-up time, fraction of \smallcaps{er}+
patients, and fraction of node positive patients
(Figure~\ref{fig:bootstrap-metabric}).  We chose \smallcaps{metabric} for this
analysis because it is one of the largest cohorts of cancer patients with
follow-up and extensive clinical annotation data available in the public domain.

\begin{figure*}
  \includegraphics{metabric-sims-19Mar2015.pdf}
  \caption[Bootstrapping experiments on the \smallcaps{metabric}
  dataset]{Bootstrapping experiments on the \num{1972} combined breast cancer
    transcriptomes of the \smallcaps{metabric} dataset.
    \textbf{A--}Distribution of the prognostic fraction in 100 samplings of 500
    expression profiles, out of the total 1972 \smallcaps{metabric} profiles.
    \textbf{B--}Effect of sample size.  \num{100} samplings (grey points) were
    assessed for each specified sample size.  \textbf{C--}Effect of follow-up
    times.  For each time $t$, \num{100} samplings were assessed for which
    patients beyond time $t$ were considered censored at time $t$.
    \textbf{D--}Effect of the fraction of \smallcaps{er}+ patients.  \num{100}
    samplings, each of 300 patients, were assessed with the respective fraction
    of \smallcaps{er}+ samples.  \textbf{E--}Effect of the fraction of node
    positive patients.  \num{100} samplings, each of 500 patients, were assessed
    with the respective fraction of node positive patients.}
  \label{fig:bootstrap-metabric}
\end{figure*}

The estimates of prognostic content were found to be markedly sensitive to
sampling variance, as suggested by the dispersal of the distribution of
estimates (confidence interval: 12\% to 23\%), when \num{100} samplings of
\num{500} random transcriptomes were examined for the fraction of genes
associated with outcome (Figure~\ref{fig:bootstrap-metabric}\emph{a}).  This
feature alone is likely to yield a significant contribution to the range of
estimates computed in our meta-analysis, as 108 of the 114 datasets analyzed
included less than \num{500} profiled tumours.  Moreover, the sensitivity of our
estimation procedure is, unsurprisingly, largely dependent on the number of
profiles included in the analysis, as shown by a bootstrapping experiment of
sample sizes towards the assessment of the prognostic fraction
(Figure~\ref{fig:bootstrap-metabric}\emph{b}).  Provocatively, the estimates of
prognostic content do not appear to level off even when the experimental sample
size reached \num{1750}---by far the largest in the field.  A sampling
experiment of sequential truncation of follow-up times was equally shown to
impact estimates of prognostic content
(Figure~\ref{fig:bootstrap-metabric}\emph{c}).  Sharply increasing predictive
fractions were observed up to the fifth year of follow-up, followed by a gradual
decrease of the estimates for higher follow-up times.  This trend suggests that,
in breast cancers, prognostic patterns of expression are optimally correlated
with short-term forms of progression of the disease, and that long-term forms of
progression are less efficiently predicted from primary tumour transcriptomes.
The modulation of the fraction of \smallcaps{er}+ transcriptomes towards
experimental samplings of our estimate has exposed a tendency congruent with the
clinical relevance of this receptor in breast cancer pathology
(Figure~\ref{fig:bootstrap-metabric}\emph{d}).  Thus, an increase of
\smallcaps{er}+ profiles to up to 50\% in our samplings leads to a corresponding
linear rise in estimates of prognostic content, at which point a further
increase in the proportion of \smallcaps{er}+ profiles yields little impact on
fraction estimates.  This observation is in line with the fact that the
predictive power of most signatures in breast cancer is mostly confined to
\smallcaps{er}+ phenotypes.\cite{weigelt_challenges_2012} A last sampling
experiment with increasing fractions of profiles from node positive patients
(Figure~\ref{fig:bootstrap-metabric}\emph{e}) also revealed an increasing
pattern of prognostic fraction estimates.  This trend could be explained by the
fact that nodal status is clinically correlated to \smallcaps{er} status in
breast cancer.

% First, repeated random selection of profiles reveals that sampling effects are   %
% substantial (prognostic fraction CI=6-17%) with 500 patients                     %
% (Fig. 2A), a size in par with the largest studies. Second, the prognostic        %
% fraction, as expected, depends on sample size (Fig. 2B). Interestingly, it shows %
% no sign of inflection as N reaches 1750, by far the largest sample size in the   %
% field. Third, we investigated the impact of follow-up time by artificially       %
% truncating follow-up data (Fig. 2C; Supplementary Methods). Shorter and longer   %
% maximum follow-up times yielded lower prognostic fractions, consistent with the  %
% fact that incomplete data, but also late events12, are not predictable from      %
% primary tumor transcriptomes. Fourth, increasing the fraction of ER+ patients up %
% to 50% increases the prognostic fraction, but has no effect beyond               %
% 50% (Fig. 2D). This is compatible with the notion that                           %
% transcriptional marker are more prognostic within the ER+ group and that the ER+ %
% patients have, on average, a better prognostic than ER- patients.13 Fifth,       %
% increasing the fraction of node positive patients, which is correlated with ER   %
% status, also increases the prognostic fraction (Fig. 2E). Sixth, lower           %
% cellularity is unexpectedly associated with a higher prognostic fraction         %
% (Fig. 2F). This may reflect the role of the microenvironment in breast cancer    %
% progression and the massive impact of cell types proportions on the              %
% transcriptomes of bulk tissues.14                                                %

Finally, dataset-specific processing details may also distort prognostic
estimates.  Consider, for instance, dataset \smallcaps{gse9893}, which exhibits
the highest prognostic fractions measured in our study
(Figure~\ref{fig:prognostic-fraction}).  A thorough reanalysis of this dataset,
detailed in the following section, reveals that its normalization was
performed in two batches and induced massive spurious correlations between
global values of expression and survival outcome.  Accordingly, a proper
single-batch normalization of the raw expression data restores a signal-to-noise
metrics to comparable values with other datasets, and decreases measurements of
prognostic content from 59\% to 19\% for \smallcaps{gse9893}.

We have shown that the fraction of prognostic single- and multi-gene biomarkers
is greater that 5\% in the majority of publicly available transcriptome
datasets.  Furthermore, we have demonstrated that the probability of a
significant single- or multi-gene marker association with outcome depends on
cohorts’ demographics, but also to a large extent on technical factors that
include sampling effects, cohort size, patient follow-up protocols, protocol
randomization and possibly other factors not addressed here.  These findings
call for the reappraisal of conclusions made by previous studies pertaining to
the implication of biological mechanisms to human cancer based on associations
of biomarkers with outcome---including low-throughput \smallcaps{pcr}-based
studies.  They also call for study-specific controls akin to those presented in
Figure~\ref{fig:prognostic-fraction} in future studies.

However, a biomarker does not need to convey relevant biological information
regarding the course of disease in order to be useful in the clinic.  Therefore,
our results have no bearing on the clinical utility of published biomarker
associations.

\subsection{Re-analysis of dataset \smallcaps{gse9893}}
\label{sec:reanalysis-gse9893}
% The following vignette details the re-analysis of the extent of prognostic
% signals in a breast cancer dataset and exposes how a normalization artifact
% can distort estimations of prognostic content.
To illustrate how procedural biases in the preprocessing of expression profiles
may impact estimates of association to outcome, we present here a detailed
re-analysis of dataset \smallcaps{gse}9893 (Table~\ref{tab:datasets}).

\smallcaps{gse}9893 is comprised of 155 samples of tamoxifen-treated primary
breast cancers.  These samples were hybridized on a homemade \mbox{70-mer} chip
containing \SI{22680} probes, mapping to \SI{21329} human specific genes.  The
original experiment was carried out to look for a gene expression signature to
predict the recurrence of tamoxifen-treated primary breast
cancer.\cite{chanrion_gene_2008}

The data-set was downloaded from \smallcaps{geo} with the \textsf{Bioconducor
  GEOquery} package, with original normalization.  The expression matrix was
then feature collapsed using a \textsf{maxMean} routine and median polished.

Among the 114 studies considered in our analysis, \smallcaps{gse}9893
shows the highest fraction of genes associated with outcome at
$p < 0.05$ (59\%).  Interestingly, nearly all \mbox{MSigDB
  \smallcaps{c2}} signatures appear associated with outcome in this
dataset (Table~\ref{tab:top-datasets}).

\begin{table}[ht]
  \begin{center}
    % \footnotesize%
    \begin{tabular}{lcc}
      \toprule
      Dataset                                  & Fraction of significant tests & Event           \\
      \midrule
      \smallcaps{GSE}9893-breast               & 0.958                         & \smallcaps{os}  \\
      \smallcaps{GSE}10846-lymphoma            & 0.856                         & \smallcaps{os}  \\
      \smallcaps{GSE}32894-bladder             & 0.847                         & \smallcaps{dss} \\
      \smallcaps{GSE}31210-lung-adenocarcinoma & 0.838                         & \smallcaps{dfs} \\
      \smallcaps{KIRC}                         & 0.821                         & \smallcaps{os}  \\
      \smallcaps{GSE}41258-colon               & 0.814                         & \smallcaps{dss} \\
      \bottomrule
    \end{tabular}
  \end{center}
  \caption[Top six studies with hightest fraction of \mbox{MSigDB
    \smallcaps{c2}} signatures associated with outcome]{Top six studies with
    highest fraction of \mbox{MSigDB \smallcaps{c2}} signatures associated with
    outcome.  Detailed information regarding each dataset can be found on to Table~\ref{tab:datasets}.}
  \label{tab:top-datasets}
\end{table}

A closer inspection of the metadata associated with the expression profiles
reveals that the arrays were scanned in two discrete time intervals during 2005
and 2006, separated by eight months (Figure~\ref{fig:array-dates}).
Surprisingly, patients whose tumours were hybridized in 2006 show a poorer
prognosis than those hybridized in 2005 (Figure~\ref{fig:km-gse9893-prior}).

\begin{marginfigure}%
  \begin{center}
    \includegraphics{gse9893-array-dates-23Feb2016.pdf}
    \caption[Time distribution of hybridization dates of the samples in
    \smallcaps{gse}9893]{Frequency distribution of hybridization dates of
      \smallcaps{gse}9893 samples relative to the first hybridization date.
      Information regarding date of hybridization of each of the 155 arrays was
      parsed from the \texttt{gpr} files downloaded from \smallcaps{geo}.  The
      dataset is composed by a batch of samples hybridized during May to July
      2005 and a second batch of samples hybridized during February and March
      2006, roughly 200 days apart.}
    \label{fig:array-dates}
  \end{center}
\end{marginfigure}

\begin{marginfigure}%
  \begin{center}
    \includegraphics{gse9893-km-prior-23Feb2016.pdf}
    \caption[Kaplan Meyer of dataset \smallcaps{gse}9893 discretized by time
    batches]{A Kaplan Meier visualization of differential overall survival,
      between patients included in \smallcaps{gse}9893 whose samples were
      hybridized in 2005 (in black) and those whose samples were hybridized in
      2006 (in red). Logrank test: $p = 1.76^{-10}$.}
    \label{fig:km-gse9893-prior}
  \end{center}
\end{marginfigure}

This observation can be explained by two facts.  First, we discovered a
normalization artifact in this dataset related to the 2005 and 2006 batches
of samples, as shown by the distribution of expression values and respective
batch associations with the first and second principal components of the global
expression matrix (Figure~\ref{fig:gse9893-prior-post-norm}, left panels).
Second, there is an enrichment in the 2006 batch of observed death events
compared with the 2005 batch (Figure~\ref{fig:km-gse9893-density-survival}).
Because the majority of observed events are linked with a subset of samples
whose global expression patterns were distorted due to the normalization
artifact, 59\% of genes in original matrix appear artificially associated with
overall survival in this dataset.

In order to correct for this bias, we downloaded the raw data \texttt{gpr} files
and proceeded to re-normalize them with the \textsf{Bioconductor limma} package.
As a result, the distribution of values of expression no longer showed a
correlation between batches of samples
(Figure~\ref{fig:gse9893-prior-post-norm}, right panels).  In addition, a
signal-to-noise quality metric based on gene-gene correlations across expression
profiles\cite{venet_measure_2012} suggests that the re-normalization of the
raw-data has significantly improved the data quality of \smallcaps{gse}9893
(Figure~\ref{fig:gse9893-snr}).  As a result, the fraction of genes associated
with outcome in this study is reduced from 58\% to 19\%, and only 74 out of the
original 4556 (2\%) \mbox{MSigDB \smallcaps{c2}} remain associated with overall
survival.

\begin{figure}%
  \begin{center}
    \includegraphics{gse9893-prior-post-normalization-28Feb2016.pdf}
    \caption[Batch effect in \smallcaps{gse}9893 prior and post
    re-normalization]{\textbf{A}--Gene expression distributions of each of the
      155 samples in \smallcaps{gse}9893. \mbox{\textbf{B}--\smallcaps{gse}9893}
      samples projected in the space of the first two principal components of
      their expression matrix.  \emph{Left,} original normalization;
      \emph{right,} quantile re-normalization on the original \texttt{gpr}
      files.  Samples in black are from the 2005 batch and samples in red are
      from the 2006 batch (See text for details).}
    \label{fig:gse9893-prior-post-norm}
  \end{center}
\end{figure}

\begin{marginfigure}%
  \begin{center}
    \includegraphics{gse9893-density-distribution-survival-events.pdf}
    \caption[Density distribution of overall survival events in
    \smallcaps{gse}9893]{Density distribution of overall survival
      events in \smallcaps{gse}9893. Censored observations are denoted
      in black and observed death events are denoted in red.  Out of
      the 116 patients in the 2005 batch, only 10 died during the
      course of the study; whereas out of the 49 patients whose
      samples where hybridized in 2006, 32 were observed events
      ($\chi^2$ test: $p = 1.42^{-12}$).}
    \label{fig:km-gse9893-density-survival}
  \end{center}
\end{marginfigure}

\begin{figure}%
  \begin{center}
    \includegraphics{gse9893-plot-hist-qual-study-1.pdf}
    \caption[Distribution of signal-to-noise quality metrics across 114 human
    cancer datasets]{Distribution of signal-to-noise quality metrics across the
      114 human cancer datasets included in our study. The
      signal-to-noise-ratios (\smallcaps{snr}) were computed with the
      \textsf{Bioconductor SNAGEE} package.  The black vertical line marks the
      value computed for \smallcaps{gse}9893.  The red vertical line shows the
      value computed for the re-normalized expression matrix of
      \smallcaps{gse}9893.  The \smallcaps{SNR} of a study is based on the
      correlation between its gene-gene correlation matrix and the expected
      matrix, and is thus a number between $-1$ and $1$.  Practically, numbers
      near or below 0 are symptomatic of seriously problematic studies
      (e.g. gene annotation problems, serious normalization issues).  Numbers
      around 20--30\% are average, depending on the platform.}
    \label{fig:gse9893-snr}
  \end{center}
\end{figure}

\clearpage

% \includepdf[pages=1-9,trim=5cm 3cm 5cm
% 3cm]{../pdf/include/supplementary-reanalysis-GSE9893.pdf}
% \includepdf[pages=1-17,trim=5cm 3cm 5cm 3cm]{../pdf/include/supplementary-reanalysis-GSE9893.pdf}

% \subsection{Datasets}
% \label{sec:results-prognostic-survival-datasets}

% A total of 114 public datasets of cancer gene expression profiles spanning 19
% types of cancer were downloaded from public repositories, manually curated and
% pre-processed for downstream analysis.  Sources for datasets include the Gene
% Expression Omnibus,\cite{edgar_gene_2002}
% InSilicoDB,\cite{taminau_insilicodb:_2011} and the \smallcaps{tcga} Research
% Network
% site.\footnote{\href{http://cancergenome.nih.gov/}{http://cancergenome.nih.gov/}}
% Individual datasets are described in the
% \hyperref[sec:methods-datasets]{\textsf{Methods}} chapter.

% \subsection{Gene sets}
% \label{sec:results-prognostic-survival-genesets}

% For each cancer dataset, we evaluated the association with clinical outcome of
% the 4722 curated gene sets from the
% \href{http://www.broadinstitute.org/gsea/msigdb/index.jsp}Molecular Signatures
% Database (\mbox{MSigDB
% \smallcaps{c2}}).\footnote{\href{http://www.broadinstitute.org/gsea/msigdb/index.jsp}{http://www.broadinstitute.org/gsea/msigdb/index.jsp}}
% \mbox{MSigDB \smallcaps{c2}} signatures are manually curated from the literature
% on gene expression and also include gene sets from curated pathways databases
% such as \smallcaps{kegg}.

% Version 4.0 of the MSigDB curated gene sets (updated on May 31, 2013) was
% downloaded and parsed to produce two lists of gene expression signatures: one
% containing the 4722 original signatures with gene symbols as identifiers, and a
% randomized list consisting of 4722 signatures of the same size as the originals,
% but composed of random samplings from the total pool of genes in the collection.

% \subsection{Association with outcome}
% \label{sec:results-prognostic-survival-association}

% Association with outcome was computed as follows.  For each cancer dataset, we
% retained the first of the following clinical outcomes available, in decreasing
% order of importance: disease-specific survival (\smallcaps{dss}); overall
% survival (\smallcaps{os}); disease-free survival (\smallcaps{dfs}); and distant
% metastasis free survival (\smallcaps{dmfs}).  The survival times for the
% selected clinical outcome specified the dependent variables.

% We then modeled the association of each gene set with survival times by fitting
% a univariate Cox proportional hazards model, using the continuous first
% principal component loading vector of the signature in the global expression
% matrix as the predictor variable.  The \emph{p}-value of the associated logrank
% test was used to estimate association of the gene set with survival times.

% A similar procedure was followed to compute the association of each single gene
% in the expression matrix with survival times, by using the respective continuous
% vector of expression as the predictor variable in the Cox model.  Here again, the
% \emph{p}-value of the logrank test was used to estimate association with
% outcome.

% \subsection{Prognostic content in human cancers}
% \label{sec:results-prognostic-survival-prognostic-content}

% Figure~\ref{fig:prognostic-fraction} shows, for each human cancer dataset
% analyzed, the fraction of \mbox{MSigDB \smallcaps{c2}} signatures, randomized
% signatures and single genes found significantly associated with outcome, at
% $p < 0.05$.

% \begin{figure}%
%   \includegraphics{fraction-significant-tests-base-R-graphics-15Jan2015.pdf}
%   \caption[Prognostic content in human cancers]{Prognostic content in human
%   cancers.  The fraction of markers significantly associated with outcome at
%   logrank $p<0.05$ is shown for single-gene markers, multi-gene markers from
%   the \mbox{MSigDB \smallcaps{c2}} database, and randomly generated
%   multi-genes markers devoid of biological meaning.  The latter was used to
%   order the datasets. Fractions were computed from death-related end-points
%   when available (\smallcaps{os} or \smallcaps{dss}), or relapse otherwise
%   (\smallcaps{dfs} or \smallcaps{dmfs}). The dotted red line marks the 0.05
%   threshold.}
%   \label{fig:prognostic-fraction}%
% \end{figure}

% \clearpage

% \subsection{Distribution of survival times}
% \label{sec:results-prognostic-survival-survival-times}

% \begin{marginfigure}%
%   \includegraphics{event-distribution-breast-os.pdf}
%   \caption[Distribution of overall survival events in thirteen breast cancer
%   datasets]{Distribution of overall survival events in 13 breast cancer
%   datasets.  The density distributions for censored observations (black) and
%   death events (red) are shown for \num{4663} breast cancer patients whose
%   tumours were profiled across 13 studies.}
%   \label{fig:distribution-survival-times}%
% \end{marginfigure}

% For each of the 13 breast cancer datasets with overall survival (\smallcaps{os})
% data in our analysis (Figure~\ref{fig:distribution-survival-times}), we computed
% the fraction of genes associated with \smallcaps{os} at logrank $p < 0.05$ for
% \num{100} permutations of the respective survival times constructs
% (Figure~\ref{fig:null-breast-os}).

% \subsection{Re-sampling of \smallcaps{metabric} dataset}
% \label{sec:results-prognostic-survival-metabric}

% The discovery and validation \smallcaps{metabric} datasets comprise altogether
% \num{1972} expression profiles of breast cancers.\cite{curtis_genomic_2012} This
% constitutes one of the largest cohorts of cancer patients with follow-up and
% extensive clinical annotation data available in the public domain.

% We ran a series of bootstrapping experiments in a merged \smallcaps{metabric}
% dataset aiming at assessing the effect of potential confounding factors in the
% quantification of the prognostic signals of cancer transcriptomes.
% Figure~\ref{fig:bootstrap-metabric} shows distributions of single-marker
% prognostic fractions for the following experimental setups: (\emph{a})~\num{100}
% samplings of size \num{500} each, out of the total \mbox{1972}
% \smallcaps{metabric} transcriptomes; (\emph{b})~\num{100} samplings of size $N$,
% for every $N \in \{100, 200, 300, 500, 750, 1000, 1250, 1500, 1750\}$;
% (\emph{c})~\num{100} samplings of size \num{500}, with survival times
% beyond time $t$ months artificially censored at time $t$, for every
% $t \in \{25, 50, 75, 100, 125, 150, 175, 200, \max(\text{follow-up time})\}$;
% (\emph{d})~\num{100} samplings, each comprised of $300 \times E$ \smallcaps{er}+
% samples and $300 \times (1-E)$ \smallcaps{er}-- samples, for every
% $E \in \{0, 0.1, 0.2, \ldots{}, 0.9, 1\}$; and (\emph{e})~\num{100} samplings,
% each comprised of $500 \times L$ lymph node positive samples and
% $500 \times (1-L)$ lymph node negative samples, for every
% $L \in \{0, 0.1, 0.2, \ldots{}, 0.9, 1\}$.

% \vspace{8\baselineskip}

% \begin{figure}
%   \includegraphics{null-breast-os.pdf}
%   %   \setfloatalignment{t}
%   \caption[Null distributions of prognostic content in 13 breast cancers cohorts
%   with randomized survival objects]{Null distributions of prognostic content in
%   13 breast cancers cohorts with randomized survival objects.  Prognostic
%   content, here defined as the fraction of single-gene markers associated with
%   overall survival at logrank \mbox{$p < 0.05$}, was computed for \num{100}
%   iterations of randomized survival times constructs for each dataset.  Each
%   boxplot shows the respective null distribution of prognostic fractions.
%   Actual prognostic fractions for each dataset are also displayed (refer to
%   Figure~\ref{fig:prognostic-fraction}), in red (if the observation is beyond
%   the 95\textsuperscript{th} quantile of the null distribution), or in grey
%   (if not).  The red horizontal dotted line shows the five percent expected
%   median of each null distribution.}
%   \label{fig:null-breast-os}%
% \end{figure}

% \begin{figure*}
%   \includegraphics{metabric-sims-19Mar2015.pdf}
%   \caption[Bootstrapping experiments on the \smallcaps{metabric}
%   dataset]{Bootstrapping experiments on the \num{1972} combined breast cancer
%   transcriptomes of the \smallcaps{metabric} dataset.
%   \textbf{A--}Distribution of the prognostic fraction in 100 samplings of 500
%   expression profiles, out of the total 1972 \smallcaps{metabric} profiles.
%   \textbf{B--}Effect of sample size.  \num{100} samplings (grey points) were
%   assessed for each specified sample size.  \textbf{C--}Effect of follow-up
%   times.  For each time $t$, \num{100} samplings were assessed for which
%   patients beyond time $t$ were considered censored at time $t$.
%   \textbf{D--}Effect of the fraction of \smallcaps{er}+ patients.  \num{100}
%   samplings, each of 300 patients, were assessed with the respective fraction
%   of \smallcaps{er}+ samples.  \textbf{E--}Effect of the fraction of node
%   positive patients.  \num{100} samplings, each of 500 patients, were assessed
%   with the respective fraction of node positive patients.}
%   \label{fig:bootstrap-metabric}
% \end{figure*}

%%%%%%%%%%%%%%%%%%%%%%%%%%%%%%%%%%%%%%%%%%%%%%%%%%%%%%%%%%%%%%%%%%%%%%%%%%%%%%%%%%%%
% example, dataset GSE9893 has one of the highest prognostic fractions (Fig. 1),   %
% but surprisingly its signal-to-noise metrics15 is very low (Supplementary        %
% Fig. S1). Examination of the data reveals that normalization was performed in    %
% two batches and induced massive spurious correlations between expression and     %
% outcome (Supplementary Fig. S1). Accordingly, a proper single-batch              %
% normalization of the raw expression data restores the signal-to-noise metrics to %
% normal and decreases the prognostic fraction from 59% to 19% (Supplementary      %
% Fig. S1).  We have shown the fraction of prognostic single- and multi-gene       %
% markers is greater that                                                          %
% 5% in the vast majority of the datasets. Furthermore, the probability of a       %
%  % significant single- or multi-genes marker association with outcome depends on %
%  % cohort’s demographics, but also to a large extent on technical factors that   %
%  % include sampling effects, cohort size, patient follow-up protocols,           %
%  % cellularity standards, protocol randomization and possibly other factors not  %
%  % addressed here. These findings call for the reappraisal of previous study     %
%  % resting on the associations of markers with outcome to support relevance to   %
%  % human cancer, including low-throughput PCR-based studies. They also call for  %
%  % study-specific controls akin to those presented Fig. 1 in future studies.     %
% A marker does not need to convey interesting biological information              %
% research-wise in order to be useful in the clinic. Therefore, our results have   %
% no bearing on the clinical utility of published marker associations.             %
%%%%%%%%%%%%%%%%%%%%%%%%%%%%%%%%%%%%%%%%%%%%%%%%%%%%%%%%%%%%%%%%%%%%%%%%%%%%%%%%%%%%

% \subsection{Datasets}
% \label{sec:results-prognostic-survival-datasets}

% A total of 114 public datasets of cancer gene expression profiles spanning 19
% types of cancer were downloaded from public repositories, manually curated and
% pre-processed for downstream analysis.  Sources for datasets include the Gene
% Expression Omnibus,\cite{edgar_gene_2002}
% InSilicoDB,\cite{taminau_insilicodb:_2011} and the \smallcaps{tcga} Research
% Network
% site.\footnote{\href{http://cancergenome.nih.gov/}{http://cancergenome.nih.gov/}}
% Individual datasets are described in the
% \hyperref[sec:methods-datasets]{\textsf{Methods}} chapter.

% \subsection{Gene sets}
% \label{sec:results-prognostic-survival-genesets}

% For each cancer dataset, we evaluated the association with clinical outcome of
% the 4722 curated gene sets from the
% \href{http://www.broadinstitute.org/gsea/msigdb/index.jsp}Molecular Signatures
% Database (\mbox{MSigDB
% \smallcaps{c2}}).\footnote{\href{http://www.broadinstitute.org/gsea/msigdb/index.jsp}{http://www.broadinstitute.org/gsea/msigdb/index.jsp}}
% \mbox{MSigDB \smallcaps{c2}} signatures are manually curated from the literature
% on gene expression and also include gene sets from curated pathways databases
% such as \smallcaps{kegg}.

% Version 4.0 of the MSigDB curated gene sets (updated on May 31, 2013) was
% downloaded and parsed to produce two lists of gene expression signatures: one
% containing the 4722 original signatures with gene symbols as identifiers, and a
% randomized list consisting of 4722 signatures of the same size as the originals,
% but composed of random samplings from the total pool of genes in the collection.

% \subsection{Association with outcome}
% \label{sec:results-prognostic-survival-association}

% Association with outcome was computed as follows.  For each cancer dataset, we
% retained the first of the following clinical outcomes available, in decreasing
% order of importance: disease-specific survival (\smallcaps{dss}); overall
% survival (\smallcaps{os}); disease-free survival (\smallcaps{dfs}); and distant
% metastasis free survival (\smallcaps{dmfs}).  The survival times for the
% selected clinical outcome specified the dependent variables.

% We then modeled the association of each gene set with survival times by fitting
% a univariate Cox proportional hazards model, using the continuous first
% principal component loading vector of the signature in the global expression
% matrix as the predictor variable.  The \emph{p}-value of the associated logrank
% test was used to estimate association of the gene set with survival times.

% A similar procedure was followed to compute the association of each single gene
% in the expression matrix with survival times, by using the respective continuous
% vector of expression as the predictor variable in the Cox model.  Here again, the
% \emph{p}-value of the logrank test was used to estimate association with
% outcome.

% \subsection{Prognostic content in human cancers}
% \label{sec:results-prognostic-survival-prognostic-content}

% Figure~\ref{fig:prognostic-fraction} shows, for each human cancer dataset
% analyzed, the fraction of \mbox{MSigDB \smallcaps{c2}} signatures, randomized
% signatures and single genes found significantly associated with outcome, at
% $p < 0.05$.

% \begin{figure}%
%   \includegraphics{fraction-significant-tests-base-R-graphics-15Jan2015.pdf}
%   \caption[Prognostic content in human cancers]{Prognostic content in human
%   cancers.  The fraction of markers significantly associated with outcome at
%   logrank $p<0.05$ is shown for single-gene markers, multi-gene markers from
%   the \mbox{MSigDB \smallcaps{c2}} database, and randomly generated
%   multi-genes markers devoid of biological meaning.  The latter was used to
%   order the datasets. Fractions were computed from death-related end-points
%   when available (\smallcaps{os} or \smallcaps{dss}), or relapse otherwise
%   (\smallcaps{dfs} or \smallcaps{dmfs}). The dotted red line marks the 0.05
%   threshold.}
%   \label{fig:prognostic-fraction}%
% \end{figure}

% \clearpage

% \subsection{Distribution of survival times}
% \label{sec:results-prognostic-survival-survival-times}

% \begin{marginfigure}%
%   \includegraphics{event-distribution-breast-os.pdf}
%   \caption[Distribution of overall survival events in thirteen breast cancer
%   datasets]{Distribution of overall survival events in 13 breast cancer
%   datasets.  The density distributions for censored observations (black) and
%   death events (red) are shown for \num{4663} breast cancer patients whose
%   tumours were profiled across 13 studies.}
%   \label{fig:distribution-survival-times}%
% \end{marginfigure}

% For each of the 13 breast cancer datasets with overall survival (\smallcaps{os})
% data in our analysis (Figure~\ref{fig:distribution-survival-times}), we computed
% the fraction of genes associated with \smallcaps{os} at logrank $p < 0.05$ for
% \num{100} permutations of the respective survival times constructs
% (Figure~\ref{fig:null-breast-os}).

% \subsection{Re-sampling of \smallcaps{metabric} dataset}
% \label{sec:results-prognostic-survival-metabric}

% The discovery and validation \smallcaps{metabric} datasets comprise altogether
% \num{1972} expression profiles of breast cancers.\cite{curtis_genomic_2012} This
% constitutes one of the largest cohorts of cancer patients with follow-up and
% extensive clinical annotation data available in the public domain.

% We ran a series of bootstrapping experiments in a merged \smallcaps{metabric}
% dataset aiming at assessing the effect of potential confounding factors in the
% quantification of the prognostic signals of cancer transcriptomes.
% Figure~\ref{fig:bootstrap-metabric} shows distributions of single-marker
% prognostic fractions for the following experimental setups: (\emph{a})~\num{100}
% samplings of size \num{500} each, out of the total \mbox{1972}
% \smallcaps{metabric} transcriptomes; (\emph{b})~\num{100} samplings of size $N$,
% for every $N \in \{100, 200, 300, 500, 750, 1000, 1250, 1500, 1750\}$;
% (\emph{c})~\num{100} samplings of size \num{500}, with survival times
% beyond time $t$ months artificially censored at time $t$, for every
% $t \in \{25, 50, 75, 100, 125, 150, 175, 200, \max(\text{follow-up time})\}$;
% (\emph{d})~\num{100} samplings, each comprised of $300 \times E$ \smallcaps{er}+
% samples and $300 \times (1-E)$ \smallcaps{er}-- samples, for every
% $E \in \{0, 0.1, 0.2, \ldots{}, 0.9, 1\}$; and (\emph{e})~\num{100} samplings,
% each comprised of $500 \times L$ lymph node positive samples and
% $500 \times (1-L)$ lymph node negative samples, for every
% $L \in \{0, 0.1, 0.2, \ldots{}, 0.9, 1\}$.

% \vspace{8\baselineskip}

% \begin{figure}
%   \includegraphics{null-breast-os.pdf}
%   %   \setfloatalignment{t}
%   \caption[Null distributions of prognostic content in 13 breast cancers cohorts
%   with randomized survival objects]{Null distributions of prognostic content in
%   13 breast cancers cohorts with randomized survival objects.  Prognostic
%   content, here defined as the fraction of single-gene markers associated with
%   overall survival at logrank \mbox{$p < 0.05$}, was computed for \num{100}
%   iterations of randomized survival times constructs for each dataset.  Each
%   boxplot shows the respective null distribution of prognostic fractions.
%   Actual prognostic fractions for each dataset are also displayed (refer to
%   Figure~\ref{fig:prognostic-fraction}), in red (if the observation is beyond
%   the 95\textsuperscript{th} quantile of the null distribution), or in grey
%   (if not).  The red horizontal dotted line shows the five percent expected
%   median of each null distribution.}
%   \label{fig:null-breast-os}%
% \end{figure}

% \begin{figure*}
%   \includegraphics{metabric-sims-19Mar2015.pdf}
%   \caption[Bootstrapping experiments on the \smallcaps{metabric}
%   dataset]{Bootstrapping experiments on the \num{1972} combined breast cancer
%   transcriptomes of the \smallcaps{metabric} dataset.
%   \textbf{A--}Distribution of the prognostic fraction in 100 samplings of 500
%   expression profiles, out of the total 1972 \smallcaps{metabric} profiles.
%   \textbf{B--}Effect of sample size.  \num{100} samplings (grey points) were
%   assessed for each specified sample size.  \textbf{C--}Effect of follow-up
%   times.  For each time $t$, \num{100} samplings were assessed for which
%   patients beyond time $t$ were considered censored at time $t$.
%   \textbf{D--}Effect of the fraction of \smallcaps{er}+ patients.  \num{100}
%   samplings, each of 300 patients, were assessed with the respective fraction
%   of \smallcaps{er}+ samples.  \textbf{E--}Effect of the fraction of node
%   positive patients.  \num{100} samplings, each of 500 patients, were assessed
%   with the respective fraction of node positive patients.}
%   \label{fig:bootstrap-metabric}
% \end{figure*

\section{Other Contributions}

This sections reports published results to which the author contributed at large
during this dissertation.  For each work, a short synopsis of the main findings
as well as the specific contributions of the author will be presented.

\subsection{Role of Epac and protein kinase A in thyrotropin-induced gene
  expression in primary thyrocytes}
\begin{figure}[h]
  \includegraphics{../pdf/include/epac-cover.pdf}
\end{figure}

This work sought to clarify which partners of the c\smallcaps{amp} cascade
regulate the \smallcaps{tsh}-induced gene expression modulation in thyrocytes:
protein kinase \smallcaps{a} and/or the \smallcaps{epac}
proteins.\cite{van_staveren_role_2012} Contingent to this objective was the
characterization of the potential role of the Epac-Rap-RapGAP pathway in thyroid
tumorigenesis.  The author contributed with data analysis, figure generation and
results discussion.

\clearpage

\subsection{5-Aza-2'-Deoxycytidine has minor effects on differentiation in human
  thyroid cancer cell lines, but modulates genes that are involved in adaptation
  \emph{in vitro}}
\begin{figure}[h]
  \includegraphics{../pdf/include/5-aza-cover.pdf}
\end{figure}

This work aimed at investigating the extent to which 5-aza-2'-deoxycytidine, a
\smallcaps{dna} demethylation agent, is able to reactivate the expression of
differentiation markers potentially repressed by epigenetic modifications in
thyroid cancer cell lines.\cite{dom_5-aza-2-deoxycytidine_2013} The author
contributed with data analysis, figure generation and results discussion.

\clearpage

\subsection{Intratumor heterogeneity and clonal evolution in an aggressive
  papillary thyroid cancer and matched metastases}
\begin{figure}[h]
  \includegraphics{../pdf/include/ptc-heterogeneity-cover.pdf}
\end{figure}

This study sought to characterize the intratmoural heterogeneity of a specimen
of aggressive papillary thyroid carcinoma, and the clonal relationships between
the primary tumour and their corresponding lymph node and distant
metastases.\cite{pennec_intratumor_2015} The author contributed with
experimental design input and results discussion.

%%% Local Variables:
%%% mode: latex
%%% TeX-master: "../thesis.tex"
%%% End:
