This chapter is comprised of two sections.  The first consists of a published
article, entitled ``A general method to derive robust organ-specific gene
expression-based differentiation indices: application to thyroid cancer
diagnostic.''\cite{tomas_general_2012} The results exposed in this manuscript
are developed in the
\hyperref[discussion-differentiation-microarrays]{\textsf{Discussion}} chapter,
under the section \emph{Differentiation and proliferation signatures in cancer
  diagnostic}.

The second section, entitled ``The extent of prognostic signals in the cancer
transcriptomes,'' details the results of a systematic analysis of 114 microarray
datasets of cancers affecting 22 different tissue types, aiming at quantifying
the range and the nature of the signals linked with differential survival in
neoplastic expression profiles.  These results are elaborated upon in the
corresponding section of the
\hyperref[sec:discussion-prognostic-microarrays]{\textsf{Discussion}} chapter.

% The results of this analysis are currently under revision.

% \resetcitations

\section{Differentiation and proliferation signatures in cancer diagnostic}
\label{sec:results-differentiation-proliferation}

See next page.

% \includepdf[pages=1-9,fitpaper=true]{../pdf/include/onc-tomas-2012.pdf}
\includepdf[pages=1-9,trim=5cm 3cm 5cm 3cm]{../pdf/include/onc-tomas-2012.pdf}
% \includegraphics{../pdf/include/onc-tomas-2012.pdf}

\section{The extent of prognostic signals in the cancer transcriptomes}
\label{sec:results-prognostic-survival}

\subsection{Datasets}
\label{sec:results-prognostic-survival-datasets}

A total of 114 public datasets of cancer gene expression profiles spanning 19
types of cancer were downloaded from public repositories, manually curated and
pre-processed for downstream analysis.  Sources for datasets include the Gene
Expression Omnibus,\cite{edgar_gene_2002}
InSilicoDB,\cite{taminau_insilicodb:_2011} and the \smallcaps{tcga} Research
Network
site.\footnote{\href{http://cancergenome.nih.gov/}{http://cancergenome.nih.gov/}}
Individual datasets are described in the
\hyperref[sec:methods-datasets]{\textsf{Methods}} chapter.

\subsection{Gene sets}
\label{sec:results-prognostic-survival-genesets}

For each cancer dataset, we evaluated the association with clinical outcome of
the 4722 curated gene sets from the
\href{http://www.broadinstitute.org/gsea/msigdb/index.jsp}Molecular Signatures
Database (\mbox{MSigDB
  \smallcaps{c2}}).\footnote{\href{http://www.broadinstitute.org/gsea/msigdb/index.jsp}{http://www.broadinstitute.org/gsea/msigdb/index.jsp}}
\mbox{MSigDB \smallcaps{c2}} signatures are manually curated from the literature
on gene expression and also include gene sets from curated pathways databases
such as \smallcaps{kegg}.

Version 4.0 of the MSigDB curated gene sets (updated on May 31, 2013) was
downloaded and parsed to produce two lists of gene expression signatures: one
containing the 4722 original signatures with gene symbols as identifiers, and a
randomized list consisting of 4722 signatures of the same size as the originals,
but composed of random samplings from the total pool of genes in the collection.

\subsection{Association with outcome}
\label{sec:results-prognostic-survival-association}

Association with outcome was computed as follows.  For each cancer dataset, we
retained the first of the following clinical outcomes available, in decreasing
order of importance: disease-specific survival (\smallcaps{dss}); overall
survival (\smallcaps{os}); disease-free survival (\smallcaps{dfs}); and distant
metastasis free survival (\smallcaps{dmfs}).  The survival times for the
selected clinical outcome specified the dependent variables.

We then modeled the association of each gene set with survival times by fitting
a univariate Cox proportional hazards model, using the continuous first
principal component loading vector of the signature in the global expression
matrix as the predictor variable.  The \emph{p}-value of the associated logrank
test was used to estimate association of the gene set with survival times.

A similar procedure was followed to compute the association of each single gene
in the expression matrix with survival times, by using the respective continuous
vector of expression as the predictor variable in the Cox model.  Here again, the
\emph{p}-value of the logrank test was used to estimate association with
outcome.

\subsection{Prognostic content in human cancers}
\label{sec:results-prognostic-survival-prognostic-content}

Figure~\ref{fig:prognostic-fraction} shows, for each human cancer dataset
analyzed, the fraction of \mbox{MSigDB \smallcaps{c2}} signatures, randomized
signatures and single genes found significantly associated with outcome, at
$p < 0.05$.

\begin{figure}%
  \includegraphics{fraction-significant-tests-base-R-graphics-15Jan2015.pdf}
  \caption[Prognostic content in human cancers]{Prognostic content in human
    cancers.  The fraction of markers significantly associated with outcome at
    logrank $p<0.05$ is shown for single-gene markers, multi-gene markers from
    the \mbox{MSigDB \smallcaps{c2}} database, and randomly generated
    multi-genes markers devoid of biological meaning.  The latter was used to
    order the datasets. Fractions were computed from death-related end-points
    when available (\smallcaps{os} or \smallcaps{dss}), or relapse otherwise
    (\smallcaps{dfs} or \smallcaps{dmfs}). The dotted red line marks the 0.05
    threshold.}
  \label{fig:prognostic-fraction}%
\end{figure}

\clearpage

\subsection{Distribution of survival times}
\label{sec:results-prognostic-survival-survival-times}

\begin{marginfigure}%
  \includegraphics{event-distribution-breast-os.pdf}
  \caption[Distribution of overall survival events in thirteen breast cancer
  datasets]{Distribution of overall survival events in 13 breast cancer
    datasets.  The density distributions for censored observations (black) and
    death events (red) are shown for \num{4663} breast cancer patients whose
    tumours were profiled across 13 studies.}
  \label{fig:distribution-survival-times}%
\end{marginfigure}

For each of the 13 breast cancer datasets with overall survival (\smallcaps{os})
data in our analysis (Figure~\ref{fig:distribution-survival-times}), we computed
the fraction of genes associated with \smallcaps{os} at logrank $p < 0.05$ for
\num{100} permutations of the respective survival times constructs
(Figure~\ref{fig:null-breast-os}).

\subsection{Re-sampling of \smallcaps{metabric} dataset}
\label{sec:results-prognostic-survival-metabric}

The discovery and validation \smallcaps{metabric} datasets comprise altogether
\num{1972} expression profiles of breast cancers.\cite{curtis_genomic_2012} This
constitutes one of the largest cohorts of cancer patients with follow-up and
extensive clinical annotation data available in the public domain.

We ran a series of bootstrapping experiments in a merged \smallcaps{metabric}
dataset aiming at assessing the effect of potential confounding factors in the
quantification of the prognostic signals of cancer transcriptomes.
Figure~\ref{fig:bootstrap-metabric} shows distributions of single-marker
prognostic fractions for the following experimental setups: (\emph{a})~\num{100}
samplings of size \num{500} each, out of the total \mbox{1972}
\smallcaps{metabric} transcriptomes; (\emph{b})~\num{100} samplings of size $N$,
for every $N \in \{100, 200, 300, 500, 750, 1000, 1250, 1500, 1750\}$;
(\emph{c})~\num{100} samplings of size \num{500}, with survival times
beyond time $t$ months artificially censored at time $t$, for every
$t \in \{25, 50, 75, 100, 125, 150, 175, 200, \max(\text{follow-up time})\}$;
(\emph{d})~\num{100} samplings, each comprised of $300 \times E$ \smallcaps{er}+
samples and $300 \times (1-E)$ \smallcaps{er}-- samples, for every
$E \in \{0, 0.1, 0.2, \ldots{}, 0.9, 1\}$; and (\emph{e})~\num{100} samplings,
each comprised of $500 \times L$ lymph node positive samples and
$500 \times (1-L)$ lymph node negative samples, for every
$L \in \{0, 0.1, 0.2, \ldots{}, 0.9, 1\}$.

\vspace{8\baselineskip}

\begin{figure}
  \includegraphics{null-breast-os.pdf}
  % \setfloatalignment{t}
  \caption[Null distributions of prognostic content in 13 breast cancers cohorts
  with randomized survival objects]{Null distributions of prognostic content in
    13 breast cancers cohorts with randomized survival objects.  Prognostic
    content, here defined as the fraction of single-gene markers associated with
    overall survival at logrank \mbox{$p < 0.05$}, was computed for \num{100}
    iterations of randomized survival times constructs for each dataset.  Each
    boxplot shows the respective null distribution of prognostic fractions.
    Actual prognostic fractions for each dataset are also displayed (refer to
    Figure~\ref{fig:prognostic-fraction}), in red (if the observation is beyond
    the 95\textsuperscript{th} quantile of the null distribution), or in grey
    (if not).  The red horizontal dotted line shows the five percent expected
    median of each null distribution.}
  \label{fig:null-breast-os}%
\end{figure}

\begin{figure*}
  \includegraphics{metabric-sims-19Mar2015.pdf}
  \caption[Bootstrapping experiments on the \smallcaps{metabric}
  dataset]{Bootstrapping experiments on the \num{1972} combined breast cancer
    transcriptomes of the \smallcaps{metabric} dataset.
    \textbf{A--}Distribution of the prognostic fraction in 100 samplings of 500
    expression profiles, out of the total 1972 \smallcaps{metabric} profiles.
    \textbf{B--}Effect of sample size.  \num{100} samplings (grey points) were
    assessed for each specified sample size.  \textbf{C--}Effect of follow-up
    times.  For each time $t$, \num{100} samplings were assessed for which
    patients beyond time $t$ were considered censored at time $t$.
    \textbf{D--}Effect of the fraction of \smallcaps{er}+ patients.  \num{100}
    samplings, each of 300 patients, were assessed with the respective fraction
    of \smallcaps{er}+ samples.  \textbf{E--}Effect of the fraction of node
    positive patients.  \num{100} samplings, each of 500 patients, were assessed
    with the respective fraction of node positive patients.}
  \label{fig:bootstrap-metabric}
\end{figure*}

%%% Local Variables:
%%% TeX-engine: xetex
%%% mode: latex
%%% TeX-master: "../thesis"
%%% End:
