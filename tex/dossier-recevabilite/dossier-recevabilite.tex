\documentclass[12pt,origdate,origdate]{lettre}
\usepackage{fontspec}
\setmainfont[Ligatures=TeX]{Minion Pro}
\usepackage[frenchb]{babel}
\usepackage{enumitem}
\setitemize{leftmargin=10pt,itemindent=10pt}

\begin{document}
\begin{letter} {\textsc{La Comission Facultaire des Doctorats} \\
    \textsc{Université Libre de Bruxelles}}
  \thispagestyle{empty}
  \pagestyle{empty}
\name{Gil Tomás}
  \address{Gil \textsc{Tomás} \\
    \textsc{IRIBHM} \\ \textsc{Universit\'{e} Libre de Bruxelles} \\
    Campus Erasme---CP602 \\ B\^{a}timent \textsc{C}, salle C.4.110 \\
    808, route de Lennik \\ 1070 \textsc{Bruxelles}}
  \lieu{Bruxelles}
  \telephone{02-555-6769}
  \nofax

  \opening{Mesdames et Messieurs les membres de la Comission
    Facultaire des doctorants,}

  Je termine actuellement ma thèse sous la supervision du Dr. Vincent Detours
  (promoteur). Je souhaiterais maintenant en effectuer la rédaction en anglais.
  La thèse est intitulée: \og \emph{Gene expression markers of proliferation and
    differentiation in cancer and the extent of prognostic signals in the cancer
    transcriptome} \fg. Ma formation doctorale a été validée en 2010 et j'ai
  présenté un poster à la Journée des Doctorants en Décembre 2013.
  Veuillez trouver ci-joint les différentes rubriques nécessaires au dossier de
  recevabilité:

\begin{itemize}
  \item Le formulaire de composition du jury de thèse rempli;
  \item Une lettre de mon promoteur soutenant ma démarche;
  \item La table des matières provisoire de ma thèse;
  \item Une copie du premier des deux articles principaux associés à cette
    thèse, paru dans le journal \emph{Oncogene}. Le deuxième article
    est en cours de rédaction;
  \item Une copie de mon \emph{Curriculum Vitae.}
  \end{itemize}

Les trois experts extrieurs proposés sont les suivant:

\begin{itemize}
  \item Ann Nowé (\texttt{ann.nowe@vub.ac.be})
  \item Jacques Van Helden (\texttt{jacques.van-helden@univ-amu.fr})
  \item Jo Vandesompele (\texttt{joke.vandesompele@ugent.be})
  \end{itemize}

Les titres de thèse annexe proposés sont les suivants:

\begin{itemize}
\item \emph{Using differentiation metagenes to infer the tissue of origin of
    cancer metastasis.}
\item \emph{The relationship between genotype and phenotype in cancer: is all
  heterogeneity within cancer cells of clonal origin?}
\item \emph{Can cancer cells induce a malignant phenotype in adjacent
    genetically normal cells through non genetic elements?}
\end{itemize}

\closing{Je vous remercie de l'attention que vous porterez \`{a} ma demande
  et vous prie d'agr\'eer, Mesdames et Messieurs, mes sinc\`eres salutations.}

\end{letter}

\end{document}