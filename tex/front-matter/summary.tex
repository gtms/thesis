\chapter*{Résumé}
\emph{To do.}
\clearpage
\blankpage
\chapter*{Summary}

Cancer is a group of genetic diseases operationally defined by uncontrolled
cellular proliferation, with consequent disruption of the organism
homeostasis.  Cancer research seeks to provide accurate diagnoses and tailored
treatments to this broad spectrum of diseases.  Microarray technology allows for
the monitoring of the expression of all transcription products in the human
genome, and thus presents with a tool to address the polygenic nature of cancer.
Microarray technology can assist both to the task of cancer class discovery and
cancer outcome prediction based on prior expression profiles.  Furthermore, the
use of gene expression signatures as surrogate markers for physiological
molecular processes enables the possibility of hypothesis testing using
microarray data.

The aim of this dissertation is two-fold: (\emph{a}) to ascertain the extent by
which gene expression markers of differentiation and proliferation may
contribute to disease classification; and (\emph{b}) to assess the extent of
prognostic signals in cancer transcriptomes.
% and (\emph{b}) to provide a contextual overview of the impact of twenty years
% of microarray technology in cancer research.

We have devised an unbiased method to derive organ-specific differentiation
signatures from gene expression data.  We then demonstrated that tissue-specific
gene expression signatures of differentiation can accurately discriminate
between challenging histopathological subtypes of cancer in a thyroid model,
therefore showing potential clinical diagnostic value.

We also show that a non-negligible fraction of cancer transcriptomes is
associated with disease outcome, on the count of the analysis of 114 cancer
cohorts spanning 19 different cancer types.  This is likely due to an extensive
correlation structure in cancer expression profiles, partly explained by
technical and biological variables.  Such evidence disavows the widespread use
of statistical association between expression markers and cancer patients
outcome in order to infer implication of particular biological mechanisms in
disease progression.

% Microarray technology has advanced cancer research by unveiling novel
% taxonomies of cancer; by uncovering novel potential therapeutic gene targets
% for particular forms of the disease; and by enabling hypothesis testing in
% cancer biospecimens by use of biologically motivated gene expression
% signatures.  However, early prospects of personalized medicine elicited by the
% assaying of global expression profiles in cancer have not been met.  This may
% be due to the lack of resolution of the technology, its inability to account
% for non-genetic determinants of cancer progression and conceptual shortcomings
% regarding the nature of cancer.

\clearpage

%%% Local Variables:
%%% mode: latex
%%% TeX-master: "../thesis.tex"
%%% End:
