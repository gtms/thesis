\section{Cancer}
\label{cancer-discussion}

% This is the \emph{cancer} section of the discussion.

\newthought{Cancer research} aims to improve the diagnosis and treatment through
better disease classification and patient stratification.  This allows for the
design of therapies that are better targeted to specific cancer subtypes and
improve the effectiveness of existing regimens, while reducing their morbidity.

In this dissertation, we present two case studies for the application of gene
expression signatures to address supervised tasks of class prediction, and
outcome prediction, based on publicly available transcriptomes of neoplastic
samples.  Each analysis addresses a problem related to the clinical management
of oncological pathologies, framed by the analytic corpus developed for
microarray data research, and grounded by the current understanding of the
biology of cancer.

% \medskip

\subsection{Molecular classification of cancer}
\label{sec:molecular-classification}

% classification
Our first contribution concerns the improvement of molecular diagnostics for
tumour classification based on expression profiles of clinical samples.
Traditionally, tumour classification rests on morphological characterization and
immunohistochemical assessment of tissue-specific antigens.  Histological types
thusly defined denote distinct clinical behaviours and responses to treatment.
Molecular characterization of tumour subtypes aims to increase the accuracy of
tumour classification based on features that escape histomorphological
assessment alone.

Following the proof-of-principle molecular classification of leukemia by Golub
et al.~in 1999,\cite{golub_molecular_1999} molecular cancer classification was
generalized to solid tumours by Ramaswamy et al. and Su et al.~in
2001.\cite{ramaswamy_multiclass_2001,su_molecular_2001} Both studies present an
experimental design where an initial number of primary carcinomas (respectively,
144 and 100), spanning most human solid cancers, was used to train a multiclass
predictor to determine the anatomical origin of a random test sample, given its
expression profile.  Reported accuracies of these classifiers ranged from 78\%
to 83\%.  Ramaswamy et al.~remarked that poorly differentiated tumours are less
amenable to molecular classification.  They interpreted this observation as
evidence for a distinct molecular pathogenesis of poorly differentiated tumours
when compared to their well differentiated counterparts.  Su et al.~noted that,
for eleven of the tumour classes they analyzed, a minimal core set of genes is
sufficient to accurately discriminate between classes.  This suggests that
expression of genes particular to the basal physiology and morphology of the
respective tissues of origin might be actionable for molecular classification.

We reasoned that these findings may well be explained by the projection of
cancer progression along axes of molecular dedifferentiation and increased
proliferation.  We then demonstrated that two biomarkers, one of thyroid
differentiation and another of proliferation, both derived from healthy tissues,
could be used to accurately classify tumour subtypes of thyroid
origin.\cite{tomas_general_2012}

In spite of a promising decade for the translation of these research findings
into relevant diagnostics with an impact on clinical management, a review of the
field by Schnabel and Erlander\cite{schnabel_gene_2012} cites only three
commercial molecular cancer classifiers currently available for clinical use
(Table~\ref{tab:molecular-classifiers}).

\begin{table}
  \footnotesize
  \centering
  \begin{tabular}{m{1.7cm}C{2.5cm}C{2.5cm}C{2.5cm}}
    % \begin{tabular}{m{1.7cm}ccc}
        %         \setlength{\extrarowheight}{20pt}
    \toprule
    & bioTheranostics Cancer \smallcaps{type} \smallcaps{ID}\textregistered{} & Rosetta
                                                                                Genomics
                                                                                mi\smallcaps{R}view met
                                                                                2\texttrademark{}
    & PathworkDiagnostics\textregistered{} Tissue of Origin\\
    \midrule
    Tissue \newline specimen & formalin-fixed and paraffin-embedded & formalin-fixed and
                                                                      paraffin-embedded &
                                                                                          formalin-fixed
                                                                                          and
                                                                                          paraffin-embedded\\
    Assay \newline technology (biomolecule) & \smallcaps{rt-pcr} (m\smallcaps{rna}) & microarray
                                                                                      (mi\smallcaps{rna}) &
                                                                                                            microarray
                                                                                                            (m\smallcaps{rna})\\
    Biomarkers assessed & 92 & 64 & 2000\\
    Tumour types classified & 54 & 42 & 15\\
    Accuracy & 87\% &  85\% & 89\% \\
    Tumour subclassification & Yes & No & No\\
    \smallcaps{fda} clearance & No & No & Yes\\
    \bottomrule
  \end{tabular}
  \caption[Commercial molecular cancer classifiers currently available for clinical use]{Commercial molecular cancer classifiers currently available for clinical use (adapted from \citealp{schnabel_gene_2012}).}
  \label{tab:molecular-classifiers}
\end{table}

Of the three, only one attempts to class molecular tumour subtypes; the other
two are designed to locate the primary tumour in patients that present with
cancers of uncertain origin in the metastatic setting.  Only the
PathworkDiagnostics\textregistered{} test has met standards for \smallcaps{fda}
clearance, yet none of these assays has properly challenged traditional
classification methods.  At best, they have found their way into clinical
management as molecular correlates to histopathological findings.

In their expert opinion piece, Schnabel and Erlander suggest that clinical
adoption of gene expression-based classifiers has been curbed by the lack of
clinical gold standards; and by the potential confounding effect of tumour
heterogeneity in molecular evaluations of mixed populations of cells.  Molecular
classifiers can, however, contribute to a clinical decision whenever
pathological assessment is uncertain or conflicting; or even rule out unlikely
tumour classes, facilitating the decision process.

Interestingly, they also raise the possibility that recent paradigm shifts in
neoplastic disease conceptualization may also contribute for the lack of reach
of molecular classifiers in the clinical setting.  Namely, the recent increased
attention given to the complex nature of heterotypic signalings in the tumour
microenvironment,\cite{weigelt_need_2014} a feature that lies beyond the scope
of microarray technology resolution, may well account for the struggle of
traditional molecular classifiers to recapitulate neoplastic disease in its
essence.

\subsection{Molecular prognostication of cancer}
\label{sec:molecular-prediction}

Cancer prognostication is one of the most challenging tasks in oncology.  Here,
the consensus is to adhere to formal anatomical staging systems, such as the the
\smallcaps{tnm} staging system,\cite{sobin_tnm:_2003} which provide a basis for
prediction of survival, choice of initial treatment, and stratification of
patients in clinical trials.\cite{ludwig_biomarkers_2005} The discovery of novel
subdivisions of traditional tumour classes---defined by exclusive biomarkers,
and presenting different clinical behaviours and therapeutic
responses---motivated the search for molecular-based models for cancer
prognostication.

Tangible success was achieved when, in early 2007, the \smallcaps{fda} cleared
MammaPrint\textcopyright{}, the first microarray-based commercial molecular
prognostic test for breast cancer.\cite{fda_2007_2007} This assay, for
node-negative women under 61 years of age with tumors less than
\SIlist{5}{\centi\metre} in diameter, is based on the 70-gene prognosis profile
of van't Veer et al.\cite{vant_veer_gene_2002}

Breast cancer is, arguably, one of the most thoroughly characterized human
cancers from the molecular point of view.  In the breast cancer model, genomics
classification methods have uncovered at least four intrinsic subtypes: the
basal-like subtype, which is estrogen receptor negative
\mbox{(\smallcaps{er--})} and \smallcaps{her2--}; the \smallcaps{her2} subtype,
characterized by increased expression of \smallcaps{her2} and of genes mapping
to the \smallcaps{her2} amplicon; and two luminal \smallcaps{er+}
subtypes---namely the luminal \smallcaps{A} subtype, characterized by high
levels of \smallcaps{er} and \smallcaps{er}-related genes; and the luminal
\smallcaps{b} subtype, characterized by lower \smallcaps{er} levels and high
expression of genes implicated in the proliferation
process.\cite{arpino_gene_2013} Evidence based on prognostic gene signatures
empirically derived to discriminate between good- and poor-prognosis cancers has
established that good prognosis \smallcaps{er+} tumours (luminal \smallcaps{a}
tumours) derive little, if any, benefit from adjuvant chemotherapy.  Conversely,
other subtypes show a greater sensitivity to multidrug chemotherapy
regimens.\cite{weigelt_challenges_2012}

While great hope was placed on first generation prognostic signatures to replace
clinicopathological parameters in therapy decision making, it was eventually
shown that these molecular classifiers largely complement the prognostic ability
of tumour size and nodal status---two of the metrics encapsulated by the
\smallcaps{tnm} staging
system.\cite{sotiriou_gene-expression_2009,reis-filho_molecular_2010}
Furthermore, the prognostic ability of these signatures is mostly restricted to
\smallcaps{er+} breast cancer, with negligible prognostic value reported for
patients with \smallcaps{er--} disease.

Microarray-based markers of predictive response to breast cancer chemotherapy
have also shown limited success in producing clinically serviceable tests.
Weigelt et al.\cite{weigelt_challenges_2012} discuss possible causes for this
poor translational output.  Among them, they cite challenges related to the
confounding impact of molecular heterogeneity of disease when deriving
prognostic and predictive markers; the lack of resolution of microarray
technology at the post-transcriptional level; and the under-explored potential
role for dynamic-response markers in predicting response to treatment.

% the complexity of the tumour's transcriptomic landscape, which may keep
% germane biological signals beyond the resolution of microarray technology; and
% the possible widespread nature of tumour resistance mechanisms, which may
% operate at different molecular levels.  and the strict requirements in tissue
% quality standards for biomarker assaying.

A large number of breast cancer prognostic signatures has been reported in the
last decade.\cite{liu_breast_2014} While their overlap is limited, their
prognostic ability has been linked with two main components of breast-derived
neoplasias' biology: their proliferative activity, and their \smallcaps{er}
signaling.  Nevertheless, the deceptive biological specificity of these
biomarkers has inspired a burgeoning literature purporting to report the
implication of biological phenomena in breast cancer progression.  This was
routinely achieved by validating the prognostic ability of a biologically
motivated gene signature in a cohort of breast cancer expression
profiles.\cite{chang_gene_2004} Venet et al.~have challenged this reasoning by
showing that most random gene expression signatures are associated with breast
cancer outcome.\footnote{Their argument was illustrated, via \emph{reductio ad
    absurdum}, with the demonstration that expression signatures of postprandial
  laughter and of mice social defeat can both predict overall survival in the
  \mbox{295-sample} \smallcaps{nki} reference cohort
  (\citealp{venet_most_2011}).}  In this dissertation, we have extended these
findings by characterizing the extent of prognostic signals in 114 cohorts of
human cancers afflicting nineteen organ systems.\footnote{Under revision.}  We
have shown that, in most cancer transcriptomes, the nature of prognostic signals
is notoriously pervasive and their assessment is highly sensitive to sampling
variance.  We describe a breast cancer case study where a sizable fraction of
the prognostic signals quantified could be ascribed to spurious correlations
related to a critical normalization artifact.  We further established that the
extent of prognostic signals in transcriptomes of breast cancer are related to
the fraction of \smallcaps{er+} neoplastic transcriptomes prognosticated and, to
a lesser extent, to the fraction of node positive patients considered in the
analysis.

\begin{table}[ht]
  \footnotesize
  \centering
  \begin{tabular}{m{3.2cm}m{1.5cm}m{1.1cm}m{2cm}m{1cm}}
    \toprule
    \multicolumn{1}{c}{Biomarker}
                                                                                    & Official gene name       & Clinical use\textsuperscript{\emph{a}} & Cancer type                & Source type \\
    \midrule
    $\alpha$-fetoprotein (\smallcaps{afp})                                          & \smallcaps{\emph{afp}}   & Stg                                    & Nonseminomatous testicular & Serum       \\
    Human chorionic gonadotropin (h\smallcaps{gc})                                  & \smallcaps{\emph{cgb}}   & Stg                                    & Testicular                 & Serum       \\
    Carbohydrate antigen 19-9 (\smallcaps{ca19-9})                                  &                          & Mnt                                    & Pancreatic                 & Serum       \\
    Carbohydrate antigen 125 (\smallcaps{ca125})                                    & \smallcaps{\emph{muc16}} & Mnt                                    & Ovarian                    & Serum       \\
    Carcinoembryonic antigen (\smallcaps{cea})                                      & \smallcaps{\emph{psg2}}  & Mnt                                    & Colorectal                 & Tissue      \\
    Epidermal growth factor receptor (\smallcaps{egfr})                             & \smallcaps{\emph{egfr}}  & Prd                                    & Colorectal                 & Tissue      \\
    v-kit Hardy-Zuckerman 4 feline sarcoma viral oncogene homolog (\smallcaps{kit}) & \smallcaps{\emph{kit}}   & Prd                                    & Gastrointestinal           & Tissue      \\
    Thyroglobulin                                                                   & \smallcaps{\emph{tg}}    & Mnt                                    & Thyroid                    & Serum       \\
    Prostate specific antigen (\smallcaps{psa})
                                                                                    & \smallcaps{\emph{klk3}}  & Scn, Mnt                               & Prostate                   & Serum       \\
    Carbohydrate antigen 15.3 (\smallcaps{ca 15.3})                                 & \smallcaps{\emph{muc1}}  & Mnt                                    & Breast                     & Serum       \\
    Carbohydrate antigen 27.29 (\smallcaps{ca27.29})                                & \smallcaps{\emph{muc1}}  & Mnt                                    & Breast                     & Serum       \\
    Estrogen receptor (\smallcaps{er})
                                                                                    & \smallcaps{\emph{esr1}}  & Prg, Prd                               & Breast                     & Tissue      \\
    Progesterone receptor (\smallcaps{pr})
                                                                                    & \smallcaps{\emph{pgr}}   & Prg, Prd                               & Breast                     & Tissue      \\
    v-erb-b2 erythroblastic leukemia viral oncogene homolog 2
    (\smallcaps{her2}-neu)                                                          & \smallcaps{\emph{erbb2}} & Prg, Prd                               & Breast                     & Tissue      \\
    Nuclear matrix protein 22 (\smallcaps{nmp-22})
                                                                                    &                          & Scn, Mnt                               & Bladder                    & Urine       \\
    Fibrin/fibrinogen degradation products (\smallcaps{fdp})                        &                          & Mnt                                    & Bladder                    & Urine       \\
    Bladder tumor antigen (\smallcaps{bta})                                         &                          & Mnt                                    & Bladder                    & Urine       \\
    High molecular \smallcaps{cea} and mucin                                        &                          & Mnt                                    & Bladder                    & Urine       \\
    \bottomrule
    \multicolumn{5}{l}{\textsuperscript{\emph{a}}~Stg: staging; Mnt:
    monitoring; Prd: prediction; Prg: prognosis; Scn: screening.}
  \end{tabular}
  \caption[\smallcaps{fda}-cleared protein cancer biomarkers]{List of the eighteen
    \smallcaps{fda}-cleared protein cancer biomarkers issued from genomic
    analyses currently available for clinical use (adapted from \citealp{pavlou_long_2013}).}
  \label{tab:fda-biomarkers}
\end{table}
% \footnotetext{Stg: staging; Mnt: monitoring; Prd: prediction; Prg: prognosis;
%   Scn: screening.}

% \newthought{Provocatively,} in a 2013 review on the impact of
% \smallcaps{fda}-approved molecular biomarkers in cancer staging, prognosis and
% treatment selection (Table~\ref{tab:fda-biomarkers}), Pavlou et al.~observed
% that, ``despite the intensified interest and investment by major stakeholders,
% including academia, industry, and government, the number of
% biomarkers\footnote{Accoding to the \smallcaps{nih}, a biomarker is defined as
%   ``a characteristic used to measure and evaluate objectively normal biological
%   processes, pathogenic processes, or pharmacological responses to a therapeutic
%   intervention.'' (\citealp{downing_biomarkers_2001})} receiving \smallcaps{us}
% Food and Drug Administration (\smallcaps{fda}) 4 clearance has declined
% substantially over the last ten years to less than one protein biomarker per
% year.''

% Among the most frequent causes for failure to validate a biomarker,
% Pavlou et al.~quoted: pre-analytical factors (e.g., patient selection bias;
% poor sample collection, handling and storage); analytical factors (e.g.,
% methodological artifacts; poor analytical methodology); statistical factors
% (e.g., small sample size; inadequate statistical analysis; data overfitting;
% non-independence of training and validation patient cohorts); and clinical
% validation factors (e.g., non-reproducible validation; poor study design; lack
% of clinical performance).

% If anything, the lackluster translational output from the bench to the bedsite
% has not refrained the appeal of high throughput genomics technologies in cancer
% research.  Following the launch of the Cancer Genome Project (\smallcaps{cgp})
% in the United Kingdom in 2000, and of The Cancer Genome Atlas (\smallcaps{tcga})
% in the United States in 2006, the International Cancer Genome Consortium
% (\smallcaps{icgc}) was created in 2007 to coordinate the generation of
% comprehensive catalogues of genome alterations in 52 different cancer
% types.\cite{watson_emerging_2013,lawrence_mutational_2013} Since then,
% comprehensive genomic characterizations of glioblastoma, colon cancer, breast
% cancer, lung cancers, renal carcinoma, gastric carcinoma, bladder carcinoma, and
% thyroid
% carcinomas,\cite{mclendon_comprehensive_2008,the_cancer_genome_atlas_network_comprehensive_2012-1,the_cancer_genome_atlas_network_comprehensive_2012,the_cancer_genome_atlas_research_network_comprehensive_2012,the_cancer_genome_atlas_research_network_comprehensive_2014,the_cancer_genome_atlas_research_network_comprehensive_2013,the_cancer_genome_atlas_research_network_comprehensive_2014-1,the_cancer_genome_atlas_research_network_comprehensive_2014-2,agrawal_integrated_2014}
% among others, have been released in the public domain.

% Since then, comprehensive genomic characterizations of
% globalism,\cite{mclendon_comprehensive_2008} colon
% cancer,\cite{the_cancer_genome_atlas_network_comprehensive_2012-1} breast
% cancer,\cite{the_cancer_genome_atlas_network_comprehensive_2012} lung
% cancers,\cite{the_cancer_genome_atlas_research_network_comprehensive_2012,the_cancer_genome_atlas_research_network_comprehensive_2014}
% renal
% carcinoma,\cite{the_cancer_genome_atlas_research_network_comprehensive_2013}
% gastric
% carcinoma,\cite{the_cancer_genome_atlas_research_network_comprehensive_2014-1}
% bladder
% carcinoma,\cite{the_cancer_genome_atlas_research_network_comprehensive_2014-2}
% and thyroid carcinomas,\cite{agrawal_integrated_2014} among others, have been
% released in the public domain.

% These % multi-disciplinary and
% large collaborative efforts bring together recent advancements in
% high-throughput technologies,\footnote{Tran et al.~provide a broad overview on
%   the latest technological developments in cancer genomics
%   (\citealp{tran_cancer_2012}).} % \cite{tran_cancer_2012}
% to integrate information provided by \smallcaps{dna} copy number arrays,
% \smallcaps{dna} methylation arrays, exome sequencing, messenger \smallcaps{rna}
% arrays, micro\smallcaps{rna} sequencing and reverse-phase protein arrays---in
% order to re-examine the molecular portraits of human cancers drawn by first
% generation microarrays fifteen years ago.  Cross-sectional global profilings of
% cancer are motivated by the premise that no single type of molecular approach
% can fully elucidate neoplastic biology,\cite{hanash_integrated_2004} and that
% intra-tumour heterogeneity is a defining feature of the molecular pathogenesis
% of cancer.\cite{yates_evolution_2012,almendro_cellular_2013}

% While aiming for a unified genomic theory of the disease, comprehensive
% molecular portraits of cancer have contributed with novel insights into
% neoplastic biology, including---the discovery of frequent somatic mutations in
% functional elements of non-coding regions of the genome, including promoter and
% enhancing regions;\cite{horn_tert_2013,huang_highly_2013} the identification of
% particular signatures of mutational processes, linked with patient's age, known
% mutagenic exposures or defects in \smallcaps{dna}
% maintenance;\cite{nik-zainal_mutational_2012} and, perhaps most enticingly, the
% possibility to recapitulate the evolutionary course of the disease, through the
% mapping of clonal and sub-clonal point and structural mutations in longitudinal
% and multi-regional samplings of tumours.\cite{watson_emerging_2013}

% the distribution of clonal and subclonal mutations and somatic copy-number
% aberrations in longitudinal and multi-region samplings.

% The technological arsenal at the service of integrated molecular catalogs of
% cancer can also be used to address the vision for personalized cancer medicine
% evoked by early microarrays.\cite{sander_genomic_2000}
% While comprehensive molecular catalogs of cancer were primarily assembled to
% seek a unified genomic theory of the disease, their technological arsenal was
% also used to address the vision for personalized cancer medicine evoked by early
% microarrays.\cite{sander_genomic_2000}
% Personalized medicine envisions to maximize treatment efficacy, and minimize its
% morbidity, through the tailoring of medical response based on individual tumour
% genomics.  Several pioneering personalized medicine programs are currently under
% way in the United States, namely at the Dana-Farber/Brigham and Womens Cancer
% Center, at the Massachusetts General Hospital, at the Memorial Sloan-Kettering
% Cancer Center, at the \smallcaps{md} Anderson Cancer Center, at the Oregon
% Health \& Science University, and at the Vanderbilt Cancer
% Center.\cite{macconaill_clinical_2011}

% Again, a familiar range of complex challenges compound to the limited
% effectiveness of global genomic approaches for diagnostic and treatment of
% cancer.  Issues related to the amount and quality of biological material
% available from patients; the high cost and slow turnaround time of
% high-throughput genomic assays; and the lack of consolidated bioinformatics
% pipelines for cancer genome diagnostics all contribute to narrow the windows of
% opportunity offered by personalized medicine programs.  But even when high
% quality data can be assembled, it is the insufficient working knowledge of the
% core dynamic biological processes driving disease progression that hinders the
% development of treatments effective at halting and receding specific forms of
% cancer progression.

% An example of such a personalized medicine program comes from the Mount Sinai
% School of Medicine, in New York, where fruit fly models bearing genetic make-ups
% based on patients' sequenced tumours are being engineered.  The flies are then
% used to screen for optimal chemotherapeutic combination of drugs tailored for a
% patient's individual cancer.\cite{warren_theres_2013} In spite of the extensive
% resources dedicated to profile genetic variation in selected individual
% patients, the use of empirically substantiated drug cocktails has been a
% recurring---if not the prevailing---strategy to innovate on cancer treatments.
% To this effect, several biological systems have been established (cancer cell
% lines; murine, zebrafish and \emph{Caenorhabditis elegans} systems, among
% others), to model the response of human cancers to selections of potential
% tumouricidal compounds.\cite{sharma_cell_2010,heyer_non-germline_2010,potts_cell_2011,gonzalez_drosophila_2013,white_zebrafish_2013}

% An equally fatalistic sense of gloom can be found on the therapeutic development
% front,

% These programs are however facing significant challenges related to the amount
% and quality of biological material available from each patient; the high cost
% and slow turnaround time of high-throughput genomic assays; the lack of
% consolidated bioinformatics pipelines for cancer genome diagnostics. and the
% lack of sufficient biological knowledge of relevant pathways, their
% relationships and interactions, to facilitate the contextual interpretation of
% the data.

% into integrated molecular portraits of each cancer.

% The volume of data produced by these pan-genomic characterization efforts has
% been staggering, and the this wealth of resources is still being actively
% mined.

% Global assessments of the expression of m\smallcaps{rna} molecules in neoplastic
% biospecimens has been gradually complemented by assays probing
% mi\smallcaps{rna}s, seeking comprehensive molecular global portraits

% On one hand, the integration of distinct sources of evidence been into a general
% systems biology narrative\cite{tarabichi_systems_2013} have

% \medskip{}

% \begin{figure}[ht]
%   \includegraphics{cancer-mortality-rate.pdf}
%   \caption[Age-adjusted death rates for selected leading causes of death: United
%   States, 1958–2010]{Age-adjusted death rates for selected leading causes of
%     death: United States, 1958–2010.  Source: \smallcaps{us} National Vital
%     Statistics Reports (\citealp{murphy_deaths:_2013}).}
%   \label{fig:cancer-mortality-rate}
% \end{figure}

% Meanwhile, targeted cancer therapies have primarily sought to address neoplastic
% progression with chemotherapeutic compounds targeting the hallmarks of
% cancer.\cite{hanahan_hallmarks_2011} These include drugs aiming at inhibiting
% proliferative signals (e.g., the monoclocal antibody \emph{Cetuximab}) and
% tumour-driven angiogenesis (e.g., the angiogenesis inhibitor
% \emph{Bevacizumab}); at preventing evasion from apoptosis (e.g., the Bcl-2
% antisense m\smallcaps{rna} inhibitor \emph{Oblimersen Sodium}); or at inhibiting
% tumor-activated kinases that regulate cancer cell proliferation and survival
% (e.g., the tyrosine-kinase inhibitor \emph{Imatinib}).\cite{pavet_towards_2011}
% Other promising therapeutic prospects include: the incitement of differentiation
% of acute myelogenous leukemic blasts into mature white blood cells via the
% administration of all-trans-retinoic acid, or \smallcaps{atra}, which targets
% the oncogenic fusion protein
% \mbox{\smallcaps{pml-rar\,$\alpha$}};\footnote{Zhen-yi Wang and Zhu Chen, of the
%   Ruijin Hospital, in Shanghai---who had the idea of using \smallcaps{atra} to
%   treat \smallcaps{aml}---, are said to have been inspired by a passage from the
%   Analects of Confucius: ``If you use laws to direct the people, and punishments
%   to control them, they will merely try to evade the punishments, and will have
%   no sense of shame. But if by virtue you guide them, and by the rites you
%   control them, there will be a sense of shame and of right.''
%   (\citealp{groopman_is_2014}).} the enlistment of the patient's immune system
% via the administration of \emph{Ipilimumab}, an immunotherapeutic drug targeting
% the \mbox{\smallcaps{t}-lymphocyte-associated antigen 4}, which has been shown
% to extend survival time in stage \smallcaps{iv} melanoma when combined with a
% peptide vaccine;\cite{hodi_improved_2010} or the attempts at recruiting the
% tumor necrosis factor-related apoptosis-inducing ligand (\smallcaps{trail})
% pathway, which has been shown to induce apoptosis in a tumour-selective
% manner.\cite{ma_novel_2009}

% In spite of this remarkable widening of the anti-cancer pharmacopoeia, little
% improvement has been made concerning the management of their toxic
% side-effects---undergoing cancer treatment is too often as much of an ordeal as
% the disease itself.\cite{cleeland_reducing_2012} This can be partially explained
% by the generic scope of most anti-cancer drugs and their inability to
% specifically target malignant neoplastic cells.  Perhaps even more pressing is
% the widening gap between the advances in fundamental biomedical cancer research
% and corresponding measurable gains in clinical research, with new treatments,
% diagnostics and prevention.  This phenomenon, oft-described as ``the valley of
% death,''\cite{butler_translational_2008} has been invoked to explain why cancer
% has virtually retained, in the United States, the same mortality rate in the
% last fifty years (Figure~\ref{fig:cancer-mortality-rate}).

% that eradicate malignant cells

% \medskip{}
% at countering immune evasion of cancer cells.

% In spite of notable improvements, the many efforts to devise chemical cocktails
% to halt and eradicate disease progression with high specificity have not been
% met with success for most cancers.

\bigskip{}

% Main problem is treatment-derived toxicity.

% The valley of death.

% Novel chemotherapeutic (genotoxic) compounds are continuously being developed,
% despite the induction of serious side effects arising from the damage caused to
% normal tissue.

% Irrespective on the accumulating knowledge on tumour-specific features, at
% present the corresponding targeted therapies have only in rare cases led to
% cure.

% None of these therapies were informed by first generation transcriptomics, of
% which high-throughput microarray technologies were at the core.

\smallskip{}

% So the paradox remains.  Of an age when tumours are being dissected at a never
% imagined resolution, and human, financial and technological resources allocated
% to the war on cancer are at their peak, our ability to restrain disease
% progression, specially in its advanced stages, has anything but stalled.  This
% maximal deployment of resources is met by shrinking expectations from the
% patient's point of view: long past are the days when a cure was seemingly
% waiting to be found around the corner.

% What avenues for cancer research?

\clearpage

% Interaction of different levels of genomic data for cancer clinical cancer
% prediction, such as opy number variants at the genome level, DNA methylation at
% the epigenome level, and gene expression and microRNA at the transcriptome level
% (Kim 2012)  On the same subject, aslo: Integrated global profiling of cancer
% (Samir Hanash, 2004).

% Resume of current therapies for cancer research---none of them informed by
% genomics technology.

% A resume on how molecular markers have impacted clinical oncological practice
% (Sidarsky).

% Ashworth discusses the Genetic Interactions in Cancer Progression and Treatment.

% But if the past is precedent, realizing this future will be exceedingly
% difficult.

% In our second contribution made in this dissertation, we show that \ldots

% Clinical outcome prediction by miRnas, Ioannidis.

% The reference model for clinical outcome prediction based on molecular
% information is breast cancer.

% hmi-kim-2014.pdf

% While the TCGA and ICGC provide many opportunities to uncover the novel
% knowledge of the molecular basis of cancer, it is crucial to address the issue
% of development of an appropriate methodological framework for data integration
% to better understand different cancer phenotypes, further providing an
% enhanced global view on the interplays between different genomic
% features. With an abundance in of multi-omics data and clinical data from
% cancer patients, relevant integration frameworks will be valuable for
% explaining the molecular pathogenesis and underlying biology in cancer,
% eventually leading to more effective screening strategies and therapeutic
% targets in many types of cancer.

% kim doesn't like michiels!

% Our results suggest that a better understanding of how best to assess classi-
% fiers in cancer prognostics is needed before generalized conclusions can be
% rendered. The utility of microarrays and other high-throughput technologies
% has been shown in clinical applications and safety assessment. It seems ap-
% parent that the excitement surrounding microarrays in general and the rush to
% publish in this area in particular have resulted in published studies that
% were not ade- quately validated. Publications such as Michiels et al.'s (19)
% have brought widespread attention to the early and perhaps premature studies
% that made unsupportable inferences. However, caution is also warranted in
% indict- ing microarrays in a broad way as unreliable, when so much work
% remains to be done in understanding how best to analyze such complex data and
% the extension of assessing the utility of a classifier derived from such data.

% Schemes for classification of tumour subtypes are notoriously less performing,

% Topics to discuss:
% \begin{itemize}
% \item expectations \emph{vs} reality
% \item solid achievements (differentiation sigs)
% \item questions raised and issues to address (extent of the prognostic signals)
% \item shortcomings of the technology regarding the translation into the
%   clinical realm (nrc-weigelt-2012.pdf)
% \end{itemize}

% \bigskip

% Where is cancer research now?
% \begin{itemize}
% \item The importance of genomics to seize the biology of cancer.
% \item Vanguard treatments and their experimental motivations.
% \item is curing cancer a realistic expectation (as it was in the mid-20th
%   century), or will we have to settle for improvement of diagnostic/prognostic
%   tools?
% \item why are genomics failing in producing a clarification of cancer as
%   a biological phenomena?
% \item paradox: maximum depth of analytical tools, minimal expectations regarding
%   clinical management of disease
% \end{itemize} tissue quality standards for biomarker

% assaying; the influence of molecular heterogeneity of the disease when deriving
% prognostic and predictive markers; the complexity of the tumour's transcriptomic
% landscape, which may keep germane biological signals beyond the resolution of
% microarray technology; and the possible widespread nature of tumour resistance
% mechanisms, which may operate at different molecular levels.

% A wide-range of breast cancer prognostic signatures has been reported in the
% last decade.\cite{liu_breast_2014} While their overlap is limited, their
% prognostic ability has been narrowed down to two main components of
% breast-derived neoplasias' biology: proliferative activity, and \smallcaps{er}
% signaling.  Nevertheless, the deceptive biological specificity of these
% biomarkers has inspired a burgeoning literature purporting to report the
% implication of biological phenomena in breast cancer progression.  This was
% routinely achieved by validating the prognostic ability of a biologically
% motivated gene signature in a cohort of breast cancer expression
% profiles.\cite{chang_gene_2004} Venet et al.~have deconstructed this reasoning
% by showing that most random gene expression signatures are associated with
% breast cancer outcome.\footnote{This fallacy was exposed, via \emph{reductio ad
%     absurdum}, with the demonstration that expression signatures of postprandial
%   laughter and of mice social defeat can both predict overall survival in the
%   \mbox{295-sample} \smallcaps{nki} reference cohort
%   (\citealp{venet_most_2011}).}  In this dissertation, we have have extended
% these findings by characterizing the extent of prognostic signals in 114 cohorts
% of human cancers afflicting nineteen organ systems.\footnote{Under revision.}
% We have shown that, in most cancer transcriptomes, the nature of prognostic
% signals is notoriously pervasive and their assessment is highly sensitive to
% sampling variance.  We describe a breast cancer case study where a sizeable
% fraction of the prognostic signals quantified could be ascribed to spurious
% correlations related to a critical normalization artifact.  We also established
% that the extent of prognostic signals in transcriptomes of breast cancer are
% also related to the fraction of \smallcaps{er+} neoplastic transcriptomes
% prognosticated and, to a lesser extent, to the fraction of node positive
% patients under analysis.

%%% Local Variables:
%%% reftex-default-bibliography: ("../../thesis.bib")
%%% mode: latex
%%% TeX-master: "../../thesis"
%%% End:
