\section{Cancer}
\label{cancer-discussion}

% This is the \emph{cancer} section of the discussion.

\newthought{Cancer research} aims to improve the diagnosis and treatment through
better disease classification and patient stratification.  This allows for the
design of therapies that are better targeted to specific cancer subtypes and
improve the effectiveness of existing regimens, while reducing their morbidity.

In this dissertation, we present two case studies for the application of gene
expression signatures to address supervised tasks of class prediction, and
outcome prediction, based on publicly available transcriptomes of neoplastic
samples.  Each analysis addresses a problem related to the clinical management
of oncological pathologies, framed by the analytic corpus developed for
microarray data research, and grounded by the current understanding of the
biology of cancer.

% \medskip

\subsection{Molecular classification of cancer}
\label{sec:molecular-classification}

% classification
Our first contribution concerns the improvement of molecular diagnostics for
tumour classification based on expression profiles of clinical samples.
Traditionally, tumour classification rests on morphological characterization and
immunohistochemical assessment of tissue-specific antigens.  Histological types
thusly defined denote distinct clinical behaviours and responses to treatment.
Molecular characterization of tumour subtypes aims to increase the accuracy of
tumour classification based on features that escape histomorphological
assessment alone.

Following the proof-of-principle molecular classification of leukemia by Golub
et al.~in 1999,\cite{golub_molecular_1999} molecular cancer classification was
generalized to solid tumours by Ramaswamy et al. and Su et al.~in
2001.\cite{ramaswamy_multiclass_2001,su_molecular_2001} Both studies present an
experimental design where an initial number of primary carcinomas (respectively
144 and 100), spanning most human solid cancers, was used to train a multiclass
predictor to determine the anatomical origin of a random test sample, given its
expression profile.  Reported accuracies of these classifiers ranged from 78\%
to 83\%.  Ramaswamy et al.~remarked that poorly differentiated tumours are less
amenable to molecular classification.  They interpreted this observation as
evidence for a distinct molecular pathogenesis of poorly differentiated tumours
when compared to their well differentiated counterparts.  Su et al.~noted that,
for eleven of the tumour classes they analyzed, a minimal core set of genes is
sufficient to accurately discriminate between classes.  This suggests that
expression of genes particular to the basal physiology and morphology of the
respective tissues of origin might be actionable for molecular classification.

We reasoned that these findings may well be explained by the projection of
cancer progression along axes of molecular dedifferentiation and increased
proliferation.  We then demonstrated that two biomarkers, one of thyroid
differentiation and another of proliferation, both derived from healthy tissues,
could be used to accurately classify tumour subtypes of thyroid
origin.\cite{tomas_general_2012}

In spite of a promising decade for the translation of these research findings
into relevant diagnostics with an impact on clinical management, a review of the
field by Schnabel and Erlander\cite{schnabel_gene_2012} cites only three
commercial molecular cancer classifiers currently available for clinical use
(Table~\ref{tab:molecular-classifiers}).

\begin{table}
  \footnotesize
  \centering
  \begin{tabular}{m{1.7cm}C{2.5cm}C{2.5cm}C{2.5cm}}
    % \begin{tabular}{m{1.7cm}ccc}
        %         \setlength{\extrarowheight}{20pt}
    \toprule
    & bioTheranostics Cancer \smallcaps{type} \smallcaps{ID}\textregistered{} & Rosetta
                                                                                Genomics
                                                                                mi\smallcaps{R}view met
                                                                                2\texttrademark{}
    & PathworkDiagnostics\textregistered{} Tissue of Origin\\
    \midrule
    Tissue \newline specimen & formalin-fixed and paraffin-embedded & formalin-fixed and
                                                                      paraffin-embedded &
                                                                                          formalin-fixed
                                                                                          and
                                                                                          paraffin-embedded\\
    Assay \newline technology (biomolecule) & \smallcaps{rt-pcr} (m\smallcaps{rna}) & microarray
                                                                                      (mi\smallcaps{rna}) &
                                                                                                            microarray
                                                                                                            (m\smallcaps{rna})\\
    Biomarkers assessed & 92 & 64 & 2000\\
    Tumour types classified & 54 & 42 & 15\\
    Accuracy & 87\% &  85\% & 89\% \\
    Tumour subclassification & Yes & No & No\\
    \smallcaps{fda} clearance & No & No & Yes\\
    \bottomrule
  \end{tabular}
  \caption[Commercial molecular cancer classifiers currently available for clinical use]{Commercial molecular cancer classifiers currently available for clinical use (adapted from \citealp{schnabel_gene_2012}).}
  \label{tab:molecular-classifiers}
\end{table}

Of the three, only one attempts to class molecular tumour subtypes; the other
two are designed to locate the primary tumour in patients that present with
cancers of uncertain origin in the metastatic setting.  Only the
PathworkDiagnostics\textregistered{} test has met standards for \smallcaps{fda}
clearance, yet none of these assays has properly challenged traditional
classification methods.  At best, they have found their way into clinical
management as molecular correlates to histopathological findings.

In their expert opinion piece, Schnabel and Erlander suggest that clinical
adoption of gene expression-based classifiers has been curbed by the lack of
clinical gold standards; and by the potential confounding effect of tumour
heterogeneity in molecular evaluations of mixed populations of cells.  Molecular
classifiers can, however, contribute to a clinical decision whenever
pathological assessment is uncertain or conflicting; or even rule out unlikely
tumour classes, facilitating the decision process.

Interestingly, they also raise the possibility that recent paradigm shifts in
neoplastic disease conceptualization may also contribute for the lack of reach
of molecular classifiers in the clinical setting.  Namely, the recent increased
attention given to the complex nature of heterotypic signalings in the tumour
microenvironment,\cite{weigelt_need_2014} a feature that lies beyond the scope
of microarray technology resolution, may well account for the struggle of
traditional molecular classifiers to recapitulate neoplastic disease in its
essence.

\subsection{Molecular outcome prognostication of cancer}
\label{sec:molecular-prediction}

Cancer prognostication is one of the most challenging tasks in oncology.  Here,
the consensus is to adhere to formal anatomical staging systems, such as the the
\smallcaps{tnm} staging system,\cite{sobin_tnm:_2003} which provide a basis for
prediction of survival, choice of initial treatment and stratification of
patients in clinical trials.\cite{ludwig_biomarkers_2005}

\clearpage

In our second contribution made in this dissertation, we show that \ldots

The reference model for clinical outcome prediction based on molecular
information is breast cancer.

hmi-kim-2014.pdf

While the TCGA and ICGC provide many opportunities to uncover the novel
knowledge of the molecular basis of cancer, it is crucial to address the issue
of development of an appropriate methodological framework for data integration
to better understand different cancer phenotypes, further providing an enhanced
global view on the interplays between different genomic features. With an
abundance in of multi-omics data and clinical data from cancer patients,
relevant integration frameworks will be valuable for explaining the molecular
pathogenesis and underlying biology in cancer, eventually leading to more
effective screening strategies and therapeutic targets in many types of cancer.

nrc-hanash-2004
Integrated global profiling of cancer

kim doesn't like michiels!

Our results suggest that a better understanding of
how best to assess classi- fiers in cancer prognostics is needed before
generalized conclusions can be rendered. The utility of microarrays and other
high-throughput technologies has been shown in clinical applications and safety
assessment. It seems ap- parent that the excitement surrounding microarrays in
general and the rush to publish in this area in particular have resulted in
published studies that were not ade- quately validated. Publications such as
Michiels et al.'s (19) have brought widespread attention to the early and
perhaps premature studies that made unsupportable inferences. However, caution
is also warranted in indict- ing microarrays in a broad way as unreliable, when
so much work remains to be done in understanding how best to analyze such
complex data and the extension of assessing the utility of a classifier derived
from such data.



% Schemes for classification of tumour subtypes are notoriously less performing,

\bigskip


% Topics to discuss:
% \begin{itemize}
% \item expectations \emph{vs} reality
% \item solid achievements (differentiation sigs)
% \item questions raised and issues to address (extent of the prognostic signals)
% \item shortcomings of the technology regarding the translation into the
%   clinical realm (nrc-weigelt-2012.pdf)
% \end{itemize}

% \bigskip

% Where is cancer research now?
% \begin{itemize}
% \item The importance of genomics to seize the biology of cancer.
% \item Vanguard treatments and their experimental motivations.
% \item is curing cancer a realistic expectation (as it was in the mid-20th
%   century), or will we have to settle for improvement of diagnostic/prognostic
%   tools?
% \item why are genomics failing in producing a clarification of cancer as
%   a biological phenomena?
% \item paradox: maximum depth of analytical tools, minimal expectations regarding
%   clinical management of disease
% \end{itemize}

\clearpage

%%% Local Variables:
%%% TeX-engine: xetex
%%% mode: latex
%%% TeX-master: "../../thesis"
%%% End:
