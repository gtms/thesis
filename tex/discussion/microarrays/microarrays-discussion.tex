\section{Microarrays}
\label{microarray-discussion}

% This is the \emph{microarrays} section of the discussion.

\newthought{Microarray technology,} through the simultaneous assessment of the
expression of thousands of genes, became the first molecular biology tool
capable of addressing the complex polygenic nature of
cancer.\cite{grant_microarrays_2004} Genomic perturbations drive cancer
progression by disturbing mechanisms for cell cycle control, differentiation,
\smallcaps{dna} repair, apoptosis, tumour vascularization, and metabolism.  A
common approach to investigate how these mechanisms are impacted during cancer
progression is through the monitoring of gene expression signatures, as
surrogates of biological processes, in expression profiles of cancer biopsies.
In this thesis, we present the result of two analyses relating to the use of
gene expression signatures as biomarkers in cancer research.  The first concerns
the use of differentiation and proliferation signatures in cancer diagnostic;
the second regards the extent of prognostic signals in cancer transcriptomes.
Here we offer a discussion of these contributions within the broader context of
the use of microarray technology in cancer research.  This section will then
conclude with some remarks on the challenges in the interpretation and analysis
of microarray data.

\subsection{Differentiation and proliferation signatures in cancer diagnostic}
\label{discussion-differentiation-microarrays}
% Monday 02Mar2015

Hello, have a look at this nice figure (Figure~\ref{fig:diff-prolif}).

\begin{marginfigure}%
  \begin{center}
    % \includegraphics[width=9cm]{microarrays-economist.jpg}
    \includegraphics{proliferation-differentiation.png}
    \caption[Differentiation proliferation in cancer]{A schematic view of
      differentiation and proliferation in cancer progression (see text for
      details).}\label{fig:diff-prolif}%
  \end{center}
\end{marginfigure}%



\subsection{The extent of prognostic signals in the cancer transcriptomes}
\label{sec:discussion-prognostic-microarrays}
% Tuesday 03Mar2015

\subsection{Microarray data interpretation and analysis}
\label{sec:discussion-interpretation-microarrays}
% Wednesday 04Mar2015

% End up with a wink to underlying models for cancer progression.  In other
% words, interpretation of microarray data is dependent on the underlying
% biological model for the system under study.

\clearpage

%%% Local Variables:
%%% TeX-engine: xetex
%%% mode: latex
%%% TeX-master: "../../thesis"
%%% End:
