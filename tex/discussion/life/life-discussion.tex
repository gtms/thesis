\section{Life}
\label{life-discussion}

\newthought{With projects} like The Cancer Genome Atlas drawing to an end,
discussion is now shifting to where should cancer research go next.  Eric
Lander, at the Broad Institute, has contended that ``completing the genomic
analysis of this disease should be a biomedical
imperative.''\cite{lawrence_discovery_2014} According to Lander, the
identification of novel cancer genes from the mining of the \smallcaps{tcga}
data provides the motivation to extend the search for genetic determinants of
the disease to even larger panels of samples.

Elsewhere, skepticism has started to be voiced regarding the value of pursuing
the entire catalog of cancer genes towards the advancement of new therapies.
Bert Vogelstein, who championed the genetic nature of cancer in the early
2000's, is less sure about whether extending the atlas project would be the best
allocation of research funds: ``There’s no question that it would be
valuable. The question is whether it’s worth it.''\cite{zimmer_catalog_2014}

\medskip

In this dissertation, we discussed applications of molecular biomarkers for
cancer diagnostics and prognosis, based on microarray technology.  We conclude
that the use of gene expression signatures in cancer research has shown great
promise (e.g., with the use of differentiation and proliferation signatures to
assist cancer diagnosis), but may have also promoted unfunded expectations
(e.g., by mis-interpreting the nature of pervasive prognostic signals in cancer
transcriptomes).  The research programs here detailed were designed to exploit
the vast compendiums of cancer expression profiles available in the public
domain, and their results must, therefore, first be contextualized within the
technical and analytic frame of the technology that enabled them.

While microarray technology allows for the quantification of m\smallcaps{rna}
products of cancer biospecimens in solution, it only lends itself to the
detection of gene products already mapped and printed in the chip of the
platform of choice.  Furthermore, eventual mutations (either nucleotide changes
or structural variants) in genes being transcribed may impair the correct
estimation of their transcription levels due to the adulteration of their
m\smallcaps{rna} sequences.  Even assuming correct estimates of the
transcriptomic load of a given biospecimen, tissue cellularity and heterogeneity
may complicate the interpretation of bulk expression profiles.  Finally, as
microarray technology only probes the Central Dogma of Biology \footnote{The
  central dogma of molecular biology, postulated by Francis Crick in 1958 and
  reasserted in 1970 (\citealp{crick_protein_1958,crick_central_1970}), pertains
  to the rules that govern the sequential flow of genetic information between
  \smallcaps{dna}, \smallcaps{rna} and proteins.  It can be summarized as
  `\smallcaps{dna} makes \smallcaps{rna} makes protein,'' which provides the
  template for the enactment of hereditary information for all living organisms,
  and frames the scope of evolutionary forces on genetic systems.} at the
transcriptional level, it provides no insight on upstream modulatory effects on
gene expression (e.g., epigenetic determinants), as well as on downstream
modulatory effects on gene products (e.g., post-transcriptional modifications).

The interpretation of microarray data is challenged by issues regarding
non-specific probe hybridization, non-consensual probe annotations, poorly
defined gene ontologies and an incomplete understanding of the dynamics of gene
expression on \emph{in vivo} biological systems.  Nevertheless, and in spite of
having fell short of the promise for personalized medicine, microarrays have
spectacularly delivered in a number of ways: by offering a first glimpse at the
previously unsuspected molecular taxonomy of many forms of cancer; by allowing
for a quantitative diagnostic of neoplastic disease; by enabling the molecular
dissection of cancer biospecimens with biologically motivated gene expression
signatures; by unveiling disease-specific molecular prognostic markers; or by
providing a framework to allow for patient stratification based on gene
expression profiles.  Looking forward, we foresee microarray technology to
retain its relevance in the clinical setting as a diagnostic tool, and for the
plethora of publicly available expression profiles to remain a valuable mining
resource, as novel insights on the biology of cancer come to the fore.

% Unsurprisingly, many of the shortcomings of microarray technology listed above
% can also be assigned to current state-of-the-art high throughput genomics
% technologies.  This is because current biomedical research is conducted under
% the paradigms prescribed by the emergence of molecular biology in the 1970's.
% With the discovery of the structure of \smallcaps{dna} and the unlocking of the
% genetic code, much of cell biology became redirected to the nucleus---not the least
% because \smallcaps{dna} is a particularly stable and amenable molecule for
% experimental manipulation.

% Up to then, medical research was fundamentally carried out by physicians with
% personal scientific interests who also treated patients.

% For all the limitations of microarrays, it might even be argued
% that, with their proven and tested pre-processing bioinformatics pipelines, they
% might still provide a solid operational platform for cancer genomics.

% Only then should their eventual contribution at the level of the current
% conceptual understanding of the disease under study be considered.

% start by discussing the results of the thesis and their shortcomings in the
% context of microarray research

% why are we failing to treat cancer?

% We know a lot about the biology of the dieases---but not nearly as much as we need.

% historical reasons

% Medical research was largely done by physician–scientists who also treated
% patients. That changed with the explosion of molecular biology in the
% 1970s. Clinical and basic research started to separate, and biomedical research
% emerged as a discipline in its own right, with its own training. The bulk of
% biomedical research is now done by highly specialized PhD scientists (see
% graph), and physician–scientists are a minority.

% conceptual reasons: Gould versus Dawkins

% \newthought{Much like} any human endeavour, the collective search to elucidate
% and defeat cancer is fraught with

% \newthought{Much like} sound models of cancer progression can inform the
% interpretation of large-scale genomic experiments, evolutionary perspectives of
% the history of life can equally be used to steer our understanding of the
% disease itself.

% This is the \emph{life} section of the discussion.

% We've learned a lot about genetic systems, but not yet enough.

% Pervasive transcription of the human genome produces thousands of previously
% unidentified long intergenic coding RNAs.  Only 2% of our genome displays
% protein-coding capacity.

% non-coding \smallcaps{rna}s (nc\smallcaps{rna}s) are an heterogeneous group of
% biomolecules that has been classified in three families concerning their size:
% (\emph{a}) molecules ranging from 18 to 25 nucleotides in length, comprising
% micro\smallcaps{rna}s and small interfering \smallcaps{rna}s; (\emph{b})
% molecules from 30 to 300 nucleotides, involving small nucleolar
% \smallcaps{rna}s, Piwi \smallcaps{rna}s (pi-\smallcaps{rna}s), transfer
% \smallcaps{rna}s, ribosomal \smallcaps{rna}s and small-ncRNAs; and (\emph{c})
% molecules larger than 300 nucleotides, referred to as long non-coding
% \smallcaps{rnao}s (nc\smallcaps{rna}s).

% For example, it has been demonstrated that some mi\samllcaps{rna}s can either
% repress or induce the transcription of a given m\smallcaps{rna}, depending on
% the proliferative status of the cell

% No one like Gould knew how to weave these evolutionary threads into a narrative
% that could so well play as metaphor for cancer itself, this ``strange eventful
% history'':\footnote{\citealp[p. 54]{gould_flamingos_1987}}

% \begin{quotation}
%   Our world is not an optimal place, fine tuned by omnipotent forces of
%   selection.  It is a quirky mass of imperfections, working well enough (often
%   admirably); a jury-rigged set of adaptations built of curious parts made
%   available by past histories in different contexts.  (\ldots) A world optimally
%   adapted to current environments is a world without history, and a world
%   without history might have been created as we find it.  History matters; it
%   confounds perfection and proves that current life transformed its own past.
% \end{quotation}

% Exacting as it may be, the unraveling of these threads may well be key to
% elucidate the nature of cancer; and of that other very peculiar wonder---life.

% The elucidation of these matters will only further our understanding of the
% nature of cancer; % \footnote{``It's bad bile.  It's bad habits.  It's bad bosses.
%   % It's bad genes.''---\emph{Mel Greaves}}
%  and of that other very peculiar wonder---life.

\clearpage

% state-of-the-art high throughput genomics
% technologies.  For all the limitations of microarrays, it might even be argued
% that, with their proven and tested pre-processing bioinformatics pipelines, they
% might still provide a solid operational platform for cancer genomics.

% Only then should their eventual contribution at the level of the current
% conceptual understanding of the disease under study be considered.

% start by discussing the results of the thesis and their shortcomings in the
% context of microarray research

% why are we failing to treat cancer?

% We know a lot about the biology of the dieases---but not nearly as much as we need.

% historical reasons

% Medical research was largely done by physician–scientists who also treated
% patients. That changed with the explosion of molecular biology in the
% 1970s. Clinical and basic research started to separate, and biomedical research
% emerged as a discipline in its own right, with its own training. The bulk of
% biomedical research is now done by highly specialized PhD scientists (see
% graph), and physician–scientists are a minority.

% conceptual reasons: Gould versus Dawkins

% \newthought{Much like} any human endeavour, the collective search to elucidate
% and defeat cancer is fraught with

% \newthought{Much like} sound models of cancer progression can inform the
% interpretation of large-scale genomic experiments, evolutionary perspectives of
% the history of life can equally be used to steer our understanding of the
% disease itself.

% This is the \emph{life} section of the discussion.

% We've learned a lot about genetic systems, but not yet enough.

% Pervasive transcription of the human genome produces thousands of previously
% unidentified long intergenic coding RNAs.  Only 2% of our genome displays
% protein-coding capacity.

% non-coding \smallcaps{rna}s (nc\smallcaps{rna}s) are an heterogeneous group of
% biomolecules that has been classified in three families concerning their size:
% (\emph{a}) molecules ranging from 18 to 25 nucleotides in length, comprising
% micro\smallcaps{rna}s and small interfering \smallcaps{rna}s; (\emph{b})
% molecules from 30 to 300 nucleotides, involving small nucleolar
% \smallcaps{rna}s, Piwi \smallcaps{rna}s (pi-\smallcaps{rna}s), transfer
% \smallcaps{rna}s, ribosomal \smallcaps{rna}s and small-ncRNAs; and (\emph{c})
% molecules larger than 300 nucleotides, referred to as long non-coding
% \smallcaps{rnao}s (nc\smallcaps{rna}s).

% For example, it has been demonstrated that some mi\samllcaps{rna}s can either
% repress or induce the transcription of a given m\smallcaps{rna}, depending on
% the proliferative status of the cell

% No one like Gould knew how to weave these evolutionary threads into a narrative
% that could so well play as metaphor for cancer itself, this ``strange eventful
% history'':\footnote{\citealp[p. 54]{gould_flamingos_1987}}

% \begin{quotation}
%   Our world is not an optimal place, fine tuned by omnipotent forces of
%   selection.  It is a quirky mass of imperfections, working well enough (often
%   admirably); a jury-rigged set of adaptations built of curious parts made
%   available by past histories in different contexts.  (\ldots) A world optimally
%   adapted to current environments is a world without history, and a world
%   without history might have been created as we find it.  History matters; it
%   confounds perfection and proves that current life transformed its own past.
% \end{quotation}

% Exacting as it may be, the unraveling of these threads may well be key to
% elucidate the nature of cancer; and of that other very peculiar wonder---life.

% The elucidation of these matters will only further our understanding of the
% nature of cancer; % \footnote{``It's bad bile.  It's bad habits.  It's bad bosses.
%   % It's bad genes.''---\emph{Mel Greaves}}
%  and of that other very peculiar wonder---life.

\clearpage

%%% Local Variables:
%%% mode: latex
%%% TeX-master: "../../thesis"
%%% End:
