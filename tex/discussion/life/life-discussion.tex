\section{Life}
\label{life-discussion}

% \newthought{Can cancer} be defeated?

% \smallskip{}

\newthought{With projects} like The Cancer Genome Atlas drawing to an end,
discussion is now shifting to where should cancer research go next.  Eric
Lander, at the Broad Institute, has argued that ``completing the genomic
analysis of this disease should be a biomedical
imperative.''\cite{lawrence_discovery_2014} According to Lander, the
identification of novel cancer genes from the mining of the \smallcaps{tcga}
data provides the motivation to extend the search for genetic determinants of
the disease to even larger panels of samples.

Elsewhere, skepticism has started to be voiced regarding the value of pursuing
the entire catalog of cancer genes towards the advancement of new therapies.
Bert Vogelstein, who championed the genetic nature of cancer in the early
2000's, is less sure about whether extending the atlas project would be the best
allocation of research funds: ``There’s no question that it would be
valuable. The question is whether it’s worth it.''\cite{zimmer_catalog_2014}

\medskip

In this dissertation, we discussed applications of molecular biomarkers for
cancer diagnostics and prognosis, based on microarray technology.  We conclude
that the use of gene expression signatures in cancer research has shown great
promise (e.g., with the use of differentiation and proliferation signatures to
assist cancer diagnosis), but may have also promoted unfunded expectations
(e.g., by mis-interpreting the nature of pervasive prognostic signals in cancer
transcriptomes).  The research programs here detailed were designed to exploit
the vast compendia of cancer expression profiles available in the public
domain, and their results must, therefore, first be contextualized within the
technical and analytic frame of the technology that enabled them.

While microarray technology allows for the quantification of m\smallcaps{rna}
products of cancer biospecimens in solution, it only lends itself to the
detection of gene products already mapped and printed in the chip of the
platform of choice.  Furthermore, eventual mutations (either nucleotide changes
or structural variants) in actively transcribed genes may impair the correct
estimation of their transcription levels due to the adulteration of their
m\smallcaps{rna} sequences.  Even assuming correct estimates of the
transcriptomic load of a given biospecimen, tissue cellularity and heterogeneity
may complicate the interpretation of bulk expression profiles.  Finally, as
microarray technology only probes the Central Dogma of Biology\footnote{The
  central dogma of molecular biology, postulated by Francis Crick in 1958 and
  reasserted in 1970 (\citealp{crick_protein_1958,crick_central_1970}), pertains
  to the rules that govern the sequential flow of genetic information between
  \smallcaps{dna}, \smallcaps{rna} and proteins.  It can be summarized as
  ``\smallcaps{dna} makes \smallcaps{rna} makes protein,'' which provides the
  template for the enactment of hereditary information for all living organisms,
  and frames the scope of evolutionary forces on genetic systems.} at the
transcriptional level, it provides no insight on upstream modulatory effects on
gene expression (e.g., epigenetic determinants), as well as on downstream
modulatory effects on gene products (e.g., post-transcriptional modifications).

The interpretation of microarray data is challenged by issues regarding
non-specific probe hybridization, non-consensual probe annotations, poorly
defined gene ontologies and an incomplete understanding of the dynamics of gene
expression on \emph{in vivo} biological systems.  Nevertheless, and in spite of
having fell short of the promise for personalized medicine, microarrays have
notably delivered in a number of ways: by offering a first glimpse at the
previously unsuspected molecular taxonomy of many forms of cancer; by allowing
for a quantitative diagnostic of neoplastic disease; by enabling the molecular
dissection of cancer biospecimens with biologically motivated gene expression
signatures; by unveiling disease-specific molecular prognostic markers; or by
providing a framework to allow for patient stratification based on gene
expression profiles.  Looking forward, we foresee microarray technology to
retain its relevance in the clinical setting as a diagnostic tool, and for the
plethora of publicly available expression profiles to remain a valuable mining
resource, as novel insights on the biology of cancer come to the fore.

\medskip{}

A more insightful discussion of the merits and limits of microarrays and
genomics in cancer research could benefit from a wider historical perspective.
% the first high throughput genomic technology used to map the genetics of
% cancer, could be made from a wider historical perspective.
Current biomedical research is conducted under the paradigms defined by the
emergence of molecular biology in the 1970's.  Following the discovery of the
structure of \smallcaps{dna} and the unlocking of the genetic code, cellular
biology turned its attention to the genome and to the mechanisms by which
hereditary information is enacted throughout the cell.  This surge of interest
in \smallcaps{dna}, complemented by the molecule's stability and amenability to
experimental manipulation, led to the development of technologies aiming for its
extensive isolation and characterization.  These technologies include molecular
cloning, polymerase chain reaction (\smallcaps{pcr}) and Southern blotting---the
precursor of microarrays.

Up to then, medical research was typically carried out by physicians, like
Sydney Farber, with a keen interest in the physiological aspects of disease.
The emergence of biomedical medicine consummated the split between clinical and
basic research, with generations of highly specialized researchers trained under
the central dogma taking over from physician-scientists.

The underlying premise of molecular and genomics approaches to tackle cancer is
the univocal relationship between genotype and a disease phenotype.\footnote{Of
  which this dissertation is an example: statistical associations between
  genetic correlates and, respectively, histo-pathological thyroid cancer types
  and differential survival times of cancer patients are at the core of the
  results here discussed.  Incidentally, this idea is typified by the concept of
  genetic penetrance, or the faction of a given phenotype that is explained by
  its underlying genotype.} This conceptualization does not take into
consideration the possibility of disease phenotypes being the result of dynamic
inter-cellular interplays (e.g., heterotypic signalings in the tumour
ecosystem), does not account for dynamically reversible neoplastic states (e.g.,
\smallcaps{emt}-\smallcaps{met} transitions), and is unconcerned with the
constrictions imposed by evolutionary forces on genetic systems (see bellow).
By seeking to apprehend cancer mainly through the prism of its genetic
alterations, molecular genomics has pushed to the background the
characterization of the many cellular states visited by neoplastic types that,
while enabled by genetic disruptions, are not fully explained by
them.\footnote{Another side-effect of the bias towards genomics in cancer
  research is the large number of biosynthesis products and enzymes remaining
  unannotated in the human genome, resulting in poorly documented ontology
  databases.}

% \footnote{Incidentally, this fact may account for \ldots{}.  These databases,
% mainly populated from results}

To illustrate this point, consider the Warburg effect.  Otto Warburg's
observation in 1924 that cancer cells metabolize glucose in a manner that is
less efficient from that of cells in normal tissues\cite{warburg_ueber_1924} has
remained a matter of contentious speculation throughout the
20\textsuperscript{th} century.  Only recent observations on the metabolism of
neoplastic cells have started to shed light into a possible link between the
uptake and incorporation of nutrients into the biomass during proliferation and
anaerobic glycolysis.\cite{heiden_understanding_2009}

Equally poorly understood are metabolic phenomena driven by nutrient conditions
of the tumour microenvironment and intercellular metabolism---much of our
understanding of metabolism was drawn from work in simple
organisms.\cite{hsu_cancer_2008} If cancer is to be portrayed has a collection
of phenotypic dynamic ranges enabled by a succession of genetic disruptions,
perhaps a thorough characterization of these metabolic states and their
inter-dependencies could provide better clues for targeted treatments than the
exhaustive characterization of all putative cancer
genes.\cite{kroemer_tumor_2008}

\medskip{}

Nevertheless, cancer remains at its core a disease of genetic deregulation.  The
evolutionary history of genetic systems provides another orthogonal perspective
when seeking a more integrated approach to the disease.

The multistep progression of cancer has been largely explained by models of
clonal evolution, driven by the acquisition of genetic variants, then pruned by
Darwinian selection.\cite{greaves_clonal_2012} These models fittingly mirror the
views of Richard Dawkins, who has strongly argued for a gene-centered
interpretation of evolution.\cite{dawkins_selfish_1976} This formulation is
however weakened by the lack of genetic correlates of the hallmarks of cancer
that could be tagged along the evolution of fitter cancer clones---the only
\emph{bona fide} cancer genes reported to date are still classical oncogenes and
tumour suppressor genes.

The paleontologist Stephen J. Gould was a vocal opponent of the gene-centric
perspective of evolution, arguing that most genes may not have a consistent
enough effect on phenotypes' fitness to drive their
evolution.\footnote{Pleiotropy (the capacity of a gene to influence two or more
  seemingly unrelated traits), epistasis (the reliance of a gene on the presence
  of one or more modifier genes to express its phenotype) and polygenic traits
  (traits that are controlled by different genes to different degrees) are given
  as examples of why natural selection lacks reach to shape genetic landscapes.}
One of Gould's main points of contention was what he described as
\emph{adaptationism}, the assertion that natural selection explains the majority
of evolved traits.\cite{gould_spandrels_1979} Instead, Gould quoted functional,
structural and genetic constraints, as well as serendipity itself, as major
evolutionary forces.  For instance, \emph{exaptation}, or the co-option of
structures with different prior functions to a new context, has been proposed to
take part in the evolution of complex traits, such as the human
eye.\cite{gould_exaptation_1982}

% The gene-centered reductionist approach is largely in line with the current genomics
% interpretation of cancer progression.  Gene fitness selection, penetrance.  If
% selection, as Gould sees it, can act only to magnify and sculpt variations
% previously embedded in genetic systems, \ldots{}.

The projection of this debate into cancer evolution has implications at many
conceptual levels.  Consider, for instance, the role of genetic instability in
cancer progression.  Here, orthodoxy posits that: ``instability leads to an
increased mutation rate and can shape the evolution of the cancer genome through
a plethora of mechanisms.''\cite{burrell_causes_2013} This is to say that
genetic instability is an \emph{advantageous} feature, and therefore is
\emph{selected for} by cancers, in order to keep producing more \emph{adaptive}
genetic variants.  This point of view is so widespread in the field that one has
to reach out to the fringes of the literature to find an alternative, more
sensible rationale: ``taking a molecular perspective, nothing can be selected
because it mutates. (\ldots{}) The evolutionary breakdown of a repair strategy
in mutagenic environments is thereby easily explained by the accumulating costs
of repair.''\cite{breivik_evolutionary_2005} In other words, mutagenic
environments in cancer occur simply because it is \emph{too costly} for
defective neoplastic cells to repair their \smallcaps{dna} \emph{and} keep
dividing at the same time.  Similarly, virtually every hallmark of cancer has
been interpreted from an adaptationist stance, without ever making proof of the
dependency of their putative genetic correlates in cancer
progression.\cite{bernards_progression_2002}

Taken together, it seems evident that a reductionist gene-centric formulation of
cancer evolution is falling short to explain the progression of the disease.
Neoplastic evolution cannot just be modeled as the relentless selection of
increasingly fitter hallmark genetic variants.  Rather, it could be refashioned
as the by-product of the loosening of the genetic stack evolved to enforce
social compliance, leaving neoplastic cells to scramble through contextual
adaptive solutions within the scope of the underlying genetic\ circuitries.  If
so, a more consequent strategy to curb cancer progression could pass by focusing
on the universal modular adaptations cancer cells depend on to thrive.  Such
adaptations are likely to be featured, in some form, in models of present-day
genetic lineages that never undertook the transition to multicellularity.

% \mediumskip{}

% Would cancer, defined by genomic instability and its phenotypic heterogeneity
% and plasticity be the best model to probe for these rules?  Perhaps such
% undertaking would best be envisaged in biological models where the deployment of
% the genetic information is faithfully mirrored by phenotypes strictly under
% control of natural selection---embryogenesis, connected with If pursuing the complete
% catalog of cancer genes remains an imperative task in cancer research,
% addressing the gap between phenome and genome.

% non-coding \smallcaps{rna}s (nc\smallcaps{rna}s) are an heterogeneous group of
% biomolecules that has been classified in three families concerning their size:
% (\emph{a}) molecules ranging from 18 to 25 nucleotides in length, comprising
% micro\smallcaps{rna}s and small interfering \smallcaps{rna}s; (\emph{b})
% molecules from 30 to 300 nucleotides, involving small nucleolar
% \smallcaps{rna}s, Piwi \smallcaps{rna}s (pi-\smallcaps{rna}s), transfer
% \smallcaps{rna}s, ribosomal \smallcaps{rna}s and small-ncRNAs; and (\emph{c})
% molecules larger than 300 nucleotides, referred to as long non-coding
% \smallcaps{rnao}s (nc\smallcaps{rna}s).

% For example, it has been demonstrated that some mi\samllcaps{rna}s can either
% repress or induce the transcription of a given m\smallcaps{rna}, depending on
% the proliferative status of the cell

% The elucidation of these matters will only further our understanding of the
% nature of cancer; % \footnote{``It's bad bile.  It's bad habits.  It's bad bosses.
%   % It's bad genes.''---\emph{Mel Greaves}}
%  and of that other very peculiar wonder---life.

% Pervasive transcription of the human genome produces thousands of previously
% unidentified long intergenic coding RNAs.  Only 2% of our genome displays
% protein-coding capacity.

% non-coding \smallcaps{rna}s (nc\smallcaps{rna}s) are an heterogeneous group of
% biomolecules that has been classified in three families concerning their size:
% (\emph{a}) molecules ranging from 18 to 25 nucleotides in length, comprising
% micro\smallcaps{rna}s and small interfering \smallcaps{rna}s; (\emph{b})
% molecules from 30 to 300 nucleotides, involving small nucleolar
% \smallcaps{rna}s, Piwi \smallcaps{rna}s (pi-\smallcaps{rna}s), transfer
% \smallcaps{rna}s, ribosomal \smallcaps{rna}s and small-ncRNAs; and (\emph{c})
% molecules larger than 300 nucleotides, referred to as long non-coding
% \smallcaps{rnao}s (nc\smallcaps{rna}s).

% For example, it has been demonstrated that some mi\samllcaps{rna}s can either
% repress or induce the transcription of a given m\smallcaps{rna}, depending on
% the proliferative status of the cell

\bigskip{}

While modern notions of the disease have dramatically departed from its humoural
depictions of a hundred and fifty years ago, many dots remain to be connected to
yield a definitive, cohesive, and actionable picture of cancer.  To this end,
engaging with historical, biological and evolutionary perspectives could provide
an invaluable contribution.  And no one quite like Gould knew how to weave these
themes into a rendering that could so well play as a metaphor for the malignancy
itself:\footnote{\citealp[p. 54]{gould_flamingos_1987}}
% , this ``strange eventful history'':

\begin{quotation}
  Our world is not an optimal place, fine tuned by omnipotent forces of
  selection.  It is a quirky mass of imperfections, working well enough (often
  admirably); a jury-rigged set of adaptations built of curious parts made
  available by past histories in different contexts.  (\ldots) A world optimally
  adapted to current environments is a world without history, and a world
  without history might have been created as we find it.  History matters; it
  confounds perfection and proves that current life transformed its own past.
\end{quotation}

Exacting as it may be, the unraveling of these threads may well be key to
elucidate the nature of cancer; and of that other very peculiar wonder---life.

\clearpage

% The elucidation of these matters will only further our understanding of the
% nature of cancer; % \footnote{``It's bad bile.  It's bad habits.  It's bad bosses.
%   % It's bad genes.''---\emph{Mel Greaves}}
%  and of that other very peculiar wonder---life.

%%% Local Variables:
%%% mode: latex
%%% TeX-master: "../../thesis"
%%% End:
