\documentclass{tufte-book}

% \usepackage{natbib}
\usepackage{xltxtra}
\usepackage{microtype}
\usepackage{booktabs}

\defaultfontfeatures{Ligatures=TeX}
\setmainfont{Minion Pro}
\setsansfont[Scale=MatchLowercase]{Gill Sans MT}
\setmonofont[Scale=MatchLowercase]{Consolas}

\usepackage{etoolbox}% provides some support for comma-separated lists

% code from:
% http://tex.stackexchange.com/questions/45868/can-i-automatically-generate-abbreviated-citations-in-tufte-documents-after-the
% Can I automatically generate abbreviated citations in Tufte documents after the first occurrence?

\makeatletter
% We'll keep track of the old/seen bibkeys here.
\def\@tufte@old@bibkeys{}

% This macro prints the full citation if it's the first time it's been used
% and a shorter citation if it's been used before.
\newcommand{\@tufte@print@margin@citation}[1]{%
  % print full citation if bibkey is not in the old bibkeys list
  \ifinlist{#1}{\@tufte@old@bibkeys}{%
    \citealp{#1}% print short entry
  }{%
    % !ALWAYS PRINT SHORT ENTRY
    \citealp{#1}% print short entry
    % \bibentry{#1}% print full entry
  }%
  % add bibkey to the old bibkeys list
  \listgadd{\@tufte@old@bibkeys}{#1}%
}

% We've modified this Tufte-LaTeX macro to call \@tufte@print@margin@citation
% instead of \bibentry.
\renewcommand{\@tufte@normal@cite}[2][0pt]{%
  % Snag the last bibentry in the list for later comparison
  \let\@temp@last@bibkey\@empty%
  \@for\@temp@bibkey:=#2\do{\let\@temp@last@bibkey\@temp@bibkey}%
  \sidenote[][#1]{%
    % Loop through all the bibentries, separating them with semicolons and spaces
    \normalsize\normalfont\@tufte@citation@font%
    \setcounter{@tufte@num@bibkeys}{0}%
    \@for\@temp@bibkeyx:=#2\do{%
      \ifthenelse{\equal{\@temp@last@bibkey}{\@temp@bibkeyx}}{%
        \ifthenelse{\equal{\value{@tufte@num@bibkeys}}{0}}{}{and\ }%
        \@tufte@trim@spaces\@temp@bibkeyx% trim spaces around bibkey
        \@tufte@print@margin@citation{\@temp@bibkeyx}%
      }{%
        \@tufte@trim@spaces\@temp@bibkeyx% trim spaces around bibkey
        \@tufte@print@margin@citation{\@temp@bibkeyx};\space
      }%
      \stepcounter{@tufte@num@bibkeys}%
    }%
  }%
}


% Calling this macro will reset the list of remembered citations. This is
% useful if you want to revert to full citations at the beginning of each
% chapter.
\newcommand{\resetcitations}{%
  \gdef\@tufte@old@bibkeys{}%
}
\makeatother

\newenvironment{docspec}{\begin{quotation}\ttfamily\parskip0pt\parindent0pt\ignorespaces}{\end{quotation}}
% command specification environment

\begin{document}

\section{Life}

\newthought{Life is a wonder} of its own.  It conceivably struck this
planet at least some 3.7~billion years ago,\cite{schopf_evidence_2007}
arguably the consequence of a phospholipidic \mbox{layer-bound}
\emph{quantum leap} in a soup of organic
precursors.\cite{miller_organic_1959} From that
singular moment on, little has been spared in guise of
amazement.

The first factual evidence of life on Earth appears inscribed in the
fossil record some 3.5~billion years ago.  It consists mainly of
microfossils and ancient rock structures in Greenland and Australia
called
stromatolites,\cite{ohtomo_evidence_2014,noffke_microbially_2013} the
product of the metabolism of photosynthesizing cyanobacteria.

\begin{quotation}
  There is grandeur in this view of life, with its several powers, having been
  originally breathed into a few forms or into one; and that, whilst this planet
  has gone cycling on according to the fixed law of gravity, from so simple a
  beginning endless forms most beautiful and most wonderful have been, and are
  being, evolved.
\end{quotation}

How ironic that one of the best models to understand this order of things is
actually one of its most harrowing violations: cancer.

\section{Cancer}

\backmatter

\bibliography{life}
\bibliographystyle{plainnat}
% \bibliographystyle{abbrvnat}
% \bibliographystyle{unsrtnat}

\end{document}