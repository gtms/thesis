\section{Life}

% \newthought{Life is a wonder} of its own.  It conceivably struck this planet at
\textsc{Life is a wonder} of its own.  It conceivably struck this planet at
least some 3.8~billion years ago,\cite{mojzsis_evidence_1996} arguably the
consequence of a phospholipidic \mbox{layer-bound} \emph{quantum leap} in a soup
of organic precursors.\cite{miller_organic_1959} From that singular moment on,
little has been spared in guise of amazement.

\bigskip

% life's timeline from wikipedia
% http://en.wikipedia.org/wiki/Timeline_of_evolutionary_history_of_life#cite_note-23

The first factual evidence of life on Earth appears inscribed in the fossil
record some 3.5~billion years ago.  It consists mainly of microfossils and
ancient rock structures in Greenland and Australia called
stromatolites,\cite{ohtomo_evidence_2014,noffke_microbially_2013} the product of
the metabolism of photosynthesizing cyanobacteria.

While the fine details of the abiogenic\footnote{From the Greek ``spontaneous
  generation'', this model posits that autocatalytic polymers able to function
  as simple molecular replicators are at the origin of life.  Alternative models
  exist, most notably panspermia---the extraterrestrial origin of life.} model
for the origin of life\footnote{Much debate has been stirred around the
  definition of life itself, see for instance \citealp{benner_defining_2010}.}
remain to be chalked, its primed instances are often depicted as
\mbox{self-replicating} units able to regulate their own chemical processes
within the confinement of a cellular membrane.  Such primeval unicellular
organisms were most likely composed of some form of nucleic acid secluded from
its external environment by a bilayer of phospholipids.

Nucleic acids, \smallcaps{rna} and \smallcaps{dna}, are the sole organic
molecules fit to direct their own replication and to act as genetic material.
\smallcaps{rna} was proposed as the original genetic system since it was shown
to be able to catalize a variety of chemical reactions, notably the
polymerization of nucleotides.\cite{bass_specific_1984} However, as the genetic
load stored within cells increased, the lesser stability and reliability of
\smallcaps{rna} molecules rendered them unsuitable for the furtherance of
genetic systems---a stage at which \smallcaps{dna} eventually took over as
universal hereditary material.

These very simple and \mbox{self-organized} systems emerged in seas densely
enriched in organic molecules and therefore didn't require sophisticated
toolkits for intake of food or generation of energy.  Only when systematic
interactions between \smallcaps{rna} and specific amino~acids became enshrined
in the genetic code\footnote{The central dogma of molecular biology, postulated
  by Francis Crick in 1958 and reasserted in 1970
  (\citealp{crick_protein_1958,crick_central_1970}), pertains to the rules that
  govern the sequential flow of genetic information between \smallcaps{dna},
  \smallcaps{rna} and proteins. It can be summarized as ``\smallcaps{dna} makes
  \smallcaps{rna} makes protein'', which provides the template for the enactment
  of hereditary information for all living organisms and frames the scope of
  evolutionary forces on genetic systems.} did evolution start to shape
concerted molecular reactions between proteins that animated cells with
metabolic activity.

Among the first elaborations of metabolic pathways are likely to have been some
simpler form of \mbox{modern-day} glycolysis, the anaerobic breakdown of glucose
to lactic acid.  Glycolysis is a fixture of all \mbox{present-day} cells and
drives the generation of \smallcaps{atp} molecules, their universal source of
energy.  This in turn enabled cellular life to fuel ever more complex metabolic
reactions, such as photosynthesis.

Photosynthesis is a metabolic reaction that gathers energy from sunlight to
synthesize organic molecules, at the expense of carbon dioxide and water, and
generating free oxygen as a \mbox{by-product}.  The emergence of
\mbox{oxygen-forming} photosynthesis in cellular life, some 2.5~billion years
ago, had two major impacts in the course of evolution.  Firstly, it liberated
cells from the need to directly access preformed organic molecules to sustain
their metabolism; and, secondly, it biologically enriched the content of Earth's
atmosphere in oxygen,\footnote{This so called ``Great Oxygenation Event'', dated
  to 2.3~billion years ago, was arguably caused by photosynthesizing organisms
  whose presence was tracked long before it occurred
  (\citealp{flannery_archean_2012}). The \smallcaps{goe} likely had a catalytic
  effect in the evolution of life also through the oxidation of exposed rocks,
  liberating phosphorus and iron that flew into the oceans, there acting as
  fertilizer (\citealp{zimmer_mystery_2013}).} which is thought to have been a
\mbox{pre-condition} for the thriving of eukariotyc life.

% http://nitro.biosci.arizona.edu/courses/EEB105/lectures/Origins_of_Life/origins.html

% The timetable
% 3.6-3.7 billion years ago: appearance of life
% 2.5 billion years ago oxygen-forming photosynthesis
% ~2.2 billion years ago: aerobic respiration
% ~1.5 billion years ago: first evidence of fossil eukaryotes

Up to that stage, all forms of life on Earth could still be modeled from the
blueprint of modern Prokaria: a \mbox{single-celled} organism devoid of any
\mbox{membrane-bound} organelles, capable of lateral \smallcaps{dna} transfer and
that mainly reproduces itself through binary fission.  The Prokaria taxon
comprises both Archae (e.g., \mbox{\emph{Thermoplasma}-like} organisms) and
Eubacteria (e.g., \mbox{\emph{Spirochaeta}-like} organisms), which fundamentally
differ at the level of the chemical composition of their cell walls; the lipidic
composition of their plasma membranes; and the number of subunits in their
\smallcaps{rna} polymerases.

An hypothetical permanent \mbox{whole-cell} fusion between members of Archaea
and Eubacteria has been proposed to be at the origin of the earliest anaerobic
Eukarya,\cite{margulis_archaeal-eubacterial_1996} of which the first fossil
record evidence could date from a 2.1~billion year black shale formation found
in Gabon.\cite{albani_large_2010} Eukarya\footnote{From the Greek,
  ``\emph{eu}''---true---and ``\emph{karyon}''---kernel.} are thus defined by
the presence of specialized organelles enclosed within membranes, namely the
presence of a discrete nucleus wherein all genetic material is confined.

The evolution of aerobic respiration, now a possibility in an oxygen rich
environment, is thought to have occurred about 2.2~billion years ago in
prokaryotic cells.  The endosymbiotic theory proposes that eukaryotic
organelles, such like chloroplasts and mitochondria, evolved from certain types
of bacteria that early eukaryotic cells engulfed through endophagocytosis and
retained in a mutualistically beneficial relationship.

The appearance of the integrated eukaryotic cell set the stage for a vast
collection of evolutionary experiments that propelled a remarkable
diversification of life forms.  Two critical features were behind this
radiation: the advent of sexual reproduction and that of multicellular life.

Sexual reproduction first evolved in a \mbox{single-celled} eukaryotic entity
some 1.2~billion years ago.\cite{bernstein_dna_2012} The exchange of genetic
information through recombination during meiosis provided cells with a new
powerful source of genetic variation, both capable to accelerate adaptation
rates to new environmental challenges as well as to supply evolution with novel
genetic backgrounds to operate on.\cite{burt_perspective:_2000}

% resilience of genetic systems

On the other hand, multicellularity is known to have recurred multiple times as
an evolutionary experiment throughout different prokaryotic and eukaryotic taxa,
including animals, fungi, plants and slime moulds.\cite{kaiser_building_2001}
The first evidence of transition between unicellular to multicellular
organization is epitomized by fossils of prokariotic filamentous and
\mbox{mat-forming} \mbox{cyanobacteria-like}, dating back to 3 to 3.5 billion
years.\cite{knoll_life_2003,schopf_microfossils_1993} However, cell
differentiation within these colonies of aggregated cyanobacteria only appears
more than 2 billion years ago.\cite{tomitani_evolutionary_2006} The first
multicellular eukaryotes might have emerged some 1 billion years
ago,\cite{knoll_eukaryotic_2006} while the most significant burst of metazoan
diversification happened some \mbox{600--700~million} years ago, at a time when
levels of oxygen in the oceans and in the atmosphere were already rising
sharply.\cite{carroll_chance_2001,king_unicellular_2004}

The main implication of multicelullarity is the integration of the cell
physiology, first at the tissue, then at the organism level---with the
concomitant progressive loss of individualization and, ultimately, independence
of the cellular unit.  This integrative process required the emergence of a
number of essential features, such as cell adhesion, \mbox{cell-cell}
communication and coordination, and programmed cell death, which are likely to
have evolved in an unicellular
context.\cite{bonner_development:_1974,bonner_first_2009,kaiser_building_2001}

From the evolutionary point of view, the appearance of the molecular apparatuses
underpinning these innovations is thus relatively recent.  Consequently, one
could expect the genetic networks supporting the essential functionalities for
multicellular life to lack the degrees of redundancy and resiliency of those
representing older, more fundamental metabolic pathways for the bearing of
unicellular life.

Furthermore, the degrees of cooperation between distinct genetic lineages during
the transition to multicellularity must also be taken into consideration.
Conflicts may arise between lineages devoted to promote the cooperative
establishment of an efficient multicellular organism and lineages focused on
selfish proliferation regardless of the integrity and performance of the
organism.\cite{buss_evolution_1987,hammerstein_genetic_2003}

The transition to multicellularity is likely to have been driven by the
advantages associated to increase in size (evasion from predation) and perhaps
more importantly to the functional specialization and division of labour
(metabolic cooperation).

A vivid example of the latter is given by the \emph{volvox}, a genus of
clorophytes.  These green algae can form colonies of up to \SI{50000} cells and
can display a degree of differentiation between internal, unflagellated germ
cells that can divide and give rise to new colonies; and flagellated, external
somatic cells that keep the colony suspended\cite{kirk_volvox:_2005} and promote
nutrient exchange.\cite{solari_multicellularity_2006}

One important distinction between the germinal and somatic lineage in the
\emph{volvox} is that the somatic cells appear to be terminally differentiated
and have a limited capacity to proliferate once they form.  This attribute
illustrates the concept of cell potency, or the cell's ability to differentiate
into distinct cell types.

Once the specification between germinal and somatic lineages was achieved, the
evolution of complexity in multicellular entities then became a function of the
orderly incorporation on genomes of the rules that guide the precise
orchestration of cell growth, differentiation and morphogenesis from a
totipotent spore or zygote into a fully functional and mature organism---a
process recapitulated in spectacular fashion during the
embriogenesis\footnote{\smallcaps{todo:} Reference to the biogenic law and
  Gould's ``\emph{Ontogeny and Phylogeny}''.}  of each \mbox{present-day}
multicellular life form.

% Grosberg & Strathmann also discuss in their review: self/nonself-recognition and
% policing; allorecognition and multicellularity; programmed cell death and
% apoptosis; maternal control in early development; germ-line sequestration

% internalization of cellular specialization into the genetic code leading to:
% transitions to ever more inclusive, hierarchically nested levels of biological
% organization

% embriogenesis

% Concomitantly, the loss of individualization and increasing social integration
% of the cellular tapestry into tissular differentiation lead to the evolution of
% specific cellular pathways

% Multicellular cyanobacteria arose early in the history of life on Earth.

% Multicellular organisms, especially long-living animals, also face the
% challenge of cancer, which occurs when cells fail to regulate their growth
% within the normal program of development. Changes in tissue morphology can be
% observed during this process. [from the Wikipedia entry on the evolution of
% multicellularity]

% \medskip

A synthesis of the major evolutionary transitions of life discussed so far has
been proposed by Maynard Smith and Szathmary in the following
terms:\cite{maynard_smith_major_1997} (\emph{a})~replicating molecules become
compartmentalized, yielding the first cells; (\emph{b})~replicating molecules
coalesce to form chromosomes; (\emph{c})~\smallcaps{dna} becomes the conveyor of
heredity and, together with \smallcaps{rna} and proteins, the agency of the
genetic code; (\emph{d})~symbiotic cells consolidate to generate the first
eukaryotic cells containing chloroplasts and mitochondria; (\emph{e})~sexual
reproduction emerges involving the production, through meiosis, of haploid
gametes and their posterior fusion; (\emph{f})~multicellular organisms evolve
from unicellular ancestors; and (\emph{g})~social groups composed of discrete
multicellular individuals become established.

All throughout theses transitions, major forces were quietly at play knitting
the tapestry of life---prime among which was natural selection.  Trait variation
in populations of conspecific individuals can be generated at the genetic level
by mutational events, and then reshuffled through recombination via sexual
reproduction.  Natural selection then shapes the evolutionary process through
the \mbox{context-dependent} differential heritability of traits across
generations.  This change of inherited characteristics of biological populations
over time can operate at every level of biological organization---from the
individual organism down to the molecular level.

\bigskip

Contemplating the diversity of living forms that surrounds us today, it is hard
not to be overwhelmed by how momentous this account of life is---a feeling best
captured by no other than Darwin himself:\cite{darwin_origin_1864}

% The forces driving this striking set of changes are noteworthy themselves.

% Notes

% A fossil record of highly organized and spatially discrete colonial living
% structures found in a formation of black shales in Gabon dating from 2.1~billion
% years ago could well be the oldest fossil evidence of such
% entities.

% The endosymbiotic theory proposes that organelles, such
% like chloroplasts and mitochondria, evolved from certain types of bacteria that
% early eukaryotic cells engulfed through endophagocytosis and retained in a
% mutualistically beneficial relationship.  The emergence eukaryotes marks compose
% a phylogenetic taxa characterized.  Origin.  Competitive advantage.  Evolution
% of sex.

% evolved from the
% first instances of unicellular organisms. These primeval \mbox{self-replicating}
% units able to regulate their own chemical processes within the confinement of a
% cellular membrane might well have of \mbox{modern-day} prokaryotes.

% Such primeval cells might well resemble modern day prokaryotes, defined by
% their lack of membrane-bound organelles, most particularly a nuclear envelop
% secluding their genetic material.  Whether representants of the Eubacteria
% phyla (defined by) or the Archea phyla (defined by), these first cells
% (metabolism).  Change of atmospheric conditions.  Evolution of Eukaryotic
% cells, definition, timing.  Specific acquired features, metabolism,
% endosymbiotic model.  Multicellular life, dramatic consequences for the
% establishment of genetic programs controlling differentiation and
% proliferation.  Concept of embryogenesis.  Cellular interdependence. Concept
% of totipotency.  Concept of natural evolution as an over-reching common thread
% to interpret and understand the evolution of life.  Darwin's quotation.
% Colonial organisms.  Modularity in biology.

% multicellularity

% http://en.wikipedia.org/wiki/Evolutionary_history_of_life

% Life can be considered to have emerged when RNA chains began to express the
% basic conditions necessary for natural selection to operate as conceived by
% Darwin: heritability, variation of type, and differential transmission of
% genetic traits constrained by competition for limited resources.

\begin{quotation}
  There is grandeur in this view of life, with its several powers, having been
  originally breathed by the Creator\footnote{This \mbox{oft-cited} final
    passage of \emph{On the Origin of Species} is frequently quoted stripped of
    this reference to a ``creator''.  Such a revisionist stance appears even
    more bizarre in the light of the fact that it was Darwin himself who added
    the reference in editions two through six---arguably to appease both the
    public and his wife (\citealp{thompson_origin_2003}).  A touch of candour
    from the man who ``politely changed the way we see the world forever''
    (\citealp{rutherford_there_2008}).} into a few forms or into one; and that,
  whilst this planet has gone cycling on according to the fixed law of gravity,
  from so simple a beginning endless forms most beautiful and most wonderful
  have been, and are being, evolved.
\end{quotation}

How ironic that one of the best models to understand this order of things is
actually one of its most harrowing violations: cancer.

\bigskip
