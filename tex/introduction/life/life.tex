\documentclass{tufte-book}

\usepackage{fontspec}
\usepackage{soul}
\usepackage{booktabs}

\defaultfontfeatures{Ligatures=TeX,Numbers=OldStyle}
\setmainfont{Minion Pro}
\setsansfont[Scale=MatchLowercase]{Gill Sans MT}
\setmonofont[Scale=MatchLowercase]{Consolas}

\usepackage{etoolbox}% provides some support for comma-separated lists

% code from:
% http://tex.stackexchange.com/questions/45868/can-i-automatically-generate-abbreviated-citations-in-tufte-documents-after-the
% Can I automatically generate abbreviated citations in Tufte documents after the first occurrence?

\makeatletter
% We'll keep track of the old/seen bibkeys here.
\def\@tufte@old@bibkeys{}

% This macro prints the full citation if it's the first time it's been used
% and a shorter citation if it's been used before.
\newcommand{\@tufte@print@margin@citation}[1]{%
  % print full citation if bibkey is not in the old bibkeys list
  \ifinlist{#1}{\@tufte@old@bibkeys}{%
    \citealp{#1}% print short entry
  }{%
    % !ALWAYS PRINT SHORT ENTRY
    \citealp{#1}% print short entry
    % \bibentry{#1}% print full entry
  }%
  % add bibkey to the old bibkeys list
  \listgadd{\@tufte@old@bibkeys}{#1}%
}

% We've modified this Tufte-LaTeX macro to call \@tufte@print@margin@citation
% instead of \bibentry.
\renewcommand{\@tufte@normal@cite}[2][0pt]{%
  % Snag the last bibentry in the list for later comparison
  \let\@temp@last@bibkey\@empty%
  \@for\@temp@bibkey:=#2\do{\let\@temp@last@bibkey\@temp@bibkey}%
  \sidenote[][#1]{%
    % Loop through all the bibentries, separating them with semicolons and spaces
    \normalsize\normalfont\@tufte@citation@font%
    \setcounter{@tufte@num@bibkeys}{0}%
    \@for\@temp@bibkeyx:=#2\do{%
      \ifthenelse{\equal{\@temp@last@bibkey}{\@temp@bibkeyx}}{%
        \ifthenelse{\equal{\value{@tufte@num@bibkeys}}{0}}{}{and\ }%
        \@tufte@trim@spaces\@temp@bibkeyx% trim spaces around bibkey
        \@tufte@print@margin@citation{\@temp@bibkeyx}%
      }{%
        \@tufte@trim@spaces\@temp@bibkeyx% trim spaces around bibkey
        \@tufte@print@margin@citation{\@temp@bibkeyx};\space
      }%
      \stepcounter{@tufte@num@bibkeys}%
    }%
  }%
}


% Calling this macro will reset the list of remembered citations. This is
% useful if you want to revert to full citations at the beginning of each
% chapter.
\newcommand{\resetcitations}{%
  \gdef\@tufte@old@bibkeys{}%
}
\makeatother

\newenvironment{docspec}{\begin{quotation}\ttfamily\parskip0pt\parindent0pt\ignorespaces}{\end{quotation}}
% command specification environment

\begin{document}

\section{Life}

\newthought{Life is a wonder} of its own.  It conceivably struck this planet at
least some 3.8~billion years ago,\cite{mojzsis_evidence_1996} arguably the
consequence of a phospholipidic \mbox{layer-bound} \emph{quantum leap} in a soup
of organic precursors.\cite{miller_organic_1959} From that singular moment on,
little has been spared in guise of amazement.\bigskip

The first factual evidence of life on Earth appears inscribed in the fossil
record some 3.5~billion years ago.  It consists mainly of microfossils and
ancient rock structures in Greenland and Australia called
stromatolites,\cite{ohtomo_evidence_2014,noffke_microbially_2013} the product of
the metabolism of photosynthesizing cyanobacteria.

While the fine details of the abiogenic model\footnote{From the Greek
  ``spontaneous generation'', this model posits that autocatalytic polymers able
  to function as simple molecular replicators are at the origin of life.
  Alternative models exist, most notably panspermia---the extraterrestrial
  origin of life.} for the origin of life\footnote{Much debate has been stirred
  around the definition of life itself, see for instance
  \citealp{benner_defining_2010}.} remain to be chalked, a wider consensus
exists to depict its primed instances as \mbox{self-replicating} units able to
regulate their own chemical processes within the confinement of a cellular
membrane.  Such primeval unicellular organisms were most likely composed of some
form of nucleic acid secluded from its external environment by a bilayer of
phospholipids.

Nucleic acids, \textsc{rna} and \textsc{dna}, are the sole organic
molecules fit to direct their own replication and to act as genetic
material.  \textsc{rna} was proposed as the original genetic system
since it was shown to be able to catalize a variety of chemical
reactions, notably the polymerization of
nucleotides.\cite{bass_specific_1984} However, as the genetic load
stored within cells increased, the lesser stability and reliability of
\textsc{rna} molecules rendered them unsuitable for the
furtherance of genetic systems, and \textsc{dna} eventually took over
as universal hereditary material.

These very simple and \mbox{self-organized} systems emerged in seas
densely enriched in organic molecules and therefore didn't require
sophisticated toolkits for intake of food or generation of energy.
Only when systematic interactions between \textsc{rna} and specific
amino~acids became enshrined in the genetic code\footnote{The central
  dogma of molecular biology, postulated by Francis Crick in 1958 and
  reasserted in 1970
  (\citealp{crick_protein_1958,crick_central_1970}), pertains to the
  rules that govern the sequential flow of genetic information between
  \textsc{dna}, \textsc{rna} and proteins. It can be summarized as
  ``\textsc{dna} makes \textsc{rna} makes protein'', which provides
  the template for the enactment of hereditary information for all
  living organisms and frames the scope of evolutionary forces on
  genetic systems.} did evolution start to shape concerted molecular
reactions between proteins that animated cells with metabolic
activity.  Among the first tinkered metabolic pathways are likely to
have been some simpler form of \mbox{modern-day} glycolysis, the
anaerobic breakdown of glucose to lactic acid, which provided cells
with their universal source of energy, \textsc{atp}.

and later when
\textsc{dna} took over the role of hereditary material

later replaced by \textsc{dna} as the holder of
hereditary information.  These very simple systems emerged in a rich.  Ordered
interactions between RNA and amino acids then evolved into the present-day
genetic code, and DNA eventually replaced RNA as the genetic material. The
central dogma of biology.

\footnote{RNA, later
  eventually replaced by DNA.}

evolved from the
first instances of unicellular organisms. These primeval \mbox{self-replicating}
units able to regulate their own chemical processes within the confinement of a
cellular membrane might well have of \mbox{modern-day} prokaryotes.

Such primeval cells might well resemble modern day
prokaryotes, defined by their lack of membrane-bound organelles, most
particularly a nuclear envelop secluding their genetic material.  Whether
representants of the Eubacteria phyla (defined by) or the Archea phyla (defined
by), these first cells (metabolism).  Change of atmospheric conditions.
Evolution of Eukaryotic cells, definition, timing.  Specific acquired features,
metabolism, endosymbiotic model.  Multicellular life, dramatic consequences for
the establishment of genetic programs controlling differentiation and
proliferation.  Concept of embryogenesis.  Cellular interdependence. Concept of
totipotency.  Concept of natural evolution as an over-reching common thread to
interpret and understand the evolution of life.  Darwin's quotation.  Colonial
organisms.  Modularity in biology.



endosymbiosis

multicellularity

Research on how life might have emerged from non-living chemicals focuses on
three possible starting points: self-replication, an organism's ability to
produce offspring that are very similar to itself; metabolism, its ability to
feed and repair itself; and external cell membranes, which allow food to enter
and waste products to leave, but exclude unwanted substances.

% http://en.wikipedia.org/wiki/Evolutionary_history_of_life

Life can be considered to have emerged when RNA chains began to express the
basic conditions necessary for natural selection to operate as conceived by
Darwin: heritability, variation of type, and differential transmission of
genetic traits constrained by competition for limited resources.

\begin{quotation}
  There is grandeur in this view of life, with its several powers, having been
  originally breathed into a few forms or into one; and that, whilst this planet
  has gone cycling on according to the fixed law of gravity, from so simple a
  beginning endless forms most beautiful and most wonderful have been, and are
  being, evolved.
\end{quotation}

How ironic that one of the best models to understand this order of things is
actually one of its most harrowing violations: cancer.\bigskip

\section{Cancer}

\newthought{Cancer is a dysfunction of}\bigskip

\section{Microarrays}

\backmatter

\bibliography{life}
\bibliographystyle{plainnat}
% \bibliographystyle{abbrvnat}
% \bibliographystyle{unsrtnat}

\end{document}