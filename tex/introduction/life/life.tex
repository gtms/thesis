%%% Local Variables:
%%% coding: utf-8
%%% mode: latex
%%% TeX-engine: xetex
%%% End:

\documentclass{tufte-book}

\usepackage{fontspec}
\usepackage{soul}
\usepackage{booktabs}
\usepackage{morefloats}

\defaultfontfeatures{Ligatures=TeX,Numbers=OldStyle}
\setmainfont{Minion Pro}
\setsansfont[Scale=MatchLowercase]{Gill Sans MT}
\setmonofont[Scale=MatchLowercase]{Consolas}

\usepackage{etoolbox}% provides some support for comma-separated lists

% code from:
% http://www.techques.com/question/31-18947/Letterspacing,-Minion-Pro,-and-Tufte-LaTeX-(problem-with-running-header)

% Set up the spacing using fontspec features
\renewcommand\allcapsspacing[1]{{\addfontfeature{LetterSpace=15}#1}}
\renewcommand\smallcapsspacing[1]{{\addfontfeature{LetterSpace=10}#1}}

% code from:
% http://tex.stackexchange.com/questions/45868/can-i-automatically-generate-abbreviated-citations-in-tufte-documents-after-the
% Can I automatically generate abbreviated citations in Tufte documents after the first occurrence?

\makeatletter
% We'll keep track of the old/seen bibkeys here.
\def\@tufte@old@bibkeys{}

% This macro prints the full citation if it's the first time it's been used
% and a shorter citation if it's been used before.
\newcommand{\@tufte@print@margin@citation}[1]{%
  % print full citation if bibkey is not in the old bibkeys list
  \ifinlist{#1}{\@tufte@old@bibkeys}{%
    \citealp{#1}% print short entry
  }{%
    % !ALWAYS PRINT SHORT ENTRY
    \citealp{#1}% print short entry
    % \bibentry{#1}% print full entry
  }%
  % add bibkey to the old bibkeys list
  \listgadd{\@tufte@old@bibkeys}{#1}%
}

% We've modified this Tufte-LaTeX macro to call \@tufte@print@margin@citation
% instead of \bibentry.
\renewcommand{\@tufte@normal@cite}[2][0pt]{%
  % Snag the last bibentry in the list for later comparison
  \let\@temp@last@bibkey\@empty%
  \@for\@temp@bibkey:=#2\do{\let\@temp@last@bibkey\@temp@bibkey}%
  \sidenote[][#1]{%
    % Loop through all the bibentries, separating them with semicolons and spaces
    \normalsize\normalfont\@tufte@citation@font%
    \setcounter{@tufte@num@bibkeys}{0}%
    \@for\@temp@bibkeyx:=#2\do{%
      \ifthenelse{\equal{\@temp@last@bibkey}{\@temp@bibkeyx}}{%
        \ifthenelse{\equal{\value{@tufte@num@bibkeys}}{0}}{}{and\ }%
        \@tufte@trim@spaces\@temp@bibkeyx% trim spaces around bibkey
        \@tufte@print@margin@citation{\@temp@bibkeyx}%
      }{%
        \@tufte@trim@spaces\@temp@bibkeyx% trim spaces around bibkey
        \@tufte@print@margin@citation{\@temp@bibkeyx};\space
      }%
      \stepcounter{@tufte@num@bibkeys}%
    }%
  }%
}


% Calling this macro will reset the list of remembered citations. This is
% useful if you want to revert to full citations at the beginning of each
% chapter.
\newcommand{\resetcitations}{%
  \gdef\@tufte@old@bibkeys{}%
}
\makeatother

\newenvironment{docspec}{\begin{quotation}\ttfamily\parskip0pt\parindent0pt\ignorespaces}{\end{quotation}}
% command specification environment

\begin{document}

\section{Life}

% \newthought{Life is a wonder} of its own.  It conceivably struck this planet at
\textsc{Life is a wonder} of its own.  It conceivably struck this planet at
least some 3.8~billion years ago,\cite{mojzsis_evidence_1996} arguably the
consequence of a phospholipidic \mbox{layer-bound} \emph{quantum leap} in a soup
of organic precursors.\cite{miller_organic_1959} From that singular moment on,
little has been spared in guise of amazement.\bigskip

The first factual evidence of life on Earth appears inscribed in the fossil
record some 3.5~billion years ago.  It consists mainly of microfossils and
ancient rock structures in Greenland and Australia called
stromatolites,\cite{ohtomo_evidence_2014,noffke_microbially_2013} the product of
the metabolism of photosynthesizing cyanobacteria.

While the fine details of the abiogenic\footnote{From the Greek ``spontaneous
  generation'', this model posits that autocatalytic polymers able to function
  as simple molecular replicators are at the origin of life.  Alternative models
  exist, most notably panspermia---the extraterrestrial origin of life.} model
for the origin of life\footnote{Much debate has been stirred around the
  definition of life itself, see for instance \citealp{benner_defining_2010}.}
remain to be chalked, its primed instances are often depicted as
\mbox{self-replicating} units able to regulate their own chemical processes
within the confinement of a cellular membrane.  Such primeval unicellular
organisms were most likely composed of some form of nucleic acid secluded from
its external environment by a bilayer of phospholipids.

Nucleic acids, \smallcaps{rna} and \smallcaps{dna}, are the sole organic molecules fit
to direct their own replication and to act as genetic material.  \smallcaps{rna}
was proposed as the original genetic system since it was shown to be able to
catalize a variety of chemical reactions, notably the polymerization of
nucleotides.\cite{bass_specific_1984} However, as the genetic load stored within
cells increased, the lesser stability and reliability of \smallcaps{rna} molecules
rendered them unsuitable for the furtherance of genetic systems, a stage at
which \smallcaps{dna} eventually took over as universal hereditary material.

These very simple and \mbox{self-organized} systems emerged in seas densely
enriched in organic molecules and therefore didn't require sophisticated
toolkits for intake of food or generation of energy.  Only when systematic
interactions between \smallcaps{rna} and specific amino~acids became enshrined in
the genetic code\footnote{The central dogma of molecular biology, postulated by
  Francis Crick in 1958 and reasserted in 1970
  (\citealp{crick_protein_1958,crick_central_1970}), pertains to the rules that
  govern the sequential flow of genetic information between \smallcaps{dna},
  \smallcaps{rna} and proteins. It can be summarized as ``\smallcaps{dna} makes
  \smallcaps{rna} makes protein'', which provides the template for the enactment of
  hereditary information for all living organisms and frames the scope of
  evolutionary forces on genetic systems.} did evolution start to shape
concerted molecular reactions between proteins that animated cells with
metabolic activity.

Among the first elaborations of metabolic pathways are likely to have been some
simpler form of \mbox{modern-day} glycolysis, the anaerobic breakdown of glucose
to lactic acid.  Glycolysis is a fixture of all \mbox{present-day} cells and
drives the generation of \smallcaps{atp} molecules, their universal source of
energy.  This in turn enabled cellular life to fuel ever more complex metabolic
reactions, prime among which was photosynthesis.

Photosynthesis is a metabolic reaction that gathers energy from sunlight to
synthesize organic molecules, at the expense of carbon dioxide and water, and
generating free oxygen as a \mbox{by-product}.  The emergence of
\mbox{oxygen-forming} photosynthesis in cellular life, some 2.5~billion years
ago, had two major impacts in the course of evolution.  Firstly, it liberated
cells from the need to directly access preformed organic molecules to sustain
their metabolism; and, secondly, it biologically enriched the content of Earth's
atmosphere in oxygen,\footnote{This so called ``Great Oxygenation Event'', dated
  to 2.3~billion years ago, was arguably caused by photosynthesizing organisms
  whose presence was tracked long before it occurred
  (\citealp{flannery_archean_2012}). The \smallcaps{goe} likely had a catalytic
  effect in the evolution of life also through the oxidation of exposed rocks,
  liberating phosphorus and iron that flew into the oceans, there acting as
  fertilizer (\citealp{zimmer_mystery_2013}).} which is thought to have been a
\mbox{pre-condition} for the thriving of eukariotyc life.

% http://nitro.biosci.arizona.edu/courses/EEB105/lectures/Origins_of_Life/origins.html

% The timetable
% 3.6-3.7 billion years ago: appearance of life
% 2.5 billion years ago oxygen-forming photosynthesis
% ~2.2 billion years ago: aerobic respiration
% ~1.5 billion years ago: first evidence of fossil eukaryotes

Up to that stage, all forms of life on Earth could still be modeled from the
blueprint of modern Prokaria: a \mbox{single-celled} organism devoid of any
\mbox{membrane-bound} organelles, capable of lateral \smallcaps{dna} transfer and
that mainly reproduces itself through binary fission.  The Prokaria taxon
comprises both Archae (e.g., \mbox{\emph{Thermoplasma}-like} organisms) and
Eubacteria (e.g., \mbox{\emph{Spirochaeta}-like} organisms), which fundamentally
differ at the level of the chemical composition of their cell walls; the lipidic
composition of their plasma membranes; and the number of subunits in their
\smallcaps{rna} polymerases.

An hypothetical permanent \mbox{whole-cell} fusion between members of Archaea
and Eubacteria has been proposed to be at the origin of the earliest anaerobic
Eukarya,\cite{margulis_archaeal-eubacterial_1996} of which the first fossil
record evidence could date from a 2.1~billion year black shale formation found
in Gabon.\cite{albani_large_2010}  Eukarya\footnote{From the Greek,
  ``\emph{eu}''---true---and ``\emph{karyon}''---kernel.} are thus defined by
the presence of specialized organelles enclosed within membranes, namely the
presence of a discrete nucleus wherein all genetic material is confined.

The evolution of aerobic respiration, now a possibility in an oxygen rich
environment, is thought to have occurred about 2.2~billion years ago in
prokaryotic cells.  The endosymbiotic theory proposes that eukaryotic
organelles, such like chloroplasts and mitochondria, evolved from certain types
of bacteria that early eukaryotic cells engulfed through endophagocytosis and
retained in a mutualistically beneficial relationship.

The appearance of the integrated eukaryotic cell set the stage for a vast
collection of evolutionary experiments that propelled a remarkable
diversification of life forms.  Two critical features were behind this
radiation: the advent of sexual reproduction and that of multicellular life.

Sexual reproduction first evolved in a \mbox{single-celled} eukaryotic entity
some 1.2~billion years ago.\cite{bernstein_dna_2012} The exchange of genetic
information through recombination during meiosis provided cells with a new
powerful source of genetic variation, both capable to accelerate adaptation
rates to new environmental challenges as well as to supply evolution with novel
genotypes to operate on.\cite{burt_perspective:_2000}

% resilience of genetic systems

On the other hand, multicellularity is known to have recurred multiple times as
an evolutionary experiment throughout different eukaryotic taxa, including
animals, fungi, plants and slime moulds.\cite{kaiser_building_2001} The first
evidence of transition between unicellular to multicellular organization is
epitomized by fossils of prokariotic filamentous and \mbox{mat-forming}
\mbox{cyanobacteria-like}, dating back to 3 to 3.5 billion
years.\cite{knoll_life_2003,schopf_microfossils_1993} However, the first
evidence of cell differentiation within these colonies of aggregated
cyanobacteria only appears more than 2 billion years
ago.\cite{tomitani_evolutionary_2006} The first multicellular eukaryotes might
have appeared some 1 billion years ago\cite{knoll_eukaryotic_2006}, while the
most significant burst of metazoan diversification happened some
\mbox{600--700~million} years ago, at a time when levels of oxygen in the oceans
and in the atmosphere were already rising
sharply.\cite{knoll_life_2003,schopf_microfossils_1993}


% Multicellular cyanobacteria arose early in the history of life on Earth.

multicellularity
volvox
colonies
specialization

internalization of cellular specialization into the genetic code leading to:
transitions to ever more inclusive, hierarchically nested levels of biological
organization

embriogenesis

requirements of multicellular evolution: cell adhesion, cell-cell communication
and coordination, programmed cell death, which likely existed in ancestral
unicellular organisms\cite{grosberg_evolution_2007}.

Concomitantly, the loss of individualization and increasing social integration
of the cellular tapestry into tissular differentiation lead to the evolution of
specific cellular pathways

% Multicellular cyanobacteria arose early in the history of life on Earth.

multicellularity
volvox
colonies
specialization

internalization of cellular specialization into the genetic code leading to:
transitions to ever more inclusive, hierarchically nested levels of biological
organization

embriogenesis

requirements of multicellular evolution: cell adhesion, cell-cell communication
and coordination, programmed cell death, which likely existed in ancestral
unicellular organisms\cite{grosberg_evolution_2007}.

Concomitantly, the loss of individualization and increasing social integration
of the cellular tapestry into tissular differentiation lead to the evolution of
specific cellular pathways

This account of the major evolutionary transitions of life was thusly
synthesized by Grosberg and Strathmann:\cite{grosberg_evolution_2007} (\emph{a})
the compartmentalization of replicating molecules, yielding the first cells;
(\emph{b}) the coalescence of replicating molecules to form chromosomes;
(\emph{c}) the use of \smallcaps{dna} and proteins as the fundamental elements
of the genetic code and replication; (\emph{d}) the consolidation of symbiotic
cells to generate the first eukaryotic cells containing chloroplasts and
mitochondria; (\emph{e}) sexual reproduction involving the production (by
meiosis) and fusion of haploid gametes; (\emph{f}) the evolution of
multicellular organisms from unicellular ancestors; and (\emph{g}) the
establishment of social groups composed of discrete multicellular individuals.

The forces driving this striking set of changes are noteworthy themselves.
Notes on natural selection, best put by Darwin himself:

% The simplest definitions of "multicellular", for example "having multiple
% cells", could include colonial cyanobacteria like Nostoc. Even a professional
% biologist's definition such as "having the same genome but different types of
% cell" would still include some genera of the green alga Volvox, which have
% cells that specialize in reproduction.  Multicellularity evolved independently
% in organisms as diverse as sponges and other animals, fungi, plants, brown
% algae, cyanobacteria, slime moulds and myxobacteria.  For the sake of brevity
% this article focuses on the organisms that show the greatest specialization of
% cells and variety of cell types, although this approach to the evolution of
% complexity could be regarded as "rather anthropocentric".

% This account of the major evolutionary transitions of life was thusly
synthesized by Grosberg and Strathmann:\cite{grosberg_evolution_2007} (\emph{a})
the compartmentalization of replicating molecules, yielding the first cells;
(\emph{b}) the coalescence of replicating molecules to form chromosomes;
(\emph{c}) the use of \smallcaps{dna} and proteins as the fundamental elements
of the genetic code and replication; (\emph{d}) the consolidation of symbiotic
cells to generate the first eukaryotic cells containing chloroplasts and
mitochondria; (\emph{e}) sexual reproduction involving the production (by
meiosis) and fusion of haploid gametes; (\emph{f}) the evolution of
multicellular organisms from unicellular ancestors; and (\emph{g}) the
establishment of social groups composed of discrete multicellular individuals.

The forces driving this striking set of changes are noteworthy themselves.
Notes on natural selection, best put by Darwin himselA slime mold solves a maze.  The mold (yellow) explored and filled the maze
% (left).  When the researchers placed sugar (red) at two separate points, the
% mold concentrated most of its mass there and left only the most efficient
% connection between the two points (right).  The initial advantages of
% multicellularity may have included: more efficient sharing of nutrients that
% are digested outside the cell, increased resistance to predators, many of
% which attacked by engulfing; the ability to resist currents by attaching to a
% firm surface; the ability to reach upwards to filter-feed or to obtain
% sunlight for photosynthesis; the ability to create an internal environment
% that gives protection against the external one; and even the opportunity for a
% group of cells to behave ``intelligently'' by sharing information.  These
% features would also have provided opportunities for other organisms to
% diversify, by creating more varied environments than flat microbial mats
% could.

% Multicellularity with differentiated cells is beneficial to the organism as a
% whole but disadvantageous from the point of view of individual cells, most of
% which lose the opportunity to reproduce themselves.  In an asexual
% multicellular organism, rogue cells which retain the ability to reproduce may
% take over and reduce the organism to a mass of undifferentiated cells. Sexual
% reproduction eliminates such rogue cells from the next generation and
% therefore appears to be a prerequisite for complex multicellularity.

% The available evidence indicates that eukaryotes evolved much earlier but
% remained inconspicuous until a rapid diversification around 1 Ga.  The only
% respect in which eukaryotes clearly surpass bacteria and archaea is their
% capacity for variety of forms, and sexual reproduction enabled eukaryotes to
% exploit that advantage by producing organisms with multiple cells that
% differed in form and function.

% Fossil evidence for multicellularity and sexual reproduction[edit] The
% Francevillian Group Fossil, dated to 2.1 Ga, is the earliest known fossil
% organism that is clearly multicellular.[23] This may have had differentiated
% cells.[122] Another early multicellular fossil, Qingshania,[note 1] dated to
% 1.7 Ga, appears to consist of virtually identical cells. The red alga called
% Bangiomorpha, dated at 1.2 Ga, is the earliest known organism that certainly
% has differentiated, specialized cells, and is also the oldest known sexually
% reproducing organism.[121] The 1.43 billion-year-old fossils interpreted as
% fungi appear to have been multicellular with differentiated cells.[100] The
% "string of beads" organism Horodyskia, found in rocks dated from 1.5 Ga to 900
% Ma, may have been an early metazoan;[8] however it has also been interpreted
% as a colonial foraminiferan.[113]

% Notes

% A fossil record of highly organized and spatially discrete colonial living
% structures found in a formation of black shales in Gabon dating from 2.1~billion
% years ago could well be the oldest fossil evidence of such
% entities.

% The endosymbiotic theory proposes that organelles, such
% like chloroplasts and mitochondria, evolved from certain types of bacteria that
% early eukaryotic cells engulfed through endophagocytosis and retained in a
% mutualistically beneficial relationship.  The emergence eukaryotes marks compose
% a phylogenetic taxa characterized.  Origin.  Competitive advantage.  Evolution
% of sex.

% evolved from the
% first instances of unicellular organisms. These primeval \mbox{self-replicating}
% units able to regulate their own chemical processes within the confinement of a
% cellular membrane might well have of \mbox{modern-day} prokaryotes.

% Such primeval cells might well resemble modern day prokaryotes, defined by
% their lack of membrane-bound organelles, most particularly a nuclear envelop
% secluding their genetic material.  Whether representants of the Eubacteria
% phyla (defined by) or the Archea phyla (defined by), these first cells
% (metabolism).  Change of atmospheric conditions.  Evolution of Eukaryotic
% cells, definition, timing.  Specific acquired features, metabolism,
% endosymbiotic model.  Multicellular life, dramatic consequences for the
% establishment of genetic programs controlling differentiation and
% proliferation.  Concept of embryogenesis.  Cellular interdependence. Concept
% of totipotency.  Concept of natural evolution as an over-reching common thread
% to interpret and understand the evolution of life.  Darwin's quotation.
% Colonial organisms.  Modularity in biology.

% multicellularity

% http://en.wikipedia.org/wiki/Evolutionary_history_of_life

% Life can be considered to have emerged when RNA chains began to express the
% basic conditions necessary for natural selection to operate as conceived by
% Darwin: heritability, variation of type, and differential transmission of
% genetic traits constrained by competition for limited resources.

\begin{quotation}
  There is grandeur in this view of life, with its several powers, having been
  originally breathed by the Creator\footnote{This \mbox{oft-cited} quote of
    \emph{On the Origin of Species by Means of Natural Selection} frequently
    appears stripped of this reference to a ``Creator''.  Such an odd
    revisionist stance appears even more bizarre in the light of the fact that
    it was Darwin himself who added the reference in editions two through
    six---arguably to appease both the public and his wife
    (\citealp{thompson_origin_2003}).  A touch of candour from the man who
    ``politely changed the way we see the world forever''
    (\citealp{rutherford_there_2008}).} into a few forms or into one; and that,
  whilst this planet has gone cycling on according to the fixed law of gravity,
  from so simple a beginning endless forms most beautiful and most wonderful
  have been, and are being, evolved.
\end{quotation}

How ironic that one of the best models to understand this order of things is
actually one of its most harrowing violations: cancer.\bigskip

\section{Cancer}

\newthought{Cancer is a dysfunction of}\bigskip

\section{Microarrays}

\backmatter

\bibliography{life}
\bibliographystyle{plainnat}
% \bibliographystyle{abbrvnat}
% \bibliographystyle{unsrtnat}

\end{document}