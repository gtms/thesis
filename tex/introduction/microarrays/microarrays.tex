\section{Microarrays}

\begin{marginfigure}%
  \includegraphics{schena-brown-fig1.jpg}
  \caption[Gene expression of \emph{Arabidopsis thaliana} monitored with
  c\smallcaps{dna} microarrays]{Gene expression of \emph{Arabidopsis thaliana}
    monitored with c\smallcaps{dna} microarrays. \textbf{A--F}: each panel shows
    the hybridization intensity of a mix of fluorescently labeled
    c\smallcaps{dna}s with a collection of \mbox{forty-five}
    \mbox{gene-specific} probes from arabidopsis, plus three controls, under
    each stated condition (see text).  Adjacent pairs of spots are experimental
    duplicates.  Negative controls were spotted on positions \emph{c}(11, 12)
    and \emph{h}(11, 12).  Positive controls were provided by adding a fixed
    diluted quantity of m\smallcaps{rna} of the human acethylcoline receptor
    gene to each sample before reverse transcription.  c\smallcaps{dna} probes
    of the positive control were printed on positions \emph{a}(1, 2) Probes for
    the \emph{\smallcaps{hat4}} gene were printed on positions \emph{e}(1, 2)
    (reproduced from \citealp{schena_quantitative_1995}).}
  \label{fig:arabidopsis-microarray}
\end{marginfigure}

\newthought{In October 1995,} a short report in \emph{Science} magazine caught
the eye with a figure showing six grids of colourful stains on a dark background
(Figure~\ref{fig:arabidopsis-microarray}).  The colour of each spot, ranging
through the visible spectrum, captured the fluorescence emitted when a 3.5 mm by
5.5 mm slide of glass was scanned with a laser.  Each slide had been previously
spotted with an array of microscopic droplets of \mbox{forty-five} clones of
c\smallcaps{dna} isolated from \emph{Arabidopsis thaliana}---a small flowering
plant with the smallest genome of any known higher eukaryote.  Before being
scanned, the slides, or microarrays, were hybridized with a solution of
fluorescently labeled, reverse transcribed c\smallcaps{dna}s, from
m\smallcaps{rna} templates extracted from the plant.

% , supplemented with fluorophores.

The vivid readouts from the microarrays were literally illuminating the
transcription patterns of the arabidopsis genome.

\bigskip

With this seminal report from Stanford, Mark Schena and Patrick Brown
demonstrated three things.  First, that microarray technology was
\emph{sensitive} and \emph{specific} enough to discriminate between distinct
m\smallcaps{rna} species in a solution (Figure~\ref{fig:arabidopsis-microarray},
\textbf{A--B}).  Second, that it could \emph{quantify} discrete levels of
m\smallcaps{rna} transcripts (Figure~\ref{fig:arabidopsis-microarray},
\textbf{C--D}).  Third, that the technology was suited to investigate gene
expression patterns in diverse tissue types
(Figure~\ref{fig:arabidopsis-microarray}, \textbf{E--F}).  As Pat Brown would
state later, microarrays were developed to ``enable a new method for relating
sequence differences in genes to complex traits in people.''  And what bigger
challenge of complexity could there be but cancer?

Four years on, Todd Golub and Eric Lander, at the Broad Institute in Boston,
reported the results of the first tackling on cancer using
microarrays.\cite{golub_molecular_1999} As a test case, they took to Farber's
acute leukemia, isolating m\smallcaps{rna} from 38 biological samples of
neoplastic bone marrow and peripheral blood. Their approach was framed as a
classification task.  Up to then, the problem of classifying cancer subtypes was
mostly left to the expertise of the pathologist,\footnote{Clinical practice for
  cancer classification would involve an experienced pathologist's
  interpretation of the tumour's morphology, histochemistry, immunophenotyping,
  and cytogenetic analysis.} and many cancers still lacked molecular markers for
their accurate definition.  Moreover, the correct distinction between acute
lymphoblastic leukemia (\smallcaps{all}, originating from lymphoid precursors)
and acute myeloid leukemia (\smallcaps{aml}, originating from myeloid
precursors) was critical for the determination of the chemotherapeutic regimen
to be used.

Using a microarray with probes reporting for 6817 human genes, they concluded
that the patterns of expression of a subset of these genes could be used to
accurately discriminate between \smallcaps{aml} and \smallcaps{all}.  They then
sought to use the expression data to blindly \emph{infer} eventual tumour
\mbox{sub-classes} among the samples.  Using a learning algorithm to build a
classifier from the data and then testing it with a cross-validation procedure,
they showed that the \mbox{\smallcaps{aml-all}} distinction could be
automatically discovered and confirmed without a biological \mbox{\emph{a
    priori}}.\footnote{These class prediction and class discovery tasks
  illustrate the distinction between \emph{supervised} and \emph{unsupervised}
  learning.  While the former derives a function from labeled training data
  (thus requiring an \mbox{\emph{a priori}} knowledge of the classes it tries to
  predict), the latter aims to produce a classification without any prior
  knowledge of the structure in the data.}  What's more, this class discovery
approach could be further refined to automatically detect the distinction
between \mbox{\smallcaps{B}-cell} and \mbox{\smallcaps{T}-cell} \smallcaps{all}.
The use of microarrays could thus enable a molecular classification of cancer,
even if this landmark experimental setup could not find a multigene expression
signature to predict response to chemotherapy.

Back in Stanford, the focus was being directed towards breast cancer.  Using a
microarray probing for \num{8102} human genes, Charles Perou, Pat Brown and
David Botstein characterized variation in gene expression patterns of breast
tumours from 42 patients.\cite{perou_molecular_2000}

\begin{marginfigure}%
  \includegraphics[width=\linewidth]{nature-the-human-genome.jpg}
  \caption{Cover of \emph{Nature} magazine of February 15, 2001.}
  \label{fig:human-genome-cover}
\end{marginfigure}

While these findings were being reported, Eric Lander was leading another major
collaborative effort that would redefine the breadth and scope of microarray
technology.  On February 2001, a \mbox{public-funded} consortium reported the
first draft of the human genome (Figure~\ref{fig:human-genome-cover}).  Prior to
this achievement, the estimated number of genes in our genome was around
\num{100000}.\cite{cox_assessing_1994} As the genome sequence quality and gene
finding methods improved, this figure was progressively revised down to an
estimated \num{20000}--\num{25000} human protein coding genes.  The prospect of
probing the expression of \emph{all} these genes in the neoplastic cell with
microarrays was now within reach---and would soon turn into a reality.  If
cancer was fundamentally a genetic disease,\footnote{``The revolution in cancer
  research can be summed up in a single sentence: cancer is, in essence, a
  genetic disease''---\emph{Bert Vogelstein}
  (\citealp{vogelstein_cancer_2004}).} then the study of the cancer genome with
microarrays would bring it into the genomic era.

% First classification results using microarray data.  Several cancers were used
% as models.  Concomitant evolution of the microarray analysis technology
% (Quackenbush and shit).  Selection bias & etc.  Public repositories for
% microarrays.  Miame consensus.

\medskip

In the early 1980's, the pathologists of the Nederlands Kanker Instituut
(\smallcaps{nki}) in Amsterdam began a frozen tissue bank of tumours from Dutch
women breast cancers.  Twenty years on, the m\smallcaps{rna} of these samples,
along with the patient's clinical histories, would be the subject of a page
turning gene expression profiling experiment.  A group including
\smallcaps{nki}'s head of molecular pathology, Laura van't Veer, head researcher
René Bernards, and Stephen Friend, a Rosetta Inpharmatics founder in Seattle,
analyzed a selection of primary tumours from 98 women younger than 55 who did
not develop lymph node metastasis.\cite{vant_veer_gene_2002}

When the tumours were originally collected, treatment standards did not require
adjuvant chemotherapy after surgery.  Thirty-four patients had since then
relapsed their cancer.  According to modern standards, approximately 95\% of the
original patients would have received chemotherapy in the \mbox{United States},
and 85\% would have been treated under European norms.  This entails that 55\%
to 65\% of the patients would have needlessly undergone a form of aggressive and
debilitating chemotherapy.

To assess whether gene expression profiles could predict metastatic relapse of
disease, the \smallcaps{nki} group profiled the tumours on a microarray
containing approximately \num{25000} genes.  Using an unsupervised, hierarchical
clustering algorithm on a subset of these features, they narrowed down a list of
70 genes whose expression levels correlated with the development of distant
metastasis.  Using the same microarray, this \mbox{70-gene} prognosis profile
was subsequently validated as a predictor on a \smallcaps{nki} cohort of 295
tumours of breast cancer.\cite{van_de_vijver_gene-expression_2002} Microarrays
were now being used to stratify cancer risk among patients based on the gene
expression profile of the tumour.\footnote{``Even though you could look under
  the microscope and they all look the same (\ldots{}) some have built into them
  programs to become aggressive''---\emph{Stephen Friend}} In breast cancer,
gene expression profiles proved better prognostic factors for the likelihood of
distant metastasis within five years than age, tumour size, status of axillary
lymph nodes, histological type of the tumour, pathological grade and
\mbox{hormone-receptor} status.

One of the originalities of the \smallcaps{nki} studies was the molecular
dissection of the list of predictor genes derived from their analysis.

\clearpage

\begin{marginfigure}%
  \includegraphics[width=\linewidth]{jci-cancer-microarrays.jpg}
  \caption{Cover of \emph{The Journal of Clinical Investigation} of June
1\textsuperscript{st}, 2005.}
  \label{fig:cancer-microarray}
\end{marginfigure}

By 2005, the partnership between microarray technology and cancer research was
in full swing (Figure~\ref{fig:cancer-microarray}).  The litany of papers
claiming (claims of progress here).

% Could it be that microarrays, that 21\textsuperscript{st} century divining
% rod,\cite{he_microarrays21st_2001} were misleading us in the war on cancer?

% 2007: Dupuy & Simon review validity and reproducibility of microarray-based
% research with a meta-analysis on 90 studies.  Quote: ``The use of microarray
% technology has generated great excitement for its potential to identify
% biomarkers for cancer outcomes, but the reproducibility and validity of
% findings based on microarray data have come under widespread challenge.''
% (nci-dupuy-2007.pdf)

% Steps to run a microarray: 1-sequence the entire genome of the organism
% you're studying.  2-use massive amount of computing power to determine where
% all the genes are in your sequence.  3-more computing power to design primer
% pairs to use PCR to make copies of every gene.  4-with those primers, copy
% every gene on the genome 5-run quality tests on all gene copies to verify
% which PCR reactions did not work 6-separate double stranded DNA into single
% stranded DNA.  7-use robots to place microscopic droplets of each
% single-stranded DNA into ordered rows and columns on a glass microscopic
% slide (microarray).

% each spot contains multiple copies of a unique DNA sequence that represents a
% single gene.

% list of terms to use:
% genetic catalog; gene expression profile.

% While microarrays quantify the m\smallcaps{rna} species extracted from a given
% tissue, they do not tell us if the subsequent protein is actually being
% properly synthesized.

% Limitations of microarrays.  They cannot:
% 1-tell which genes went bad to cause a disease.
% 2-cure a disease
% 3-identify every genes that is behaving inappropriately

% This microarray technology had brought cancer research from the
% 20\textsuperscript{th} century genetics into 21\textsuperscript{th} century
% genomics.

% human acetylcholine receptor

\bigskip

% Cancer\footnote{``It's bad bile.  It's bad habits.  It's bad bosses.  It's bad
% genes.''---\emph{Mel Greaves}}
