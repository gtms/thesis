\section{Microarrays}

\newthought{In October 1995,} a short report in \emph{Science} magazine caught
the eye with a figure showing six grids of colourful stains on a dark background
(Figure~\ref{fig:arabidopsis-microarray}).  The colour of each spot %, ranging
% through the visible spectrum,
captured the fluorescence emitted when a 3.5 mm by
5.5 mm slide of glass was scanned with a laser.  Each slide had been previously
spotted with an array of microscopic droplets of \mbox{forty-five} clones of
c\smallcaps{dna} isolated from \emph{Arabidopsis thaliana}---a small flowering
plant with the smallest genome of any known higher eukaryote.  Before being
scanned, the slides, or microarrays, were hybridized with a solution of
fluorescently labeled, reverse transcribed c\smallcaps{dna}s, synthesized from
m\smallcaps{rna} templates extracted from the plant.

\begin{marginfigure}%
  \includegraphics{schena-brown-fig1.jpg}
  \caption[Gene expression of \emph{Arabidopsis thaliana} monitored with
  c\smallcaps{dna} microarrays]{Gene expression of \emph{Arabidopsis thaliana}
    monitored with c\smallcaps{dna} microarrays. \textbf{A--F}: each panel shows
    the hybridization intensity of a mix of fluorescently labeled
    c\smallcaps{dna}s with a collection of \mbox{forty-five}
    \mbox{gene-specific} probes from arabidopsis, plus three controls, under
    each stated condition (see text).  Adjacent pairs of spots are experimental
    duplicates.  Negative controls were spotted on positions \emph{c}(11, 12)
    and \emph{h}(11, 12).  Positive controls were provided by adding a fixed
    diluted quantity of m\smallcaps{rna} of the human acethylcoline receptor
    gene to each sample before reverse transcription.  c\smallcaps{dna} probes
    of the positive control were printed on positions \emph{a}(1, 2). Probes for
    the \emph{\smallcaps{hat4}} gene were printed on positions \emph{e}(1, 2)
    (reproduced from \citealp{schena_quantitative_1995}).}
  \label{fig:arabidopsis-microarray}
\end{marginfigure}

The vivid readouts from the microarrays were literally illuminating the
transcription patterns of the arabidopsis genome.

\bigskip

With this seminal report from Stanford, Mark Schena and Patrick Brown
demonstrated three things.  First, that microarray technology was
\emph{sensitive} and \emph{specific} enough to discriminate between
distinct m\smallcaps{rna} species in solution
(Figure~\ref{fig:arabidopsis-microarray}, \textbf{A--B}).  Second,
that it could \emph{quantify} levels of m\smallcaps{rna} transcripts
(Figure~\ref{fig:arabidopsis-microarray}, \textbf{C--D}).  Third, that
the technology was suited to investigate gene expression patterns in
diverse tissue types (Figure~\ref{fig:arabidopsis-microarray},
\textbf{E--F}).  As Patrick Brown would state later, microarrays were
developed to ``enable a new method for relating sequence differences
in genes to complex traits in people.''  And what bigger challenge of
complexity could there be but cancer?

\subsection{Cancer class discovery with microarrays}
\label{sec:class-discovery}

Four years on, Todd Golub and Eric Lander, at the Broad Institute in Boston,
reported the results of the first tackling on cancer using
microarrays.\cite{golub_molecular_1999} As a test case, they took to Farber's
acute leukemia, isolating m\smallcaps{rna} from 38 biological samples of
neoplastic bone marrow and peripheral blood. Their approach was framed as a
classification task.  Up to then, the problem of classifying cancer subtypes was
mostly left to the expertise of the pathologist,\footnote{Clinical practice for
  cancer classification would involve an experienced pathologist's
  interpretation of the tumour's morphology, histochemistry, immunophenotyping,
  and cytogenetic analysis.} and many cancers still lacked molecular markers for
their accurate definition.  Moreover, the correct distinction between acute
lymphoblastic leukemia (\smallcaps{all}, originating from lymphoid precursors)
and acute myeloid leukemia (\smallcaps{aml}, originating from myeloid
precursors) was critical for the determination of the chemotherapeutic regimen
to be used.

Using a microarray with probes reporting for 6817 human genes, they concluded
that the patterns of expression of a subset of these genes could be used to
accurately discriminate between \smallcaps{aml} and \smallcaps{all}.  They then
sought to use the expression data to blindly \emph{infer} eventual tumour
\mbox{sub-classes} among the samples.  Using a learning algorithm to build a
classifier from the data and then testing it with a cross-validation procedure,
they showed that the \mbox{\smallcaps{aml-all}} distinction could be
automatically discovered and confirmed without a biological \mbox{\emph{a
    priori}}.\footnote{These class prediction and class discovery tasks
  illustrate the distinction between \emph{supervised} and \emph{unsupervised}
  learning.  While the former derives a function from labeled training data
  (thus requiring an \mbox{\emph{a priori}} knowledge of the classes it tries to
  predict), the latter aims to produce a classification without any prior
  knowledge of the structure in the data.}  What's more, this class discovery
approach could be further refined to automatically detect the distinction
between \mbox{\smallcaps{B}-cell} and \mbox{\smallcaps{T}-cell} \smallcaps{all}.
The use of microarrays could thus enable a molecular classification of cancer,
even if this landmark experimental setup failed to find a multigene expression
signature to predict response to chemotherapy.

Back in Stanford, the focus was being directed towards breast cancer.
With a microarray probing for \num{8102} human genes, Charles Perou,
Patrick Brown and David Botstein studied the variation in gene
expression patterns of breast tumours from 42
patients.\cite{perou_molecular_2000} Using an unsupervised
hierarchical clustering algorithm, they were able to identify at least
five distinct molecular breast cancer classes.

\begin{marginfigure}%
  \includegraphics[width=\linewidth]{nature-the-human-genome.jpg}
  \caption[Cover of \emph{Nature} magazine of February 15, 2001]{Cover of
    \emph{Nature} magazine of February 15, 2001.}
  \label{fig:human-genome-cover}
\end{marginfigure}

Two types of epithelial cells are found in the human mammary gland: basal cells
(the outer layer of myoepithelial cells in the mammary duct) and luminal cells
(at the apical surface of the ducts, with secretory proprieties).  In Perou's
study, cancers of luminal origin were found to cluster in two previously
unrecognized groups, termed luminal \smallcaps{a} (of lower grade) and luminal
\smallcaps{b} (of higher grade).  Cancers with a \mbox{basal-like} phenotype,
\mbox{over-expressing} the \smallcaps{her2} receptor, or with a normal-like
phenotype were each found to cluster together in their respective group.
However, the most robust distinction was observed between the transcriptome of
breast cancers expressing the estrogen receptor (\smallcaps{ER+}) and those that
did not (\smallcaps{ER--}).  This pioneering study showed that a new taxonomy of
breast cancer could be based on its molecular features---a classification that
would be challenged, extended and refined throughout the ensuing
decade.\cite{sorlie_gene_2001,sorlie_repeated_2003,hu_molecular_2006,pusztai_molecular_2006,rakha_basal-like_2008,parker_supervised_2009,gusterson_basal-like_2009,weigelt_contribution_2010,prat_deconstructing_2011}

% Distinct breast cancer genotypes were first characterized.  Multiparameter
% gene expression assays for early-stage breast cancer.  A new taxonomy of
% breast cancer based on their molecular features.  The gene expression
% microarray-based class discovery studies pioneered by the Stanford group have
% led to the identification of at least five molecular breast cancer subtypes:
% luminal \smallcaps{A}, luminal \smallcaps{B}, normal breast-like, HER2, and
% basal-like.  Indeed, the most robust distinction observed by microarray
% analysis is between the transcriptome of breast cancers expressing the
% estrogen receptor (\smallcaps{ER+}) and estrogen receptor negative
% (\smallcaps{ER--}).

While these findings were being reported, Eric Lander was leading another
collaborative effort that would redefine the breadth of microarray technology.
On February 2001, a \mbox{public-funded} consortium reported the first draft of
the human genome (Figure~\ref{fig:human-genome-cover}).  Prior to this
achievement, the estimated number of genes in our genome was around
\num{100000}.\cite{cox_assessing_1994} As the genome sequence quality and gene
finding methods improved, this figure was progressively revised down to an
estimated \numrange{20000}{25000} human protein coding genes.  The prospect of
measuring the \emph{entire} transcriptome of the neoplastic cell with
microarrays was now within reach---and would soon turn into a reality.  If
cancer was fundamentally a genetic disease,\footnote{``The revolution in cancer
  research can be summed up in a single sentence: cancer is, in essence, a
  genetic disease''---\emph{Bert Vogelstein}
  (\citealp{vogelstein_cancer_2004}).} then the study of the cancer genome with
microarrays would bring it into the genomic era.

It was with microarrays probing for most of the then reported human genome that
the expression profiles of lung adenocarcinomas\cite{garber_diversity_2001},
hepatocellular carcinomas,\cite{chen_gene_2002} and gastric
cancers\cite{leung_phospholipase_2002} were interrogated.  Time and again,
unsupervised classification methods were highlighting clinical subtypes that
recapitulated morphological categorizations, underlined tumour differentiation
stages, or even uncovered tentative progression markers.  Each of these
portraits revealed a wide diversity in tumour profiles, both at the \mbox{intra-
  and} \mbox{inter-patient} sampling level, and a relatively minimal variation
in normal tissue profiles.  While nuanced, the transcriptomes of these different
cancers were still remarkably consistent within each disease and largely
reminiscent of the expression profiles of the normal tissues from which they
were derived.\cite{botstein_genomic_2003}

The use of class discovery algorithms, mostly as descriptive techniques,
dominated the early genomic approach to cancer
biology.\cite{matros_genomic_2004,eschrich_dna_2004} However, the problem of
predicting cancer progression, response to treatment or survival time remained
an elusive one.  In order to provide sound evidence for predictive genomic
markers, an experimental setting with a systematic, long term follow-up of
cancer patients was in demand.

\medskip

% Mass reporting of cancer transcription profiles.  Which cancers.  Which genes
% are deferentially expressed.  First public microarray databases.  Microarray
% meta-analysis.  Single cell gene expression profiling.  A compelling case was
% made.  Presently.

% Wide diversity in tumour profiles and relatively minimal variation in normal
% tissue profiles have been found not only for breast cancers, but also for
% lung, and gastric cancers.  While nuanced, the global patterns of gene
% expression of different cancers where still remarkably consistent within each
% malady and largely reminiscent of the transcription profiles of the normal
% tissues from which they are derived.  (Botstein, Fig.4).  Furthermore, tumour
% samples from the same breast cancer patient, either by repeated surgical
% sampling or from lymph node metastases, tend to have profiles very similar to
% each other; similar results were reported in lung and liver tumours.  Gene
% discovery.

% Concomitantly, a statistical and computational theory for the analysis of the
% millions of data points that result from these experiments was starting to
% take shape.\cite{quackenbush_computational_2001,irizarry_summaries_2003}

% First classification results using microarray data.  Several cancers were used
% as models.  Concomitant evolution of the microarray analysis technology
% (Quackenbush and shit).  Selection bias & etc.  Public repositories for
% microarrays.  Miame consensus.

\subsection{Cancer outcome prediction with microarrays}
\label{sec:outcome-prediction}

\newthought{In the early 1980's}, the pathologists of the Nederlands Kanker
Instituut (\smallcaps{nki}) in Amsterdam began a frozen tissue bank of tumours
from Dutch women breast cancers.  Twenty years on, the m\smallcaps{rna} of these
samples, along with the patient's clinical histories, would be the subject of a
page turning gene expression profiling experiment.  A group including
\smallcaps{nki}'s head of molecular pathology, Laura van't Veer, head researcher
René Bernards, and Stephen Friend, a Weinberg trainee turned Rosetta
Inpharmatics founder in Seattle, analyzed a selection of primary tumours from 98
women younger than 55 who did not develop lymph node
metastasis.\cite{vant_veer_gene_2002}

When the tumours were originally resected, treatment standards did not require
adjuvant chemotherapy after surgery.  Thirty-four patients, or roughly one third
of the women in the study, would later relapse of of their cancer.  According to
modern guidelines, approximately 95\% of the original patients would have
received chemotherapy in the \mbox{United States}, and 85\% would have been
treated under European norms.  This entails that 55\% to 65\% of the patients
would have needlessly undergone an aggressive and debilitating form of
chemotherapy.  To assess whether gene expression profiles could predict
metastatic relapse of disease, the \smallcaps{nki} team profiled the tumours on
a microarray containing approximately \num{25000} genes.  Using a supervised
iterative learning procedure on a subset of these features, they narrowed down a
list of 70 genes whose expression levels correlated with the development of
distant metastasis.  The robustness of this \mbox{70-gene} prognosis profile was
subsequently validated on a wider cohort of 295 breast cancer tumours with
either positive or negative nodal
status.\cite{van_de_vijver_gene-expression_2002} In this study, the classifier
accurately predicted overall survival and distant metastasis in stratified
univariate analyses, and was the strongest predictor of distant metastasis in a
multivariate model that included traditional breast cancer
predictors.\footnote{Classical prognosticators for breast cancer include age,
  tumour size, status of axillary lymph nodes, histological type of the tumour,
  pathological grade and \mbox{hormone-receptor} status.}  Microarrays were now
being used to stratify cancer risk among patients based on the gene expression
profile of the tumour.

The genes included in the predictor were scrutinized for potential ``insight
into the underlying biological mechanism leading to rapid metastasis.''  The
van't Veer article in \emph{Nature} reports that ``genes involved in cell cycle,
invasion and metastasis, angiogenesis, and signal transduction are significantly
upregulated in the poor prognosis signature.''  Not only did this work produce
the first genomic predictor to inform treatment decisions, it also paved the way
for an alternative to infer physiological mechanisms in human cancers.  If the
tumour transcriptome already contained information regarding disease
progression,\footnote{``Even though you could look under the microscope and they
  all look the same (\ldots{}) some have built into them programs to become
  aggressive''---\emph{Stephen Friend}} then querying for biologically motivated
collections of genes among the predictive features could make proof of the
implication of particular genetic programs in cancer biology.

% \FloatBarrier

It was with a similar reasoning in mind that the Stanford group presented the
argument for the link between a wound healing genetic program and cancer
progression.\cite{chang_gene_2004} The argument begins with the recognition of
the similarities between the tumour microenvironment and normal wound healing.
It then proceeds by characterizing a gene expression profile of fibroblast serum
response (which physiologically only occurs in the context of a local injury),
in a cell culture model profiled by microarray.  Finally, this signature profile
is used to test specific hypotheses using publicly available gene expression
data from human cancers.

Accordingly, they demonstrated that, in a cohort of 51 breast cancer patients
with equal treatment, those with a higher expression of the core serum response
signature were significantly more likely to develop metastasis and to die in a
\mbox{5-year} \mbox{follow-up} period.  Similar results were obtained by
segregating the \mbox{295-sample} \smallcaps{nki} cohort along an axis of
expression of the serum response signature.  The signature was also shown to be
predictive of outcome in a dataset of 62 patients with \mbox{stage
  \smallcaps{i}} and \mbox{stage \smallcaps{ii}} lung
adenocarcinomas\cite{garber_diversity_2001} and a dataset of 42 patients with
\mbox{stage \smallcaps{iii}} gastric carcinomas.\cite{leung_phospholipase_2002}
This formulation established a novel framework to infer biological determinants
of cancer progression based on gene expression profiles of clinical samples,
obviating the need for experimental setups on \emph{in vivo} models.

The rehashing of this strategy would prove exceptionally prolific.  In the wake
of the Chang et al. publication, links between cancer progression and various
biological signature markers were reported---including gene expression programs
of stem
\mbox{cell-ness};\cite{glinsky_microarray_2005,ben-porath_embryonic_2008}
p53-status;\cite{miller_expression_2005} stromal
component;\cite{west_determination_2005} response to
hypoxia;\cite{chi_gene_2006} chromosomal
instability;\cite{carter_signature_2006,buffa_large_2010} loss of
\emph{\smallcaps{PTEN}} expression;\cite{saal_poor_2007} \smallcaps{EMT}
transition;\cite{welm_macrophage-stimulating_2007,taube_core_2010}
\emph{\smallcaps{E2F1}} perturbation;\cite{hallstrom_e2f1-dependent_2008}
% mi\smallcaps{R}-31 targets;\cite{valastyan_tumor_2011}
among many more.
% bromodomain 4;\cite{crawford_bromodomain_2008}
% retinoic acid receptor;\cite{hua_genomic_2009} anchorage-independent
% growth;\cite{mori_anchorage-independent_2009} among many more.

\medskip

\begin{marginfigure}%
  \includegraphics[width=\linewidth]{jci-cancer-microarrays.jpg}
  \caption[Cover of \emph{The Journal of Clinical Investigation} of June
  1\textsuperscript{st}, 2005]{Cover of \emph{The Journal of Clinical
      Investigation} of June 1\textsuperscript{st}, 2005.}
  \label{fig:cancer-microarray}
\end{marginfigure}

\subsection{State of the art of microarray technology}
\label{sec:soa-microarray}

\newthought{By 2005,} ten years after the arabidopsis report, the partnership
between microarray technology and cancer research was in full swing
(Figure~\ref{fig:cancer-microarray}).  Reflecting the increasing appeal of the
technology, at least half a dozen vendors were then marketing whole genome
microarrays, each relying on their own specifics
(Table~\ref{tab:WholeGenomeArray}).

Microarrays can typically be designed in a \mbox{two-colour}
(\mbox{dual-channel}) or in a \mbox{single-colour} (\mbox{single-channel})
setup.  In \mbox{dual-channel} microarrays, c\smallcaps{dna}s prepared from two
samples (usually diseased \emph{versus} healthy tissue) are each labeled with
its own fluorophore, then mixed and hybridized on the same microarray.  The
ratios of the measurements of the fluorescence emission in the wavelength of
each fluorophore is then used to estimate the relative abundance of individual
transcripts in the two samples.  \mbox{Single-channel} microarrays measure the
hybridization intensities of a single population of c\smallcaps{dna}s labeled
with a unique fluorophore and, therefore, express the relative abundance of
transcript expression across biological samples processed in the same
experiment.  Oligonucleotide microarrays often carry control probes designed to
hybridize with \smallcaps{rna} \mbox{spike-ins}.  The degree of hybridization
between the \mbox{spike-ins} and the control probes is used to normalize the
hybridization measurements for the target probes.

Other platform specific attributes include the probe manufacturing process (made
\mbox{\emph{in situ}} by photolithographic or ink-jet methods, or by standard
oligonucleotide synthesis protocols followed by attachment to various
substrates); the probe substrates (activated glass slides, silicon chips, or
membranes); the probe design and location (most probes are derived from the 3'
end of the gene coding sequences to accommodate the fact that target labeling
usually begins at the 3' end of m\smallcaps{RNA}s); probe size and number per
array (Table~\ref{tab:WholeGenomeArray}); and the proper probe annotation (as
sequence databases were still in state of flux, probe annotations were
constantly being revised and did not necessarily target their designated
gene).\cite{kawasaki_end_2006}

% The great majority of these studies were
% done without internal controls or standards, which rendered comparison of
% results in independent experimental settings difficult or impossible.

\begin{table}[ht]
  \small
  \centering
  % \fontfamily{ppl}%\selectfont
  \begin{tabular}[c]{lcS[table-format=6.0]S[table-format=6.0]}
    \toprule
    \multicolumn{1}{c}{Vendor} & \multicolumn{1}{c}{Probe Size} &
    \multicolumn{1}{c}{\# Probesets} & \multicolumn{1}{c}{\# Probes per array}\\
    \midrule
    \smallcaps{ABI} & 60mer & 33000 & 33000 \\
    Affymetrix & 25mer & 54000 & 1000000 \\
    Agilent & 60mer & 44000 & 44000 \\
    \smallcaps{GE} Amersham & 30mer & 57000 & 57000 \\
    Illumina & 50mer & 46000 & 1500000 \\
    Microarrays, Inc. & 70mer & 49000 & 49000 \\
    NimbleGen & 60mer & 38000 & 380000 \\
    Phalanx Biotech & 60mer & 30000 & 30000 \\
    ``Home brew'' & \begin{tabular}[c]
      {@{}c@{}}
      \vspace{-.1cm}
      50mer--\emph{n}70mer\\or c\smallcaps{DNA}s
    \end{tabular}&
    40000 & 40000 \\
    \bottomrule
  \end{tabular}
  \caption[Technical attributes of principal commercial microarray platforms]{Technical attributes of the principal commercial microarray
    platforms by 2005.  A probeset constitutes a collection of probes
    targeting a specific gene
    (adapted from \citealp{kawasaki_end_2006}).}
  \label{tab:WholeGenomeArray}
  \vspace{0cm}
\end{table}

\medskip

In addition, the expression data resulting from a microarray experiment can be
influenced by a number of experimental factors, like target c\smallcaps{dna}
synthesis (linearly amplified \smallcaps{rna} may contain biases in the original
m\smallcaps{rna}s ratio);\cite{nygaard_options_2006} target labelling (different
fluorescent dyes present distinct stabilities, quantum efficiencies and
wavelengths for stimulation and emission); hybridization and washing protocols
(every commercial platform abides by its own methodology); and the imaging of
the arrays (usually done by confocal and non-confocal scanners---yet variables
like laser power, pixel sizes or scan time are not standardized).

These sources of technical variation were making comparison of results obtained
from different microarray platforms difficult or even impossible.  Some studies
were reporting poor correlations between expression levels measured with
different platforms.\cite{tan_evaluation_2003,shi_cross-platform_2005} In order
to improve the reliability and concordance of microarray data, international
consortia and technical study groups were assembled to determine a core set of
\emph{Minimum Information About a Microarray Experiment} (\smallcaps{miame})
standards\cite{brazma_minimum_2001} and, in 2006, the \emph{MicroArray Quality
  Control} (\smallcaps{maqc}) project was
launched.\cite{maqc_consortium_microarray_2006} As a result of these concerted
efforts, the National Center for Biotechnology Information in the United States
created, in 2002, the \emph{Gene Expression Omnibus}
(\smallcaps{geo}),\footnote{\url{http://www.ncbi.nlm.nih.gov/geo/}
  (\citealp{edgar_gene_2002})} an online repository for \mbox{high-throughput}
gene expression data.  In 2003, the European Bioinformatics Institute started
\emph{ArrayExpress},\footnote{\url{http://www.ebi.ac.uk/arrayexpress}
  (\citealp{brazma_arrayexpress--public_2003})} a public database of microarray
gene expression data.

Gene expression profiling studies are also challenged by biological
idiosyncrasies.  For one, global gene expression patterns are a function of
fluid states in coordinated cellular ecosystems in constant readjustment.
Microarray experiments consist of snapshots of such dynamic ranges, which may
account for some of the variation across experiments.  Distinct synthesis and
degradation rates of the probed m\smallcaps{rna} transcripts may further nuance
expression readings.  Even traditional housekeeping genes (fundamental to the
basic biology of the cell and thus considered gold standards) have been shown to
differ across cell types and experimental conditions.\cite{thorrez_using_2008}
What is more, biological heterogeneity in the biopsy sample can significantly
bias reports of gene expression, as distinct cellular types may be differently
represented in distinct samples.

But by and large, the most contentious aspect of the application of microarrays
in cancer research concerns the methodological analysis of the experimental
data.  In 2007, a critical detailed review of \mbox{forty-two} studies for
cancer outcome appeared in the \emph{Journal of the National Cancer Institute}
by Alain Dupuy and Richard Simon, from the Hôpital Saint-Louis in Paris and the
\smallcaps{nci} in Maryland.\cite{dupuy_critical_2007} They
identified three endemic analytic flaws permeating the reviewed studies.

Microarray experiments typically aim for one or more of the following
objectives: (\emph{a}) to identify individual genes (transcripts) whose
expression is correlated with a phenotypic trait; (\emph{b}) to identify
multiple genes interactively involved in regulatory networks and in mediating
biological phenomena or disease pathogenesis; (\emph{c}) to discover potential
targets for drug development; and (\emph{d}) to identify molecular markers that
can be used as tools for disease diagnosis and prognosis or as predictors of
clinical outcome.\cite{kim_expectations_2010} In \mbox{outcome-related}
microarray experiments, these aims can be approached with statistical tools
addressing three kinds of tasks: finding genes correlated with outcome, class
discovery, and supervised prediction.

For the outcome-related gene finding task, Dupuy and Simon identified a trend
for an inadequate, unclear, or unstated method for controlling the number of
\mbox{false-positive} differentially expressed genes.  Because microarray
analysis involves making inferences about each gene whose expression is measured
on the array, the large number of hypotheses being tested can yield a higher
than desirable proportion of false positives.  Statistical procedures to correct
for excess of false positives in multiple testing, like the false discovery
rate, are thus necessary.\cite{benjamini_controlling_1995,noble_how_2009}

Concerning the class discovery task, they recognized a tendency to credit
expression clusters with biological meaning when the clustering procedure was
itself based on genes selected for their correlation with outcome.  This causes
non-independent evidence that outcome can be predicted based on expression
levels---a statistical misconception known as feature selection
bias.\cite{ambroise_selection_2002}

For the supervised prediction task, they flagged analytic lapses causing a
biased estimation of the prediction accuracy through incorrect cross-validation
procedures.  The more common of these experimental design errors involved the
violation of the principle of separation of the training and testing sets during
the validation of the classifier.  The models derived thusly were likely prone
to data overfitting.\footnote{Overfitting occurs when a classifier describes
  random error or noise pertaining to the training set instead of the underlying
  structure of the data.}

With the emerging developments on the computational analysis of microarray
data\cite{quackenbush_computational_2001,irizarry_summaries_2003} and the
concomitant accrued sensitivity to its technical specifics, the stream of
published microarray studies started to be tempered by a series of reviews
questioning the validity, reproducibility and biological significance of the
results.\cite{michiels_prediction_2005,tinker_challenges_2006,kawasaki_end_2006,cahan_meta-analysis_2007,gusnanto_identification_2007,mathoulin-pelissier_survival_2008,kim_expectations_2010}

In iconic fashion, John Ioannidis, now a Professor of Medicine at Stanford,
chastised the field with a dire assessment on the lack of reproducibility of
some \mbox{high-profile} microarray research
findings.\cite{ioannidis_repeatability_2009} Upon independent dissection of the
analysis protocols of eighteen studies published in \mbox{\emph{Nature
    Genetics}} during 2005 and 2006, his team concluded that ten of them could
not be reproduced at all.  The main reason for failure to reproduce was data
unavailability, and discrepancies were mostly due to incomplete data annotation
or specification of data processing and analysis.

In the meantime, prognostic gene expression signatures of clinical outcome of
breast cancer were accumulating in the literature at a steady
rate.\cite{vant_veer_gene_2002,paik_multigene_2004,ma_two-gene_2004,wang_gene-expression_2005,chang_robustness_2005,miller_expression_2005,glinsky_microarray_2005,foekens_multicenter_2006,naderi_gene-expression_2006,teschendorff_consensus_2006,sotiriou_gene_2006,liu_prognostic_2007}
Intriguingly, very few genes were showing in common among the distinct
prognostic markers.  On a thorough review on the correspondence between these
biomarkers and the clinicopathological features of breast cancer, Christos
Sotiriou and Lajos Putszai, from the Institut Jules Bordet in Brussels and the
University of Texas in Houston, sought to form a synthesis of the evidence
supporting genomic prognostic signals.\cite{sotiriou_gene-expression_2009}
Sotiriou and Putszai reasoned that the absence of common markers between
signatures could be a feature of complex gene expression systems entertaining a
large number of correlated variables.  They also stressed the general tendency
for prognostic signatures to perform better among \smallcaps{ER+} tumours, as
they best discriminate \mbox{low-proliferation} \mbox{luminal \smallcaps{a}}
tumours from \mbox{high-proliferation} \mbox{luminal \smallcaps{b}} tumours,
whereas they mostly classify \smallcaps{ER--} tumours (comprising the basal-like
and \mbox{\smallcaps{her}-positive} phenotypes) as \mbox{high-risk}.  This could
be explained by the ability of prognostic signatures to capture molecular
features of tumour differentiation and tumour grade, both linked with cancer
progression and metastatic spread.  They summed up by proposing that models for
breast cancer prognostication should include both genomic and clinical variables
for best accuracy.

% tectonic readjustments

In 2011, David Venet and Vincent Detours, at the Université Libre de Bruxelles,
further added to the debate concerning the biological interpretation of
prognostic genomic signals in breast cancer.\cite{venet_most_2011} Probing the
\mbox{295-sample} \smallcaps{nki} reference cohort, they compared the prognostic
ability of \mbox{fourty-seven} published breast cancer outcome signatures with
signatures made of random genes.  They showed that 60\% of them were not
significantly better outcome predictors than random signatures of identical
size.  In addition, more than 90\% of random signatures with more than 100 genes
were shown to be significant outcome predictors.  Interestingly, they observed
that adjusting breast cancer expression data for a proliferation marker
abrogated most of the outcome association of published and random signatures.
By systematically exploring outcome associations in the \smallcaps{nki}-295
cohort, Venet and Detours exposed a wider and more pervasive range of prognostic
signals than previously anticipated.  They obtained similar results by
replicating the analysis in expression profiles of an independent cohort of 380
breast cancer patients from another study.\cite{loi_definition_2007} A
\mbox{decade worth} of biological extrapolations based on the transcription
profiles of clinical samples was eventually challenged by a more stringent
formulation of experimental controls.

\medskip

Microarrays, once hailed as the ``21\textsuperscript{st} century divining
rod,''\cite{he_microarrays21st_2001} have highlighted exciting new avenues to
engage the neoplastic cell.  Still, on the battlefield, cancer remained as
deceptive a foe as ever.  Translational research with direct impact on clinical
practice resulting from the microarray boom has proved tentative and timid at
best.  The most significant contributions were made in the context of predicting
a patient's prognosis by interpreting a panel of specific \mbox{tumour-related}
genes.  The first \smallcaps{FDA} clearance of \mbox{microarray-based} gene
profiling reagents was obtained in May
2011.\footnote{\url{http://investor.affymetrix.com/phoenix.zhtml?c=116408&p=irol-newsArticle&ID=1561100}}
As of 2013,\cite{kittaneh_molecular_2013} three genomic assays were commercially
available for prognostication in early stage breast cancer: Oncotype
\smallcaps{dx} (consisting of a \mbox{21-gene} profile narrowed down from a list
of 250 candidate genes that were analyzed in a total of 447 patients from 3
separate studies); MammaPrint\textcopyright{} (based on the \mbox{70-gene}
\smallcaps{nki} predictor; \smallcaps{fda} approved in 2007); and
\smallcaps{pam50} (a \mbox{50-gene} set used for standardizing subtype
classification).

The mining of the wealth of data yielded by cancer expression profiling has been
as much a source of promising guidance as of humbling reappraisal.

% Other issues troubling outome-related microarray analysis include poor
% definition of survival end points\cite{mathoulin-pelissier_survival_2008}

% There's a kernel of truth

% \mbox{Outcome-related} microarray analysis\footnote{\mbox{Outcome-related}
% studies seek to establish a statistical association between gene expression
% levels and a particular clinical outcome, such as relapse of disease, death,
% or therapeutic response.} allows for all these and, in their review, Dupuy and
% Simon schematized these objectives into three categories of statistical
% analysis: finding genes correlated with outcome, class discovery, and
% supervised prediction.

% results, list of possible biases. batch specific biases.

% 2007: Dupuy & Simon review validity and reproducibility of microarray-based
% research with a meta-analysis on 90 studies.  Quote: ``The use of microarray
% technology has generated great excitement for its potential to identify
% biomarkers for cancer outcomes, but the reproducibility and validity of
% findings based on microarray data have come under widespread challenge.''
% (nci-dupuy-2007.pdf)

% Steps to run a microarray: 1-sequence the entire genome of the organism
% you're studying.  2-use massive amount of computing power to determine where
% all the genes are in your sequence.  3-more computing power to design primer
% pairs to use PCR to make copies of every gene.  4-with those primers, copy
% every gene on the genome 5-run quality tests on all gene copies to verify
% which PCR reactions did not work 6-separate double stranded DNA into single
% stranded DNA.  7-use robots to place microscopic droplets of each
% single-stranded DNA into ordered rows and columns on a glass microscopic
% slide (microarray).

% each spot contains multiple copies of a unique DNA sequence that represents a
% single gene.

% list of terms to use:
% genetic catalog; gene expression profile.

% While microarrays quantify the m\smallcaps{rna} species extracted from a given
% tissue, they do not tell us if the subsequent protein is actually being
% properly synthesized.

% Limitations of microarrays.  They cannot:
% 1-tell which genes went bad to cause a disease.
% 2-cure a disease
% 3-identify every genes that is behaving inappropriately

% From:
% http://en.wikipedia.org/wiki/Breast_cancer_classification#cite_ref-pmid20065178_43-0
% Several commercially marketed DNA microarray tests analyze clusters of genes
% and may help decide which possible treatment is most effective for a
% particular cancer.  The use of these assays in breast cancers is supported by
% Level II evidence or Level III evidence.  No tests have been verified by Level
% I evidence, which is rigorously defined as being derived from a prospective,
% randomized controlled trial where patients who used the test had a better
% outcome than those who did not.  Acquiring extensive Level I evidence would be
% clinically and ethically challenging.  However, several validation approaches
% are being actively pursued.

% tying up:
% Some evidence suggest that these assays provide comparable
% prognostic information: Concordance among Gene-Expression-Based Predictors for
% Breast Cancer

% Affymetrix Achieves First FDA Clearance of Microarray-Based Gene Profiling
% Reagents
% http://investor.affymetrix.com/phoenix.zhtml?c=116408&p=irol-newsArticle&ID=1561100

\clearpage

% Cancer\footnote{``It's bad bile.  It's bad habits.  It's bad bosses.  It's bad
% genes.''---\emph{Mel Greaves}}

%%% Local Variables:
%%% TeX-engine: xetex
%%% mode: latex
%%% TeX-master: "../../thesis"
%%% End:
