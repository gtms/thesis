\section{Cancer}

% introduction with a challenging perspective

\newthought{Cancer is a rupture} of the social contract engaged by cells of the
somatic lineage of multicellular organisms.  This defection is caused by a
collection of critical failures of the genetic systems evolved to ensure the
correct and timely integration of the cellular unit's physiology at the tissue
and organism's level.  Neoplastic cells are operationally defined by their
ability to sustain chronic proliferation, to invade tissues and to set up
satellite growths in other organs.  When left unchecked, these features can
compromise the host organism's ability to sustain homeostatic
balance,\footnote{While the concept of homeostasis emphasises the stability of
  the internal milieu toward perturbation, perhaps a more accurate formulation
  could be that of \mbox{\emph{homeodynamics}}---a concept that seeks to account
  for the diverse behaviour exhibited by biological systems, including all its
  emergent characteristics, i.e., bistable switches, thresholds, waves,
  gradients, mutual entrainment, and periodic as well as chaotic behaviour
  (\citealp{lloyd_why_2001}).} and eventually lead to its systemic failure---and
death.

\bigskip

% defining attributes of cancer; functional and morphological features;
% conceptual framework to understand cancer

% Quoting Emmanuel Farber in his 1973 address to the American Cancer Society:
% \begin{quotation}
%   ``Cancer'' is an imprecise descriptive term to encompass all the many
%   conditions in which cells proliferate for whatever reason in a more or less
%   uncontrolled manner, invade tissues, and set up satellite growths in other
%   organs.  The overall result of such a process, if left undisturbed, is
%   almost always the death of the
%   host.\cite{farber_carcinogenesiscellular_1973}
% \end{quotation}

Far from a static physiological state specified by a particular set of cellular
dysfunctions, cancer is rather a progression along the course of a dynamic
evolutionary history.  Sixty years ago, Armitage and
Doll\cite{armitage_age_1954} developed a multistage theory to analyze rates of
cancer progression.  That theory rested on the premise that cancer builds upon a
sequence of cellular systems' cumulative failures.  Each such failure, for
instance the abrogation of a critical \smallcaps{dna} repair pathway or the loss
of control over cellular death, moves the system one step along the progression
to disease.\cite{frank_dynamics_2007}

Models of neoplasia evolution historically identify the seminal transforming
event with an alteration on a single somatic cell\footnote{\emph{Ominis cellula
    e cellula}---every cell originates from a cell alike---, is the epigram
  popularized by Rudolph Carl Virchow, stating a shift from the tenet of
  spontaneous generation that dominated the 19\textsuperscript{th} century
  school of thought concerning cancer's origins.} that triggers cancer
progression.  Initiation events contributing to the early stages of neoplastic
transition are caused by mutations in specific genes whose output is either
enhanced (oncogenes) or repressed (tumour supressor genes).\footnote{Alterations
  in nearly 500 of such genes have been linked to cancer initiation and
  progression (\citealp{forbes_catalogue_2008}).}  Such genetic mutations can be
structural, including nucleotide substitutions or mutations resulting from gene
fusion,\cite{konopka_cell_1985} juxtaposition to enhancer
elements,\cite{tsujimoto_t14;18_1985} or by amplification.  Alterations that
imply a discrete change of output in gene expression, such as translocations or
other structural mutations, can occur as initiating
events\cite{finger_common_1986} or during tumour progression, whereas
amplification usually occurs during progression.\cite{croce_oncogenes_2008}

However, a single genetic change is rarely sufficient for the development of a
neoplasia.  Since the term \emph{neoplasia} is generally used to refer to any
new, abnormal growth of tissue,\footnote{Concomitantly, \emph{malignant}
  neoplasia is defined by the acquired capacity of neoplastic cells to invade
  locally and metastasize.} the original oncogenic hit is usually associated
with a mutation disrupting the balance between proliferation and cell death.
From then on, cancer progression is modeled as a reiterative process of clonal
expansion, with sequential subclonal
selection.\cite{nowell_clonal_1976,greaves_clonal_2012} The dynamics of this
evolution are dictated by successive genetic and epigenetic\footnote{In this
  text, \emph{epigenetics} will refer to the range of global modifications in
  gene expression that are not under control of the genetic code itself.  The
  modifiable and reversible nature of certain cancer programs can largely be
  explained by epigenetics.} changes in the neoplasm, and are constrained by the
ecological context in which the tumour is developing.

The prevailing mode of clonal evolution is through the gradual emergence of
selectively advantageous ``driver'' genetic injuries against a complex
background of mostly deleterious and selectively neutral ``passenger'' lesions.
Alternatively, or perhaps concurrently, another mode of tumoural evolution
contemplates the possibility of a few drastic events generating multiple lesions
at once across the genome.  These dramatic punctuated changes can be prompted by
an acute insult or a single catastrophic pan-genomic event---of which
chromothripsis, at the chromosome level, is an
example.\cite{stephens_massive_2011}\footnote{The argument of gradualism versus
  punctuated equilibrium (\citealp{gould_punctuated_1993}) is a longstanding
  debate in species evolution and is another example of how much our current
  conceptualization of cancer progression owes to the developments of the theory
  of evolution in the second half of the 20\textsuperscript{th} century.}

The time frame of somatic evolution is a function of the mutational rate of
neoplasms.  While events like chromothripsis and
kataegis\footnote{\emph{Kataegis}, a term derived from the ancient Greek word
  for ``thunder'', refers to a pattern of localized hypermutation identified in
  some cancer genomes (\mbox{\citealp{nik-zainal_mutational_2012}}).} have the
potential to provide nearly instant triggers for the onset of disease, the high
frequency of clinically covert \mbox{pre-maligant}
lesions\cite{sakr_frequency_1993} suggests that transformation of somatic cells
is a far more frequent event than advised by incidence curves.  Furthermore, the
fact that the majority of cancers only become clinically relevant at old age is
a testament to both the prevalence of cancer-suppressing mechanisms as to the
relatively slow rates of mutational accretion in neoplasms.  Intriguingly, the
rate of epigenetic change has been reported to be orders of magnitude higher
than that of genetic change,\cite{siegmund_inferring_2009} but its role in
clonal evolution is not yet completely understood.

% The interaction of a tumour with its milieu is a complex, dynamic and
% reciprocal affair.

While the evolution of neoplasms is driven by their underlying genetic and
epigenetic lability, it is their tissue ecosystems that provide the adaptive
landscape for clonal fitness selection.\cite{greaves_clonal_2012} Systemic
regulators, such as hormones, growth factors, immune and inflammatory cells as
well as cytokines may conspire either to counteract or promote neoplastic
growth.\cite{bierie_tumour_2006,hanahan_hallmarks_2011} Architectural
constraints, in the form of physical compartments, basement membranes and
confined metastatic niches, restrict the growth of tumoural masses and set a
boundary for neoplastic microevolution.  But perhaps most striking is the
ability of tumours to remodel tissue micro-environments to their competitive
advantage---illustrated by the capacity of transformed cells to promote
neovascularization in response to anoxia or to incite malignant phenotypes in
their adjacent stromal cells.\cite{lathia_deadly_2011}

At the phenotipic level, the course of neoplastic evolution is tagged along two
major defining axis: one concerning increasing proliferation rates and another
reporting the loss of morphological and physiological differentiation at the
cellular level.\cite{tarabichi_systems_2013}  The paradox resides then in
explaining how transformed genomes, torn by their karyotypic instability and
ragged by waves of disrupting mutations, are able to cope with the formidable
challenges evolved by multicellular organisms to control cellular defection.

In order to become clinically conspicuous, endogenously proliferating neoplastic
masses must be able to evade sophisticated \mbox{built-in} genetic programs
evolved during the establishment of multicellularity, precisely to enforce
compliance with societal rules.  These include programs designed to eradicate
deviant phenotypes, such as apoptosis, senescence and necrosis; a range of
\mbox{cell-to-cell} signaling programs designed to control and suppress
unwarranted growth; and, perhaps most challenging, cancer cells must dynamically
and persistently deceive the ever adapting propositions of the host organism's
immune system.

% self-reliant and highly resilient neoplasms

In addition, neoplasms can also acquire the astonishing capacity to invade
surrounding tissues and to disseminate the organism, a feature responsible for
most cancer-related deaths.  This complex process, termed the
\mbox{invasion-metastasis} cascade,\cite{valastyan_tumor_2011} requires an array
of well coordinated genetic and epigenetic adaptions.  It can be schematized as
a sequence of discrete steps, starting with local invasion, followed by
intravasion by cancer cells into nearby blood and lymphatic vessels.  Tumour
cells in transit through the lymphatic and hematogenous systems then escape
the lumina of these vessels into the parenchyma of distant tissues (extravasion)
and eventually establish micrometastatic lesions that may fully develop into
macroscopic tumours.

To reconcile the outstanding increase in phenotypic resilience and plasticity of
neoplastic cells---in spite of their degrading genomic integrity---with the
canonical multistep model of cancer
progression,\cite{land_cellular_1983,vogelstein_multistep_1993} carcinogenesis
is conventionally portrayed as a linear Darwinian evolutionary
process.\cite{merlo_cancer_2006,polyak_tumor_2014} According to this
view,\cite{podlaha_evolution_2012} inheritance, environmental factors and
spontaneous errors in \smallcaps{dna} replication cause mutations or
epimutations in critical caretaker genes, fomenting genetic and epigenetic
instability.  This, in turn, promotes mutations or epimutations in specific
gatekeeper genes\footnote{Among \mbox{cancer-susceptibility} genes, the
  distinction between \emph{caretaker} and \emph{gatekeeper} genes is a subtle
  yet conceptually important one.  While the former are responsible for
  maintaining the integrity of the genome, the latter directly control cellular
  proliferation.  This is epitomized by the \mbox{breast-cancer-susceptibility}
  genes \smallcaps{\emph{brca1}} and \smallcaps{\emph{brca2}}, that can
  functionally play both roles at different points of breast cancer progression
  (\citealp{kinzler_gatekeepers_1997}).}  (oncogenes or tumour supressor genes),
triggering uncontrolled growth.  The ensuing genomic instability creates a
feedback loop that increases evolutionary and proliferation
rates.\cite{sieber_genomic_2003} This results in heritable variation in the form
of clonal diversity, upon which Darwinian selection operates.  Clones that
progressively acquire the biological hallmark capabilities of
cancer\footnote{These include sustaining proliferative signaling; evading growth
  suppressors; resisting cell death; enabling replicative immortality; inducing
  angiogenesis; and activating invasion and metastasis
  (\citealp{hanahan_hallmarks_2011}).} gain a competitive edge and become more
dominant and aggressive, promoting a form of neoplastic progression that, at the
macroscopic level, is consistent with a multistep
development.\footnote{Interestingly, this modern consensus on tumoural evolution
  is thus not very distant from what Peter Nowell proposed in his 1976 landmark
  paper (\citealp{nowell_clonal_1976}) postulating that ``neosplasms arise from
  a single cell of origin, and tumour progression results from acquired genetic
  variability within the original clone allowing sequential selection of more
  aggressive sublines.''}

This prevailing reasoning is supported by the observed \mbox{step-wise}
progression of most solid tumours.  For instance, the gradual evolution of colon
cancer is well documented.\cite{vogelstein_multistep_1993} Here, the first
manifestations of neoplasia occur in the colorectal epithelium, in the form of
small benign adenomas.  Such tumours are reasonably confined and are almost
normal in their intra- and intercellular organisation.  With time, adenomas
start to grow (proliferation) and become morphologically and physiologically
disorganized (dedifferentiation).  Eventually the tumour evolves into an
aggressive neoplasia (carcinoma), presumably because one of the cells in the
adenoma has acquired a sufficient number of mutations to drive the process of
invasion and metastasis.

The concept of gatekeeper gene is central to this understanding.  Consider for
instance the \smallcaps{\emph{tp53}} gene, the first \mbox{tumour-suppressor}
gene to be identified in 1979.  This gene encodes for the p53 protein, a master
transcription factor with an overarching role in the maintenance of the
integrity of the genome.\cite{efeyan_p53:_2007} Under normal circumstances, p53
is functionally inactive due to its rapid degradation.  However, upon the
infliction of virtually any form of cellular stress, p53 degradation is halted,
and the protein gains full competence in transcriptional activation.  The
regulatory networks under its control are associated with several critical
mechanisms for cancer progression, namely apoptosis, \mbox{cell-cycle}
inhibition, genome stability and inhibition of
angiogenesis.\cite{vogelstein_surfing_2000} Unsurprisingly, about 50\% of all
human cancers have lost p53 expression or express an inactive mutant of the
protein.\cite{toledo_regulating_2006}

The evolutionary origins of the \smallcaps{\emph{tp53}} gene can be inferred
from modern-day descendants of both the single cell choanoflagellates and the
early multicellular sea anemone.  The function of the homolog to this ancestral
gene in the sea anemone is to protect the germ-line gametes from \smallcaps{dna}
damage.\cite{belyi_origins_2010} Over the course of the last billion years, this
function has not only been conserved, but enhanced through
pleiotropy\footnote{The ability of a single gene to influence multiple,
  seemingly unrelated phenotypic traits.} to set \smallcaps{\emph{tp53}} as a
major enforcer of somatic conformity in multicellular organisms.

Another recurrent oncogene with an established role in \mbox{cell-cycle}
progression and apoptosis, \smallcaps{\emph{myc}}, has had its evolutionary
roots traced back to at least 600 million years ago.\cite{hartl_stem_2010} The
fact that many cancer-susceptibility genes are ancient, highly conserved and may
have taken a role in the transition to
multicellularity\cite{srivastava_amphimedon_2010} has been interpreted as
evidence that they play a pivotal role in regulating the normal physiology of
the somatic cell.\cite{weinberg_oncogenes_1983,weinberg_biology_2013}  Their
disruption during carcinogenesis could then symbolize the unshackling of the
transformed cell from its social bindings.

This narrative paints the neoplastic cell as an egotistic rogue cell set free to
thrive through uncontrolled replication in a hostile environment.  It becomes a
metaphor for the unicellular evolutionary stage, with cancer lineages competing
with each other and with normal cells for survival.\cite{merlo_cancer_2006} The
success of any one lineage is dependent on its \mbox{step-wise} acquisition of
cancer hallmarks through Darwinian evolution.  However appealing and widespread,
this idea fails to account for a number of
observations.\cite{davies_cancer_2011}

First, it falls short to explain the rather high degree of cooperative
organization among cancer cells.  Increasingly, tumours are being recognized as
ecosystems with complex and dynamic interactions between neoplastic cells and
their microenvironment.\cite{polyak_co-evolution_2009} These heterotypic
interchanges include the stimulation via paracrine signaling of normal stromal
cells to produce mitogenic signals; the required signaling to induce
neoangiogenesis; and the disparate cell-to-cell interactions during the
invasion-metastasis cascade.\cite{axelrod_evolution_2006} Clearly, a very
specific minimal set of communication skills must be acquired from early on in
order for cancer cells to prosper.  It is hard to conceive renegade cells
evolving such intricate adaptations independently from the ground up; rather, a
more prosaic explanation could involve a shift or modulation of the original set
of rules somatic cells use to engage with their partners.

Second, it fails to elucidate the quasi teleological way by which advanced
tumours consistently resort to the same array of sophisticated
adaptations\cite{hanahan_hallmarks_2011} to match the challenges posted by their
host organisms.  At their end stage, malignant carcinomas could almost be taken
for entities bestowed with a volition of their own, capable not only to evade
the host organism's defenses, but to shrewdly overtake it.  The conventional way
to account for this comprehensive arsenal of adaptations is to invoke a
Darwinian evolutionary process among competing lineages of rogue
cells.\cite{merlo_cancer_2006} However, some adaptations are particularly
challenging to frame in this conceptual view.  Take for instance the progression
puzzle highlighted by Bernards and Weinberg: how can mutations promoting
invasion and metastasis add a competitive advantage to the primary tumour in its
original ecological context?\footnote{To untie this knot, the proposed solution
  was to invoke a ``pre-ordained'' correlation between some previously acquired
  advantageous genes and the genes that facilitate metastasis
  (\citealp{bernards_progression_2002}).}  Additionally, given that the majority
of variants randomly arising on a genetic system are deleterious, the odds of
any neoplastic cell incrementally acquiring all the required hallmark mutations
before collapsing to a lethal one is exceptionally low.

Third, it fails to properly integrate the role of genetic instability in cancer
progression.  Because advantageous mutations are so rare, rogue cells are
thought to promote genomic instability in order to accelerate evolutionary rates
and increase the odds of, via Darwinian selection, produce a fully malignant
phenotype.\cite{sieber_genomic_2003} Such mutational arms race among rogue cell
lineages to reach the jackpot of complete neoplastic competence is riddled by a
paradox: too few mutations in the mix and the cell won't evade genetic controls;
too many and it dies.  Especially troublesome are the \mbox{pan-genomic}
mutations that lead to gross structural changes, including aberrant chromosomes
and aneuploid cells.\footnote{``If you look at most solid tumours in adults, it
  looks like someone set off a bomb in the nucleus''---\emph{William C. Hahn}}
Breivik proposed an elegant solution to this
inconsistency.\cite{breivik_evolutionary_2005} Rather than postulating genomic
instability as a pre-requisite to cancer progression, he refashioned it as a
\mbox{by-product} of the lack of competitive fitness of reparing-phenotypes in
the tumour environment.\footnote{``Don't stop for repairs in a war zone'' is the
  metaphor used by Breivik to explain why a mutagenic environment would increase
  the fitness of the non-repairing phenotype.  This conceptual reformulation
  could have implications on other central controversies in cancer research---as
  discussed bellow.}  Still, it remains thought-provoking to note that the
neoplastic cells with the most deranged genomes are precisely those with the
most competitive phenotypes.

A tentative hypothesis to address these quandaries has been put forward by
Davies and Lineweaver.\cite{davies_cancer_2011} Instead of modeling the
transformed cell as a free-agent requiring the independent \emph{acquisition} of
enabling properties to obtain malignant status, it seeks to explain hallmark
adaptations as the reenactment of atavic genetic systems already embedded in the
genomes of somatic cells.  According to this view, the disruption of high-order
gatekeeper genes drives the neoplastic cell to reconfigure its regulatory
networks around a \emph{pre-existing} toolkit of primitive adaptations
reminiscent of those evolved during early transition to multicellularity.
Although its original formulation has been either ignored\footnote{At the date
  of this writing, Google Scholar reported a total of 36 citations of the
  original article.} or thoroughly
dismissed,\cite{pettit_cancer_2012,myers_aaargh!_2012} this idea is not entirely
without merits---and could provide a basis for a reevaluation of some core
features of cancer.

Consider the Warburg effect\cite{warburg_origin_1956} for instance.  The
majority of cancer cells favour a metabolism based on anaerobic glycolysis
followed by lactic acid fermentation in the cytoplasm.  Because glycolysis is
far less efficient than oxidative respiration for \smallcaps{atp} production,
this metabolic shift has been implied to be an essential adaptation to cancer
progression---perhaps in response to intermittent hypoxia in malignant
lesions.\cite{gatenby_why_2004} A more economical interpretation could be that
transformed cells, without gatekeeper genes to enforce the proper regulation of
the high-performing metabolic mode of somatic cells, are now forced to fall back
to a more basic, yet reliable, metabolic outlet for energy production.  While
the former interpretation taxes neoplastic progression with another requirement
and begs for a rationale for the selective fitness of anaerobic metabolism, the
latter simply conceives the cancer cell as a system seeking a dynamic
equilibrium in a novel, stressful and unstable environment.

More recently, another much debated concept in cancer research has been the
cancer stem cell (\smallcaps{csc}).\footnote{These cells are operationally
  defined by their ``capacity for self-renewal, the potential to develop into
  any cell in the overall tumor population, and the proliferative ability to
  drive continued expansion of the population of malignant cells.''
  (\citealp{jordan_cancer_2006})} While the idea of a unique, genetically
homogeneous population of determined cancer stem cells is still
contested,\cite{marotta_cancer_2009} the existence of a particular class of
tumour propagating cells within neoplastic masses is well accepted.  Here, the
crux of the matter concerns the ontology of the \smallcaps{csc} phenotype.  In
short, \smallcaps{csc}s have been proposed to arise from mutations in either
derivatives of developing stem or progenitor lineages, or from mutations in
differentiated cells that acquire stem-like
attributes.\cite{wicha_cancer_2006,lobo_biology_2007} As niche-specific stem
cells are subjected to higher turnover rates to seed their respective tissues,
they become prime targets for neoplastic-inducing replication defects.  The
potential of a seeder neoplastic cell at the apex of the tumoural clonal
hierarchy is luring both from the diagnostic and treatment perspective.
Moreover, this conceptualization seems especially fitting for the established
linear Darwinian clonal progression model.  However, there is evidence that
these cells do not constitute one homogeneous population.  Alternatively, the
\smallcaps{ctc} phenotype could represent one of many possible configurations to
be adopted by the neoplastic cell, given the selective pressures it faces in its
milieu, and the stochastic events that govern its internal homeodynamics.  The
\smallcaps{ctc} would then be, rather than a deterministic, a dynamically
reversible phenotype; and represent, instead of qualitative, a quantitative
modulation.\cite{maenhaut_cancer_2010,tarabichi_systems_2013}

Another example of the phenotypic modularity of the neoplastic cell is the
epithelial-mesenchymal transition program (\smallcaps{emt})---and its reverse
process, the mesenchymal-epithelial transition (\smallcaps{met}).  As with the
\smallcaps{ctc} phenotype, this mechanism is borrowed from embryogenesis, where
it partakes in gastrulation, neural crest formation, heart valve formation,
palatogenesis and myogenesis.\cite{thiery_epithelial-mesenchymal_2009} This
comprehensive genetic program globally shifts the physiology and morphology of
the cell between an epithelial phenotype (characterized by a well defined
polarity, tight junction of the cells and a their binding to a basal lamina) and
a mesenchymal phenotype (characterized by a lack of polarity, spindle-shape
morphology and loose cell-to-cell interaction).\cite{thiery_complex_2006} Under
normal physiological circumstances, the dynamics of the \smallcaps{emt-met}
program are under strict and orderly genetic control, even if it can be
recruited exceptionally, for instance in response to injury.  In pathological
conditions, the unwarranted activation of the \smallcaps{emt-met} program can
cause organ fibrosis.  But most importantly it provides cancer cells with an
outlet to break free from the primary tumour and embark the invasion-metastasis
cascade under the cover of the mesenchymal phenotype.  Conversely, the
colonization of new micrometastatic niche is facilitated by a transition back to
an epithelial phenotype.  This constitutes the most thorough illustration yet of
how neoplastic cells can recruit built-in genetic apparatuses to deploy complex
adaptations.

Thus, cancer cell lineages could evolve by exploiting the realm of new
homeodynamic equilibriums made available by the breaching of higher-order
somatic regulatory systems.  Because these regulatory circuitries are kept in
check by gatekeeper genes, this would entail that the number of required
breaking points for the potential emergence of full neoplastic competence is
considerably smaller than anticipated. The acquisition of the hallmark
adaptations could simply be the outcome of a stochastic process of readjustments
of regulatory networks of the cell.

Even Weinberg flirted with such possibility: ``Maybe the information for
inducing cancer was already in the normal cell genome, waiting to be
unmasked.''\footnote{\citealp{weinberg_biology_2013}, \emph{p} 79} Could this
idea be further developed?  Consider the main criticism to the hypothesis of
Davies and Lineweaver: their invocation of an \emph{atavistic} genetic system,
in itself largely responsible for the cancer phenotype, already packed in our
genomes and ready to be unleashed with the disruption of gatekeeper
genes.\footnote{``In short, proto-metazoans, which we dub Metazoans 1.0, were
  tumor-like neoplasms.'' (\citealp{davies_cancer_2011})} This proposition is
perhaps too easy to caricature and to dismiss, but it does share one aspect in
common with the linear Darwinian model for tumourigenesis: both seek to explain
cancer almost exclusively through its genetic determinants.  The two theories
differ, however, in the nature of the qualitative changes required for tumor
progression.  While clonal evolution posits for an \emph{ab initio} generation
of genetic determinants for cancer progression, the Metazoan theory
short-circuits this precondition by allocating the required determinants in the
genome of the somatic cell.  None of these perspectives fully addresses the
perplexities of cancer; both seem to ignore the epigenetic determinants for
cancer progression.

In reality, the predicament of the neoplastic cell might be much more complex.
Evolution has concocted a vast collection of modular genetic programs to provide
the somatic cell with adaptive dynamic physiological ranges within which to
carry its activities.  In order for the orderly integration of the somatic cell
at the tissue and organism level, such physiological ranges must be under a
tight genetic control of a higher-order.  The breakdown of this level of control
leaves the somatic cell hapless on how to keep its metabolism balanced and free
to stochastically reconfigure its genetic regulation.  This may lead to the
co-optation of genetic programs via natural selection as they increase the
fitness of the transformed cell in particular ecological contexts.  Because
these programs usually end up serving an adaptive role other than the one they
were evolved to provide, perhaps cancer adaptations ought to be better described
as \emph{exaptations}\cite{gould_exaptation_1982}---a concept that leaves at
rest the teleological interpretation of cancer hallmarks.

But let us speculate a bit further.  The evolutionary success of the neoplastic
cell is only bounded by its ability to self-replicate.  Consequently, with the
exception of the genetic systems that support the cell cycle and the remaining
context-specific hallmark adaptations, the vast majority of the transformed
genome could be at the mercy of disruption through genetic instability.  This
alone could explain the second major axis\footnote{The first being increased
  proliferation.} of cancer progression: the gradual loss of morphological and
physiological differentiation of neoplasms as they evolve through the ranks of
malignancy.  But there is another neglected driver of cancer evolution, one that
could prove to be the hidden iceberg upholding the shapeshifting nature of
cancer.

Very little is know about the non-genetic, or epigenetic, determinants of
cellular metabolism.  Chromatin remodeling and \smallcaps{dna} methylation are
known to modulate to a considerable degree patterns of gene expression;
\smallcaps{rna} transcripts and their encoded proteins may retroactively
modulate, direct or indirectly, the activity of certain
genes;\footnote{\smallcaps{rna} mollecules may even directly spread directly to
  other cells or nuclei by diffusion.} a variety of less known species of
\smallcaps{rna} are animated with catalytic activity of their own and their
transcriptional output has been historically
underestimated.\cite{huttenhofer_principles_2006,ptashne_use_2007}  This family
of non-protein-coding \smallcaps{rna}s are thought to be relics of a primordial
``\smallcaps{rna} world'', in which \smallcaps{rna} served both as the carrier
of genetic information and as the catalytic agent.  If cancer cells are able to
resourcefully and adaptively tap on the genetic resources of the somatic cell to
shape their evolution, why would they neglect the rich diversity of epigenetic
tools at their disposal?

Perhaps most provocatively, as these epigenetic systems are not necessarily
constrained by the dynamics of Darwinian evolution,\footnote{And could indeed
  abide by the rules of Lamarckian evolution.} their evolutionary rates could be
orders of magnitude higher than those observed in genetic systems---even those
riddled by genetic instability.  This could provide a potent mechanism driving
cancer evolution, one that might potentially account for the extraordinary
degree of resilience of advanced cancers.

% These considerations could have profound implications in the way cancer
% research, diagnostic, prognostic and treatment are conducted.

\medskip

Unlocking these mysteries may well require a lot of ingenuity, but without doing
so it will be hard to elucidate the intricate nature of life---and that of
its dark horse, cancer.

% Genetic pleiotropy.  Answer: rather than a multistep progression driven by
% linear accretion of driver mutations supporting the discrete and independent
% acquisition of malignancy hallmarks, a global transcriptional shift
% encompassing the activation of ldots{} Concept of Pleiotropy.  Concept of
% \emph{exaptations}.

% \smallcaps{\emph{tp53}} and \smallcaps{\emph{myc}}.

% The resilience, plasticity and insidiousness of malignant neoplastic cells is
% eerily reminiscent of the \ldots{}

% Final argument about Darwinian evolution, hallmarks of cancer, and eventual
% atavic genetic circuitry engaged once system gateway controlers, such as p53,
% fail.  Frame cancer as an evolutionary problem To discuss: Darwinian evolution
% of cancers, multistep model of cancer progression, hallmarks of cancer,
% critics of the hallmarks, oncogenes as evolutionary gatekeepers, examples
% centered around \smallcaps{\emph{tp53}} and \smallcaps{\emph{tp53}} oncogenes,
% integration of cancer programs, Warburg effect, a progressin puzzle: example
% with \smallcaps{EMT} program, metastasis is a highly inefficient process,
% similarities between cancer phenotype and early developmental stages, concept
% of cancer stem cell, concept of \emph{exaptation}.  Example: Evolutionary
% history of the retinoblstoma gene from rchea to eukaria.  Implications for
% both the understanding of multicellularity from an evolutionary perspective as
% well as for the understanding of the biology of cancer.

% However these complex considerations on the biology

% Cancer dynamics are bounded by genetic and epigenetic
% alterations.\footnote{From Nowell, 1976: reversibility of transformation in
% certain culture systems suggests that cancer initiation could result from
% altered gene expression rather than structural mutation}

% Cells of the neoplastic lineage are defined by their ability to sustain
% chronic proliferation, irrespectively of the social cues conveyed by its
% tissue context.  At the morphological level, however, the most conspicuous
% feature of cancer has to be the diversity of shapes and forms these
% proliferating masses can take when departing from the neatly organized
% architectural tissue types they arise from.  An integrative framework to
% apprehend this remarkable phenotypic plasticity has been proposed by Hanahan
% and Weinberg,\cite{hanahan_hallmarks_2000,hanahan_hallmarks_2011} who proposed
% six essential and complementary capabilities for tumour growth and metastatic
% dissemination.  These include self-suficiency in growth signals, insensitivity
% to growth-inhibitory signals, evasion of programmed cell-death, limitless
% replicative potential, sustained angiogenesis, and tissue invasion and
% metastasis.

% Cellular transformation, the process through which the neoplastic phenotype
% arises, is caused by genetic and epigenetic alterations in somatic cells.
% Under regular circumstances, most somatic cells exhibiting behaviours beyond
% physiological ranges end up being singled out and targeted for removal, either
% by eliciting an immune response or by triggering self-induced cellular death
% (\emph{note about apoptosis here}).  In order for a neoplasm to become
% clinically relevant, it has thus to acquire a number of alterations that
% consign it with the capacity to evade its host organisms' regulatory control
% mechanisms against unicellular defection.

% an operating outside of normal physiological levels cells abnormal behaviour
% exhibited by most early neoplastic cells may be sufficient for them to be
% recognized and targeted for removal, either by eliciting an immune response or
% by triggering self-induced cellular death (\emph{note about apoptosis here}).

% Clonality; multistep model for the nature of cancer;

% Specific genes, termed oncogenes, have the potential to induce transformation
% when disrupted in particular circumstances.  Alterations in nearly 500 of such
% genes have been linked to cancer initiation and
% progression.\cite{forbes_catalogue_2008} The multistep model for the nature of
% cancer posits that several such alterations are cumulatively required in order
% to initiate tumourigenesis and to evolve increasingly more aggressive and
% invasive tumour phenotypes.\cite{vogelstein_multistep_1993}

% The fundamental and defining characteristic of the cancer cell is, arguably,
% it's ability to sustain chronic proliferation.

% The number and patterns of somatic alterations vary dramatically across cancer
% types.  At one extreme, childhood medulloblastomas can harbour fewer than ten
% genomic alterations, whereas over \SI{50000} somatic changes have been
% observed in primary lung adenocarcinoma samples.

% In biological systems the instanciation of information is \smallcaps{DNA}.

% a phenotype eerily reminiscent of that seen in loosely cooperative unicellular
% life forms, at a time when multicellularity was still being attempted at.

% From the prologue of 'The Emperor of All Maladies': Malignant and normal
% growth are so genetically intertwined that unbraiding the two might be one of
% the most significant scientific challenges faced by our species. Cancer is
% built into our genomes: the genes that unmoor normal cell division are not
% foreign to our bodies, but rather mutated, distorted versions of the very
% genes that perform vital cellular functions.  individual alterations in
% oncogenes are seen as necessary but not sufficient to give rise to cancer.
% Considered to arise sequentially and to give rise to the progressively more
% aggressive and invasive phenotypes observed during tumourigenesis.

% which these control mechanisms are compromised lead by mutations on specific
% genes which these mechanisms are hindered is through the accumulation of
% somatic mutations of oncogenes


% Cells of normal tissues collectively control their growth rate by regulating
% the production and release of paracrine \mbox{growth-promoting} signals that
% direct entry and progression through the cell \mbox{growth-and-division}
% cycle.  A first requirement for the acquisition of the neoplastic identity
% must then be the ability to generate endogenous mitogenic signals that result
% in autocrine proliferative stimulation.\footnote{This endeavour can also be
% achieved through the emission of signals that stimulate surrounding normal
% cells to feed cancer cells back with growth factors
% (citealp{Cheng-2008,Bhowmick-2004}); specific somatic mutations that
% constitutively activate pathways usually triggered by activated growth factor
% repectors; or through disruptions of \mbox{negative-feedback} mechanisms that
% attenuate proliferative signaling.}  In addition, the cancer cell must also
% become insensitive to social cues designed to control unrestricted cell
% proliferation.

% From jcb-weinberg-1983.pdf:

% This leads to the realization that these [onco]genes are of extremely ancient
% lineage---their precursors were already present in similar form in the
% primitive metazoans taht served as common ancestors to chordates and
% arthropodes.  Such conservation indicates that these genes have served vital,
% indispensable function in normal cellular and organismic physiology, and that
% their role in carcinogenesis represents only an unusual and aberrant diversion
% from their usual functions.

% by caused by steered by the
% engagement of genetic routines of the form of an atavistic state engagement of
% genetic routines active at an earlier stage of development that are
% inappropriately reactivated in the mature organism as a result of some sort of
% insult of damage.  Cancer is an atavistic state of multicellular life.  Cells
% relieved of the molecular constrains that subordinate them to societal
% discipline.

% Causing cancer cells' metabolism to default to more fundamental modes of
% functioning not unlike those conceivably typified by loosely societal
% cellular sorts.

% These six hallmarks of cancer---distinctive and complementary capabilities
% that enable tumour growth and metastatic dissemination

% Further evidence that supports our theory comes from experiments in which the
% nuclei of egg cells are replaced with cancer-cell nuclei.  Astonishingly,
% embryos start to develop normally.  But abnormalities eventually appear, at
% earlier stages when the cancer is more malignant (advanced). This inverse
% correlation of cancer stage with embryo stage is consistent with our theory.

% Cancer follows a well-defined progression of within a host organism of
% accelerating growth (proliferation), mobility, spread and colonization.

% To better understand cancer, including its place in the great sweep of
% evolutionary history.

% ideias to discuss:

% causes of cancer

% field cancerization (tarabichi)

% 1--For multicellularity to become a viable strategy, a number of strategies
% must have evolved in order to maximize cooperation and minimize conflict
% between individual cells of an organism.

% 2--One such strategy is the imposition of a genetic bottleneck per generation
% in the course of the reproductive cycle of multicellular organisms in order to
% ensure clonality among constituent cells of an organism.

% 3--As cooperation creates new levels of fitness, it creates the opportunity
% for conflict between levels as deleterious mutants arise and spread within the
% group.  Fundamental to the emergence of a new higher-level unit is the
% mediation of conflict among lower-level units in favor of the higher-level
% unit.

% 4--Michod et al. (2003) lists a number of conflict-mediating adaptations that
% can be useful in engendering cooperation among physically associated cells.
% These include the existence of a germ line, the tendency for unnecessary or
% even defecting cells to undergo apoptosis, the potential for self policing,
% display of relatively low mutation rates, canalization of growth (of which one
% mechanism is "determinate growth"), etc.

% 5--Take the \emph{p53} gene as example.  \emph{p53} has likely been the most
% studied gene for over a decade.

% cancer dormancy is a also perplexing

% pleiotropy of the key oncogenic gatekeepers

% quote the process of ``exaptation'' (Gould & Vrba, 1982)
% the role of epigenetics

% The conventional argument to explain the deployment of this \mbox{swiss-knife}
% armory of solutions is to invoke Darwinian evolution between \mbox
% {sub-clones} of the original neoplasm.\cite{Merlo-2006} According to this
% hypothesis, the evolution of the required survival traits for the success of
% the tumour would be attributed mostly to random mutations and the trial and
% error of normal Darwinian evolution.  Rapid mutation rate within tumours +
% strong selective pressure as organism ``fights back'' or patient undergoes
% chemotherapy.

% This understanding of cancer as a feature of a particular evolutionary
% mechanism

% Rewrite discussion on multicellularity in the life section according to the
% lines discussed here:
% http://www.biologyaspoetry.com/textbooks/microbes_and_evolution/greater_size_problems_in_multicellularity.html

% NOTES from tmm-floor-2012.pdf: hallmarks of cancer: of all cancer cells, all
% the time?

% The hallmarks of cancer
% 1-self-sufficiency in growth signals
% 2-insensitivity in anti-growth signals
% 3-evasion of apoptosis
% 4-limitless replicative potential
% 5-sustained angiogenesis
% 6-tissue invasion
% 7-metastasis
% 8-metabolic reprogramming
% 9-evasion of the immune system

% Strangely, one fundamental characteristic of cancer cells, the loss of
% differentiation, was not considered a distinct hallmark.  This characteristic
% is essential because it is a primary difference between benign and malignant
% tumours (e.g., autonomous adenomas as opposed to carcinoma of the thyroid).
% It is supported by a measurable expression program switch that affects all
% gene categories.  We argue that this should be another, if not the major,
% ``hallmark'' of a cancerous cell, and it has been shown to be a robust
% diagnostic index for thyroid tumours.

% new hallmarks of cancer must be able

% cancer progression is characterized by a succession of oncogenic events

% The hallmarks of cancer (cel-hanahan-2011)
% The hallmarks constitute an organizing principle for rationalizing the
% complexities of neoplastic disease.  They include:
% 1--sustainment of proliferative signalling
% 2--evasion of growth supressors
% 3--resistance to cell death
% 4--enabling of replicative immortality
% 5--inducing of angiogenesis
% 6--activation of invasion and metastasis

% The genomes of nearly all healthy human cells, containing the entirety of an
% individual's inherited information, evidently come pre-loaded with a ``cancer
% sub-routine'' that is normally idle but can be triggered into action by a wide
% variety of insults, such as chemicals, radiation and inflammation.

% Breivik 2005

% Cancer incidence increases with age, and in countries with high life
% expectancy, aproximately 40\% of the population experiences some kind of
% cancer in their lifetime.  Adiotionally, a significant proportion of the
% autopsies reveal undiagnosed maliganacies, and virtually everybody carry some
% kind of in situ carcinomas after the age of 50.

\bigskip

% Clinical definition of cancer. A vast collection of diseases (almost 100,
% according to http://medical-dictionary.thefreedictionary.com/Cancer).

\newthought{But cancer is also} a leading cause of death worldwide, accounting
for 8.2 million deaths in 2012.\cite{ferlay_globocan_2014}

Cancer incidence\footnote{Incidence is the number of new cases arising in a
  given period in a specified population. This information is collected
  routinely by cancer registries. It can be expressed as an absolute number of
  cases per year or as a rate per 100,000 persons per year (see Crude rate and
  ASR below). The rate provides an approximation of the average risk of
  developing a cancer.} and mortality\footnote{Mortality is the number of deaths
  occurring in a given period in a specified population. It can be expressed as
  an absolute number of deaths per year or as a rate per \SI{100000} persons per
  year.}
