\section{Cancer}

% introduction with a challenging perspective

\newthought{Cancer is a rupture} of the social contract engaged by cells of the
somatic lineage of multicellular organisms.  This defection is caused by a
collection of critical failures of the genetic systems evolved to ensure the
correct and timely integration of the cellular unit's physiology at the tissue
and organism's level.  Neoplastic cells are operationally defined by their
ability to sustain chronic proliferation, to invade tissues and to set up
satellite growths in other organs.  When left unchecked, these features can
compromise the host organism's ability to uphold homeostatic
balance,\footnote{While the concept of homeostasis emphasises the stability of
  the internal milieu toward perturbation, perhaps a more accurate formulation
  could be that of \mbox{\emph{homeodynamics}}---a concept that seeks to account
  for the diverse behaviour exhibited by biological systems, including all its
  emergent characteristics, i.e., bistable switches, thresholds, waves,
  gradients, mutual entrainment, and periodic as well as chaotic behaviour
  (\citealp{lloyd_why_2001}).} and eventually lead to its systemic failure---and
death.

\bigskip

% defining attributes of cancer; functional and morphological features;
% conceptual framework to understand cancer

% Quoting Emmanuel Farber in his 1973 address to the American Cancer Society:
% \begin{quotation}
%   ``Cancer'' is an imprecise descriptive term to encompass all the many
%   conditions in which cells proliferate for whatever reason in a more or less
%   uncontrolled manner, invade tissues, and set up satellite growths in other
%   organs.  The overall result of such a process, if left undisturbed, is
%   almost always the death of the
%   host.\cite{farber_carcinogenesiscellular_1973}
% \end{quotation}

\subsection{The dynamics of cancer}
\label{sec:dynamics-cancer}

Sixty years ago, Armitage and Doll developed a multistage theory to analyze
rates of cancer progression.\cite{armitage_age_1954} They reasoned that cancer
builds upon a sequence of cellular systems' cumulative failures.  Each such
failure, for instance the abrogation of a critical \smallcaps{dna} repair
pathway or the loss of control over cellular death, moves the system one step
closer to the onset of disease.\cite{frank_dynamics_2007} Rather than a static
physiological condition specified by a unique set of cellular dysfunctions,
cancer is depicted as a progression along the course of a dynamic evolutionary
history.

Models of neoplasia evolution historically identify the seminal transforming
event with an alteration on a single somatic cell that triggers cancer
progression.  Initiation events contributing to the early stages of neoplastic
transition are caused by mutations in specific genes whose output is either
enhanced (oncogenes) or repressed (tumour supressor genes).\footnote{Alterations
  in nearly 500 of such genes have been linked to cancer initiation and
  progression (\citealp{forbes_catalogue_2008}).}  Such genetic mutations can be
structural, including nucleotide substitutions or mutations resulting from gene
fusion,\cite{konopka_cell_1985} juxtaposition to enhancer
elements,\cite{tsujimoto_t14;18_1985} or by amplification.  Alterations that
imply a discrete change of output in gene expression, such as translocations or
other structural mutations, can occur as initiating
events\cite{finger_common_1986} or during tumour progression, whereas
amplification usually occurs during progression.\cite{croce_oncogenes_2008}

However, a single genetic change is rarely sufficient to trigger the development
of a neoplasia.  Since the term \emph{neoplasia} is generally used to refer to
any new, abnormal growth of tissue,\footnote{Concomitantly, \emph{malignant}
  neoplasia is defined by the acquired capacity of neoplastic cells to invade
  locally and metastasize.} the original oncogenic hit is usually associated
with a mutation disrupting the balance between proliferation and cell death.
From then on, cancer progression is modeled as a reiterative process of clonal
expansion, with sequential subclonal
selection.\cite{nowell_clonal_1976,greaves_clonal_2012} The dynamics of this
evolution are dictated by successive genetic and epigenetic\footnote{In this
  text, \emph{epigenetics} will refer to the range of global modifications in
  gene expression that are not under control of the genetic code itself.  The
  modifiable and reversible nature of certain cancer programs can largely be
  explained by epigenetics.} changes in the neoplasm, and are constrained by the
ecological context in which the tumour is developing.

The prevailing mode of clonal evolution is through the gradual emergence of
selectively advantageous ``driver'' genetic injuries against a complex
background of mostly deleterious and selectively neutral ``passenger'' lesions.
Alternatively, or perhaps concurrently, another mode of tumoural evolution
contemplates the possibility of a few drastic events generating multiple lesions
at once across the genome.  These dramatic punctuated changes can be prompted by
an acute insult or a single catastrophic pan-genomic event---of which
chromothripsis, at the chromosome level, is an
example.\cite{stephens_massive_2011}\footnote{The argument of gradualism versus
  punctuated equilibrium (\citealp{gould_punctuated_1993}) is a longstanding
  debate in species evolution and is another example of how much our current
  conceptualization of cancer progression owes to the developments of the theory
  of evolution in the second half of the 20\textsuperscript{th} century.}

The time frame of somatic evolution is a function of the mutational rate of
neoplasms.  While events like chromothripsis and
kataegis\footnote{\emph{Kataegis}, a term derived from the ancient Greek word
  for ``thunder'', refers to a pattern of localized hypermutation identified in
  some cancer genomes (\mbox{\citealp{nik-zainal_mutational_2012}}).} have the
potential to provide nearly instant triggers for the onset of disease, the high
frequency of clinically covert \mbox{pre-maligant}
lesions\cite{sakr_frequency_1993} suggests that transformation of somatic cells
is a far more frequent event than suggested by incidence curves.  Furthermore, the
fact that the majority of cancers only become clinically relevant at old age is
a testament to both the prevalence of cancer-suppressing mechanisms as to the
relatively slow rates of mutational accretion in neoplasms.  Intriguingly, the
rate of epigenetic change has been reported to be orders of magnitude higher
than that of genetic change,\cite{siegmund_inferring_2009} but its role in
clonal evolution is not yet completely understood.

% The interaction of a tumour with its milieu is a complex, dynamic and
% reciprocal affair.

While the evolution of neoplasms is driven by their underlying genetic and
epigenetic lability, it is their tissue ecosystems that provide the adaptive
landscape for clonal fitness selection.\cite{greaves_clonal_2012} Systemic
regulators, such as hormones, growth factors, immune and inflammatory cells as
well as cytokines may conspire either to counteract or promote neoplastic
growth.\cite{bierie_tumour_2006,hanahan_hallmarks_2011} Architectural
constraints, in the form of physical compartments, basement membranes and
confined metastatic niches, restrict the growth of tumoural masses and set a
boundary for neoplastic microevolution.  But perhaps most striking is the
ability of tumours to remodel tissue micro-environments to their competitive
advantage---illustrated by the capacity of transformed cells to promote
neovascularization in response to anoxia or to incite malignant phenotypes in
their adjacent stromal cells.\cite{lathia_deadly_2011}

\subsection{The linear model of cancer progression}
\label{sec:models-cancer-progression}

At the phenotypic level, the course of neoplastic evolution is tagged along two
major defining axis: one concerning increasing proliferation rates and another
reporting the loss of morphological and physiological differentiation at the
cellular level.\cite{tarabichi_systems_2013}

% The paradox resides then in explaining how transformed genomes, torn by their
% karyotypic instability and ragged by waves of disrupting mutations, are able
% to cope with the formidable challenges evolved by multicellular organisms to
% control cellular defection.

In order to grow and become clinically conspicuous, neoplastic masses
must shut down \mbox{built-in} genetic programs acquired during the
establishment of multicellularity, to enforce compliance with societal
rules.  These include programs evolved to eradicate deviant
phenotypes, such as apoptosis, senescence and necrosis and a range of
\mbox{cell-to-cell} signaling programs designed to control and
suppress unwarranted growth.  Cancer cells must also persistently
evade the immune response of the host organism in its diverse
ecological contexts.

For neoplasms can also invade surrounding tissues and disseminate in the
organism, a feature responsible for most \mbox{cancer-related} deaths.  This
process is termed the \mbox{invasion-metastasis}
cascade\cite{valastyan_tumor_2011} and requires an array of well coordinated
genetic and epigenetic adaptations.  It can be schematized as a sequence of
discrete steps that starts with the local invasion of the surrounding tissues.
It then follows with the intravasion of cancer cells into nearby blood and
lymphatic vessels---and with their extravasion into the parenchyma of distant
tissues.  Eventually, new micrometastatic lesions are established with the
potential to develop into macroscopic tumours.

% , followed by the intravasion of cancer cells into nearby blood and lymphatic
% vessels.  Cancer cells in transit through the lymphatic and hematogenous
% systems then extravase these vessels into the parenchyma of distant tissues
% and eventually establish micrometastatic lesions---that may fully develop into
% macroscopic tumours.

To reconcile this increase in phenotypic resilience and plasticity of neoplastic
cells with the canonical multistep model of cancer
progression,\cite{land_cellular_1983,vogelstein_multistep_1993} carcinogenesis
is envisioned as a linear Darwinian evolutionary
process.\cite{merlo_cancer_2006,polyak_tumor_2014} According to this
view,\cite{podlaha_evolution_2012} inheritance, environmental factors and
spontaneous errors in \smallcaps{dna} replication cause mutations or
epimutations in critical caretaker genes, entailing genetic and epigenetic
instability.  This, in turn, promotes mutations or epimutations in specific
gatekeeper genes\footnote{Among \mbox{cancer-susceptibility} genes, the
  distinction between \emph{caretaker} and \emph{gatekeeper} genes is a subtle
  yet conceptually important one.  While the former are responsible for
  maintaining the integrity of the genome, the latter directly control cellular
  proliferation.  This is epitomized by the \mbox{breast-cancer-susceptibility}
  genes \smallcaps{\emph{brca1}} and \smallcaps{\emph{brca2}}, that can
  functionally play both roles at different points of breast cancer progression
  (\citealp{kinzler_gatekeepers_1997}).}  (oncogenes or tumour supressor genes),
triggering uncontrolled growth.  The ensuing genomic instability creates a
feedback loop that increases evolutionary and proliferation
rates.\cite{sieber_genomic_2003} This results in heritable variation in the form
of clonal diversity, upon which Darwinian selection operates.  Clones that
progressively acquire the biological hallmark capabilities of
cancer\footnote{These include sustaining proliferative signaling; evading growth
  suppressors; resisting cell death; enabling replicative immortality; inducing
  angiogenesis; and activating invasion and metastasis
  (\citealp{hanahan_hallmarks_2011}).} gain a competitive edge and become
prevalent, promoting a form of neoplastic progression that, at the macroscopic
level, is consistent with a multistep development.

% \footnote{Interestingly, this
%   modern consensus on tumoural evolution is thus not very distant from what
%   Peter Nowell proposed in his 1976 landmark paper
%   (\citealp{nowell_clonal_1976}) postulating that ``neosplasms arise from a
%   single cell of origin, and tumour progression results from acquired genetic
%   variability within the original clone allowing sequential selection of more
%   aggressive sublines.''}

This reasoning is supported by the observed \mbox{stepwise} progression of most
solid tumours.  For instance, the gradual evolution of colon cancer is well
documented.\cite{vogelstein_multistep_1993} Here, the first manifestations of
neoplasia occur in the colorectal epithelium, in the form of small benign
adenomas.  Such tumours are reasonably confined and are almost normal in their
intra- and intercellular organisation.  With time, adenomas start to grow
(proliferation) and become morphologically and physiologically disorganized
(dedifferentiation).  Eventually the tumour evolves into an aggressive neoplasia
(carcinoma), presumably because one of the cells in the adenoma has acquired a
sufficient number of mutations to drive the process of invasion and metastasis.

The concept of gatekeeper gene is central to this understanding.  Consider for
instance the \smallcaps{\emph{tp53}} gene, the first \mbox{tumour-suppressor}
gene to be identified in 1979.  This gene encodes for the p53 protein, a master
transcription factor with an overarching role in the maintenance of the
integrity of the genome.\cite{efeyan_p53:_2007} Under normal circumstances, p53
is functionally inactive due to its rapid degradation.  However, upon the
infliction of virtually any form of cellular stress, p53 degradation is halted,
and the protein gains full competence in transcriptional activation.  The
regulatory networks under its control are associated with several critical
mechanisms for cancer progression, namely apoptosis, \mbox{cell-cycle}
inhibition, genome stability and inhibition of
angiogenesis.\cite{vogelstein_surfing_2000} Unsurprisingly, about 50\% of all
human cancers have lost p53 expression or express an inactive mutant of the
protein.\cite{toledo_regulating_2006}

The evolutionary origins of the \smallcaps{\emph{tp53}} gene can be inferred
from modern-day descendants of both the single cell choanoflagellates and the
early multicellular sea anemone.  The function of the homolog to this ancestral
gene in the sea anemone is to protect the germ-line gametes from \smallcaps{dna}
damage.\cite{belyi_origins_2010} Over the course of the last billion years, this
function has not only been conserved, but enhanced through
pleiotropy\footnote{The ability of a single gene to influence multiple,
  seemingly unrelated phenotypic traits.} to set \smallcaps{\emph{tp53}} as a
major enforcer of somatic conformity in multicellular organisms.

Another recurrent oncogene with an established role in \mbox{cell-cycle}
progression and apoptosis, \smallcaps{\emph{myc}}, has had its evolutionary
roots traced back to at least 600 million years ago.\cite{hartl_stem_2010} The
fact that many cancer-susceptibility genes are ancient, highly conserved and may
have taken a role in the transition to
multicellularity\cite{srivastava_amphimedon_2010} has been interpreted as
evidence that they play a pivotal role in regulating the normal physiology of
the somatic cell.\cite{weinberg_oncogenes_1983,weinberg_biology_2013} Their
disruption during carcinogenesis could then symbolize the unshackling of the
transformed cell from its social bindings.

\subsection{Shortcomings of the linear model of cancer progression}
\label{sec:shortcomings-canonical-model}

This model describes the neoplastic cell as a unit set free to thrive
through uncontrolled replication in a hostile environment.  It becomes
a metaphor for the unicellular evolutionary stage, with cancer
lineages competing with each other and with normal cells for
survival.\cite{merlo_cancer_2006} The success of any one lineage is
dependent on its \mbox{step-wise} acquisition of cancer hallmarks
through Darwinian evolution.  However, this formulation does not fully
account for a number of observations.\cite{davies_cancer_2011}

First, it falls short to explain the high degree of cooperative organization
among cancer cells.  Increasingly, tumours are being recognized as ecosystems
with complex and dynamic interactions between neoplastic cells and their
microenvironment.\cite{polyak_co-evolution_2009} These heterotypic
reciprocations include the stimulation via paracrine signaling of normal stromal
cells to produce mitogenic signals; the required signaling to induce
neoangiogenesis; and the diverse cell-to-cell interactions during the
invasion-metastasis cascade.\cite{axelrod_evolution_2006} Thus, in order for
transformed cells to prosper, they must acquire a very specific minimal set of
communication skills from early on.  Rather than conceiving transformed cells
evolving such intricate adaptations independently via adaptive mutations, a more
prosaic explanation could involve a shift or modulation of the original set of
rules somatic cells use to engage with their partners.

Second, it fails to elucidate the quasi teleological way by which advanced
tumours consistently resort to the same set of sophisticated adaptations to
deceive their host organisms.\cite{hanahan_hallmarks_2011}
% At their end stage, malignant carcinomas could almost be taken for entities
% bestowed with a volition of their own, capable not only to evade the host
% organism's defenses, but to shrewdly overtake it.
This deceptively deliberate neoplastic progression is conventionally accounted
for by a Darwinian evolutionary process among competing lineages of rogue
cells.\cite{merlo_cancer_2006} However, some adaptations are particularly
challenging to frame in this view.  A notorious example is the progression
puzzle suggested by Bernards and Weinberg: how can mutations promoting invasion
and metastasis add a competitive edge to the primary tumour in its original
ecological context?\footnote{To solve the paradox, Bernards and Weinberg
  suggested a mechanistic model according to which ``the tendency to metastasize
  is largely determined by the identities of mutant alleles that are acquired
  relatively early during multistep tumourigenesis''
  (\citealp{bernards_progression_2002}).}  In addition, as the majority of
variants randomly arising on a genetic system are expected to be deleterious,
the odds of any neoplastic cell incrementally acquiring all the required
hallmark mutations before collapsing to a lethal one is exceptionally low.

Third, it doesn't fully integrate the role of genetic instability in cancer
progression.  Because advantageous mutations are rare, rogue cells are thought
to promote genomic instability in order to accelerate evolutionary rates and
increase the odds of, via Darwinian selection, produce a fully malignant
phenotype.\cite{sieber_genomic_2003} Yet, this mutational arms race among rogue
cell lineages to reach the jackpot of complete neoplastic competence is riddled
by a paradox: too few mutations in the mix and the cell won't escape genetic
controls; too many and it dies.  Especially troublesome are the
\mbox{pan-genomic} mutations that lead to gross structural changes, including
aberrant chromosomes and aneuploid cells.\footnote{``If you look at most solid
  tumours in adults, it looks like someone set off a bomb in the
  nucleus''---\emph{William C. Hahn}} It is thought-provoking to note that
the neoplastic cells with the most deranged genomes are precisely those with the
most competitive phenotypes.  Jarle Breivik proposed an elegant solution to this
inconsistency.\cite{breivik_evolutionary_2005} Rather than postulating genomic
instability as a pre-requisite to cancer progression, he refashioned it as a
\mbox{by-product} of the lack of competitive fitness of reparing-phenotypes in
the tumour environment.\footnote{``Don't stop for repairs in a war zone'' is the
  metaphor used by Breivik to explain why a mutagenic environment would increase
  the fitness of the non-repairing phenotype.  % This conceptual
  % reformulation could have implications on other central controversies in cancer
  % research---as discussed bellow.
}

A tentative proposition to address these inconsistencies has been put forward by
Davies and Lineweaver.\cite{davies_cancer_2011} Instead of modeling the
transformed cell as a free-agent requiring the cumulative \emph{acquisition} of
enabling properties to obtain malignant status, it seeks to explain hallmark
adaptations as the reenactment of atavic genetic systems already embedded in the
genomes of somatic cells.  According to this view, the disruption of high-order
gatekeeper genes drives the neoplastic cell to reconfigure its regulatory
networks around a \emph{pre-existing} toolkit of primitive adaptations
reminiscent of those evolved during early transition to multicellularity.
Although its original formulation has been either ignored\footnote{At the date
  of this writing, Google Scholar reported a total of 36 citations of the
  original article.} or thoroughly
dismissed,\cite{pettit_cancer_2012,myers_aaargh!_2012} this idea could provide a
basis for a reevaluation of some core features of cancer.

Consider the Warburg effect\cite{warburg_origin_1956} for instance.  The
majority of cancer cells favour a metabolism based on anaerobic glycolysis
followed by lactic acid fermentation in the cytoplasm.  Because glycolysis is
far less efficient than oxidative respiration for \smallcaps{atp} production,
this metabolic shift has been implied to be an essential adaptation to cancer
progression---perhaps in response to intermittent hypoxia in malignant
lesions.\cite{gatenby_why_2004} A more economical interpretation could be that
transformed cells, without gatekeeper genes to enforce the proper regulation of
the high-performing metabolic mode of somatic cells, are now forced to fall back
to a more basic, yet dependable, metabolic outlet for energy production.  While
the former interpretation taxes neoplastic progression with another requirement
and begs for a rationale for the selective fitness of anaerobic metabolism, the
latter simply conceives the cancer cell as a system seeking a dynamic
equilibrium in a novel and unstable environment.

More recently, another key concept in cancer research has been the cancer stem
cell.  Tissue-specific stem cells have been identified at the top of the
differentiation hierarchy of many organs.  These stem cells are functionally
defined by their long-term self-renewal capacity and their ability to
differentiate into one or more tissue lineages.  This hierarchical organization
of tissue differentiation has been co-opted to explain tumour growth and
heterogeneity, with cancer-specific stem cells responsible for the maintenance
and growth of tumours.  Cancer stem cells (\smallcaps{csc}s), or tumour initiating
cells, are operationally defined by their ability to re-form the parental tumour
on transplantation into immunodeficient mice.  They have been isolated from a
range of solid tumours, such as breast cancer, brain tumours, colorectal cancer,
and others.\cite{beck_unravelling_2013} Fittingly, this model complies with the
requirements of clonal evolution, as tumour progression can be explained by
derivatives of \smallcaps{csc}s bearing different mutational signatures competing
with each other.  This interpretation suggests that all cancer cells in a tumour
share a unique genetic and epigenetic history, in accordance with a linear
progression.  However, the reported coexistence of multiple genetic clones
during acute lymphocytic leukemia progression suggests a more dynamic and
modular clonal architecture
instead.\cite{anderson_genetic_2011,notta_evolution_2011} % Evidence of
% intra-tumoural heterogeneity does not necessarily inform the underlying model
% of cancer progression.
This evidence, together with the fact that phenotypic conversion also occurs
among non-hierarchically organized tumour cells in
melanoma,\cite{quintana_phenotypic_2010} indicates that the \smallcaps{csc}
phenotype may be a transient response to the selective pressures of the tumour's
milieu and of the stochastic events that govern its internal
homeodynamics.\cite{visvader_cancer_2012,aktipis_life_2013} The \smallcaps{csc}
would then be, rather than a deterministic, a dynamically reversible phenotype;
and represent, instead of qualitative, a quantitative
modulation.\cite{maenhaut_cancer_2010,tarabichi_systems_2013}

% Stem cells play a crucial role not only during development, but during tissue
% homeostasis and repair as well.  These cells are operationally defined by
% their ``capacity for self-renewal, the potential to develop into any cell in
% the overall tumour population, and the proliferative ability to drive
% continued expansion of the population of malignant cells.''
% (\citealp{jordan_cancer_2006}) While the idea of a unique, genetically
% homogeneous population of determined cancer stem cells is still
% contested,\cite{marotta_cancer_2009} the existence of a particular class of
% tumour propagating cells within neoplastic masses is well accepted.  Here, the
% crux of the matter concerns the ontology of the \smallcaps{csc} phenotype.  In
% short, \smallcaps{csc}s have been proposed to arise from mutations in either
% derivatives of developing stem or progenitor lineages, or from mutations in
% differentiated cells that acquire stem-like
% attributes.\cite{wicha_cancer_2006,lobo_biology_2007} As niche-specific stem
% cells are subjected to higher turnover rates to seed their respective tissues,
% they become prime targets for neoplastic-inducing replication defects.  The
% potential of a seeder neoplastic cell at the apex of the tumoural clonal
% hierarchy is luring both from the diagnostic and treatment perspective.
% Moreover, this conceptualization seems especially fitting for the established
% linear Darwinian clonal progression model.  However, there is evidence that
% these cells do not constitute one homogeneous population.

Another example of the phenotypic modularity of the neoplastic cell is the
epithelial-mesenchymal transition program (\smallcaps{emt})---and its reverse
process, the mesenchymal-epithelial transition (\smallcaps{met}).  As with the
\smallcaps{csc} phenotype, this mechanism is borrowed from embryogenesis, where
it takes part in gastrulation, neural crest formation, heart valve formation,
palatogenesis and myogenesis.\cite{thiery_epithelial-mesenchymal_2009} This
comprehensive genetic program globally shifts the physiology and morphology of
the cell between an epithelial phenotype (characterized by a well defined
polarity, tight junction of the cells and a their binding to a basal lamina) and
a mesenchymal phenotype (characterized by a lack of polarity, spindle-shape
morphology and loose cell-to-cell interaction).\cite{thiery_complex_2006} Under
normal physiological circumstances, the dynamics of the \smallcaps{emt-met}
program are under strict and orderly genetic control, even if it can be
re-enacted in a post tissue differentiation context---for instance in response
to injury.  In pathological conditions, the unwarranted activation of the
\smallcaps{emt-met} program can cause organ fibrosis.  But most importantly, it
provides cancer cells with an outlet to break free from the primary tumour and
embark the invasion-metastasis cascade under the cover of the mesenchymal
phenotype.  Conversely, the colonization of new micrometastatic niche is
facilitated by a transition back to an epithelial phenotype.  This constitutes
the most thorough illustration of how neoplastic cells can recruit built-in
genetic apparatuses to deploy complex adaptations.

% Thus, cancer cell lineages could evolve by exploiting the realm of new
% homeodynamic equilibriums made available by the breaching of higher-order
% somatic regulatory systems.  Because these regulatory circuitries are kept in
% check by gatekeeper genes, this would entail that the number of required
% breaking points for the potential emergence of full neoplastic competence is
% considerably smaller than anticipated. The acquisition of the hallmark
% adaptations could simply be the outcome of a stochastic process of
% readjustments of regulatory networks of the cell.

% Even Weinberg flirted with such possibility: ``Maybe the information for
% inducing cancer was already in the normal cell genome, waiting to be
% unmasked.''\footnote{\citealp{weinberg_biology_2013}, \emph{p} 79} Could this
% idea be further developed?  Consider the main criticism to the hypothesis of
% Davies and Lineweaver: their invocation of an \emph{atavistic} genetic system,
% in itself largely responsible for the cancer phenotype, already packed in our
% genomes and ready to be unleashed with the disruption of gatekeeper
% genes.\footnote{``In short, proto-metazoans, which we dub Metazoans 1.0, were
% tumor-like neoplasms.'' (\citealp{davies_cancer_2011})} This proposition is
% perhaps too easy to caricature and to dismiss, but it does share one aspect in
% common with the linear Darwinian model for tumourigenesis: both seek to
% explain cancer almost exclusively through its genetic determinants.  The two
% theories differ, however, in the nature of the qualitative changes required
% for tumor progression.  While clonal evolution posits for an \emph{ab initio}
% generation of genetic determinants for cancer progression, the Metazoan theory
% short-circuits this precondition by allocating the required determinants in
% the genome of the somatic cell.  None of these perspectives fully addresses
% the perplexities of cancer; both seem to ignore the epigenetic determinants
% for cancer progression.

% In reality, the predicament of the neoplastic cell might be much more complex.
% Evolution has concocted a vast collection of modular genetic programs to
% provide the somatic cell with adaptive dynamic physiological ranges within
% which to carry its activities.  In order for the orderly integration of the
% somatic cell at the tissue and organism level, such physiological ranges must
% be under a tight genetic control of a higher-order.  The breakdown of this
% level of control leaves the somatic cell hapless on how to keep its metabolism
% balanced and free to stochastically reconfigure its genetic regulation.  This
% may lead to the co-optation of genetic programs via natural selection as they
% increase the fitness of the transformed cell in particular ecological
% contexts.  Because these programs usually end up serving an adaptive role
% other than the one they were evolved to provide, perhaps cancer adaptations
% ought to be better described as
% \emph{exaptations}\cite{gould_exaptation_1982}---a concept that leaves at rest
% the teleological interpretation of cancer hallmarks.

% But let us speculate a bit further.  The evolutionary success of the
% neoplastic cell is only bounded by its ability to self-replicate.
% Consequently, with the exception of the genetic systems that support the cell
% cycle and the remaining context-specific hallmark adaptations, the vast
% majority of the transformed genome could be at the mercy of disruption through
% genetic instability.  This alone could explain the second major
% axis\footnote{The first being increased proliferation.} of cancer progression:
% the gradual loss of morphological and physiological differentiation of
% neoplasms as they evolve through the ranks of malignancy.  But there is
% another neglected driver of cancer evolution, one that could prove to be the
% hidden iceberg upholding the shapeshifting nature of cancer.

% Very little is know about the non-genetic, or epigenetic, determinants of
% cellular metabolism.  Chromatin remodeling and \smallcaps{dna} methylation are
% known to modulate to a considerable degree patterns of gene expression;
% \smallcaps{rna} transcripts and their encoded proteins may retroactively
% modulate, direct or indirectly, the activity of certain
% genes;\footnote{\smallcaps{rna} mollecules may even directly spread directly
% to other cells or nuclei by diffusion.} a variety of less known species of
% \smallcaps{rna} are animated with catalytic activity of their own and their
% transcriptional output has been historically
% underestimated.\cite{huttenhofer_principles_2006,ptashne_use_2007} This family
% of non-protein-coding \smallcaps{rna}s are thought to be relics of a
% primordial ``\smallcaps{rna} world'', in which \smallcaps{rna} served both as
% the carrier of genetic information and as the catalytic agent.  If cancer
% cells are able to resourcefully and adaptively tap on the genetic resources of
% the somatic cell to shape their evolution, why would they neglect the rich
% diversity of epigenetic tools at their disposal?

% Perhaps most provocatively, as these epigenetic systems are not necessarily
% constrained by the dynamics of Darwinian evolution,\footnote{And could indeed
% abide by the rules of Lamarckian evolution.} their evolutionary rates could be
% orders of magnitude higher than those observed in genetic systems---even those
% riddled by genetic instability.  This could provide a potent mechanism driving
% cancer evolution, one that might potentially account for the extraordinary
% degree of resilience of advanced cancers.

% These considerations could have profound implications in the way cancer
% research, diagnostic, prognostic and treatment are conducted.

% Unlocking these mysteries may well require a lot of ingenuity, but without
% doing so it will be hard to elucidate the intricate nature of life---and that
% of its dark horse, cancer.

% Genetic pleiotropy.  Answer: rather than a multistep progression driven by
% linear accretion of driver mutations supporting the discrete and independent
% acquisition of malignancy hallmarks, a global transcriptional shift
% encompassing the activation of ldots{} Concept of Pleiotropy.  Concept of
% \emph{exaptations}.

% \smallcaps{\emph{tp53}} and \smallcaps{\emph{myc}}.

% The resilience, plasticity and insidiousness of malignant neoplastic cells is
% eerily reminiscent of the \ldots{}

% Final argument about Darwinian evolution, hallmarks of cancer, and eventual
% atavic genetic circuitry engaged once system gateway controlers, such as p53,
% fail.  Frame cancer as an evolutionary problem To discuss: Darwinian evolution
% of cancers, multistep model of cancer progression, hallmarks of cancer,
% critics of the hallmarks, oncogenes as evolutionary gatekeepers, examples
% centered around \smallcaps{\emph{tp53}} and \smallcaps{\emph{tp53}} oncogenes,
% integration of cancer programs, Warburg effect, a progressin puzzle: example
% with \smallcaps{EMT} program, metastasis is a highly inefficient process,
% similarities between cancer phenotype and early developmental stages, concept
% of cancer stem cell, concept of \emph{exaptation}.  Example: Evolutionary
% history of the retinoblstoma gene from rchea to eukaria.  Implications for
% both the understanding of multicellularity from an evolutionary perspective as
% well as for the understanding of the biology of cancer.

% However these complex considerations on the biology

% Cancer dynamics are bounded by genetic and epigenetic
% alterations.\footnote{From Nowell, 1976: reversibility of transformation in
% certain culture systems suggests that cancer initiation could result from
% altered gene expression rather than structural mutation}

% Cells of the neoplastic lineage are defined by their ability to sustain
% chronic proliferation, irrespectively of the social cues conveyed by its
% tissue context.  At the morphological level, however, the most conspicuous
% feature of cancer has to be the diversity of shapes and forms these
% proliferating masses can take when departing from the neatly organized
% architectural tissue types they arise from.  An integrative framework to
% apprehend this remarkable phenotypic plasticity has been proposed by Hanahan
% and Weinberg,\cite{hanahan_hallmarks_2000,hanahan_hallmarks_2011} who proposed
% six essential and complementary capabilities for tumour growth and metastatic
% dissemination.  These include self-suficiency in growth signals, insensitivity
% to growth-inhibitory signals, evasion of programmed cell-death, limitless
% replicative potential, sustained angiogenesis, and tissue invasion and
% metastasis.

% Cellular transformation, the process through which the neoplastic phenotype
% arises, is caused by genetic and epigenetic alterations in somatic cells.
% Under regular circumstances, most somatic cells exhibiting behaviours beyond
% physiological ranges end up being singled out and targeted for removal, either
% by eliciting an immune response or by triggering self-induced cellular death
% (\emph{note about apoptosis here}).  In order for a neoplasm to become
% clinically relevant, it has thus to acquire a number of alterations that
% consign it with the capacity to evade its host organisms' regulatory control
% mechanisms against unicellular defection.

% an operating outside of normal physiological levels cells abnormal behaviour
% exhibited by most early neoplastic cells may be sufficient for them to be
% recognized and targeted for removal, either by eliciting an immune response or
% by triggering self-induced cellular death (\emph{note about apoptosis here}).

% Clonality; multistep model for the nature of cancer;

% Specific genes, termed oncogenes, have the potential to induce transformation
% when disrupted in particular circumstances.  Alterations in nearly 500 of such
% genes have been linked to cancer initiation and
% progression.\cite{forbes_catalogue_2008} The multistep model for the nature of
% cancer posits that several such alterations are cumulatively required in order
% to initiate tumourigenesis and to evolve increasingly more aggressive and
% invasive tumour phenotypes.\cite{vogelstein_multistep_1993}

% The fundamental and defining characteristic of the cancer cell is, arguably,
% it's ability to sustain chronic proliferation.

% The number and patterns of somatic alterations vary dramatically across cancer
% types.  At one extreme, childhood medulloblastomas can harbour fewer than ten
% genomic alterations, whereas over \SI{50000} somatic changes have been
% observed in primary lung adenocarcinoma samples.

% In biological systems the instanciation of information is \smallcaps{DNA}.

% a phenotype eerily reminiscent of that seen in loosely cooperative unicellular
% life forms, at a time when multicellularity was still being attempted at.

% From the prologue of 'The Emperor of All Maladies': Malignant and normal
% growth are so genetically intertwined that unbraiding the two might be one of
% the most significant scientific challenges faced by our species. Cancer is
% built into our genomes: the genes that unmoor normal cell division are not
% foreign to our bodies, but rather mutated, distorted versions of the very
% genes that perform vital cellular functions.  individual alterations in
% oncogenes are seen as necessary but not sufficient to give rise to cancer.
% Considered to arise sequentially and to give rise to the progressively more
% aggressive and invasive phenotypes observed during tumourigenesis.

% which these control mechanisms are compromised lead by mutations on specific
% genes which these mechanisms are hindered is through the accumulation of
% somatic mutations of oncogenes


% Cells of normal tissues collectively control their growth rate by regulating
% the production and release of paracrine \mbox{growth-promoting} signals that
% direct entry and progression through the cell \mbox{growth-and-division}
% cycle.  A first requirement for the acquisition of the neoplastic identity
% must then be the ability to generate endogenous mitogenic signals that result
% in autocrine proliferative stimulation.\footnote{This endeavour can also be
% achieved through the emission of signals that stimulate surrounding normal
% cells to feed cancer cells back with growth factors
% (citealp{Cheng-2008,Bhowmick-2004}); specific somatic mutations that
% constitutively activate pathways usually triggered by activated growth factor
% repectors; or through disruptions of \mbox{negative-feedback} mechanisms that
% attenuate proliferative signaling.}  In addition, the cancer cell must also
% become insensitive to social cues designed to control unrestricted cell
% proliferation.

% From jcb-weinberg-1983.pdf:

% This leads to the realization that these [onco]genes are of extremely ancient
% lineage---their precursors were already present in similar form in the
% primitive metazoans taht served as common ancestors to chordates and
% arthropodes.  Such conservation indicates that these genes have served vital,
% indispensable function in normal cellular and organismic physiology, and that
% their role in carcinogenesis represents only an unusual and aberrant diversion
% from their usual functions.

% by caused by steered by the engagement of genetic routines of the form of an
% atavistic state engagement of genetic routines active at an earlier stage of
% development that are inappropriately reactivated in the mature organism as a
% result of some sort of insult of damage.  Cancer is an atavistic state of
% multicellular life.  Cells relieved of the molecular constrains that
% subordinate them to societal discipline.

% Causing cancer cells' metabolism to default to more fundamental modes of
% functioning not unlike those conceivably typified by loosely societal cellular
% sorts.

% These six hallmarks of cancer---distinctive and complementary capabilities
% that enable tumour growth and metastatic dissemination

% Further evidence that supports our theory comes from experiments in which the
% nuclei of egg cells are replaced with cancer-cell nuclei.  Astonishingly,
% embryos start to develop normally.  But abnormalities eventually appear, at
% earlier stages when the cancer is more malignant (advanced). This inverse
% correlation of cancer stage with embryo stage is consistent with our theory.

% Cancer follows a well-defined progression of within a host organism of
% accelerating growth (proliferation), mobility, spread and colonization.

% To better understand cancer, including its place in the great sweep of
% evolutionary history.

% ideias to discuss:

% causes of cancer

% field cancerization (tarabichi)

% 1--For multicellularity to become a viable strategy, a number of strategies
% must have evolved in order to maximize cooperation and minimize conflict
% between individual cells of an organism.

% 2--One such strategy is the imposition of a genetic bottleneck per generation
% in the course of the reproductive cycle of multicellular organisms in order to
% ensure clonality among constituent cells of an organism.

% 3--As cooperation creates new levels of fitness, it creates the opportunity
% for conflict between levels as deleterious mutants arise and spread within the
% group.  Fundamental to the emergence of a new higher-level unit is the
% mediation of conflict among lower-level units in favor of the higher-level
% unit.

% 4--Michod et al. (2003) lists a number of conflict-mediating adaptations that
% can be useful in engendering cooperation among physically associated cells.
% These include the existence of a germ line, the tendency for unnecessary or
% even defecting cells to undergo apoptosis, the potential for self policing,
% display of relatively low mutation rates, canalization of growth (of which one
% mechanism is "determinate growth"), etc.

% 5--Take the \emph{p53} gene as example.  \emph{p53} has likely been the most
% studied gene for over a decade.

% cancer dormancy is a also perplexing

% pleiotropy of the key oncogenic gatekeepers

% quote the process of ``exaptation'' (Gould & Vrba, 1982) the role of
% epigenetics

% The conventional argument to explain the deployment of this \mbox{swiss-knife}
% armory of solutions is to invoke Darwinian evolution between \mbox
% {sub-clones} of the original neoplasm.\cite{Merlo-2006} According to this
% hypothesis, the evolution of the required survival traits for the success of
% the tumour would be attributed mostly to random mutations and the trial and
% error of normal Darwinian evolution.  Rapid mutation rate within tumours +
% strong selective pressure as organism ``fights back'' or patient undergoes
% chemotherapy.

% This understanding of cancer as a feature of a particular evolutionary
% mechanism

% Rewrite discussion on multicellularity in the life section according to the
% lines discussed here:
% http://www.biologyaspoetry.com/textbooks/microbes_and_evolution/greater_size_problems_in_multicellularity.html

% NOTES from tmm-floor-2012.pdf: hallmarks of cancer: of all cancer cells, all
% the time?

% The hallmarks of cancer 1-self-sufficiency in growth signals 2-insensitivity
% in anti-growth signals 3-evasion of apoptosis 4-limitless replicative
% potential 5-sustained angiogenesis 6-tissue invasion 7-metastasis 8-metabolic
% reprogramming 9-evasion of the immune system

% Strangely, one fundamental characteristic of cancer cells, the loss of
% differentiation, was not considered a distinct hallmark.  This characteristic
% is essential because it is a primary difference between benign and malignant
% tumours (e.g., autonomous adenomas as opposed to carcinoma of the thyroid).
% It is supported by a measurable expression program switch that affects all
% gene categories.  We argue that this should be another, if not the major,
% ``hallmark'' of a cancerous cell, and it has been shown to be a robust
% diagnostic index for thyroid tumours.

% new hallmarks of cancer must be able

% cancer progression is characterized by a succession of oncogenic events

% The hallmarks of cancer (cel-hanahan-2011) The hallmarks constitute an
% organizing principle for rationalizing the complexities of neoplastic disease.
% They include: 1--sustainment of proliferative signalling 2--evasion of growth
% supressors 3--resistance to cell death 4--enabling of replicative immortality
% 5--inducing of angiogenesis 6--activation of invasion and metastasis

% The genomes of nearly all healthy human cells, containing the entirety of an
% individual's inherited information, evidently come pre-loaded with a ``cancer
% sub-routine'' that is normally idle but can be triggered into action by a wide
% variety of insults, such as chemicals, radiation and inflammation.

% Breivik 2005

% Cancer incidence increases with age, and in countries with high life
% expectancy, aproximately 40\% of the population experiences some kind of
% cancer in their lifetime.  Adiotionally, a significant proportion of the
% autopsies reveal undiagnosed maliganacies, and virtually everybody carry some
% kind of in situ carcinomas after the age of 50.

% Clinical definition of cancer. A vast collection of diseases (almost 100,
% according to http://medical-dictionary.thefreedictionary.com/Cancer).

\clearpage

\subsection{Cancer epidemiology}
\label{cancer-epidemiology}

\newthought{cancer is} a leading cause of death worldwide, accounting
for 8.2 million deaths in 2012.  From the clinical point of view, cancer is a
general designation for a group of more than 100 diseases.  Lung, liver,
stomach, colorectal and breast cancers are responsible for the majority of
cancer deaths each year.  The most frequent types of cancer, as well as their
incidence, differ between women and men (Figure~\ref{fig:globocan}).

\begin{figure}[ht]
  \includegraphics{globocan-13Feb2015.pdf}
  \caption[Global estimates of cancer incidence and mortality by
  sex][6pt]{Global estimates of cancer incidence and mortality by sex.
    \mbox{Age-standardized} rate per \SI{100000} population (2012).  Source:
    \mbox{\smallcaps{globocan}} (\citealp{ferlay_globocan_2014}).}
  \label{fig:globocan}
\end{figure}

The list of factors involved in the causation of cancer is wide and diverse.
Heritable genetic susceptibility, in the form of highly penetrant, dominant
allelic variants, could account for \numrange{2}{5}\% of fatal cancers.
Environmental factors, including smoking, alcohol consumption, dietary habits
and infectious agents of viral (e.g., \smallcaps{hpv}) or bacterial (e.g.,
\emph{Helicobacter pylori}) origin, are responsible for varying degrees of
cancer susceptibility.\cite{cassidy_oxford_2010} According to the estimates of
the American Cancer Society, approximately 40\% of cancer deaths in 1998 were
due to tobacco and excessive alcohol use.  The 1996 Harvard Report on Cancer
Prevention concluded that over 90\% of malignant melanoma is attributable to
solar radiation.  While other exposures, such as radiation and environmental
pollutants, could account for up to 5\% of the cancer burden, few causal links
with other potential carcinogens have been firmly established.

The medical act of assessing the degree of development and spreading of the
neoplastic disease is called staging.  Correct cancer staging is critical
because treatment (in the form of pre-operative therapy and/or adjuvant therapy)
and disease prognosis are based on this evaluation.  Staging systems are
specific for each type of cancer and are usually sanctioned by international
organizations, like the \smallcaps{uicc} and the
\smallcaps{ajcc}.\footnote{Respectively, the Union for International Cancer
  Control and the American Joint Committee on Cancer.}  The most prevalent
staging system mirrors cancer progression and classifies solid cancers according
to their surgical tractability: Stage 0 represents a tumour confined \emph{in
  situ}; stage \smallcaps{i}, a localized, still surgically removable tumour;
stage \smallcaps{ii} and stage \smallcaps{iii} describe locally advanced cancers
(the specifics depending on the type of cancer being staged); and stage
\smallcaps{iv} marks a metastatic cancer, spread to other organs throughout the
body.\cite{greene_ajcc_2002} Another staging system, the \smallcaps{tnm}
classification of malignant tumours, also applies to the majority of solid
cancers. It relies on the size and extension of the primary tumour; on its
lymphatic involvement; and on the presence of metastases to classify cancer
malignancy and to inform treatment decisions.\cite{denoix_enquete_1946} For
breast cancer, the Nottingham modification of the Bloom-Richardson scale is most
commonly used. This grading scale classifies each cancer in a scale from 1 to
3---each describing, respectively, low-, intermediate- and high-grade
neoplasias.\footnote{This classification system positions cancer along a
  differentiation axis: from low-grade, well differentiated tumours; to
  high-grade, poorly differentiated ones.}

\subsection{Cancer research}
\label{cancer-research}

Most of these classification systems have been implemented during the second
half of the last century and reflect a compromise between the need to find
sensible and universal guidelines for cancer treatment and our unfolding
understanding of the disease.  Medical cancer research has come a long way since
the times when nearly every disease was attributed to the workings of some
invisible force such as bile, miasmas or bad humours (Table~\ref{tab:200years}).

\begin{table}[ht]
  \small
  \centering
  % \fontfamily{ppl}%\selectfont
  \begin{tabular}{lm{6.5cm}m{1.5cm}}
    \toprule
    Year & Discovery or Event                                             & Relative \mbox{Survival Rate} \\
    \midrule
    1863 & Cellular origin of cancer (Virchow)                            &                               \\
    1889 & Seed-and-soil hypothesis (Paget)                               &                               \\
    1912 & Transplantable rodent tumours                                  &                               \\
    1914 & Chromosomal mutations in cancer (Boveri)                       &                               \\
    1928 & Head and neck cancer cured                                     &                               \\
         & by fractionated radiotherapy                                   &                               \\
    1950 & Experimental evidence links lung cancer                        &                               \\
         & to smoking                                                     &                               \\
    1953 & Report on structure of \smallcaps{DNA}                         & 35\%                          \\
    1961 & Breaking of the genetic code                                   &                               \\
    1967 & Proof of principle: drug cures for Hodgkin’s disease           &                               \\
         & and childhood leukemia                                         &                               \\
    1974 & Adjuvant chemotherapy for breast cancer                        &                               \\
    1976 & Cellular origin of retroviral oncogenes                        & 50\%                          \\
         & Link discovered between \smallcaps{hpv} and cervical cancer    &                               \\
    1979 & Epidermal growth factor and receptor                           &                               \\
    1981 & Suppression of tumour growth by p53                            &                               \\
    1984 & G proteins and cell signaling                                  &                               \\
    1985 & First effective cancer immunotherapy                           &                               \\
         & with interleukin-2                                             &                               \\
    1986 & Retinoblastoma gene                                            &                               \\
    1990 & First decrease in cancer incidence and mortality               &                               \\
    1991 & Association between mutation in \emph{\smallcaps{APC}} gene    &                               \\
         & and colorectal cancer                                          &                               \\
    1994 & Association between \emph{\smallcaps{BRCA1}} and breast cancer &                               \\
    1996 & Proof of principle: targeted therapy                           &                               \\
         & with imatinib for \smallcaps{cml}                              &                               \\
    1998 & Tamoxifen reduces breast-cancer incidence                      &                               \\
    2000 & Sequencing of the human genome                                 &                               \\
         & \smallcaps{fda} approves \smallcaps{hpv} vaccine to prevent    &                               \\
         & cervical cancer                                                &                               \\
    2002 & Epigenetics in cancer                                          &                               \\
         & micro \smallcaps{RNA}s in cancer                               &                               \\
    2005 & First decrease in total number of                              & 68\%                          \\
         & deaths from cancer                                             &                               \\
    2006 & Tumour-stromal interaction                                     &                               \\
    \bottomrule
  \end{tabular}
  \caption[Landmarks of 200 years of cancer research]{Some of the landmarks in the last 200 years of cancer research
    (adapted from \citealp{devita_two_2012}).}
  \label{tab:200years}
  % \zsavepos{pos:normaltab}
\end{table}

It was in 1863 that Rudolph Virchow, through the lens of a microscope, deduced
the cellular origin of cancer.\cite{virchow_cellular_1863} At once, cancer was
being recasted as the quintessential disease of hyperplasia and rescued back to
the somatic realm.\footnote{\emph{Omnis cellula e cellula}---every cell
  originates from a cell alike---, is the epigram popularized by Virchow,
  stating a shift from the tenet of spontaneous generation that dominated the
  19\textsuperscript{th} century school of thought concerning cancer's origins.
  Noting that this form of cellular multiplication was fundamentally novel and
  inexplicable, he coined it \emph{neo}plasia.}  Albeit localized in its origin,
cancer was nonetheless perceived as a humoural disease and a systemic illness.
The supposition that cancer spreads in a centrifugal fashion from the primary
tumour to adjacent structures was the foundation for William Halsted to
introduce, in 1894, the radical mastectomy for breast cancer.\footnote{This
  surgical procedure demands that the breast, the underlying chest muscles and
  the lymph nodes of the axila be removed (\citealp{halsted_i._1894}).  From
  1895 to the mid-1970's, about 90\% of the women treated for breast cancer in
  the \smallcaps{usa} underwent radical mastectomy.}

During the 19\textsuperscript{th} century, surgery was the only known way to
treat cancer.  The first example of a cancer cure by surgery happened in 1809
with the removal of an ovarian tumour without anesthesia.  Surgery protocols
were subsequently enhanced with the use of anesthesia, first reported in
1846,\cite{warren_inhalation_1846} and the introduction of antisepsis, in
1867.\cite{lister_antiseptic_1867} Halsted's radical mastectomy advocated the
\emph{en bloc} resection of the surrounding tissue to remove all cancer cells.
As cancers kept on relapsing locally after surgery, Halsted reasoned that more
and more tissue had to be extirpated in order to root out the last of malignant
cells.  This pushed radical mastectomy into the ``super-radical'' and then into
the ``\mbox{ultra-radical}.''\footnote{This was an extraordinarily morbid,
  disfiguring procedure in which surgeons removed the breast, the pectoral
  muscles, the axillary nodes, the chest wall, and occasionally the ribs, parts
  of the sternum, the clavicle, and the lymph nodes inside the chest
  (\citealp{mukherjee_emperor_2011}).}  By the turn of the century, \emph{en
  bloc} resection became known as ``the cancer operation'' and turned into the
standard approach for the removal of all other cancers.

This surgical tradition came to be challenged in 1968 by Bernard Fisher, a
surgeon from Philadelphia.  Against the prevailing consensus, Fisher conducted a
series of clinical trials in the 1960's to compare the performance of radical
mastectomy with localized surgery (a ``lumpectomy''), supplemented with
radiation.  The results of these trials showed that \emph{en bloc} surgery was
no more effective to treat early-stage breast cancer than the combination of
surgical extraction of the tumour mass and radiation
therapy.\cite{fisher_five-year_1985,fisher_ten-year_1985} Radical mastectomy was
a fallacy that pushed too far when the tumour was localized, and not enough when
the cancer had turned metastatic.

In 1928, Henry Coutard, a radiologist from the Institut Curie in Paris, showed
that fractioned radiation treatments could be used to cure head and neck
cancers.\cite{coutard_roentgen_1932} In those days, the treatment was called
Roentgen therapy, after Wilhelm Röntgen, a lecturer at the Würzburg Institute in
Germany.  While working with an electron tube in 1895, Röntgen discovered a form
of radiant energy he called \smallcaps{x}-rays.  The discovery of radium in 1898
by Pierre and Marie Curie further opened the door to the use of radiation to
kill cancer by ``burning'' it.\footnote{It was not just cancer cancer cells that
  were being burned.  Marie Curie died of leukemia in July 1934.  Emil Grubbe,
  the first American to use \smallcaps{X}-rays in the treatment of cancer, had
  his fingers amputated to remove necrotic and gangrenous bones and his face cut
  up in repeated operations to remove \mbox{radiation-induced} tumours and
  \mbox{pre-malignant} warts. He died at the age of eighty-five to metastatic
  cancer (\citealp{mukherjee_emperor_2011}).}  All throughout the first half of
the 20\textsuperscript{th} century, this use of radiation therapy mirrored the
advent of the atomic age, with ``cyclotrons'', ``supervoltage rays'', ``neutron
beams'' and ``millions of tiny bullets of energy'' all harnessed to eradicate
what the surgeon's knife could not reach.\cite{mukherjee_emperor_2011} The
pinnacle of this era came in 1968, in Stanford, when Henry Kaplan demonstrated,
with what was to become one of the first controlled medical trials in oncology,
that \mbox{Gamma-knife} radiosurgery could significantly increase the survival
rate of early-stage Hodgkin's disease.\cite{kaplan_clinical_1968}

These advances in surgery and radiotherapy were most beneficial to patients with
early, localized forms of cancer.  Metastatic disease, on the other hand,
requires a systemic cure.  Furthermore, not all tumours respond equally to
generic treatments, underscoring a fundamental aspect of cancer's biology---its
diversity.  When, a hundred years ago, the German chemist Paul Ehrlich
launched the first systematic attempt to find chemical substances with specific
affinity to malignant cells in order to poison them, he was in essence devising
a new form of treating cancer. He called it chemotherapy.\footnote{``Give up all
  hope ye who enter''---was the reading on Ehrlich's lab door
  (\citealp{devita_history_2008}).}

To maximize the efficiency of anti-cancer drug screening, two conceptual
advances had to be achieved.  First, a cancer model was needed in which the
impact of therapy could be effectively quantified.  Second, a form of
circumventing the ethical limitations of testing human patients had to be found.
The first issue was addressed by Sydney Farber at the Children's Hospital in
Boston when he turned his attention to childhood's leukemia in the
1940's.\cite{devita_history_2008} In cancer medicine, leukemia is a particularly
appealing model because it offers the possibility to actually count the number
of cancer cells flowing in the blood---and thus measure the response to the
treatment.  The recruitment of murine models to test drug response in grafted
tumours provided the second piece of the puzzle needed to spur the hunt for
anti-cancer drugs.\cite{clowes_further_1905}

In 1955, a national screening effort for the development and testing of
chemotherapeutic drugs was launched in the \smallcaps{usa}.  This lead to the
concoction of highly toxic cocktails of drugs aimed at ``maximal, intermittent,
intensive, upfront'' chemotherapy to vanquish the
disease.\cite{frei_curative_1985} One of such high-dose, life-threatening
regimens, known as \smallcaps{vamp},\footnote{\smallcaps{vamp} is based on a
  combination of four drugs: vincristine, amethopterin, mercaptopurine and
  prednisone.} was tested on children with acute lymphoblastic leukemia
(\smallcaps{aml}), in a trial based at the \smallcaps{nci} in 1961.  The
morbidity of the treatment was egregious and only two of the fifteen children
subjected to the initial protocol survived it\footnote{``If we didn't kill the
  tumour, we killed the patient''---\emph{William Moloney on the early days of
    chemotherapy} (\citealp{moloney_pioneering_1997}).}---one of which was still
alive in 2008.\cite{mukherjee_emperor_2011} In spite of all their complications,
the \smallcaps{vamp}\cite{frei_effectiveness_1965} and the
\smallcaps{mopp}\cite{devita_combination_1970} (an equally aggressive regimen to
treat advanced Hodgkin's disease) trials proved that cancer could be cured by
chemotherapy.

All these experimental drugs were selected from lists of synthetic chemicals,
fermentation products and plant derivatives.  Their anti-cancer ability was
purely deduced on an empirical basis and the only feature they shared was their
rather indiscriminate effect as cell cycle inhibitors.  The first case of a
chemotherapeutic drug being reasoned from its biological underpinnings occurred
in 1969, with the confirmation that tamoxifen could bring metastatic breast
cancer into remission.\cite{cole_new_1971} Tamoxifen, a molecular mimicker of
estrogen, was first synthesized in 1962 in the \smallcaps{uk} with the goal of
being marketed as a hormonal contraceptive.  However, tamoxifen turned out to be
an estrogen antagonist instead: by binding to the estrogen receptor, it deprives
the cell from the necessary signaling to trigger its cell
cycle.\cite{jordan_effects_1977} Adjuvant chemotherapy for the treatment of
breast cancer, i.e., the use of chemotherapy after surgical extraction of the
primary tumour, was shown to decrease the rate of relapse for the first time, in
a trial launched in 1974.\cite{bonadonna_combination_1976}

By 1990, the three-pronged approach of surgery, radiotherapy and chemotherapy,
complemented with prevention campaigns and improved diagnostic tools for early
diagnosis, led to the first reported decrease of cancer incidence and
mortality.\cite{devita_two_2012} It was a worthy achievement, but one that fell
short of the haughty rhetoric to ``cure'',\footnote{``We have a cure for breast
  cancer''---\emph{Emil Frei to a colleague}, summer of 1982
  (\citealp{mukherjee_emperor_2011}).} ``conquer'',\footnote{``Why don't we try
  to conquer cancer by America's 200\textsuperscript{th} birthday? What a
  holiday that would be!''---advertisement published in the \emph{New York
    Times} in December 1969.} or win ``the war on cancer''\footnote{The National
  Cancer Act was signed by Richard Nixon on December 23, 1971.} that was voiced
throughout the 20\textsuperscript{th} century from several corners of the cancer
research establishment.  In fact, the field was starting to come to terms with
the evidence that, no matter how aggressive the treatment,\footnote{In order to
  allow for an otherwise intolerably high chemotherapeutic dosage, Emil Frei
  devised in 1982 a trial for advanced breast cancer treatment contemplating an
  autologous bone marrow transplantation.  The re-implanted frozen bone marrow
  cells would thus be spared the excessively high drug dosage.  This regimen,
  known as \smallcaps{stamp}, became embroiled in controversy during the next
  twenty years.  It was finally put to rest in 2011, when proof was published
  that it added no discernible benefit to patient's overall survival
  (\citealp{berry_high-dose_2011}).} our capacity to arrest the nearly
monomaniacal progression of cancer had been pushed to its limits.  The most
expressive sign of this tacit acquiescence was perhaps the rise in prominence of
palliative medicine during the 1980's---prolonging life at any cost no longer
was the purpose of cancer medicine.

% ``poison'',\footnote{\citealp{shorter_health_1987}, \emph{p} 189}

\medskip

In 1997, John Bailar published a review article in the \emph{New England Journal
  of Medicine} entitled ``Cancer undefeated''.\cite{bailar_cancer_1997} He
concluded:

\begin{quotation}
  The war against cancer is far from over.  Observed changes in mortality due to
  cancer primarily reflect changing incidence or early detection.  The effect of
  new treatments for cancer on mortality has been largely disappointing.  The
  most promising approach to the control of cancer is a national commitment to
  prevention, with a concomitant rebalancing of the focus and funding of
  research.
\end{quotation}

The call for a more fundamental understanding of the neoplastic cell had been
made.  But, in order to embrace it, complementary approaches to microscopy,
\smallcaps{x}-ray scans, or mice models, would be needed.

\clearpage

% \clearpage

% Cellular origin of cancer.  Microscopy (tools).  Surgery.  X-rays.
% Chemotherapy.

% Ehrlich (p.130): ``To target the abnormal cell, one would need to decipher the
% biology of the normal cell.''

% Every drug, the sixteenth-century physician Paracelsus once opined, is a
% poison in disguise.  Cancer chemotherapy, consumed by its ery obsession to
% obliterate the cancer cell, found its roots in the obverse logic: every poison
% might be a drug in disguise. (p.133) Argument for cancer selection of
% chemotherapy.

% By 1955, this effort, called the Cancer Chemotherapy National Service Center
% (CCNSC), was in full swing.  Between 1954 and 1964, this unit would test
% 82,700 synthetic chemicals, 115,000 fermentation products, and 17,200 plant
% derivatives and treat nearly 1 million mice every year with various chemicals
% to find an ideal drug. (p.176)

% When I went through the avalanche of chemotherapy drugs that would be used
% over the next two years to treat her, she repeated the names softly after me
% under her breath, like a child discovering a new tongue twister:
% ``Cyclophosphamide, cytarabine, prednisone, asparaginase, Adriamycin,
% thioguanine, vincristine, 6- mercaptopurine, methotrexate.'' (p.184)

% Experimental regimen of combination of drugs.  Grafting of cancers in animal
% models.  Skipper learned that he could halt this effusive cell division by
% administering chemotherapy to the leukemia-engrafted mouse. By charting the
% life and death of leukemia cells as they responded to drugs in these mice,
% Skipper emerged with two pivotal findings.  First, he found that chemotherapy
% typically killed a fixed percentage of cells at any given instance no matter
% what the total number of cancer cells was.  This percentage was a unique,
% cardinal number particular to every drug.  Killing leukemia was an iterative
% process, like halving a monster's body, then halving the half, and halving the
% remnant half.  Second, Skipper found that by adding drugs in combination, he
% could often get synergistic effects on killing. (p.202)

% The notable common feature that linked all these drugs was that they were all
% rather indiscriminate inhibitors of cellular growth.  Nitrogen mustard, for
% instance, damages DNA and kills nearly all dividing cells; it kills cancer
% cells somewhat preferentially because cancer cells divide most
% actively. (p.232)

% Doctors are men who prescribe medicines of which they know little, to cure
% diseases of which they know less, in human beings of whom they know
% nothing.---Voltaire (p.204)

% If we didn't kill the tumor, we killed the patient.---William Moloney on the
% early days of chemotherapy.  VAMP trials on acute lymphoblastic leukaemia
% (ALL).  The terror of VAMP was death by infection.  Chemotherapy could cure
% cancer.  Of the fifteen patients treated on the initial protocol, only two
% still survived. (p.237) Perhaps the most disturbing side effect of
% chemotherapy would emerge nearly a decade later.  Several young men and women,
% cured of Hodgkin's disease, would relapse with a second cancer---typically an
% aggressive, drug-resistant leukemia---caused by the prior treatment with MOPP
% chemotherapy.  As with radiation, cytotoxic chemotherapy would thus turn out
% to be a double-edged sword: cancer-curing on one hand, and cancer-causing on
% the other. (p.236)

% ``A revolution [has been] set in motion,'' DeVita wrote.  Kenneth Endicott,
% the NCI director, concurred: ``The next step---the complete cure---is almost
% sure to follow.'' (p.244)

% Cancer is not one single disease!  It felt---nearly twenty five hundred years
% after Hippocrates had naively coined the overarching term karkinos---that
% modern oncology was hardly any more sophisticated in its taxonomy of cancer.
% Orman's lymphoma and Sorenson's pancreatic cancer were both, of course,
% ``cancers,'' malignant proliferations of cells.  But the diseases could not
% have been further apart in their trajectories and personalities.  Even
% referring to them by the same name, cancer, felt like some sort of medical
% anachronism, like the medieval habit of using apoplexy to describe anything
% from a stroke to a hemorrhage to a seizure.  Like Hippocrates, it was as if
% we, too, had naively lumped the lumps.

% Radiotherapy and Hodgkin's lymphoma in Stanford (p.228): The trials that
% Kaplan designed still rank among the classics of study design. In the first
% set, called the L1 trials, he assigned equal numbers of patients to either
% extended field radiation or to limited ``involved field'' radiation and
% plotted relapse-free survival curves.  The answer was definitive.  Extended
% field radiation---``meticulous radiotherapy'' as one doctor described
% it---drastically diminished the relapse rate of Hodgkin's disease.

% A hundred instances of Hodgkin's disease, even though pathologically
% classified as the same entity, were a hundred variants around a common theme.
% Cancers possessed temperaments, personalities---behaviors.  And biological
% heterogeneity demanded therapeutic heterogeneity; the same treatment could not
% indiscriminately be applied to all.  But even if Kaplan understood it fully in
% 1963 and made an example of it in treating Hodgkin's disease, it would take
% decades for a generation of oncologists to come to the same
% realization. (p.229)

% The somatic theory of cancer argued that environmental carcinogens such as
% soot or radium somehow permanently altered the structure of the cell and thus
% caused cancer. (p.247)

% The cure before the cause, i.e., treatment before the understanding of the
% disease.

% James Watson, who had discovered the structure of DNA, unloosed a verbal
% rampage against the Senate bill.  ``Doing `research' is not necessarily doing
% `good' research,'' Watson would later write.  ``In particular we must reject
% the notion that we will be lucky\ldots Instead we will be witnessing a massive
% expansion of well-intentioned mediocrity.'' (p.267)

% X-rays and surgery---1924 Keynes (p.278)

% Cancer didn't move centrifugally by whirling through larger and larger ordered
% spirals; its spread was more erratic and unpredictable. (p.280)

% This is the point at which a clear understanding of invasiveness of the
% primary tumour started driving the extent of surgical approach---no more
% radical and ultra-radical surgery.

% 1928: Jerzy Neyman and Egon Pearson provided a systematic method to evaluate a
% negative systematic claim---the statistical concept of power. (p.280)

% The `power' of orthodoxy. (p.282)

% 1973: tamoxifen/ER receptor and estrogen-dependent growth (p.306) Adjuvant
% chemotherapy.

% 1972: at the NCI, first effort to integrate chemotherapy with surgery. The
% Istituto Tumori (Milan) trial (p.310) Anti-hormone therapy for prostate and
% breast cancer.

% 1980: paliative medicine (not care) in oncology (p.316) The movement to
% restore sanity and sanctity to the end-of-life care of cancer patients emerged
% predictably, not for the cure obsessed America but from Europe.  Cecile
% Saunders.  1967: first hospice in London to take care of the terminally ill,
% away from the oncology wards. (p.318)

% The ``More doctors smoke Camel'' ad.  Incidentally, one particular event in
% cancer history spurred more than any other the need to \emph{understand} the
% mechanism of action of carcinogens in causing cancer.  It was the strong and
% dishonest rebutal of the tobacco industry in accepting the epidemiological
% studies as effective proof of cancer cause. (p.360)

% Cancer demonstrates a spectrum of behaviour.

%%% Local Variables:
%%% TeX-engine: xetex
%%% mode: latex
%%% TeX-master: "../../thesis"
%%% End:
