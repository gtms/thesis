%%% Local Variables:
%%% coding: utf-8
%%% mode: latex
%%% TeX-engine: xetex
%%% End:

\documentclass{tufte-book}

\usepackage{fontspec}
\usepackage{soul}
\usepackage{booktabs}
\usepackage{morefloats}
\usepackage{siunitx}

\defaultfontfeatures{Ligatures=TeX,Numbers=OldStyle}
\setmainfont{Minion Pro}
\setsansfont[Scale=MatchLowercase]{Gill Sans MT}
% \setsansfont[Scale=MatchLowercase]{Gill Sans Std}
\setmonofont[Scale=MatchLowercase]{Consolas}

\usepackage{etoolbox}% provides some support for comma-separated lists

% Set up the spacing using fontspec features
\renewcommand\allcapsspacing[1]{{\addfontfeature{LetterSpace=15}#1}}
\renewcommand\smallcapsspacing[1]{{\addfontfeature{LetterSpace=10}#1}}

\makeatletter
% We'll keep track of the old/seen bibkeys here.
\def\@tufte@old@bibkeys{}

% This macro prints the full citation if it's the first time it's been used
% and a shorter citation if it's been used before.
\newcommand{\@tufte@print@margin@citation}[1]{%
  \citealp{#1}% print short entry
  % add bibkey to the old bibkeys list
  \listgadd{\@tufte@old@bibkeys}{#1}%
}

% We've modified this Tufte-LaTeX macro to call \@tufte@print@margin@citation
% instead of \bibentry.
\renewcommand{\@tufte@normal@cite}[2][0pt]{%
  % Snag the last bibentry in the list for later comparison
  \let\@temp@last@bibkey\@empty%
  \@for\@temp@bibkey:=#2\do{\let\@temp@last@bibkey\@temp@bibkey}%
  \sidenote[][#1]{%
    % Loop through all the bibentries, separating them with semicolons and spaces
    \normalsize\normalfont\@tufte@citation@font%
    \setcounter{@tufte@num@bibkeys}{0}%
    \@for\@temp@bibkeyx:=#2\do{%
      \ifthenelse{\equal{\@temp@last@bibkey}{\@temp@bibkeyx}}{%
        \ifthenelse{\equal{\value{@tufte@num@bibkeys}}{0}}{}{and\ }%
        \@tufte@trim@spaces\@temp@bibkeyx% trim spaces around bibkey
        \@tufte@print@margin@citation{\@temp@bibkeyx}%
      }{%
        \@tufte@trim@spaces\@temp@bibkeyx% trim spaces around bibkey
        \@tufte@print@margin@citation{\@temp@bibkeyx};\space
      }%
      \stepcounter{@tufte@num@bibkeys}%
    }%
  }%
}

% Calling this macro will reset the list of remembered citations. This is
% useful if you want to revert to full citations at the beginning of each
% chapter.
\newcommand{\resetcitations}{%
  \gdef\@tufte@old@bibkeys{}%
}
\makeatother

\begin{document}

\section{Cancer}

\newthought{Cancer is a rupture} of the social contract engaged by cells of the
somatic lineage of multicellular organisms.  This defection is caused by a
collection of critical failures of the genetic systems evolved to ensure the
correct and timely integration of the cellular unit's physiology at the tissue
and organism's level.\footnote{Homeostasis, is the property of a system in which
  variables are regulated so that internal conditions remain stable and
  relatively constant.}

\bigskip

Cellular transformation, the process through which the neoplastic phenotype
arises, is caused by genetic and epigenetic alterations in somatic cells.
Fortunately, most somatic cells exhibiting behaviours beyond physiological
ranges end up being singled out and targeted for removal, either by eliciting an
immune response or by triggering self-induced cellular death (\emph{note about
  apoptosis here}).  In order for a neoplasm to become clinically relevant, it
has thus to acquire a number of alterations that consign it with the capacity to
evade its host organisms' regulatory control mechanisms against unicellular
defection.

% an operating outside of normal physiological levels cells abnormal behaviour
% exhibited by most early neoplastic cells may be sufficient for them to be
% recognized and targeted for removal, either by eliciting an immune response or
% by triggering self-induced cellular death (\emph{note about apoptosis here}).

% Clonality; multistep model for the nature of cancer;

From a functional point of view, the defining attribute of a neoplastic lineage
is, arguably, its ability to sustain chronic proliferation, irrespectively of
the social cues conveyed by its tissular context.  Beyond that, the most
conspicuous feature of cancer has to be the diversity of shapes and forms these
proliferating masses can take when departing from the neatly organized
architectural tissue types they arise from.  An integrative framework to grasp
this remarkable phenotypic plasticity has been proposed by Hanahan and
Weinberg,\cite{hanahan_hallmarks_2000,hanahan_hallmarks_2011} who proposed six
essential and complementary capabilities for tumour growth and metastatic
dissemination.  These include self-suficiency in growth signals, insensitivity
to growth-inhibitory (antigrowth) signals, evasion of programmed cell-death
(apoptosis), limitless replicative potential, sustained angiogenesis, and tissue
invasion and metastasis.

Specific genes, termed oncogenes, have the potential to induce transformation
when disrupted in particular circumstances.  Alterations in nearly 500 of such
genes have been linked to cancer initiation and
progression.\cite{forbes_catalogue_2008} The multistep model for the nature of
cancer posits that several such alterations are cumulatively required in order
to initiate tumorigenesis and to evolve increasingly more aggressive and
invasive tumour phenotypes.\cite{vogelstein_multistep_1993}

% The fundamental and defining characteristic of the cancer cell is, arguably,
% it's ability to sustain chronic proliferation.

The number and patterns of somatic alterations vary dramatically across cancer
types.  At one extreme, childhood medulloblastomas can harbour fewer than ten
genomic alterations, whereas over \SI{50000} somatic changes have been observed
in primary lung adenocarcinoma samples.

In biological systems the instanciation of information is \smallcaps{DNA}.

a phenotype eerily reminiscent of that seen in loosely cooperative unicellular
life forms, at a time when multicellularity was still being attempted at.

% individual alterations in oncogenes are seen as necessary but not
% sufficient to give rise to cancer.  Considered to arise sequentially and to give
% rise to the progressively more aggressive and invasive phenotypes observed
% during tumorigenesis.

% which these control mechanisms are compromised lead by mutations on specific genes
% which these mechanisms are hindered is through the accumulation of somatic
% mutations of oncogenes


% Cells of normal tissues collectively control their growth rate by regulating the
% production and release of paracrine \mbox{growth-promoting} signals that direct
% entry and progression through the cell \mbox{growth-and-division} cycle.  A
% first requirement for the acquisition of the neoplastic identity must then be
% the ability to generate endogenous mitogenic signals that result in autocrine
% proliferative stimulation.\footnote{This endeavour can also be achieved through
%   the emission of signals that stimulate surrounding normal cells to feed cancer
%   cells back with growth factors (citealp{Cheng-2008,Bhowmick-2004}); specific
%   somatic mutations that constitutively activate pathways usually triggered by
%   activated growth factor repectors; or through disruptions of
%   \mbox{negative-feedback} mechanisms that attenuate proliferative signaling.}
% In addition, the cancer cell must also become insensitive to social cues
% designed to control unrestricted cell proliferation.

% From jcb-weinberg-1983.pdf:

% This leads to the realization that these [onco]genes are of extremely ancient
% lineage---their precursors were already present in similar form in the
% primitive metazoans taht served as common ancestors to chordates and
% arthropodes.  Such conservation indicates that these genes have served vital,
% indispensable function in normal cellular and organismic physiology, and that
% their role in carcinogenesis represents only an unusual and aberrant diversion
% from their usual functions.

% by caused by steered by the
% engagement of genetic routines of the form of an atavistic state engagement of
% genetic routines active at an earlier stage of development that are
% inappropriately reactivated in the mature organism as a result of some sort of
% insult of damage.  Cancer is an atavistic state of multicellular life.  Cells
% relieved of the molecular constrains that subordinate them to societal
% discipline.

% Causing cancer cells' metabolism to default to more fundamental modes of
% functioning not unlike those conceivably typified by loosely societal
% cellular sorts.

% These six hallmarks of cancer---distinctive and complementary capabilities taht
% enable tumour growth and metastatic dissemination

% Further evidence that supports our theory comes from experiments in which the
% nuclei of egg cells are replaced with cancer-cell nuclei.  Astonishingly,
% embryos start to develop normally.  But abnormalities eventually appear, at
% earlier stages when the cancer is more malignant (advanced). This inverse
% correlation of cancer stage with embryo stage is consistent with our theory.

% Cancer follows a well-defined progression of within a host organism of
% accelerating growth (proliferation), mobility, spread and colonization.

% To better understand cancer, including its place in the great sweep of
% evolutionary history.

% ideias to discuss:

% causes of cancer

% field cancerization (tarabichi)

% 1--For multicellularity to become a viable strategy, a number of strategies must
% have evolved in order to maximize cooperation and minimize conflict between
% individual cells of an organism.

% 2--One such strategy is the imposition of a genetic bottleneck per generation in
% the course of the reproductive cycle of multicellular organisms in order to ensure
% clonality among constituent cells of an organism.

% 3--As cooperation creates new levels of fitness, it creates the opportunity for
% conflict between levels as deleterious mutants arise and spread within the
% group.  Fundamental to the emergence of a new higher-level unit is the mediation
% of conflict among lower-level units in favor of the higher-level unit.

% 4--Michod et al. (2003) lists a number of conflict-mediating adaptations that
% can be useful in engendering cooperation among physically associated cells.
% These include the existence of a germ line, the tendency for unnecessary or even
% defecting cells to undergo apoptosis, the potential for self policing, display
% of relatively low mutation rates, canalization of growth (of which one mechanism
% is "determinate growth"), etc.

% 5--Take the \emph{p53} gene as example.  \emph{p53} has likely been the most
% studied gene for over a decade.

% cancer dormancy is a also perplexing

% pleiotropy of the key oncogenic gatekeepers

% quote the process of ``exaptation'' (Gould & Vrba, 1982)
% the role of epigenetics

% The conventional argument to explain the deployment of this \mbox{swiss-knife}
% armory of solutions is to invoke Darwinian evolution between \mbox {sub-clones}
% of the original neoplasm.\cite{Merlo-2006} According to this hypothesis, the
% evolution of the required survival traits for the success of the tumour would be
% attributed mostly to random mutations and the trial and error of normal
% Darwinian evolution.  Rapid mutation rate within tumours + strong selective
% pressure as organism ``fights back'' or patient undergoes chemotherapy.

% This understanding of cancer as a feature of a particular evolutionary mechanism

% Rewrite discussion on multicellularity in the life section according to the
% lines discussed here:
% http://www.biologyaspoetry.com/textbooks/microbes_and_evolution/greater_size_problems_in_multicellularity.html

% NOTES from tmm-floor-2012.pdf: hallmarks of cancer: of all cancer cells, all
% the time?

% The hallmarks of cancer
% 1-self-sufficiency in growth signals
% 2-insensitivity in anti-growth signals
% 3-evasion of apoptosis
% 4-limitless replicative potential
% 5-sustained angiogenesis
% 6-tissue invasion
% 7-metastasis
% 8-metabolic reprogramming
% 9-evasion of the immune system

% Strangely, one fundamental characteristic of cancer cells, the loss of
% differentiation, was not considered a distinct hallmark.  This characteristic is
% essential because it is a primary difference between benign and malignant tumors
% (e.g., autonomous adenomas as opposed to carcinoma of the thyroid).  It is
% supported by a measurable expression program switch that affects all gene
% categories.  We argue that this should be another, if not the major,
% ``hallmark'' of a cancerous cell, and it has been shown to be a robust
% diagnostic index for thyroid tumors.

% new hallmarks of cancer must be able

% cancer progression is characterized by a succession of oncogenic events

% The hallmarks of cancer (cel-hanahan-2011)
% The hallmarks constitute an organizing principle for rationalizing the
% complexities of neoplastic disease.  They include:
% 1--sustainment of proliferative signalling
% 2--evasion of growth supressors
% 3--resistance to cell death
% 4--enabling of replicative immortality
% 5--inducing of angiogenesis
% 6--activation of invasion and metastasis

% The genomes of nearly all healthy human cells, containing the entirety of an
% individual's inherited information, evidently come pre-loaded with a ``cancer
% sub-routine'' that is normally idle but can be triggered into action by a wide
% variety of insults, such as chemicals, radiation and inflammation.

% Breivik 2005

% Cancer incidence increases with age, and in countries with high life expectancy,
% aproximately 40\% of the population experiences some kind of cancer in their
% lifetime.  Adiotionally, a significant proportion of the autopsies reveal
% undiagnosed maliganacies, and virtually everybody carry some kind of in situ
% carcinomas after the age of 50.

\bigskip

Clinical definition of cancer. A vast collection of diseases (almost 100,
according to http://medical-dictionary.thefreedictionary.com/Cancer).

\backmatter

\bibliography{cancer}
\bibliographystyle{plainnat}

\end{document}
