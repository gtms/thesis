\section{Motivation \& Contributions of this Thesis}

The work contributed in this thesis is anchored in the analytic corpus developed
during ten years of gene expression assaying of cancer biopsies with
microarray technology.  Its motivations are framed by the current understanding
of the fundamental biology of cancer, primed by its two major axis of
progression: differentiation and proliferation.

\subsection{Differentiation}
We first explored the differentiation axis by investigating the potential of
gene expression differentiation signatures to chart cancer progression using
whole genome microarray profiles of tumour samples.  As most molecular
classifiers of cancer rely on the variant features of the transformed neoplastic
transcriptome, we reasoned that a complementary approach could consist of
leveraging the invariant features specified by the unique transcriptional
signatures of each tissue of origin.

As a case study, we took to thyroid cancer.  Neoplasms of the thyrocyte cell are
characterized by a well defined linear progression from benign, fully
differentiated tumour types, up to one of the most lethal human cancers, the
anaplastic thyroid carcinoma (Figure~\ref{fig:thyroid-carcinogenesis}).  We
aimed to build a genomic marker of thyroid cancer progression based on a gene
expression signature of healthy thyrocytes.  As a result, we devised a general
method to derive robust \mbox{organ-specific} gene \mbox{expression-based}
differentiation indices, published in the journal
\emph{Oncogene}.\cite{tomas_general_2012}

\begin{figure}[ht]
  \includegraphics{fie-thyroid-carcinogenesis-2012-thyroid.png}%
  \caption[Step model of thyroid carcinogenesis]{Step model of thyroid
    carcinogenesis.  Thyroid epithelial cells may undergo transformation via
    alterations in different oncogenes and tumour supressor genes, giving rise
    to \mbox{well-differentiated} papillary or follicular carcinomas.
    Additional mutational load can cause progression of a differentiated tumour
    into a poorly differentiated one, and eventually into an anaplastic
    carcinoma.  Particularly challenging, from the histopathological point of
    view, is the distinction between follicular adenomas and follicular
    carcinomas; and between follicular variants of papillary carcinomas and
    their classical counterpart (reproduced from
    \citealp{sastre-perona_role_2012}).}
  \label{fig:thyroid-carcinogenesis}
\end{figure}

Contributions made in the context of this work include:
\begin{itemize}{}{}
\item An unbiased procedure to derive \mbox{organ-specific} differentiation
  markers from gene expression profiles of healthy tissues.
\item Proof of concept of the clinical utility of differentiation signatures in
  cancer diagnosis, featuring thyroid cancer as a test case.  Specifically, we
  demonstrated that, in a panel of expression profiles composed of thyroid
  cancers of distinct subtypes and normal thyroid samples:
  \begin{enumerate}
  \item The expression of a \mbox{thyroid-specific} differentiation biomarker,
    consisting of 15 genes, is inversely correlated with that of a proliferation
    biomarker, also independently derived from expression profiles of healthy
    tissues.  Conversely, the differentiation biomarker is positively correlated
    with the proliferation biomarker in expression profiles of a
    \mbox{time-course} experiment where thyrocytes in culture were treated with
    \smallcaps{tsh} hormone (the \mbox{thyroid-stimulating} hormone,
    \smallcaps{tsh}, induces both the metabolic activity and proliferation of
    thyrocytes).  These observations support the independence of the two
    biomarkers and prove that the differentiation biomarker does capture a
    transcriptome signature particular to the epithelial thyroid cell.
  \item A multidimensional scaling analysis representation of the profiled
    clinical samples exposes a non-overlapping continuum of thyroid tumours of
    increasing malignancy.
  \item The differentiation biomarker can accurately discriminate between
    follicular adenomas and follicular carcinomas; and between follicular
    variants of papillary carcinomas and classical papillary carcinomas---two
    challenging histopathological diagnosis.  Moreover, the accuracy of the
    differentiation biomarker in this supervised classification task was not
    significantly different from the accuracy of two supervised machine learning
    classifiers trained within the whole gene expression space of the tested
    samples.
  \end{enumerate}
\end{itemize}

\subsection{Proliferation}
Uncontrolled proliferation is not just a hallmark of
cancer\cite{hanahan_hallmarks_2011}---but its very own operational definition.
Using a proliferation biomarker consisting of 129 genes derived from healthy
tissues, Venet and Detours\cite{venet_most_2011} showed that most of the
prognostic content found in the reference \smallcaps{nki}-295 dataset was
linked to a pervasive proliferative signal in the neoplastic transcriptome.

We took to a wider collection of 102 distinct \mbox{outcome-related} cancer
cohorts, spanning 22 types of cancer to (\emph{a}) evaluate the extent of
prognostic signals in human cancers transcriptomes; and (\emph{b}) dissect the
potential technical and biological variables linked to prognostic content in
different cancer.  The results of this work are currently under submission and
will be thoroughly detailed in the \emph{results}\footnote{Link to results
  section here.} section.

\medskip

Contributions made in the context of this work include:
\begin{itemize}
\item Evidence of an extensive correlation structure in cancer transcriptomes as
  assayed by microarrays.  This is concluded on the count that, in 76\% of the
  cancer cohorts analyzed, more than 5\% of random gene expression signatures is
  associated either with patient death or relapse of disease.
\item Demonstration that variables of both technical and biological essence are
  linked with the heterogeneity in prognostic content observed across 33 breast
  cancer cohorts.  This was determined from a thorough statistical multivariate
  analysis of the \mbox{clinico-pathological} variables associated in each study
  with the prognostic fraction of the transcriptome.
\item Corroboration of the significance of proliferation as the biological
  program comprising most of the prognostic content in cancer transcriptomes.
  Serial deconvolution of distinct biologically motivated biomarkers was
  undertaken across the studied cohorts to yield this result.
\end{itemize}

% A proliferation metagene captures a significant fraction of the
% pan-transcriptomic pervasive signals associated with outcome; other biological
% programs may also account for prognostic content to various degrees.

%%% Local Variables:
%%% mode: latex
%%% TeX-master: "../../thesis"
%%% End:
