%%%%%%%%%%%%%%%%%%%%%%%%%%%%%%%%%%%%%%%%%%%%%%%%%%%%%%%%%%%%%%%%%%%%%%%%%%%%%%%%
%% PREAMBLE BEGINS HERE
%%%%%%%%%%%%%%%%%%%%%%%%%%%%%%%%%%%%%%%%%%%%%%%%%%%%%%%%%%%%%%%%%%%%%%%%%%%%%%%%

\documentclass[a4paper,twoside]{tufte-book}
% \documentclass[a4paper,sfsidenotes,twoside]{tufte-book}


%%%%%%%%%%%%%%%%%%%%%%%%%%%%%%%%%%%%%%%%%%%%%%%%%%%%%%%%%%%%%%%%%%%%%%%%%%%%%%%%
%% Loads packages
%%%%%%%%%%%%%%%%%%%%%%%%%%%%%%%%%%%%%%%%%%%%%%%%%%%%%%%%%%%%%%%%%%%%%%%%%%%%%%%%

% \usepackage{fontspec}
\usepackage[MnSymbol]{mathspec}
% \usepackage{parskip}
\usepackage{soul}
\usepackage{booktabs}
\usepackage{morefloats}
\usepackage{siunitx}
\usepackage{microtype}
\usepackage{pdfpages} % to insert pages of external documents
\usepackage{etoolbox} % provides some support for comma-separated lists
\usepackage{booktabs}
% \usepackage{units} % superseded by siunitx
\usepackage{xspace} % prints trailing space in a smart way
\usepackage{makeidx}
\usepackage{fancyvrb}
\usepackage{wasysym} % for circles in fig:censorship
\usepackage{dcolumn} % for decimal alignment in columns
\usepackage{lipsum} % for dummy text
\usepackage[hyperfootnotes=false]{hyperref}


%%%%%%%%%%%%%%%%%%%%%%%%%%%%%%%%%%%%%%%%%%%%%%%%%%%%%%%%%%%%%%%%%%%%%%%%%%%%%%%%
%% Sets up fonts
%%%%%%%%%%%%%%%%%%%%%%%%%%%%%%%%%%%%%%%%%%%%%%%%%%%%%%%%%%%%%%%%%%%%%%%%%%%%%%%%

\defaultfontfeatures{Ligatures=TeX}
\setprimaryfont{Minion Pro}
\setmainfont[Numbers=OldStyle]{Minion Pro}
% \setsansfont[Scale=MatchLowercase]{Gill Sans} % pc15
% \setsansfont[Scale=MatchLowercase]{Gill Sans Std} % HP-ENVY14
\setsansfont[Scale=MatchLowercase]{Gill Sans MT Pro Book} % HP-ENVY14
\setmonofont[Scale=MatchLowercase]{Consolas}


%%%%%%%%%%%%%%%%%%%%%%%%%%%%%%%%%%%%%%%%%%%%%%%%%%%%%%%%%%%%%%%%%%%%%%%%%%%%%%%%
%% Defines font size of verbatim environments (package fancyvrb)
%%%%%%%%%%%%%%%%%%%%%%%%%%%%%%%%%%%%%%%%%%%%%%%%%%%%%%%%%%%%%%%%%%%%%%%%%%%%%%%%

\fvset{fontsize=\normalsize}


%%%%%%%%%%%%%%%%%%%%%%%%%%%%%%%%%%%%%%%%%%%%%%%%%%%%%%%%%%%%%%%%%%%%%%%%%%%%%%%%
%% Defines settings to typeset quantities in a consistent way (package siunitx)
%%%%%%%%%%%%%%%%%%%%%%%%%%%%%%%%%%%%%%%%%%%%%%%%%%%%%%%%%%%%%%%%%%%%%%%%%%%%%%%%

\sisetup{detect-all=true}%


%%%%%%%%%%%%%%%%%%%%%%%%%%%%%%%%%%%%%%%%%%%%%%%%%%%%%%%%%%%%%%%%%%%%%%%%%%%%%%%%
%% Sets up the spacing using fontspec features
%%%%%%%%%%%%%%%%%%%%%%%%%%%%%%%%%%%%%%%%%%%%%%%%%%%%%%%%%%%%%%%%%%%%%%%%%%%%%%%%

\renewcommand\allcapsspacing[1]{{\addfontfeature{LetterSpace=15}#1}}
\renewcommand\smallcapsspacing[1]{{\addfontfeature{LetterSpace=10}#1}}


%%%%%%%%%%%%%%%%%%%%%%%%%%%%%%%%%%%%%%%%%%%%%%%%%%%%%%%%%%%%%%%%%%%%%%%%%%%%%%%%
%% Specifies settings for graphics/images
%%%%%%%%%%%%%%%%%%%%%%%%%%%%%%%%%%%%%%%%%%%%%%%%%%%%%%%%%%%%%%%%%%%%%%%%%%%%%%%%

\setkeys{Gin}{width=\linewidth,totalheight=\textheight,keepaspectratio}
\graphicspath{{../graphics/}}


%%%%%%%%%%%%%%%%%%%%%%%%%%%%%%%%%%%%%%%%%%%%%%%%%%%%%%%%%%%%%%%%%%%%%%%%%%%%%%%%
%% Redefines \cite command
%%%%%%%%%%%%%%%%%%%%%%%%%%%%%%%%%%%%%%%%%%%%%%%%%%%%%%%%%%%%%%%%%%%%%%%%%%%%%%%%

%% As per instructions taken from:
%% http://tex.stackexchange.com/questions/46012/how-do-i-get-full-or-abbreviated-citations-to-appear-automatically-in-tufte-foot

\makeatletter
% We'll keep track of the old/seen bibkeys here.
\def\@tufte@old@bibkeys{}

% This macro prints the full citation if it's the first time it's been used
% and a shorter citation if it's been used before.
\newcommand{\@tufte@print@margin@citation}[1]{%
  \citealp{#1}% print short entry
  % add bibkey to the old bibkeys list
  \listgadd{\@tufte@old@bibkeys}{#1}%
}

% We've modified this Tufte-LaTeX macro to call \@tufte@print@margin@citation
% instead of \bibentry.
\renewcommand{\@tufte@normal@cite}[2][0pt]{%
  % Snag the last bibentry in the list for later comparison
  \let\@temp@last@bibkey\@empty%
  \@for\@temp@bibkey:=#2\do{\let\@temp@last@bibkey\@temp@bibkey}%
  \sidenote[][#1]{%
    % Loop through all the bibentries, separating them with semicolons and spaces
    \normalsize\normalfont\@tufte@citation@font%
    \setcounter{@tufte@num@bibkeys}{0}%
    \@for\@temp@bibkeyx:=#2\do{%
      \ifthenelse{\equal{\@temp@last@bibkey}{\@temp@bibkeyx}}{%
        \ifthenelse{\equal{\value{@tufte@num@bibkeys}}{0}}{}{and\ }%
        \@tufte@trim@spaces\@temp@bibkeyx% trim spaces around bibkey
        \@tufte@print@margin@citation{\@temp@bibkeyx}%
      }{%
        \@tufte@trim@spaces\@temp@bibkeyx% trim spaces around bibkey
        \@tufte@print@margin@citation{\@temp@bibkeyx};\space
      }%
      \stepcounter{@tufte@num@bibkeys}%
    }%
  }%
}

% Calling this macro will reset the list of remembered citations. This is
% useful if you want to revert to full citations at the beginning of each
% chapter.
\newcommand{\resetcitations}{%
  \gdef\@tufte@old@bibkeys{}%
}
\makeatother


%%%%%%%%%%%%%%%%%%%%%%%%%%%%%%%%%%%%%%%%%%%%%%%%%%%%%%%%%%%%%%%%%%%%%%%%%%%%%%%%
%% Random hacks from the preamble of tufte-latex-3.5.0 (sample-book.tex)
%%%%%%%%%%%%%%%%%%%%%%%%%%%%%%%%%%%%%%%%%%%%%%%%%%%%%%%%%%%%%%%%%%%%%%%%%%%%%%%%

%%
% Prints argument within hanging parentheses (i.e., parentheses that take
% up no horizontal space).  Useful in tabular environments.
\newcommand{\hangp}[1]{\makebox[0pt][r]{(}#1\makebox[0pt][l]{)}}

%%
% Prints an asterisk that takes up no horizontal space.
% Useful in tabular environments.
\newcommand{\hangstar}{\makebox[0pt][l]{*}}

% Prints the month name (e.g., January) and the year (e.g., 2008)
\newcommand{\monthyear}{%
  \ifcase\month\or January\or February\or March\or April\or May\or June\or
  July\or August\or September\or October\or November\or
  December\fi\space\number\year
}

% Prints an epigraph and speaker in sans serif, all-caps type.
\newcommand{\openepigraph}[2]{%
  %\sffamily\fontsize{14}{16}\selectfont
  \begin{fullwidth}
  \sffamily\large
  \begin{doublespace}
  \noindent\allcaps{#1}\\% epigraph
  \noindent\allcaps{#2}% author
  \end{doublespace}
  \end{fullwidth}
}

% Inserts a blank page
\newcommand{\blankpage}{\newpage\hbox{}\thispagestyle{empty}\newpage}

% Typesets the font size, leading, and measure in the form of 10/12x26 pc.
\newcommand{\measure}[3]{#1/#2$\times$\unit[#3]{pc}}


%%%%%%%%%%%%%%%%%%%%%%%%%%%%%%%%%%%%%%%%%%%%%%%%%%%%%%%%%%%%%%%%%%%%%%%%%%%%%%%%
%% Defines book metadata
%%%%%%%%%%%%%%%%%%%%%%%%%%%%%%%%%%%%%%%%%%%%%%%%%%%%%%%%%%%%%%%%%%%%%%%%%%%%%%%%

\title[Proliferation and Differentiation in cancer]{%
  \setlength{\parindent}{0pt}%
  % Proliferation \& \par
  Proliferation and \par
  Differentiation \par
  in Cancer
  \vspace{1cm}
}
\author[Gil Tomás]{Gil Tomás}
\publisher{Université Libre de Bruxelles}


%%%%%%%%%%%%%%%%%%%%%%%%%%%%%%%%%%%%%%%%%%%%%%%%%%%%%%%%%%%%%%%%%%%%%%%%%%%%%%%%
%% Generates the index
%%%%%%%%%%%%%%%%%%%%%%%%%%%%%%%%%%%%%%%%%%%%%%%%%%%%%%%%%%%%%%%%%%%%%%%%%%%%%%%%

\makeindex


%%%%%%%%%%%%%%%%%%%%%%%%%%%%%%%%%%%%%%%%%%%%%%%%%%%%%%%%%%%%%%%%%%%%%%%%%%%%%%%%
%% PREAMBLE ENDS HERE
%%%%%%%%%%%%%%%%%%%%%%%%%%%%%%%%%%%%%%%%%%%%%%%%%%%%%%%%%%%%%%%%%%%%%%%%%%%%%%%%


\begin{document}

\frontmatter

\blankpage

% \maketitle

\newpage
\begin{fullwidth}
~\vfill
\thispagestyle{empty}
\setlength{\parindent}{0pt}
\setlength{\parskip}{\baselineskip}
Copyright \copyright\ \the\year\ \thanklessauthor

\par\smallcaps{Published by \thanklesspublisher}

\par\smallcaps{tufte-latex.googlecode.com}

\par Licensed under the Apache License, Version 2.0 (the ``License''); you may
not use this file except in compliance with the License. You may obtain a copy
of the License at \url{http://www.apache.org/licenses/LICENSE-2.0}. Unless
required by applicable law or agreed to in writing, software distributed under
the License is distributed on an \smallcaps{``AS IS'' BASIS, WITHOUT WARRANTIES
OR CONDITIONS OF ANY KIND}, either express or implied. See the License for the
specific language governing permissions and limitations under the
License.\index{license}

\par\textit{First printing, \monthyear}
\end{fullwidth}


\tableofcontents
\listoffigures
\listoftables


\chapter{Introduction}
\label{chap:introduction}

\documentclass{tufte-book}
\usepackage{luatextra}
\defaultfontfeatures{Ligatures=TeX}
\setmainfont{Minion Pro}
\setsansfont[Scale=MatchLowercase]{Gill Sans MT}
\setmonofont[Scale=MatchLowercase]{Consolas}
\usepackage{booktabs}

\newenvironment{docspec}{\begin{quotation}\ttfamily\parskip0pt\parindent0pt\ignorespaces}{\end{quotation}}
% command specification environment


\begin{document}

\section{Life}

\newthought{Life is a wonder} of its own.  It conceivably struck this planet some
3.7~billion years ago [citation], as the consequence of a phospholipidic
layer-bound \emph{quantum leap} in a soup of organic precursors
[citation]. From that singular moment on, little has been spared in guise of
amazement.

The first factual evidence of life on Earth appears inscribed in the fossil
record some 3.4~billion years ago. It consists mainly

\begin{quotation}
There is grandeur in this view of life, with its several powers, having been
originally breathed into a few forms or into one; and that, whilst this planet
has gone cycling on according to the fixed law of gravity, from so simple a
beginning endless forms most beautiful and most wonderful have been, and are
being, evolved.
\end{quotation}

How ironic that one of the best models to understand this order of things is
actually one of its most harrowing violations: cancer.

\section{Cancer}

\end{document}

\section{Cancer}

% introduction with a challenging perspective

\newthought{Cancer is a rupture} of the social contract engaged by cells of the
somatic lineage of multicellular organisms.  This defection is caused by a
collection of critical failures of the genetic systems evolved to ensure the
correct and timely integration of the cellular unit's physiology at the tissue
and organism's level.  Neoplastic cells are operationally defined by their
ability to sustain chronic proliferation, to invade tissues and to set up
satellite growths in other organs.  When left unchecked, these features can
compromise the host organism's ability to sustain homeostatic
balance,\footnote{While the concept of homeostasis emphasises the stability of
  the internal milieu toward perturbation, perhaps a more accurate formulation
  could be that of \mbox{\emph{homeodynamics}}---a concept that seeks to account
  for the diverse behaviour exhibited by biological systems, including all its
  emergent characteristics, i.e., bistable switches, thresholds, waves,
  gradients, mutual entrainment, and periodic as well as chaotic behaviour
  (\citealp{lloyd_why_2001}).} and eventually lead to its systemic failure---and
death.

\bigskip

% defining attributes of cancer; functional and morphological features;
% conceptual framework to understand cancer

% Quoting Emmanuel Farber in his 1973 address to the American Cancer Society:
% \begin{quotation}
%   ``Cancer'' is an imprecise descriptive term to encompass all the many
%   conditions in which cells proliferate for whatever reason in a more or less
%   uncontrolled manner, invade tissues, and set up satellite growths in other
%   organs.  The overall result of such a process, if left undisturbed, is
%   almost always the death of the
%   host.\cite{farber_carcinogenesiscellular_1973}
% \end{quotation}

\subsection{The dynamics of cancer}
\label{sec:dynamics-cancer}

Sixty years ago, Armitage and Doll developed a multistage theory to analyze
rates of cancer progression.\cite{armitage_age_1954} They reasoned that cancer
builds upon a sequence of cellular systems' cumulative failures.  Each such
failure, for instance the abrogation of a critical \smallcaps{dna} repair
pathway or the loss of control over cellular death, moves the system one step
closer to the onset of disease.\cite{frank_dynamics_2007} Rather than a static
physiological condition specified by a unique set of cellular dysfunctions,
cancer is depicted as a progression along the course of a dynamic evolutionary
history.

Models of neoplasia evolution historically identify the seminal transforming
event with an alteration on a single somatic cell that triggers cancer
progression.  Initiation events contributing to the early stages of neoplastic
transition are caused by mutations in specific genes whose output is either
enhanced (oncogenes) or repressed (tumour supressor genes).\footnote{Alterations
  in nearly 500 of such genes have been linked to cancer initiation and
  progression (\citealp{forbes_catalogue_2008}).}  Such genetic mutations can be
structural, including nucleotide substitutions or mutations resulting from gene
fusion,\cite{konopka_cell_1985} juxtaposition to enhancer
elements,\cite{tsujimoto_t14;18_1985} or by amplification.  Alterations that
imply a discrete change of output in gene expression, such as translocations or
other structural mutations, can occur as initiating
events\cite{finger_common_1986} or during tumour progression, whereas
amplification usually occurs during progression.\cite{croce_oncogenes_2008}

However, a single genetic change is rarely sufficient to trigger the development
of a neoplasia.  Since the term \emph{neoplasia} is generally used to refer to
any new, abnormal growth of tissue,\footnote{Concomitantly, \emph{malignant}
  neoplasia is defined by the acquired capacity of neoplastic cells to invade
  locally and metastasize.} the original oncogenic hit is usually associated
with a mutation disrupting the balance between proliferation and cell death.
From then on, cancer progression is modeled as a reiterative process of clonal
expansion, with sequential subclonal
selection.\cite{nowell_clonal_1976,greaves_clonal_2012} The dynamics of this
evolution are dictated by successive genetic and epigenetic\footnote{In this
  text, \emph{epigenetics} will refer to the range of global modifications in
  gene expression that are not under control of the genetic code itself.  The
  modifiable and reversible nature of certain cancer programs can largely be
  explained by epigenetics.} changes in the neoplasm, and are constrained by the
ecological context in which the tumour is developing.

The prevailing mode of clonal evolution is through the gradual emergence of
selectively advantageous ``driver'' genetic injuries against a complex
background of mostly deleterious and selectively neutral ``passenger'' lesions.
Alternatively, or perhaps concurrently, another mode of tumoural evolution
contemplates the possibility of a few drastic events generating multiple lesions
at once across the genome.  These dramatic punctuated changes can be prompted by
an acute insult or a single catastrophic pan-genomic event---of which
chromothripsis, at the chromosome level, is an
example.\cite{stephens_massive_2011}\footnote{The argument of gradualism versus
  punctuated equilibrium (\citealp{gould_punctuated_1993}) is a longstanding
  debate in species evolution and is another example of how much our current
  conceptualization of cancer progression owes to the developments of the theory
  of evolution in the second half of the 20\textsuperscript{th} century.}

The time frame of somatic evolution is a function of the mutational rate of
neoplasms.  While events like chromothripsis and
kataegis\footnote{\emph{Kataegis}, a term derived from the ancient Greek word
  for ``thunder'', refers to a pattern of localized hypermutation identified in
  some cancer genomes (\mbox{\citealp{nik-zainal_mutational_2012}}).} have the
potential to provide nearly instant triggers for the onset of disease, the high
frequency of clinically covert \mbox{pre-maligant}
lesions\cite{sakr_frequency_1993} suggests that transformation of somatic cells
is a far more frequent event than advised by incidence curves.  Furthermore, the
fact that the majority of cancers only become clinically relevant at old age is
a testament to both the prevalence of cancer-suppressing mechanisms as to the
relatively slow rates of mutational accretion in neoplasms.  Intriguingly, the
rate of epigenetic change has been reported to be orders of magnitude higher
than that of genetic change,\cite{siegmund_inferring_2009} but its role in
clonal evolution is not yet completely understood.

% The interaction of a tumour with its milieu is a complex, dynamic and
% reciprocal affair.

While the evolution of neoplasms is driven by their underlying genetic and
epigenetic lability, it is their tissue ecosystems that provide the adaptive
landscape for clonal fitness selection.\cite{greaves_clonal_2012} Systemic
regulators, such as hormones, growth factors, immune and inflammatory cells as
well as cytokines may conspire either to counteract or promote neoplastic
growth.\cite{bierie_tumour_2006,hanahan_hallmarks_2011} Architectural
constraints, in the form of physical compartments, basement membranes and
confined metastatic niches, restrict the growth of tumoural masses and set a
boundary for neoplastic microevolution.  But perhaps most striking is the
ability of tumours to remodel tissue micro-environments to their competitive
advantage---illustrated by the capacity of transformed cells to promote
neovascularization in response to anoxia or to incite malignant phenotypes in
their adjacent stromal cells.\cite{lathia_deadly_2011}

\subsection{The linear model of cancer progression}
\label{sec:models-cancer-progression}

At the phenotipic level, the course of neoplastic evolution is tagged along two
major defining axis: one concerning increasing proliferation rates and another
reporting the loss of morphological and physiological differentiation at the
cellular level.\cite{tarabichi_systems_2013}

% The paradox resides then in explaining how transformed genomes, torn by their
% karyotypic instability and ragged by waves of disrupting mutations, are able
% to cope with the formidable challenges evolved by multicellular organisms to
% control cellular defection.

In order to grow and become clinically conspicuous, neoplastic masses must shut
down \mbox{built-in} genetic programs acquired during the establishment of
multicellularity, to enforce compliance with societal rules.  These include
programs evolved to eradicate deviant phenotypes, such as apoptosis, senescence
and necrosis and a range of \mbox{cell-to-cell} signaling programs designed to
control and suppress unwarranted growth.  Perhaps most challenging, cancer cells
must also persistently evade the predatory immune response of the host organism
in its diverse ecological contexts.

For neoplasms can also invade surrounding tissues and disseminate in the
organism, a feature responsible for most \mbox{cancer-related} deaths.  This
process is termed the \mbox{invasion-metastasis}
cascade\cite{valastyan_tumor_2011} and requires an array of well coordinated
genetic and epigenetic adaptations.  It can be schematized as a sequence of
discrete steps that starts with the local invasion of the surrounding tissues.
It then follows with the intravasion of cancer cells into nearby blood and
lymphatic vessels---and with their extravasion into the parenchyma of distant
tissues.  Eventually, new micrometastatic lesions are established with the
potential to develop into macroscopic tumours.

% , followed by the intravasion of cancer cells into nearby blood and lymphatic
% vessels.  Cancer cells in transit through the lymphatic and hematogenous
% systems then extravase these vessels into the parenchyma of distant tissues
% and eventually establish micrometastatic lesions---that may fully develop into
% macroscopic tumours.

To reconcile this increase in phenotypic resilience and plasticity of neoplastic
cells with the canonical multistep model of cancer
progression,\cite{land_cellular_1983,vogelstein_multistep_1993} carcinogenesis
is envisioned as a linear Darwinian evolutionary
process.\cite{merlo_cancer_2006,polyak_tumor_2014} According to this
view,\cite{podlaha_evolution_2012} inheritance, environmental factors and
spontaneous errors in \smallcaps{dna} replication cause mutations or
epimutations in critical caretaker genes, fomenting genetic and epigenetic
instability.  This, in turn, promotes mutations or epimutations in specific
gatekeeper genes\footnote{Among \mbox{cancer-susceptibility} genes, the
  distinction between \emph{caretaker} and \emph{gatekeeper} genes is a subtle
  yet conceptually important one.  While the former are responsible for
  maintaining the integrity of the genome, the latter directly control cellular
  proliferation.  This is epitomized by the \mbox{breast-cancer-susceptibility}
  genes \smallcaps{\emph{brca1}} and \smallcaps{\emph{brca2}}, that can
  functionally play both roles at different points of breast cancer progression
  (\citealp{kinzler_gatekeepers_1997}).}  (oncogenes or tumour supressor genes),
triggering uncontrolled growth.  The ensuing genomic instability creates a
feedback loop that increases evolutionary and proliferation
rates.\cite{sieber_genomic_2003} This results in heritable variation in the form
of clonal diversity, upon which Darwinian selection operates.  Clones that
progressively acquire the biological hallmark capabilities of
cancer\footnote{These include sustaining proliferative signaling; evading growth
  suppressors; resisting cell death; enabling replicative immortality; inducing
  angiogenesis; and activating invasion and metastasis
  (\citealp{hanahan_hallmarks_2011}).} gain a competitive edge and become
prevalent, promoting a form of neoplastic progression that, at the macroscopic
level, is consistent with a multistep development.

% \footnote{Interestingly, this
%   modern consensus on tumoural evolution is thus not very distant from what
%   Peter Nowell proposed in his 1976 landmark paper
%   (\citealp{nowell_clonal_1976}) postulating that ``neosplasms arise from a
%   single cell of origin, and tumour progression results from acquired genetic
%   variability within the original clone allowing sequential selection of more
%   aggressive sublines.''}

This reasoning is supported by the observed \mbox{step-wise} progression of most
solid tumours.  For instance, the gradual evolution of colon cancer is well
documented.\cite{vogelstein_multistep_1993} Here, the first manifestations of
neoplasia occur in the colorectal epithelium, in the form of small benign
adenomas.  Such tumours are reasonably confined and are almost normal in their
intra- and intercellular organisation.  With time, adenomas start to grow
(proliferation) and become morphologically and physiologically disorganized
(dedifferentiation).  Eventually the tumour evolves into an aggressive neoplasia
(carcinoma), presumably because one of the cells in the adenoma has acquired a
sufficient number of mutations to drive the process of invasion and metastasis.

The concept of gatekeeper gene is central to this understanding.  Consider for
instance the \smallcaps{\emph{tp53}} gene, the first \mbox{tumour-suppressor}
gene to be identified in 1979.  This gene encodes for the p53 protein, a master
transcription factor with an overarching role in the maintenance of the
integrity of the genome.\cite{efeyan_p53:_2007} Under normal circumstances, p53
is functionally inactive due to its rapid degradation.  However, upon the
infliction of virtually any form of cellular stress, p53 degradation is halted,
and the protein gains full competence in transcriptional activation.  The
regulatory networks under its control are associated with several critical
mechanisms for cancer progression, namely apoptosis, \mbox{cell-cycle}
inhibition, genome stability and inhibition of
angiogenesis.\cite{vogelstein_surfing_2000} Unsurprisingly, about 50\% of all
human cancers have lost p53 expression or express an inactive mutant of the
protein.\cite{toledo_regulating_2006}

The evolutionary origins of the \smallcaps{\emph{tp53}} gene can be inferred
from modern-day descendants of both the single cell choanoflagellates and the
early multicellular sea anemone.  The function of the homolog to this ancestral
gene in the sea anemone is to protect the germ-line gametes from \smallcaps{dna}
damage.\cite{belyi_origins_2010} Over the course of the last billion years, this
function has not only been conserved, but enhanced through
pleiotropy\footnote{The ability of a single gene to influence multiple,
  seemingly unrelated phenotypic traits.} to set \smallcaps{\emph{tp53}} as a
major enforcer of somatic conformity in multicellular organisms.

Another recurrent oncogene with an established role in \mbox{cell-cycle}
progression and apoptosis, \smallcaps{\emph{myc}}, has had its evolutionary
roots traced back to at least 600 million years ago.\cite{hartl_stem_2010} The
fact that many cancer-susceptibility genes are ancient, highly conserved and may
have taken a role in the transition to
multicellularity\cite{srivastava_amphimedon_2010} has been interpreted as
evidence that they play a pivotal role in regulating the normal physiology of
the somatic cell.\cite{weinberg_oncogenes_1983,weinberg_biology_2013} Their
disruption during carcinogenesis could then symbolize the unshackling of the
transformed cell from its social bindings.

\subsection{Shortcomings of the linear model of cancer progression}
\label{sec:shortcomings-canonical-model}

This model describes the neoplastic cell as an egotistic rogue cell set free to
thrive through uncontrolled replication in a hostile environment.  It becomes a
metaphor for the unicellular evolutionary stage, with cancer lineages competing
with each other and with normal cells for survival.\cite{merlo_cancer_2006} The
success of any one lineage is dependent on its \mbox{step-wise} acquisition of
cancer hallmarks through Darwinian evolution.  However, this formulation does
not fully account for a number of observations.\cite{davies_cancer_2011}

First, it falls short to explain the high degree of cooperative organization
among cancer cells.  Increasingly, tumours are being recognized as ecosystems
with complex and dynamic interactions between neoplastic cells and their
microenvironment.\cite{polyak_co-evolution_2009} These heterotypic
reciprocations include the stimulation via paracrine signaling of normal stromal
cells to produce mitogenic signals; the required signaling to induce
neoangiogenesis; and the diverse cell-to-cell interactions during the
invasion-metastasis cascade.\cite{axelrod_evolution_2006} Thus, in order for
transformed cells to prosper, they must acquire a very specific minimal set of
communication skills from early on.  Rather than conceiving renegade cells
evolving such intricate adaptations independently via adaptive mutations, a more
prosaic explanation could involve a shift or modulation of the original set of
rules somatic cells use to engage with their partners.

Second, it fails to elucidate the quasi teleological way by which advanced
tumours consistently resort to the same set of sophisticated adaptations to
deceive their host organisms.\cite{hanahan_hallmarks_2011}
% At their end stage, malignant carcinomas could almost be taken for entities
% bestowed with a volition of their own, capable not only to evade the host
% organism's defenses, but to shrewdly overtake it.
This deceptively deliberate neoplastic progression is conventionally accounted
for by a Darwinian evolutionary process among competing lineages of rogue
cells.\cite{merlo_cancer_2006} However, some adaptations are particularly
challenging to frame in this view.  A notorious example is the progression
puzzle suggested by Bernards and Weinberg: how can mutations promoting invasion
and metastasis add a competitive edge to the primary tumour in its original
ecological context?\footnote{To solve the paradox, Bernards and Weinberg
  suggested a mechanistic model according to which ``the tendency to metastasize
  is largely determined by the identities of mutant alle- les that are acquired
  relatively early during multistep tumourigenesis''
  (\citealp{bernards_progression_2002}).}  In addition, as the majority of
variants randomly arising on a genetic system are expected to be deleterious,
the odds of any neoplastic cell incrementally acquiring all the required
hallmark mutations before collapsing to a lethal one is exceptionally low.

Third, it doesn't fully integrate the role of genetic instability in cancer
progression.  Because advantageous mutations are rare, rogue cells are thought
to promote genomic instability in order to accelerate evolutionary rates and
increase the odds of, via Darwinian selection, produce a fully malignant
phenotype.\cite{sieber_genomic_2003} Yet, this mutational arms race among rogue
cell lineages to reach the jackpot of complete neoplastic competence is riddled
by a paradox: too few mutations in the mix and the cell won't escape genetic
controls; too many and it dies.  Especially troublesome are the
\mbox{pan-genomic} mutations that lead to gross structural changes, including
aberrant chromosomes and aneuploid cells.\footnote{``If you look at most solid
  tumours in adults, it looks like someone set off a bomb in the
  nucleus''---\emph{William C. Hahn}} It is thought-provoking to note that
the neoplastic cells with the most deranged genomes are precisely those with the
most competitive phenotypes.  Jarle Breivik proposed an elegant solution to this
inconsistency.\cite{breivik_evolutionary_2005} Rather than postulating genomic
instability as a pre-requisite to cancer progression, he refashioned it as a
\mbox{by-product} of the lack of competitive fitness of reparing-phenotypes in
the tumour environment.\footnote{``Don't stop for repairs in a war zone'' is the
  metaphor used by Breivik to explain why a mutagenic environment would increase
  the fitness of the non-repairing phenotype.  % This conceptual
  % reformulation could have implications on other central controversies in cancer
  % research---as discussed bellow.
}

A tentative proposition to address these inconsistencies has been put forward by
Davies and Lineweaver.\cite{davies_cancer_2011} Instead of modeling the
transformed cell as a free-agent requiring the cumulative \emph{acquisition} of
enabling properties to obtain malignant status, it seeks to explain hallmark
adaptations as the reenactment of atavic genetic systems already embedded in the
genomes of somatic cells.  According to this view, the disruption of high-order
gatekeeper genes drives the neoplastic cell to reconfigure its regulatory
networks around a \emph{pre-existing} toolkit of primitive adaptations
reminiscent of those evolved during early transition to multicellularity.
Although its original formulation has been either ignored\footnote{At the date
  of this writing, Google Scholar reported a total of 36 citations of the
  original article.} or thoroughly
dismissed,\cite{pettit_cancer_2012,myers_aaargh!_2012} this idea could provide a
basis for a reevaluation of some core features of cancer.

Consider the Warburg effect\cite{warburg_origin_1956} for instance.  The
majority of cancer cells favour a metabolism based on anaerobic glycolysis
followed by lactic acid fermentation in the cytoplasm.  Because glycolysis is
far less efficient than oxidative respiration for \smallcaps{atp} production,
this metabolic shift has been implied to be an essential adaptation to cancer
progression---perhaps in response to intermittent hypoxia in malignant
lesions.\cite{gatenby_why_2004} A more economical interpretation could be that
transformed cells, without gatekeeper genes to enforce the proper regulation of
the high-performing metabolic mode of somatic cells, are now forced to fall back
to a more basic, yet dependable, metabolic outlet for energy production.  While
the former interpretation taxes neoplastic progression with another requirement
and begs for a rationale for the selective fitness of anaerobic metabolism, the
latter simply conceives the cancer cell as a system seeking a dynamic
equilibrium in a novel and unstable environment.

More recently, another key concept in cancer research has been the cancer stem
cell.  Tissue-specific stem cells have been identified at the top of the
differentiation hierarchy of many organs.  These stem cells are functionally
defined by their long-term self-renewal capacity and their ability to
differentiate into one or more tissue lineages.  This hierarchical organization
of tissue differentiation has been co-opted to explain tumour growth and
heterogeneity, with cancer-specific stem cells responsible for the maintenance
and growth of tumours.  Cancer stem cells (\textsc{csc}s), or tumour initiating
cells, are operationally defined by their ability to re-form the parental tumour
on transplantation into immunodeficient mice.  They have been isolated from a
range of solid tumours, such as breast cancer, brain tumours, colorectal cancer,
and others.\cite{beck_unravelling_2013} Fittingly, this model complies with the
requirements of clonal evolution, as tumour progression can be explained by
derivatives of \textsc{csc}s bearing different mutational signatures competing
with each other.  This interpretation suggests that all cancer cells in a tumour
share a unique genetic and epigenetic history, in accordance with a linear
progression.  However, the reported coexistence of multiple genetic clones
during acute lymphocytic leukemia progression suggests a more dynamic and
modular clonal architecture
instead.\cite{anderson_genetic_2011,notta_evolution_2011} % Evidence of
% intra-tumoural heterogeneity does not necessarily inform the underlying model
% of cancer progression.
This evidence, together with the fact that phenotypic conversion also occurs
among non-hierarchically organized tumour cells in
melanoma,\cite{quintana_phenotypic_2010} indicates that the \smallcaps{csc}
phenotype may be a transient response to the selective pressures of the tumour's
milieu and of the stochastic events that govern its internal
homeodynamics.\cite{visvader_cancer_2012,aktipis_life_2013} The \smallcaps{csc}
would then be, rather than a deterministic, a dynamically reversible phenotype;
and represent, instead of qualitative, a quantitative
modulation.\cite{maenhaut_cancer_2010,tarabichi_systems_2013}

% Stem cells play a crucial role not only during development, but during tissue
% homeostasis and repair as well.  These cells are operationally defined by
% their ``capacity for self-renewal, the potential to develop into any cell in
% the overall tumour population, and the proliferative ability to drive
% continued expansion of the population of malignant cells.''
% (\citealp{jordan_cancer_2006}) While the idea of a unique, genetically
% homogeneous population of determined cancer stem cells is still
% contested,\cite{marotta_cancer_2009} the existence of a particular class of
% tumour propagating cells within neoplastic masses is well accepted.  Here, the
% crux of the matter concerns the ontology of the \smallcaps{csc} phenotype.  In
% short, \smallcaps{csc}s have been proposed to arise from mutations in either
% derivatives of developing stem or progenitor lineages, or from mutations in
% differentiated cells that acquire stem-like
% attributes.\cite{wicha_cancer_2006,lobo_biology_2007} As niche-specific stem
% cells are subjected to higher turnover rates to seed their respective tissues,
% they become prime targets for neoplastic-inducing replication defects.  The
% potential of a seeder neoplastic cell at the apex of the tumoural clonal
% hierarchy is luring both from the diagnostic and treatment perspective.
% Moreover, this conceptualization seems especially fitting for the established
% linear Darwinian clonal progression model.  However, there is evidence that
% these cells do not constitute one homogeneous population.

Another example of the phenotypic modularity of the neoplastic cell is the
epithelial-mesenchymal transition program (\smallcaps{emt})---and its reverse
process, the mesenchymal-epithelial transition (\smallcaps{met}).  As with the
\smallcaps{csc} phenotype, this mechanism is borrowed from embryogenesis, where
it takes part in gastrulation, neural crest formation, heart valve formation,
palatogenesis and myogenesis.\cite{thiery_epithelial-mesenchymal_2009} This
comprehensive genetic program globally shifts the physiology and morphology of
the cell between an epithelial phenotype (characterized by a well defined
polarity, tight junction of the cells and a their binding to a basal lamina) and
a mesenchymal phenotype (characterized by a lack of polarity, spindle-shape
morphology and loose cell-to-cell interaction).\cite{thiery_complex_2006} Under
normal physiological circumstances, the dynamics of the \smallcaps{emt-met}
program are under strict and orderly genetic control, even if it can be
re-enacted in a post tissue differentiation context---for instance in response
to injury.  In pathological conditions, the unwarranted activation of the
\smallcaps{emt-met} program can cause organ fibrosis.  But most importantly, it
provides cancer cells with an outlet to break free from the primary tumour and
embark the invasion-metastasis cascade under the cover of the mesenchymal
phenotype.  Conversely, the colonization of new micrometastatic niche is
facilitated by a transition back to an epithelial phenotype.  This constitutes
the most thorough illustration of how neoplastic cells can recruit built-in
genetic apparatuses to deploy complex adaptations.

% Thus, cancer cell lineages could evolve by exploiting the realm of new
% homeodynamic equilibriums made available by the breaching of higher-order
% somatic regulatory systems.  Because these regulatory circuitries are kept in
% check by gatekeeper genes, this would entail that the number of required
% breaking points for the potential emergence of full neoplastic competence is
% considerably smaller than anticipated. The acquisition of the hallmark
% adaptations could simply be the outcome of a stochastic process of
% readjustments of regulatory networks of the cell.

% Even Weinberg flirted with such possibility: ``Maybe the information for
% inducing cancer was already in the normal cell genome, waiting to be
% unmasked.''\footnote{\citealp{weinberg_biology_2013}, \emph{p} 79} Could this
% idea be further developed?  Consider the main criticism to the hypothesis of
% Davies and Lineweaver: their invocation of an \emph{atavistic} genetic system,
% in itself largely responsible for the cancer phenotype, already packed in our
% genomes and ready to be unleashed with the disruption of gatekeeper
% genes.\footnote{``In short, proto-metazoans, which we dub Metazoans 1.0, were
% tumor-like neoplasms.'' (\citealp{davies_cancer_2011})} This proposition is
% perhaps too easy to caricature and to dismiss, but it does share one aspect in
% common with the linear Darwinian model for tumourigenesis: both seek to
% explain cancer almost exclusively through its genetic determinants.  The two
% theories differ, however, in the nature of the qualitative changes required
% for tumor progression.  While clonal evolution posits for an \emph{ab initio}
% generation of genetic determinants for cancer progression, the Metazoan theory
% short-circuits this precondition by allocating the required determinants in
% the genome of the somatic cell.  None of these perspectives fully addresses
% the perplexities of cancer; both seem to ignore the epigenetic determinants
% for cancer progression.

% In reality, the predicament of the neoplastic cell might be much more complex.
% Evolution has concocted a vast collection of modular genetic programs to
% provide the somatic cell with adaptive dynamic physiological ranges within
% which to carry its activities.  In order for the orderly integration of the
% somatic cell at the tissue and organism level, such physiological ranges must
% be under a tight genetic control of a higher-order.  The breakdown of this
% level of control leaves the somatic cell hapless on how to keep its metabolism
% balanced and free to stochastically reconfigure its genetic regulation.  This
% may lead to the co-optation of genetic programs via natural selection as they
% increase the fitness of the transformed cell in particular ecological
% contexts.  Because these programs usually end up serving an adaptive role
% other than the one they were evolved to provide, perhaps cancer adaptations
% ought to be better described as
% \emph{exaptations}\cite{gould_exaptation_1982}---a concept that leaves at rest
% the teleological interpretation of cancer hallmarks.

% But let us speculate a bit further.  The evolutionary success of the
% neoplastic cell is only bounded by its ability to self-replicate.
% Consequently, with the exception of the genetic systems that support the cell
% cycle and the remaining context-specific hallmark adaptations, the vast
% majority of the transformed genome could be at the mercy of disruption through
% genetic instability.  This alone could explain the second major
% axis\footnote{The first being increased proliferation.} of cancer progression:
% the gradual loss of morphological and physiological differentiation of
% neoplasms as they evolve through the ranks of malignancy.  But there is
% another neglected driver of cancer evolution, one that could prove to be the
% hidden iceberg upholding the shapeshifting nature of cancer.

% Very little is know about the non-genetic, or epigenetic, determinants of
% cellular metabolism.  Chromatin remodeling and \smallcaps{dna} methylation are
% known to modulate to a considerable degree patterns of gene expression;
% \smallcaps{rna} transcripts and their encoded proteins may retroactively
% modulate, direct or indirectly, the activity of certain
% genes;\footnote{\smallcaps{rna} mollecules may even directly spread directly
% to other cells or nuclei by diffusion.} a variety of less known species of
% \smallcaps{rna} are animated with catalytic activity of their own and their
% transcriptional output has been historically
% underestimated.\cite{huttenhofer_principles_2006,ptashne_use_2007} This family
% of non-protein-coding \smallcaps{rna}s are thought to be relics of a
% primordial ``\smallcaps{rna} world'', in which \smallcaps{rna} served both as
% the carrier of genetic information and as the catalytic agent.  If cancer
% cells are able to resourcefully and adaptively tap on the genetic resources of
% the somatic cell to shape their evolution, why would they neglect the rich
% diversity of epigenetic tools at their disposal?

% Perhaps most provocatively, as these epigenetic systems are not necessarily
% constrained by the dynamics of Darwinian evolution,\footnote{And could indeed
% abide by the rules of Lamarckian evolution.} their evolutionary rates could be
% orders of magnitude higher than those observed in genetic systems---even those
% riddled by genetic instability.  This could provide a potent mechanism driving
% cancer evolution, one that might potentially account for the extraordinary
% degree of resilience of advanced cancers.

% These considerations could have profound implications in the way cancer
% research, diagnostic, prognostic and treatment are conducted.

% Unlocking these mysteries may well require a lot of ingenuity, but without
% doing so it will be hard to elucidate the intricate nature of life---and that
% of its dark horse, cancer.

% Genetic pleiotropy.  Answer: rather than a multistep progression driven by
% linear accretion of driver mutations supporting the discrete and independent
% acquisition of malignancy hallmarks, a global transcriptional shift
% encompassing the activation of ldots{} Concept of Pleiotropy.  Concept of
% \emph{exaptations}.

% \smallcaps{\emph{tp53}} and \smallcaps{\emph{myc}}.

% The resilience, plasticity and insidiousness of malignant neoplastic cells is
% eerily reminiscent of the \ldots{}

% Final argument about Darwinian evolution, hallmarks of cancer, and eventual
% atavic genetic circuitry engaged once system gateway controlers, such as p53,
% fail.  Frame cancer as an evolutionary problem To discuss: Darwinian evolution
% of cancers, multistep model of cancer progression, hallmarks of cancer,
% critics of the hallmarks, oncogenes as evolutionary gatekeepers, examples
% centered around \smallcaps{\emph{tp53}} and \smallcaps{\emph{tp53}} oncogenes,
% integration of cancer programs, Warburg effect, a progressin puzzle: example
% with \smallcaps{EMT} program, metastasis is a highly inefficient process,
% similarities between cancer phenotype and early developmental stages, concept
% of cancer stem cell, concept of \emph{exaptation}.  Example: Evolutionary
% history of the retinoblstoma gene from rchea to eukaria.  Implications for
% both the understanding of multicellularity from an evolutionary perspective as
% well as for the understanding of the biology of cancer.

% However these complex considerations on the biology

% Cancer dynamics are bounded by genetic and epigenetic
% alterations.\footnote{From Nowell, 1976: reversibility of transformation in
% certain culture systems suggests that cancer initiation could result from
% altered gene expression rather than structural mutation}

% Cells of the neoplastic lineage are defined by their ability to sustain
% chronic proliferation, irrespectively of the social cues conveyed by its
% tissue context.  At the morphological level, however, the most conspicuous
% feature of cancer has to be the diversity of shapes and forms these
% proliferating masses can take when departing from the neatly organized
% architectural tissue types they arise from.  An integrative framework to
% apprehend this remarkable phenotypic plasticity has been proposed by Hanahan
% and Weinberg,\cite{hanahan_hallmarks_2000,hanahan_hallmarks_2011} who proposed
% six essential and complementary capabilities for tumour growth and metastatic
% dissemination.  These include self-suficiency in growth signals, insensitivity
% to growth-inhibitory signals, evasion of programmed cell-death, limitless
% replicative potential, sustained angiogenesis, and tissue invasion and
% metastasis.

% Cellular transformation, the process through which the neoplastic phenotype
% arises, is caused by genetic and epigenetic alterations in somatic cells.
% Under regular circumstances, most somatic cells exhibiting behaviours beyond
% physiological ranges end up being singled out and targeted for removal, either
% by eliciting an immune response or by triggering self-induced cellular death
% (\emph{note about apoptosis here}).  In order for a neoplasm to become
% clinically relevant, it has thus to acquire a number of alterations that
% consign it with the capacity to evade its host organisms' regulatory control
% mechanisms against unicellular defection.

% an operating outside of normal physiological levels cells abnormal behaviour
% exhibited by most early neoplastic cells may be sufficient for them to be
% recognized and targeted for removal, either by eliciting an immune response or
% by triggering self-induced cellular death (\emph{note about apoptosis here}).

% Clonality; multistep model for the nature of cancer;

% Specific genes, termed oncogenes, have the potential to induce transformation
% when disrupted in particular circumstances.  Alterations in nearly 500 of such
% genes have been linked to cancer initiation and
% progression.\cite{forbes_catalogue_2008} The multistep model for the nature of
% cancer posits that several such alterations are cumulatively required in order
% to initiate tumourigenesis and to evolve increasingly more aggressive and
% invasive tumour phenotypes.\cite{vogelstein_multistep_1993}

% The fundamental and defining characteristic of the cancer cell is, arguably,
% it's ability to sustain chronic proliferation.

% The number and patterns of somatic alterations vary dramatically across cancer
% types.  At one extreme, childhood medulloblastomas can harbour fewer than ten
% genomic alterations, whereas over \SI{50000} somatic changes have been
% observed in primary lung adenocarcinoma samples.

% In biological systems the instanciation of information is \smallcaps{DNA}.

% a phenotype eerily reminiscent of that seen in loosely cooperative unicellular
% life forms, at a time when multicellularity was still being attempted at.

% From the prologue of 'The Emperor of All Maladies': Malignant and normal
% growth are so genetically intertwined that unbraiding the two might be one of
% the most significant scientific challenges faced by our species. Cancer is
% built into our genomes: the genes that unmoor normal cell division are not
% foreign to our bodies, but rather mutated, distorted versions of the very
% genes that perform vital cellular functions.  individual alterations in
% oncogenes are seen as necessary but not sufficient to give rise to cancer.
% Considered to arise sequentially and to give rise to the progressively more
% aggressive and invasive phenotypes observed during tumourigenesis.

% which these control mechanisms are compromised lead by mutations on specific
% genes which these mechanisms are hindered is through the accumulation of
% somatic mutations of oncogenes


% Cells of normal tissues collectively control their growth rate by regulating
% the production and release of paracrine \mbox{growth-promoting} signals that
% direct entry and progression through the cell \mbox{growth-and-division}
% cycle.  A first requirement for the acquisition of the neoplastic identity
% must then be the ability to generate endogenous mitogenic signals that result
% in autocrine proliferative stimulation.\footnote{This endeavour can also be
% achieved through the emission of signals that stimulate surrounding normal
% cells to feed cancer cells back with growth factors
% (citealp{Cheng-2008,Bhowmick-2004}); specific somatic mutations that
% constitutively activate pathways usually triggered by activated growth factor
% repectors; or through disruptions of \mbox{negative-feedback} mechanisms that
% attenuate proliferative signaling.}  In addition, the cancer cell must also
% become insensitive to social cues designed to control unrestricted cell
% proliferation.

% From jcb-weinberg-1983.pdf:

% This leads to the realization that these [onco]genes are of extremely ancient
% lineage---their precursors were already present in similar form in the
% primitive metazoans taht served as common ancestors to chordates and
% arthropodes.  Such conservation indicates that these genes have served vital,
% indispensable function in normal cellular and organismic physiology, and that
% their role in carcinogenesis represents only an unusual and aberrant diversion
% from their usual functions.

% by caused by steered by the engagement of genetic routines of the form of an
% atavistic state engagement of genetic routines active at an earlier stage of
% development that are inappropriately reactivated in the mature organism as a
% result of some sort of insult of damage.  Cancer is an atavistic state of
% multicellular life.  Cells relieved of the molecular constrains that
% subordinate them to societal discipline.

% Causing cancer cells' metabolism to default to more fundamental modes of
% functioning not unlike those conceivably typified by loosely societal cellular
% sorts.

% These six hallmarks of cancer---distinctive and complementary capabilities
% that enable tumour growth and metastatic dissemination

% Further evidence that supports our theory comes from experiments in which the
% nuclei of egg cells are replaced with cancer-cell nuclei.  Astonishingly,
% embryos start to develop normally.  But abnormalities eventually appear, at
% earlier stages when the cancer is more malignant (advanced). This inverse
% correlation of cancer stage with embryo stage is consistent with our theory.

% Cancer follows a well-defined progression of within a host organism of
% accelerating growth (proliferation), mobility, spread and colonization.

% To better understand cancer, including its place in the great sweep of
% evolutionary history.

% ideias to discuss:

% causes of cancer

% field cancerization (tarabichi)

% 1--For multicellularity to become a viable strategy, a number of strategies
% must have evolved in order to maximize cooperation and minimize conflict
% between individual cells of an organism.

% 2--One such strategy is the imposition of a genetic bottleneck per generation
% in the course of the reproductive cycle of multicellular organisms in order to
% ensure clonality among constituent cells of an organism.

% 3--As cooperation creates new levels of fitness, it creates the opportunity
% for conflict between levels as deleterious mutants arise and spread within the
% group.  Fundamental to the emergence of a new higher-level unit is the
% mediation of conflict among lower-level units in favor of the higher-level
% unit.

% 4--Michod et al. (2003) lists a number of conflict-mediating adaptations that
% can be useful in engendering cooperation among physically associated cells.
% These include the existence of a germ line, the tendency for unnecessary or
% even defecting cells to undergo apoptosis, the potential for self policing,
% display of relatively low mutation rates, canalization of growth (of which one
% mechanism is "determinate growth"), etc.

% 5--Take the \emph{p53} gene as example.  \emph{p53} has likely been the most
% studied gene for over a decade.

% cancer dormancy is a also perplexing

% pleiotropy of the key oncogenic gatekeepers

% quote the process of ``exaptation'' (Gould & Vrba, 1982) the role of
% epigenetics

% The conventional argument to explain the deployment of this \mbox{swiss-knife}
% armory of solutions is to invoke Darwinian evolution between \mbox
% {sub-clones} of the original neoplasm.\cite{Merlo-2006} According to this
% hypothesis, the evolution of the required survival traits for the success of
% the tumour would be attributed mostly to random mutations and the trial and
% error of normal Darwinian evolution.  Rapid mutation rate within tumours +
% strong selective pressure as organism ``fights back'' or patient undergoes
% chemotherapy.

% This understanding of cancer as a feature of a particular evolutionary
% mechanism

% Rewrite discussion on multicellularity in the life section according to the
% lines discussed here:
% http://www.biologyaspoetry.com/textbooks/microbes_and_evolution/greater_size_problems_in_multicellularity.html

% NOTES from tmm-floor-2012.pdf: hallmarks of cancer: of all cancer cells, all
% the time?

% The hallmarks of cancer 1-self-sufficiency in growth signals 2-insensitivity
% in anti-growth signals 3-evasion of apoptosis 4-limitless replicative
% potential 5-sustained angiogenesis 6-tissue invasion 7-metastasis 8-metabolic
% reprogramming 9-evasion of the immune system

% Strangely, one fundamental characteristic of cancer cells, the loss of
% differentiation, was not considered a distinct hallmark.  This characteristic
% is essential because it is a primary difference between benign and malignant
% tumours (e.g., autonomous adenomas as opposed to carcinoma of the thyroid).
% It is supported by a measurable expression program switch that affects all
% gene categories.  We argue that this should be another, if not the major,
% ``hallmark'' of a cancerous cell, and it has been shown to be a robust
% diagnostic index for thyroid tumours.

% new hallmarks of cancer must be able

% cancer progression is characterized by a succession of oncogenic events

% The hallmarks of cancer (cel-hanahan-2011) The hallmarks constitute an
% organizing principle for rationalizing the complexities of neoplastic disease.
% They include: 1--sustainment of proliferative signalling 2--evasion of growth
% supressors 3--resistance to cell death 4--enabling of replicative immortality
% 5--inducing of angiogenesis 6--activation of invasion and metastasis

% The genomes of nearly all healthy human cells, containing the entirety of an
% individual's inherited information, evidently come pre-loaded with a ``cancer
% sub-routine'' that is normally idle but can be triggered into action by a wide
% variety of insults, such as chemicals, radiation and inflammation.

% Breivik 2005

% Cancer incidence increases with age, and in countries with high life
% expectancy, aproximately 40\% of the population experiences some kind of
% cancer in their lifetime.  Adiotionally, a significant proportion of the
% autopsies reveal undiagnosed maliganacies, and virtually everybody carry some
% kind of in situ carcinomas after the age of 50.

% Clinical definition of cancer. A vast collection of diseases (almost 100,
% according to http://medical-dictionary.thefreedictionary.com/Cancer).

\clearpage

\subsection{The epidemiology of cancer}
\label{epidemiology-cancer}

\newthought{But cancer is also} a leading cause of death worldwide, accounting
for 8.2 million deaths in 2012.  From the clinical point of view, cancer is a
general designation for a group of more than 100 diseases.  Lung, liver,
stomach, colorectal and breast cancers are responsible for the majority of
cancer deaths each year.  The most frequent types of cancer, as well as their
incidence, differ between women and men (Figure~\ref{fig:globocan}).

\begin{figure}[ht]
  \includegraphics{globocan.pdf}
  \caption[Global estimates of cancer incidence and mortality by
  sex][6pt]{Global estimates of cancer incidence and mortality by sex.
    \mbox{Age-standardized} rate per \SI{100000} population (2012).  Source:
    \mbox{\smallcaps{globocan}} (\citealp{ferlay_globocan_2014}).}
  \label{fig:globocan}
\end{figure}

The list of factors involved in the causation of cancer is wide and diverse.
Heritable genetic susceptibility, in the form of highly penetrant, dominant
allelic variants, could account for 2--5\% of fatal cancers.  Environmental
factors, including smoking, alcohol consumption, dietary habits and infectious
agents of viral (e.g., \smallcaps{hpv}) or bacterial (e.g., \emph{Helicobacter
  pylori}) origin, are responsible for varying degrees of cancer
susceptibility.\cite{cassidy_oxford_2010} According to the estimates of the
American Cancer Society, approximately 40\% of cancer deaths in 1998 were due to
tobacco and excessive alcohol use.  The 1996 Harvard Report on Cancer Prevention
concluded that over 90\% of malignant melanoma is attributable to solar
radiation.  While other exposures, such as radiation and environmental
pollutants, could account for up to 5\% of the cancer burden, few causal links
with other potential carcinogens have been firmly established.

The medical act of assessing the degree of development and spreading of the
neoplastic disease is called staging.  Correct cancer staging is critical
because treatment (in the form of pre-operative therapy and/or adjuvant therapy)
and disease prognosis are based on this evaluation.  Staging systems are
specific for each type of cancer and are usually sanctioned by international
organizations, like the \smallcaps{uicc} and the
\smallcaps{ajcc}.\footnote{Respectively, the Union for International Cancer
  Control and the American Joint Committee on Cancer.}  The most prevalent
staging system mirrors cancer progression and classifies solid cancers according
to their surgical tractability: Stage 0 represents a tumour confined \emph{in
  situ}; stage \smallcaps{i}, a localized, still surgically removable tumour;
stage \smallcaps{ii} and stage \smallcaps{iii} describe locally advanced cancers
(the specifics depending on the type of cancer being staged); and stage
\smallcaps{iv} marks a metastatic cancer, spread to other organs throughout the
body.\cite{greene_ajcc_2002} Another staging system, the \smallcaps{tnm}
classification of malignant tumours, also applies to the majority of solid
cancers. It relies on the size and extension of the primary tumour; on its
lymphatic involvement; and on the presence of metastases to classify cancer
malignancy and to inform treatment proceedings.\cite{denoix_enquete_1946} For
breast cancer, the Nottingham modification of the Bloom-Richardson scale is most
commonly used. This grading scale classifies each cancer in a scale from 1 to
3---each describing, respectively, low-, intermediate- and high-grade
neoplasias.\footnote{This classification system positions cancer along a
  differentiation axis: from low-grade, well differentiated tumours; to
  high-grade, poorly differentiated ones.}

\subsection{Cancer research}
\label{cancer-research}

Most of these classification systems have been implemented during the second
half of the last century and reflect a compromise between the need to find
sensible and universal guidelines for cancer treatment and our unfolding
understanding of the disease.  Medical cancer research has come a long way since
the times when nearly every disease was attributed to the workings of some
invisible force such as bile, miasmas or bad humours (Table~\ref{tab:200years}).

\begin{table}[ht]
  \small
  \centering
  % \fontfamily{ppl}%\selectfont
  \begin{tabular}{lm{6.5cm}m{1.5cm}}
    \toprule
    Year & Discovery or Event & Relative \mbox{Survival Rate}\\
    \midrule
    1863 & Cellular origin of cancer (Virchow) & \\
    1889 & Seed-and-soil hypothesis (Paget) & \\
    1914 & Chromosomal mutations in cancer (Boveri) & \\
    1937 & Founding of \smallcaps{nci} & \\
    1944 & Transmission of cellular information by \smallcaps{DNA} & \\
         & (Avery) & \\
    1950 & Availability of cancer drugs through & \\
         & Cancer Chemotherapy National Service Center & \\
    1953 & Report on structure of \smallcaps{DNA} & 35\%\\
    1961 & Breaking of the genetic code & \\
    1970 & Reverse transcriptase & \\
    1971 & Restriction enzymes & \\
         & Passage of National Cancer Act--- & \\
         & ``War on Cancer'' declared by Nixon in the \smallcaps{usa}& \\
    1975 & Hybridomas and monoclonal antibodies & 50\%\\
         & Tracking of cancer statistics by \smallcaps{SEER} program & \\
    1976 & Cellular origin of retroviral oncogenes & \\
    1979 & Epidermal growth factor and receptor & \\
    1981 & Suppression of tumour growth by p53 & \\
    1984 & G proteins and cell signaling & \\
    1986 & Retinoblastoma gene & \\
    1990 & First decrease in cancer incidence and mortality & \\
    1991 & Association between mutation in \emph{\smallcaps{APC}} gene & \\
         & and colorectal cancer & \\
    1994 & Genetic cancer syndromes & \\
         & Association between \emph{\smallcaps{BRCA1}} and breast cancer & \\
    2000 & Sequencing of the human genome & \\
    2002 & Epigenetics in cancer & \\
         & micro \smallcaps{RNA}s in cancer & \\
    2005 & First decrease in total number of & \\
         & deaths from cancer & 68\%\\
    2006 & Tumour-stromal interaction & \\
    \bottomrule
  \end{tabular}
  \caption[Landmarks of 200 years of cancer research]{Some of the landmarks in the last 200 years of cancer research
    (adapted from \citealp{devita_two_2012}).}
  \label{tab:200years}
  % \zsavepos{pos:normaltab}
\end{table}

It was in 1863 that Rudolph Virchow, through the lens of a microscope, deduced
the cellular origin of cancer.\cite{virchow_cellular_1863} At once, cancer was
being recasted as the quintessential disease of hyperplasia and rescued back to
the somatic realm.\footnote{\emph{Omnis cellula e cellula}---every cell
  originates from a cell alike---, is the epigram popularized by Virchow,
  stating a shift from the tenet of spontaneous generation that dominated the
  19\textsuperscript{th} century school of thought concerning cancer's origins.
  Noting that this form of cellular multiplication was fundamentally novel and
  inexplicable, he coined it \emph{neo}plasia.}  Albeit localized in its origin,
cancer was nonetheless understood as a humoural disease, a systemic illness.
The supposition that cancer spreads in a centrifugal fashion from the primary
tumour to adjacent structures was the foundation for William Halsted to
introduce, in 1894, the radical mastectomy for breast cancer.\footnote{This
  surgical procedure demands that the breast, the underlying chest muscles and
  the lymph nodes of the axila be removed (\citealp{halsted_i._1894}).  From
  1895 to the mid-1970's, about 90\% of the women treated for breast cancer in
  the \smallcaps{usa} underwent radical mastectomy.}

During the 19\textsuperscript{th} century, surgery was the only known way to
treat cancer.  The first example of a cancer cure by surgery happened in 1809
with the remotion of an ovarian tumour without anesthesia.  Surgery protocols
were subsequently enhanced with the use of anesthesia, first reported in
1846,\cite{warren_inhalation_1846} and the introduction of antisepsis, in
1867.\cite{lister_antiseptic_1867} Halsted's radical mastectomy advocated the
\emph{en bloc} resection of the surrounding tissue to remove all cancer cells.
As cancers kept on relapsing locally after surgery, Halsted reasoned that more
and more tissue had to be extirpated in order to root out the last of malignant
cells.  This crept radical mastectomy into the ``super-radical'' and then into
the ``\mbox{ultra-radical}.''\footnote{This was an extraordinarily morbid,
  disfiguring procedure in which surgeons removed the breast, the pectoral
  muscles, the axillary nodes, the chest wall, and occasionally the ribs, parts
  of the sternum, the clavicle, and the lymph nodes inside the chest
  (\citealp{mukherjee_emperor_2011}).}  By the turn of the century, \emph{en
  bloc} resection became know as ``the cancer operation'' and turned into the
standard approach for the removal of all other cancers.

This surgical tradition came to be challenged in 1968 by Bernard Fisher, a
surgeon from Philadelphia.  Against the prevailing consensus, Fisher conducted a
series of clinical trials in the 1960's to compare the performance of radical
mastectomy with localized surgery (a ``lumpectomy''), supplemented with
radiation.  The results of these trials showed that \emph{en bloc} surgery was
no more effective to treat early-stage breast cancer than the combination of
surgical extraction of the tumour mass and radiation
therapy.\cite{fisher_five-year_1985,fisher_ten-year_1985} Radical mastectomy was
a fallacy that pushed too far when the tumour was localized, and not enough when
the cancer had turned metastatic.

In 1928, Henry Coutard, a radiologist from the Institut Curie in Paris, showed
that fractioned radiation treatments could be used to cure head and neck
cancers.\cite{coutard_roentgen_1932} In those days, the treatment was called
Roentgen therapy, after Wilhelm Röntgen, a lecturer at the Würzburg Institute in
Germany.  While working with an electron tube in 1895, Röntgen discovered a form
of radiant energy he called \smallcaps{x}-rays.  The discovery of radium in 1898
by Pierre and Marie Curie further opened the door to the use of radiation to
kill cancer by ``burning'' it.\footnote{It was not just cancer cancer cells that
  were being burned.  Marie Curie died of leukemia in July 1934.  Emil Grubbe,
  the first American to use \smallcaps{X}-rays in the treatment of cancer, had
  his fingers amputated to remove necrotic and gangrenous bones and his face cut
  up in repeated operations to remove \mbox{radiation-induced} tumours and
  \mbox{pre-malignant} warts. He died at the age of eighty-five to metastatic
  cancer (\citealp{mukherjee_emperor_2011}).}  All throughout the first half of
the 20\textsuperscript{th} century, this use of radiation therapy mirrored the
advent of the atomic age, with ``cyclotrons'', ``supervoltage rays'', ``neutron
beams'' and ``millions of tiny bullets of energy'' all harnessed to eradicate
what the surgeon's knife could not reach.\cite{mukherjee_emperor_2011} The
pinnacle of this era came in 1968, in Stanford, when Henry Kaplan demonstrated,
with what was to become one of the first controlled medical trials in oncology,
that \mbox{Gamma-knife} radiosurgery could significantly increase the survival
rate of early-stage Hodgkin's disease.\cite{kaplan_clinical_1968}

These advances in surgery and radiotherapy were most beneficial to patients with
early, localized forms of cancer.  Metastatic disease, on the other hand,
requires a systemic cure.  Furthermore, not all tumours respond equally to
generic treatments, underscoring a fundamental aspect of cancer's biology---its
heterogeneity.  When, a hundred years ago, the German chemist Paul Ehrlich
launched the first systematic attempt to find chemical substances with specific
affinity to malignant cells in order to poison them, he was in essence devising
a new form of treating cancer. He called it chemotherapy.\footnote{``Give up all
  hope ye who enter''---was the reading on Ehrlich's lab door
  (\citealp{devita_history_2008}).}

To maximize the efficiency of anti-cancer drug screening, two conceptual
advances had to be achieved.  First, a cancer model was needed where the impact
of therapy could be effectively quantified.  Second, a form of circumventing the
ethical limitations of testing human patients had to be found.  The first issue
was addressed by Sydney Farber at the Children's Hospital in Boston when he
turned his attention to childhood's leukemia in the
1940's.\cite{devita_history_2008} In cancer medicine, leukemia is a particularly
appealing model because it offers the possibility to actually count the number
of cancer cells flowing in the blood---and thus measure the response to the
treatment.  The recruitment of murine models to test drug response in grafted
tumours provided the second piece of the puzzle needed to spur the hunt for
anti-cancer drugs.\cite{clowes_further_1905}

In 1955, a national screening effort for the development and testing of
chemotherapeutic drugs was launched in the \smallcaps{usa}.  This lead to the
concoction of highly toxic cocktails of drugs aimed at ``maximal, intermittent,
intensive, upfront'' chemotherapy to vanquish the
disease.\cite{frei_curative_1985} One of such high-dose, life-threatening
regimens, known as \smallcaps{vamp},\footnote{\smallcaps{vamp} is based on a
  combination of four drugs: vincristine, amethopterin, mercaptopurine and
  prednisone.} was tested on children with acute lymphoblastic leukemia
(\smallcaps{aml}), in a trial based at the \smallcaps{nci} in 1961.  The
morbidity of the treatment was egregious and only two of the fifteen children
subjected to the initial protocol survived it\footnote{``If we didn't kill the
  tumour, we killed the patient''---\emph{William Moloney on the early days of
    chemotherapy} (\citealp{moloney_pioneering_1997}).}---one of which was still
alive in 2008.\cite{mukherjee_emperor_2011} In spite of all their complications,
the \smallcaps{vamp}\cite{frei_effectiveness_1965} and the
\smallcaps{mopp}\cite{devita_combination_1970} (an equally aggressive regimen to
treat advanced Hodgkin's disease) trials made proof of cure of cancer by
chemotherapy.

All these experimental drugs were selected from lists of synthetic chemicals,
fermentation products and plant derivatives.  Their anti-cancer ability was
purely deduced on an empirical basis and the only feature they shared was their
rather indiscriminate effect as cell cycle inhibitors.  The first case of a
chemotherapeutic drug being reasoned from its biological underpinnings occurred
in 1969, with the confirmation that tamoxifen could bring metastatic breast
cancer into remission.\cite{cole_new_1971} Tamoxifen, a molecular mimicker of
estrogen, was first synthesized in 1962 in the \smallcaps{uk} with the goal of
being marketed as a hormonal contraceptive.  However, tamoxifen turned out to be
an estrogen antagonist instead: by binding to the estrogen receptor, it deprives
the cell from the necessary signaling to trigger its cell
cycle.\cite{jordan_effects_1977} Adjuvant chemotherapy for the treatment of
breast cancer, i.e., the use of chemotherapy after surgical extraction of the
primary tumour, was shown to decrease the rate of relapse for the first time in
a trial launched in 1974.\cite{bonadonna_combination_1976}

By 1990, the three-pronged approach of surgery, radiotherapy and chemotherapy,
complemented with prevention campaigns and improved diagnostic tools for early
diagnosis, led to the first reported decrease of cancer incidence and
mortality.\cite{devita_two_2012} It was a worthy achievement, but one that fell
short of the haughty rhetoric to ``cure'',\footnote{``We have a cure for breast
  cancer''---\emph{Emil Frei to a colleague}, summer of 1982
  (\citealp{mukherjee_emperor_2011}).} ``conquer'',\footnote{``Why don't we try
  to conquer cancer by America's 200\textsuperscript{th} birthday? What a
  holiday that would be!''---advertisement published in the \emph{New York
    Times} in December 1969.} or win ``the war on cancer''\footnote{The National
  Cancer Act was signed by Richard Nixon on December 23, 1971.} that was voiced
throughout the 20\textsuperscript{th} century from several corners of the cancer
research establishment.  In fact, the field was starting to come to terms with
the evidence that, no matter how aggressive the treatment,\footnote{In order to
  allow for an otherwise intolerably high chemotherapeutic dosage, Emil Frei
  devised in 1982 a trial for advanced breast cancer treatment contemplating an
  autologous bone marrow transplantation.  The re-implanted frozen bone marrow
  cells would thus be spared the excessively high drug dosage.  This regimen,
  known as \smallcaps{stamp}, became embroiled in controversy during the next
  twenty years.  It was finally put to rest in 2011, when proof was published
  that it added no discernible benefit to patient's overall survival
  (\citealp{berry_high-dose_2011}).} our capacity to arrest the nearly
monomaniacal progression of cancer had been pushed to its limits.  The most
expressive sign of this tacit acquiescence was perhaps the rise in prominence of
palliative medicine during the 1980's---prolonging life at any cost no longer
was the purpose of cancer medicine.

% ``poison'',\footnote{\citealp{shorter_health_1987}, \emph{p} 189}

\medskip

In 1997, John Bailar published a review article in the \emph{New England Journal
  of Medicine} entitled ``Cancer undefeated''.\cite{bailar_cancer_1997} He
concluded thusly:

\begin{quotation}
  The war against cancer is far from over.  Observed changes in mortality due to
  cancer primarily reflect changing incidence or early detection.  The effect of
  new treatments for cancer on mortality has been largely disappointing.  The
  most promising approach to the control of cancer is a national commitment to
  prevention, with a concomitant rebalancing of the focus and funding of
  research.
\end{quotation}

The call for a more fundamental understanding of the neoplastic cell had been
made.  But, in order to embrace it, something more telling than microscopes,
\smallcaps{x}-ray scans, or mice models would be needed.

\clearpage

% \clearpage

% Cellular origin of cancer.  Microscopy (tools).  Surgery.  X-rays.
% Chemotherapy.

% Ehrlich (p.130): ``To target the abnormal cell, one would need to decipher the
% biology of the normal cell.''

% Every drug, the sixteenth-century physician Paracelsus once opined, is a
% poison in disguise.  Cancer chemotherapy, consumed by its ery obsession to
% obliterate the cancer cell, found its roots in the obverse logic: every poison
% might be a drug in disguise. (p.133) Argument for cancer selection of
% chemotherapy.

% By 1955, this effort, called the Cancer Chemotherapy National Service Center
% (CCNSC), was in full swing.  Between 1954 and 1964, this unit would test
% 82,700 synthetic chemicals, 115,000 fermentation products, and 17,200 plant
% derivatives and treat nearly 1 million mice every year with various chemicals
% to find an ideal drug. (p.176)

% When I went through the avalanche of chemotherapy drugs that would be used
% over the next two years to treat her, she repeated the names softly after me
% under her breath, like a child discovering a new tongue twister:
% ``Cyclophosphamide, cytarabine, prednisone, asparaginase, Adriamycin,
% thioguanine, vincristine, 6- mercaptopurine, methotrexate.'' (p.184)

% Experimental regimen of combination of drugs.  Grafting of cancers in animal
% models.  Skipper learned that he could halt this effusive cell division by
% administering chemotherapy to the leukemia-engrafted mouse. By charting the
% life and death of leukemia cells as they responded to drugs in these mice,
% Skipper emerged with two pivotal findings.  First, he found that chemotherapy
% typically killed a fixed percentage of cells at any given instance no matter
% what the total number of cancer cells was.  This percentage was a unique,
% cardinal number particular to every drug.  Killing leukemia was an iterative
% process, like halving a monster's body, then halving the half, and halving the
% remnant half.  Second, Skipper found that by adding drugs in combination, he
% could often get synergistic effects on killing. (p.202)

% The notable common feature that linked all these drugs was that they were all
% rather indiscriminate inhibitors of cellular growth.  Nitrogen mustard, for
% instance, damages DNA and kills nearly all dividing cells; it kills cancer
% cells somewhat preferentially because cancer cells divide most
% actively. (p.232)

% Doctors are men who prescribe medicines of which they know little, to cure
% diseases of which they know less, in human beings of whom they know
% nothing.---Voltaire (p.204)

% If we didn't kill the tumor, we killed the patient.---William Moloney on the
% early days of chemotherapy.  VAMP trials on acute lymphoblastic leukaemia
% (ALL).  The terror of VAMP was death by infection.  Chemotherapy could cure
% cancer.  Of the fifteen patients treated on the initial protocol, only two
% still survived. (p.237) Perhaps the most disturbing side effect of
% chemotherapy would emerge nearly a decade later.  Several young men and women,
% cured of Hodgkin's disease, would relapse with a second cancer---typically an
% aggressive, drug-resistant leukemia---caused by the prior treatment with MOPP
% chemotherapy.  As with radiation, cytotoxic chemotherapy would thus turn out
% to be a double-edged sword: cancer-curing on one hand, and cancer-causing on
% the other. (p.236)

% ``A revolution [has been] set in motion,'' DeVita wrote.  Kenneth Endicott,
% the NCI director, concurred: ``The next step---the complete cure---is almost
% sure to follow.'' (p.244)

% Cancer is not one single disease!  It felt---nearly twenty five hundred years
% after Hippocrates had naively coined the overarching term karkinos---that
% modern oncology was hardly any more sophisticated in its taxonomy of cancer.
% Orman's lymphoma and Sorenson's pancreatic cancer were both, of course,
% ``cancers,'' malignant proliferations of cells.  But the diseases could not
% have been further apart in their trajectories and personalities.  Even
% referring to them by the same name, cancer, felt like some sort of medical
% anachronism, like the medieval habit of using apoplexy to describe anything
% from a stroke to a hemorrhage to a seizure.  Like Hippocrates, it was as if
% we, too, had naively lumped the lumps.

% Radiotherapy and Hodgkin's lymphoma in Stanford (p.228): The trials that
% Kaplan designed still rank among the classics of study design. In the first
% set, called the L1 trials, he assigned equal numbers of patients to either
% extended field radiation or to limited ``involved field'' radiation and
% plotted relapse-free survival curves.  The answer was definitive.  Extended
% field radiation---``meticulous radiotherapy'' as one doctor described
% it---drastically diminished the relapse rate of Hodgkin's disease.

% A hundred instances of Hodgkin's disease, even though pathologically
% classified as the same entity, were a hundred variants around a common theme.
% Cancers possessed temperaments, personalities---behaviors.  And biological
% heterogeneity demanded therapeutic heterogeneity; the same treatment could not
% indiscriminately be applied to all.  But even if Kaplan understood it fully in
% 1963 and made an example of it in treating Hodgkin's disease, it would take
% decades for a generation of oncologists to come to the same
% realization. (p.229)

% The somatic theory of cancer argued that environmental carcinogens such as
% soot or radium somehow permanently altered the structure of the cell and thus
% caused cancer. (p.247)

% The cure before the cause, i.e., treatment before the understanding of the
% disease.

% James Watson, who had discovered the structure of DNA, unloosed a verbal
% rampage against the Senate bill.  ``Doing `research' is not necessarily doing
% `good' research,'' Watson would later write.  ``In particular we must reject
% the notion that we will be lucky\ldots Instead we will be witnessing a massive
% expansion of well-intentioned mediocrity.'' (p.267)

% X-rays and surgery---1924 Keynes (p.278)

% Cancer didn't move centrifugally by whirling through larger and larger ordered
% spirals; its spread was more erratic and unpredictable. (p.280)

% This is the point at which a clear understanding of invasiveness of the
% primary tumour started driving the extent of surgical approach---no more
% radical and ultra-radical surgery.

% 1928: Jerzy Neyman and Egon Pearson provided a systematic method to evaluate a
% negative systematic claim---the statistical concept of power. (p.280)

% The `power' of orthodoxy. (p.282)

% 1973: tamoxifen/ER receptor and estrogen-dependent growth (p.306) Adjuvant
% chemotherapy.

% 1972: at the NCI, first effort to integrate chemotherapy with surgery. The
% Istituto Tumori (Milan) trial (p.310) Anti-hormone therapy for prostate and
% breast cancer.

% 1980: paliative medicine (not care) in oncology (p.316) The movement to
% restore sanity and sanctity to the end-of-life care of cancer patients emerged
% predictably, not for the cure obsessed America but from Europe.  Cecile
% Saunders.  1967: first hospice in London to take care of the terminally ill,
% away from the oncology wards. (p.318)

% The ``More doctors smoke Camel'' ad.  Incidentally, one particular event in
% cancer history spurred more than any other the need to \emph{understand} the
% mechanism of action of carcinogens in causing cancer.  It was the strong and
% dishonest rebutal of the tobacco industry in accepting the epidemiological
% studies as effective proof of cancer cause. (p.360)

% Cancer demonstrates a spectrum of behaviour.

%%% Local Variables:
%%% TeX-engine: xetex
%%% mode: latex
%%% TeX-master: "../../thesis"
%%% End:

\section{Microarrays}

\newthought{In October 1995,} a short report in \emph{Science} magazine caught
the eye with a figure showing six grids of colourful stains on a dark background
(Figure~\ref{fig:arabidopsis-microarray}).  The colour of each spot, ranging
through the visible spectrum, captured the fluorescence emitted when a 3.5 mm by
5.5 mm slide of glass was scanned with a laser.  Each slide had been previously
spotted with an array of microscopic droplets of \mbox{forty-five} clones of
c\smallcaps{dna} isolated from \emph{Arabidopsis thaliana}---a small flowering
plant with the smallest genome of any known higher eukaryote.  Before being
scanned, the slides, or microarrays, were hybridized with a solution of
fluorescently labeled, reverse transcribed c\smallcaps{dna}s, synthesized from
m\smallcaps{rna} templates extracted from the plant.

\begin{marginfigure}%
  \includegraphics{schena-brown-fig1.jpg}
  \caption[Gene expression of \emph{Arabidopsis thaliana} monitored with
  c\smallcaps{dna} microarrays]{Gene expression of \emph{Arabidopsis thaliana}
    monitored with c\smallcaps{dna} microarrays. \textbf{A--F}: each panel shows
    the hybridization intensity of a mix of fluorescently labeled
    c\smallcaps{dna}s with a collection of \mbox{forty-five}
    \mbox{gene-specific} probes from arabidopsis, plus three controls, under
    each stated condition (see text).  Adjacent pairs of spots are experimental
    duplicates.  Negative controls were spotted on positions \emph{c}(11, 12)
    and \emph{h}(11, 12).  Positive controls were provided by adding a fixed
    diluted quantity of m\smallcaps{rna} of the human acethylcoline receptor
    gene to each sample before reverse transcription.  c\smallcaps{dna} probes
    of the positive control were printed on positions \emph{a}(1, 2) Probes for
    the \emph{\smallcaps{hat4}} gene were printed on positions \emph{e}(1, 2)
    (reproduced from \citealp{schena_quantitative_1995}).}
  \label{fig:arabidopsis-microarray}
\end{marginfigure}

The vivid readouts from the microarrays were literally illuminating the
transcription patterns of the arabidopsis genome.

\bigskip

With this seminal report from Stanford, Mark Schena and Patrick Brown
demonstrated three things.  First, that microarray technology was
\emph{sensitive} and \emph{specific} enough to discriminate between distinct
m\smallcaps{rna} species in solution (Figure~\ref{fig:arabidopsis-microarray},
\textbf{A--B}).  Second, that it could \emph{quantify} discrete levels of
m\smallcaps{rna} transcripts (Figure~\ref{fig:arabidopsis-microarray},
\textbf{C--D}).  Third, that the technology was suited to investigate gene
expression patterns in diverse tissue types
(Figure~\ref{fig:arabidopsis-microarray}, \textbf{E--F}).  As Pat Brown would
state later, microarrays were developed to ``enable a new method for relating
sequence differences in genes to complex traits in people.''  And what bigger
challenge of complexity could there be but cancer?

Four years on, Todd Golub and Eric Lander, at the Broad Institute in Boston,
reported the results of the first tackling on cancer using
microarrays.\cite{golub_molecular_1999} As a test case, they took to Farber's
acute leukemia, isolating m\smallcaps{rna} from 38 biological samples of
neoplastic bone marrow and peripheral blood. Their approach was framed as a
classification task.  Up to then, the problem of classifying cancer subtypes was
mostly left to the expertise of the pathologist,\footnote{Clinical practice for
  cancer classification would involve an experienced pathologist's
  interpretation of the tumour's morphology, histochemistry, immunophenotyping,
  and cytogenetic analysis.} and many cancers still lacked molecular markers for
their accurate definition.  Moreover, the correct distinction between acute
lymphoblastic leukemia (\smallcaps{all}, originating from lymphoid precursors)
and acute myeloid leukemia (\smallcaps{aml}, originating from myeloid
precursors) was critical for the determination of the chemotherapeutic regimen
to be used.

Using a microarray with probes reporting for 6817 human genes, they concluded
that the patterns of expression of a subset of these genes could be used to
accurately discriminate between \smallcaps{aml} and \smallcaps{all}.  They then
sought to use the expression data to blindly \emph{infer} eventual tumour
\mbox{sub-classes} among the samples.  Using a learning algorithm to build a
classifier from the data and then testing it with a cross-validation procedure,
they showed that the \mbox{\smallcaps{aml-all}} distinction could be
automatically discovered and confirmed without a biological \mbox{\emph{a
    priori}}.\footnote{These class prediction and class discovery tasks
  illustrate the distinction between \emph{supervised} and \emph{unsupervised}
  learning.  While the former derives a function from labeled training data
  (thus requiring an \mbox{\emph{a priori}} knowledge of the classes it tries to
  predict), the latter aims to produce a classification without any prior
  knowledge of the structure in the data.}  What's more, this class discovery
approach could be further refined to automatically detect the distinction
between \mbox{\smallcaps{B}-cell} and \mbox{\smallcaps{T}-cell} \smallcaps{all}.
The use of microarrays could thus enable a molecular classification of cancer,
even if this landmark experimental setup failed to find a multigene expression
signature to predict response to chemotherapy.

Back in Stanford, the focus was being directed towards breast cancer.  With a
microarray probing for \num{8102} human genes, Charles Perou, Pat Brown and
David Botstein studied the variation in gene expression patterns of breast
tumours from 42 patients.\cite{perou_molecular_2000} Using an unsupervised
hierarchical clustering algorithm, they were able to identify at least five
distinct molecular breast cancer classes.

\begin{marginfigure}%
  \includegraphics[width=\linewidth]{nature-the-human-genome.jpg}
  \caption[Cover of \emph{Nature} magazine of February 15, 2001]{Cover of
    \emph{Nature} magazine of February 15, 2001.}
  \label{fig:human-genome-cover}
\end{marginfigure}

Two types of epithelial cells are found in the human mammary gland: basal cells
(the outer layer of myoepithelial cells in the mammary duct) and luminal cells
(at the apical surface of the ducts, with secretory proprieties).  In Perou's
study, cancers of luminal origin were found to cluster in two previously
unrecognized groups, termed luminal \smallcaps{a} (of lower grade) and luminal
\smallcaps{b} (of higher grade).  Cancers with a \mbox{basal-like} phenotype;
\mbox{over-expressing} the \smallcaps{her2} receptor; or with a normal-like
phenotype were each found to cluster together in their respective group.
However, the most robust distinction was observed between the transcriptome of
breast cancers expressing the estrogen receptor (\smallcaps{ER+}) and those that
did not (\smallcaps{ER--}).  This pioneering study showed that a new taxonomy of
breast cancer could be based on its molecular features---a classification that
would be challenged, extended and refined throughout the ensuing
decade.\cite{sorlie_gene_2001,sorlie_repeated_2003,hu_molecular_2006,pusztai_molecular_2006,rakha_basal-like_2008,parker_supervised_2009,gusterson_basal-like_2009,weigelt_contribution_2010,prat_deconstructing_2011}

% Distinct breast cancer genotypes were first characterized.  Multiparameter
% gene expression assays for early-stage breast cancer.  A new taxonomy of
% breast cancer based on their molecular features.  The gene expression
% microarray-based class discovery studies pioneered by the Stanford group have
% led to the identification of at least five molecular breast cancer subtypes:
% luminal \smallcaps{A}, luminal \smallcaps{B}, normal breast-like, HER2, and
% basal-like.  Indeed, the most robust distinction observed by microarray
% analysis is between the transcriptome of breast cancers expressing the
% estrogen receptor (\smallcaps{ER+}) and estrogen receptor negative
% (\smallcaps{ER--}).

While these findings were being reported, Eric Lander was leading another
collaborative effort that would redefine the breadth of microarray technology.
On February 2001, a \mbox{public-funded} consortium reported the first draft of
the human genome (Figure~\ref{fig:human-genome-cover}).  Prior to this
achievement, the estimated number of genes in our genome was around
\num{100000}.\cite{cox_assessing_1994} As the genome sequence quality and gene
finding methods improved, this figure was progressively revised down to an
estimated \num{20000}--\num{25000} human protein coding genes.  The prospect of
measuring the \emph{entire} transcriptome of the neoplastic cell with
microarrays was now within reach---and would soon turn into a reality.  If
cancer was fundamentally a genetic disease,\footnote{``The revolution in cancer
  research can be summed up in a single sentence: cancer is, in essence, a
  genetic disease''---\emph{Bert Vogelstein}
  (\citealp{vogelstein_cancer_2004}).} then the study of the cancer genome with
microarrays would bring it into the genomic era.

It was with microarrays probing for most of the then reported human genome that
the expression profiles of lung adenocarcinomas\cite{garber_diversity_2001},
hepatocellular carcinomas,\cite{chen_gene_2002} and gastric
cancers,\cite{leung_phospholipase_2002} were interrogated.  Time and again,
unsupervised classification methods were highlighting clinical subtypes that
recapitulated morphological categorizations, underlined tumour differentiation
stages, or even uncovered tentative progression markers.  Each of these
portraits revealed a wide diversity in tumour profiles, both at the \mbox{intra-
  and} \mbox{inter-patient} sampling level, and a relatively minimal variation
in normal tissue profiles.  While nuanced, the transcriptomes of these different
cancers were still remarkably consistent within each disease and largely
reminiscent of the expression profiles of the normal tissues from which they
were derived.\cite{botstein_genomic_2003}

The use of class discovery algorithms, mostly as descriptive techniques,
dominated the early genomic approach to cancer
biology.\cite{matros_genomic_2004,eschrich_dna_2004} However, the problem of
predicting cancer progression, response to treatment or survival time remained
an elusive one.  In order to provide sound evidence for predictive genomic
markers, an experimental setting with a systematic, long term follow-up of
cancer patients was in demand.

\medskip

% Mass reporting of cancer transcription profiles.  Which cancers.  Which genes
% are deferentially expressed.  First public microarray databases.  Microarray
% meta-analysis.  Single cell gene expression profiling.  A compelling case was
% made.  Presently.

% Wide diversity in tumour profiles and relatively minimal variation in normal
% tissue profiles have been found not only for breast cancers, but also for
% lung, and gastric cancers.  While nuanced, the global patterns of gene
% expression of different cancers where still remarkably consistent within each
% malady and largely reminiscent of the transcription profiles of the normal
% tissues from which they are derived.  (Botstein, Fig.4).  Furthermore, tumour
% samples from the same breast cancer patient, either by repeated surgical
% sampling or from lymph node metastases, tend to have profiles very similar to
% each other; similar results were reported in lung and liver tumours.  Gene
% discovery.

% Concomitantly, a statistical and computational theory for the analysis of the
% millions of data points that result from these experiments was starting to
% take shape.\cite{quackenbush_computational_2001,irizarry_summaries_2003}

% First classification results using microarray data.  Several cancers were used
% as models.  Concomitant evolution of the microarray analysis technology
% (Quackenbush and shit).  Selection bias & etc.  Public repositories for
% microarrays.  Miame consensus.

\newthought{In the early 1980's}, the pathologists of the Nederlands Kanker
Instituut (\smallcaps{nki}) in Amsterdam began a frozen tissue bank of tumours
from Dutch women breast cancers.  Twenty years on, the m\smallcaps{rna} of these
samples, along with the patient's clinical histories, would be the subject of a
page turning gene expression profiling experiment.  A group including
\smallcaps{nki}'s head of molecular pathology, Laura van't Veer, head researcher
René Bernards, and Stephen Friend, a Weinberg trainee turned Rosetta
Inpharmatics founder in Seattle, analyzed a selection of primary tumours from 98
women younger than 55 who did not develop lymph node
metastasis.\cite{vant_veer_gene_2002}

When the tumours were originally resected, treatment standards did not require
adjuvant chemotherapy after surgery.  Thirty-four patients, or roughly one third
of the women in the study, had since then relapsed their cancer.  According to
modern guidelines, approximately 95\% of the original patients would have
received chemotherapy in the \mbox{United States}, and 85\% would have been
treated under European norms.  This entails that 55\% to 65\% of the patients
would have needlessly undergone an aggressive and debilitating form of
chemotherapy.  To assess whether gene expression profiles could predict
metastatic relapse of disease, the \smallcaps{nki} team profiled the tumours on
a microarray containing approximately \num{25000} genes.  Using a supervised
iterative learning procedure on a subset of these features, they narrowed down a
list of 70 genes whose expression levels correlated with the development of
distant metastasis.  The robustness of this \mbox{70-gene} prognosis profile was
subsequently validated on a wider cohort of 295 breast cancer tumours with
either positive or negative nodal
status.\cite{van_de_vijver_gene-expression_2002} In this study, the classifier
accurately predicted overall survival and distant metastasis in stratified
univariate analyses and was the strongest predictor of distant metastasis in a
multivariate model that included traditional breast cancer
predictors.\footnote{Classical prognosticators for breast cancer include age,
  tumour size, status of axillary lymph nodes, histological type of the tumour,
  pathological grade and \mbox{hormone-receptor} status.}  Microarrays were now
being used to stratify cancer risk among patients based on the gene expression
profile of the tumour.

The genes included in the predictor were scrutinized for potential ``insight
into the underlying biological mechanism leading to rapid metastasis.''  The
van't Veer article in \emph{Nature} reports that ``genes involved in cell cycle,
invasion and metastasis, angiogenesis, and signal transduction are significantly
upregulated in the poor prognosis signature.''  Not only did this work produce
the first genomic predictor to inform treatment decisions, it also paved the way
for an alternative to infer physiological mechanisms in human cancers.  If the
tumour transcriptome already contained information regarding disease
progression,\footnote{``Even though you could look under the microscope and they
  all look the same (\ldots{}) some have built into them programs to become
  aggressive''---\emph{Stephen Friend}} then querying for biologically motivated
collections of genes among the predictive features could make proof of the
implication of particular genetic programs in cancer biology.

% \FloatBarrier

\begin{marginfigure}%
  \includegraphics[width=\linewidth]{jci-cancer-microarrays.jpg}
  \caption[Cover of \emph{The Journal of Clinical Investigation} of June
  1\textsuperscript{st}, 2005]{Cover of \emph{The Journal of Clinical
      Investigation} of June 1\textsuperscript{st}, 2005.}
  \label{fig:cancer-microarray}
\end{marginfigure}

It was with a similar reasoning in mind that the Stanford group presented the
argument for the link between a wound healing genetic program and cancer
progression.\cite{chang_gene_2004} The argument begins with the recognition of
the similarities between the tumour microenvironment and normal wound healing.
It then proceeds by characterizing a gene expression profile of fibroblast serum
response (which physiologically only occurs in the context of a local injury),
in a cell culture model profiled by microarray.  Finally, this signature profile
is used to test specific hypotheses using publicly available gene expression
data from human cancers.

Accordingly, they demonstrated that, in a cohort of 51
breast cancer patients with equal treatment, those with a higher expression of
the core serum response signature were significantly more likely to develop
metastasis and to die in a \mbox{5-year} \mbox{follow-up} period.  Similar
results were obtained by segregating the \mbox{295-sample} \smallcaps{nki}
cohort along an axis of expression of the serum response signature.  The
signature was also shown to be predictive of outcome in a dataset of 62 patients
with \mbox{stage \smallcaps{i}} and \mbox{stage \smallcaps{ii}} lung
adenocarcinomas\cite{garber_diversity_2001} and a dataset of 42 patients with
\mbox{stage \smallcaps{iii}} gastric carcinomas.\cite{leung_phospholipase_2002}
This formulation established a novel framework to infer biological determinants
of cancer progression based on gene expression profiles of clinical samples,
obviating the need of experimental setups on \emph{in vivo} models.

The rehashing of this strategy would prove exceptionally prolific.  In the wake
of the Chang et al. publication, links between cancer progression and various
biological signature markers were reported---including gene expression programs
of stem
\mbox{cell-ness};\cite{glinsky_microarray_2005,ben-porath_embryonic_2008}
p53-status;\cite{miller_expression_2005} stromal
component;\cite{west_determination_2005} response to
hypoxia;\cite{chi_gene_2006} chromosomal
instability;\cite{carter_signature_2006,buffa_large_2010} loss of
\emph{\smallcaps{PTEN}} expression;\cite{saal_poor_2007} \smallcaps{EMT}
transition;\cite{welm_macrophage-stimulating_2007,taube_core_2010}
\emph{\smallcaps{E2F1}} perturbation;\cite{hallstrom_e2f1-dependent_2008}
mi\smallcaps{R}-31 targets;\cite{valastyan_tumor_2011} among many more.
% bromodomain 4;\cite{crawford_bromodomain_2008}
% retinoic acid receptor;\cite{hua_genomic_2009} anchorage-independent
% growth;\cite{mori_anchorage-independent_2009} among many more.

\medskip

\newthought{By 2005,} ten years after the arabidopsis report, the partnership
between microarray technology and cancer research was in full swing
(Figure~\ref{fig:cancer-microarray}).  Reflecting the increasing appeal of the
technology, at least half a dozen vendors were then marketing whole genome
microarrays, each relying on their own specifics
(Table~\ref{tab:WholeGenomeArray}).

Microarrays can typically be designed in a \mbox{two-colour}
(\mbox{dual-channel}) or in a \mbox{single-colour} (\mbox{single-channel})
setup.  In \mbox{dual-channel} microarrays, c\smallcaps{dna}s prepared from two
samples (usually diseased \emph{versus} healthy tissue) are each labeled with
its own fluorophore, then mixed and hybridized on the same microarray.  The
ratios of the measurements of the fluorescence emission in the wavelength of
each fluorophore is then used to estimate the relative abundance of individual
transcripts in the two samples.  \mbox{Single-channel} microarrays measure the
hybridization intensities of a single population of c\smallcaps{dna}s labeled
with a unique fluorophore and, therefore, express the relative abundance of
transcript expression across biological samples processed in the same
experiment.  Oligonucleotide microarrays often carry control probes designed to
hybridize with \smallcaps{rna} \mbox{spike-ins}.  The degree of hybridization
between the \mbox{spike-ins} and the control probes is used to normalize the
hybridization measurements for the target probes.

Other platform specific attributes include the probe manufacturing process (made
\mbox{\emph{in situ}} by photolithographic or ink-jet methods, or by standard
oligonucleotide synthesis protocols followed by attachment to various
substrates); the probe substrates (activated glass slides, silicon chips, or
membranes); the probe design and location (most probes are derived from the 3'
end of the gene coding sequences to accommodate the fact that target labeling
usually begins at the 3' end of m\smallcaps{RNA}s); probe size and number per
array (Table~\ref{tab:WholeGenomeArray}); and the proper probe annotation (as
sequence databases were still in state of flux, probe annotations were
constantly being revised and did not necessarily target their designated
gene).\cite{kawasaki_end_2006}

% The great majority of these studies were
% done without internal controls or standards, which rendered comparison of
% results in independent experimental settings difficult or impossible.

\begin{table}[ht]
  \small
  \centering
  % \fontfamily{ppl}%\selectfont
  \begin{tabular}[c]{lcS[table-format=6.0]S[table-format=6.0]}
    \toprule
    \multicolumn{1}{c}{Vendor} & \multicolumn{1}{c}{Probe Size} &
    \multicolumn{1}{c}{\# Probesets} & \multicolumn{1}{c}{\# Probes per array}\\
    \midrule
    \smallcaps{ABI} & 60mer & 33000 & 33000 \\
    Affymetrix & 25mer & 54000 & 1000000 \\
    Agilent & 60mer & 44000 & 44000 \\
    \smallcaps{GE} Amersham & 30mer & 57000 & 57000 \\
    Illumina & 50mer & 46000 & 1500000 \\
    Microarrays, Inc. & 70mer & 49000 & 49000 \\
    NimbleGen & 60mer & 38000 & 380000 \\
    Phalanx Biotech & 60mer & 30000 & 30000 \\
    ``Home brew'' & \begin{tabular}[c]
      {@{}c@{}}
      \vspace{-.1cm}
      50mer--\emph{n}70mer\\or c\smallcaps{DNA}s
    \end{tabular}&
    40000 & 40000 \\
    \bottomrule
  \end{tabular}
  \caption[Technical attributes of principal commercial microarray platforms]{Technical attributes of the principal commercial microarray
    platforms by 2005.  A probeset constitutes a collection of probes
    targeting a specific gene
    (adapted from \citealp{kawasaki_end_2006}).}
  \label{tab:WholeGenomeArray}
  \vspace{0cm}
\end{table}

\medskip

In addition, the expression data resulting from a microarray experiment can be
influenced by a number of experimental factors, such like target
c\smallcaps{dna} synthesis (linearly amplified \smallcaps{rna} may contain
biases in the original m\smallcaps{rna}s ratio);\cite{nygaard_options_2006}
target labelling (different fluorescent dyes present distinct stabilities,
quantum efficiencies and wavelengths for stimulation and emission);
hybridization and washing protocols (every commercial platform abides by its own
methodology); and the imaging of the arrays (usually done by confocal and
non-confocal scanners---yet variables like laser power, pixel sizes or scan time
are not standardized).

These sources of technical variation were making comparison of results obtained
from different microarray platforms difficult or even impossible.  Some studies
were reporting poor correlations between expression levels measured with
different platforms.\cite{tan_evaluation_2003,shi_cross-platform_2005} In order
to improve the reliability and concordance of microarray data, international
consortia and technical study groups were assembled to determine a core set of
\emph{Minimum Information About a Microarray Experiment} (\smallcaps{miame})
standards\cite{brazma_minimum_2001} and, in 2006, the \emph{MicroArray Quality
  Control} (\smallcaps{maqc}) project was
launched.\cite{maqc_consortium_microarray_2006} As a result of these concerted
efforts, the National Center for Biotechnology Information in the United States
created, in 2002, the \emph{Gene Expression Omnibus}
(\smallcaps{geo}),\footnote{\url{http://www.ncbi.nlm.nih.gov/geo/}
  (\citealp{edgar_gene_2002})} an online repository for \mbox{high-throughput}
gene expression data.  In 2003, the European Bioinformatics Institute started
\emph{ArrayExpress},\footnote{\url{http://www.ebi.ac.uk/arrayexpress}
  (\citealp{brazma_arrayexpress--public_2003})} a public database of microarray
gene expression data.

Gene expression profiling studies are also challenged by biological
idiosyncrasies.  For one, gene expression patterns are a function of fluid
cellular states in constant readjustment.  Microarray experiments consist of
snapshots of such dynamic ranges, which may account for some of the variation
across experiments.  Distinct synthesis and degradation rates of the probed
m\smallcaps{rna} transcripts may further nuance expression readings.  Even
traditional housekeeping genes (fundamental to the basic biology of the cell and
thus considered gold standards) have been shown to differ across tissues and
experimental conditions.\cite{thorrez_using_2008} What is more, biological
heterogeneity in the biopsy sample can significantly bias reports of gene
expression, as distinct cellular types may be differently represented in
distinct samples.

But by and large, the most contentious aspect of the application of microarrays
in cancer research concerns the methodological analysis of the experimental
data.  In 2007, a critical detailed review of \mbox{forty-two} studies for
cancer outcome appeared in the \emph{Journal of the National Cancer Institute}
by Alain Dupuy and Richard Simon, from the Hôpital Saint-Louis in Paris and the
\smallcaps{nci} in Maryland.\cite{dupuy_critical_2007} They
identified three endemic analytic flaws permeating the reviewed studies.

Microarray experiments typically aim for one or more of the following
objectives: (\emph{a}) to identify individual genes (transcripts) whose
expression is correlated with a phenotypic trait; (\emph{b}) to identify
multiple genes interactively involved in regulatory networks and in mediating
biological phenomena or disease pathogenesis; (\emph{c}) to discover potential
targets for drug development; and (\emph{d}) to identify molecular markers that
can be used as tools for disease diagnosis and prognosis or as predictors of
clinical outcome.\cite{kim_expectations_2010} In \mbox{outcome-related}
microarray experiments, these aims can be approached with statistical tools
addressing three kinds of tasks: finding genes correlated with outcome, class
discovery, and supervised prediction.

For the outcome-related gene finding task, Dupuy and Simon identified a trend
for an inadequate, unclear, or unstated method for controlling the number of
\mbox{false-positive} differentially expressed genes.  Because microarray
analysis involves making inferences about each gene whose expression is measured
on the array, the large number of hypotheses being tested can yield a higher
than desirable proportion of false positives.  Statistical procedures to correct
for excess of false positives in multiple testing, such like the false discovery
rate, are thus necessary.\cite{benjamini_controlling_1995,noble_how_2009}

Concerning the class discovery task, they recognized a tendency to credit
expression clusters with biological meaning when the clustering procedure was
itself based on genes selected for their correlation with outcome.  This causes
non-independent evidence that outcome can be predicted based on expression
levels---a statistical misconception known as feature selection
bias.\cite{ambroise_selection_2002}

For the supervised prediction task, they flagged analytic lapses causing a
biased estimation of the prediction accuracy through incorrect cross-validation
procedures.  The more common of these experimental design errors involved the
violation of the principle of separation of the training and testing sets during
the validation of the classifier.  The models derived thusly were likely prone
to data overfitting.\footnote{Overfitting occurs when a classifier describes
  random error or noise pertaining to the training set instead of the underlying
  structure of the data.}

With the emerging developments on the computational analysis of microarray
data\cite{quackenbush_computational_2001,irizarry_summaries_2003} and the
concomitant accrued sensitivity to its technical specifics, the stream of
published microarray studies started to be tempered by a series of reviews
questioning the validity, reproducibility and biological significance of the
results.\cite{michiels_prediction_2005,tinker_challenges_2006,kawasaki_end_2006,cahan_meta-analysis_2007,gusnanto_identification_2007,mathoulin-pelissier_survival_2008,kim_expectations_2010}

In iconic fashion, John Ioannidis, now a Professor of Medicine at Stanford,
chastised the field with a dire assessment on the lack of reproducibility of
some \mbox{high-profile} microarray research
findings.\cite{ioannidis_repeatability_2009} Upon independent dissection of the
analysis protocols of eighteen studies published in \mbox{\emph{Nature
    Genetics}} during 2005 and 2006, his team concluded that ten of them could
not be reproduced at all.  The main reason for failure to reproduce was data
unavailability, and discrepancies were mostly due to incomplete data annotation
or specification of data processing and analysis.

In the meantime, prognostic gene expression signatures of clinical outcome of
breast cancer were accumulating in the literature at a steady
rate.\cite{vant_veer_gene_2002,paik_multigene_2004,ma_two-gene_2004,wang_gene-expression_2005,chang_robustness_2005,miller_expression_2005,glinsky_microarray_2005,foekens_multicenter_2006,naderi_gene-expression_2006,teschendorff_consensus_2006,sotiriou_gene_2006,liu_prognostic_2007}
Intriguingly, very few genes were showing in common among the distinct
prognostic markers.  On a thorough review on the correspondence between these
biomarkers and the clinicopathological features of breast cancer, Christos
Sotiriou and Lajos Putszai, from the Institut Jules Bordet in Brussels and the
University of Texas in Houston, sought to form a synthesis of the evidence
supporting genomic prognostic signals.\cite{sotiriou_gene-expression_2009}
Sotiriou and Putszai reasoned that the absence of common markers between
signatures could be a feature of complex gene expression systems entertaining a
large number of correlated variables.  They also stressed the general tendency
for prognostic signatures to perform better among \smallcaps{ER+} tumours, as
they best discriminate \mbox{low-proliferation} \mbox{luminal \smallcaps{a}}
tumours from \mbox{high-proliferation} \mbox{luminal \smallcaps{b}} tumours,
whereas they mostly classify \smallcaps{ER--} tumours (comprising the basal-like
and \mbox{\smallcaps{her}-positive} phenotypes) as \mbox{high-risk}.  This could
be explained by the ability of prognostic signatures to capture molecular
features of tumour differentiation and tumour grade, both linked with cancer
progression and metastatic spread.  They summed up by proposing that models for
breast cancer prognostication should include both genomic and clinical variables
for best accuracy.

% tectonic readjustments

In 2011, David Venet and Vincent Detours, at the Université Libre de Bruxelles,
further added to the debate concerning the biological interpretation of
prognostic genomic signals in breast cancer.\cite{venet_most_2011} Probing the
\mbox{295-sample} \smallcaps{nki} reference cohort, they compared the prognostic
ability of \mbox{fourty-seven} published breast cancer outcome signatures with
signatures made of random genes.  They showed that 60\% of them were not
significantly better outcome predictors than random signatures of identical
size.  In addition, more than 90\% of random signatures with more than 100 genes
were shown to be significant outcome predictors.  Interestingly, they observed
that adjusting breast cancer expression data for a proliferation marker
abrogated most of the outcome association of published and random signatures.
By systematically exploring outcome associations in the \smallcaps{nki}-295
cohort, Venet and Detours exposed a wider and more pervasive range of prognostic
signals than previously anticipated.  They obtained similar results by
replicating the analysis in expression profiles of an independent cohort of 380
breast cancer patients from another study.\cite{loi_definition_2007} A
\mbox{decade worth} of biological extrapolations based on the transcription
profiles of clinical samples was eventually challenged by a more stringent
formulation of experimental controls.

\medskip

Microarrays, once hailed as the ``21\textsuperscript{st} century divining
rod,''\cite{he_microarrays21st_2001} have highlighted exciting new avenues to
engage the neoplastic cell.  Still, on the battlefield, cancer remained as
deceptive a foe as ever.  Translational research with direct impact on clinical
practice resulting from the microarray boom has proved tentative and timid at
best.  The most significant contributions were made in the context of predicting
a patient's prognosis by interpreting a panel of specific \mbox{tumour-related}
genes.  The first \smallcaps{FDA} clearance of \mbox{microarray-based} gene
profiling reagents was obtained in May
2011.\footnote{\url{http://investor.affymetrix.com/phoenix.zhtml?c=116408&p=irol-newsArticle&ID=1561100}}
As of 2013,\cite{kittaneh_molecular_2013} three genomic assays were commercially
available for prognostication in early stage breast cancer: Oncotype
\smallcaps{dx} (consisting of a \mbox{21-gene} profile narrowed down from a list
of 250 candidate genes that were analyzed in a total of 447 patients from 3
separate studies); MammaPrint (based on the \mbox{70-gene} \smallcaps{nki}
predictor; \smallcaps{fda} approved in 2007); and \smallcaps{pam50} (a
\mbox{50-gene} set used for standardizing subtype classification).

The mining of the wealth of data yielded by cancer expression profiling has been
as much a source of promising guidance as of humbling reappraisal.

% Other issues troubling outome-related microarray analysis include poor
% definition of survival end points\cite{mathoulin-pelissier_survival_2008}

% There's a kernel of truth

% \mbox{Outcome-related} microarray analysis\footnote{\mbox{Outcome-related}
% studies seek to establish a statistical association between gene expression
% levels and a particular clinical outcome, such as relapse of disease, death,
% or therapeutic response.} allows for all these and, in their review, Dupuy and
% Simon schematized these objectives into three categories of statistical
% analysis: finding genes correlated with outcome, class discovery, and
% supervised prediction.

% results, list of possible biases. batch specific biases.

% 2007: Dupuy & Simon review validity and reproducibility of microarray-based
% research with a meta-analysis on 90 studies.  Quote: ``The use of microarray
% technology has generated great excitement for its potential to identify
% biomarkers for cancer outcomes, but the reproducibility and validity of
% findings based on microarray data have come under widespread challenge.''
% (nci-dupuy-2007.pdf)

% Steps to run a microarray: 1-sequence the entire genome of the organism
% you're studying.  2-use massive amount of computing power to determine where
% all the genes are in your sequence.  3-more computing power to design primer
% pairs to use PCR to make copies of every gene.  4-with those primers, copy
% every gene on the genome 5-run quality tests on all gene copies to verify
% which PCR reactions did not work 6-separate double stranded DNA into single
% stranded DNA.  7-use robots to place microscopic droplets of each
% single-stranded DNA into ordered rows and columns on a glass microscopic
% slide (microarray).

% each spot contains multiple copies of a unique DNA sequence that represents a
% single gene.

% list of terms to use:
% genetic catalog; gene expression profile.

% While microarrays quantify the m\smallcaps{rna} species extracted from a given
% tissue, they do not tell us if the subsequent protein is actually being
% properly synthesized.

% Limitations of microarrays.  They cannot:
% 1-tell which genes went bad to cause a disease.
% 2-cure a disease
% 3-identify every genes that is behaving inappropriately

% From:
% http://en.wikipedia.org/wiki/Breast_cancer_classification#cite_ref-pmid20065178_43-0
% Several commercially marketed DNA microarray tests analyze clusters of genes
% and may help decide which possible treatment is most effective for a
% particular cancer.  The use of these assays in breast cancers is supported by
% Level II evidence or Level III evidence.  No tests have been verified by Level
% I evidence, which is rigorously defined as being derived from a prospective,
% randomized controlled trial where patients who used the test had a better
% outcome than those who did not.  Acquiring extensive Level I evidence would be
% clinically and ethically challenging.  However, several validation approaches
% are being actively pursued.

% tying up:
% Some evidence suggest that these assays provide comparable
% prognostic information: Concordance among Gene-Expression-Based Predictors for
% Breast Cancer

% Affymetrix Achieves First FDA Clearance of Microarray-Based Gene Profiling
% Reagents
% http://investor.affymetrix.com/phoenix.zhtml?c=116408&p=irol-newsArticle&ID=1561100

\clearpage

% Cancer\footnote{``It's bad bile.  It's bad habits.  It's bad bosses.  It's bad
% genes.''---\emph{Mel Greaves}}

%%% Local Variables:
%%% mode: latex
%%% TeX-master: "../../thesis"
%%% End:

\section{Motivation \& Contributions of this Thesis}

The work contributed in this thesis is anchored in the analytic corpus developed
during twenty years of gene expression assaying of cancer biopsies with
microarray technology.  Its motivations are framed by the current understanding
of the fundamental biology of cancer, primed by its two major axis of
progression: dedifferentiation and proliferation.

\subsection{Differentiation}
We first explored the differentiation axis by investigating the potential of
gene expression differentiation signatures to chart cancer progression using
whole genome microarray profiles of tumour samples.  As most molecular
classifiers of cancer rely on the variant features of the transformed neoplastic
transcriptome, we reasoned that a complementary approach could consist of
leveraging the invariant features specified by the unique transcriptional
signatures of each tissue of origin.

As a case study, we took to thyroid cancer.  Neoplasms of the thyrocyte cell are
characterized by a well defined linear progression from benign, fully
differentiated tumour types, up to one of the most lethal human cancers, the
anaplastic thyroid carcinoma (Figure~\ref{fig:thyroid-carcinogenesis}).  We
aimed to build a genomic marker of thyroid cancer progression based on a gene
expression signature of healthy thyrocytes.  As a result, we devised a general
method to derive robust \mbox{organ-specific} gene \mbox{expression-based}
differentiation indices, published in the journal
\emph{Oncogene}.\cite{tomas_general_2012}

\begin{figure}[ht]
  \includegraphics{fie-thyroid-carcinogenesis-2012-thyroid.png}%
  \caption[Step model of thyroid carcinogenesis]{Step model of thyroid
    carcinogenesis.  Thyroid epithelial cells may undergo transformation via
    alterations in different oncogenes and tumour supressor genes, giving rise
    to \mbox{well-differentiated} papillary or follicular carcinomas.
    Additional mutational load can cause progression of a differentiated tumour
    into a poorly differentiated one, and eventually into an anaplastic
    carcinoma.  Particularly challenging, from the histopathological point of
    view, is the distinction between follicular adenomas and follicular
    carcinomas; and between follicular variants of papillary carcinomas and
    their classical counterpart (reproduced from
    \citealp{sastre-perona_role_2012}).}
  \label{fig:thyroid-carcinogenesis}
\end{figure}

Contributions made in the context of this work include:
\begin{itemize}{}{}
\item An unbiased procedure to derive \mbox{organ-specific} differentiation
  markers from gene expression profiles of healthy tissues.
\item Proof of concept of the clinical utility of differentiation signatures in
  cancer diagnosis, featuring thyroid cancer as a test case.  Specifically, we
  demonstrated that, in a panel of expression profiles composed of thyroid
  cancers of distinct subtypes and normal thyroid samples:
  \begin{enumerate}
  \item The expression of a \mbox{thyroid-specific} differentiation biomarker,
    consisting of 15 genes, is inversely correlated with that of a proliferation
    biomarker, also independently derived from expression profiles of healthy
    tissues.  Conversely, the differentiation biomarker is positively correlated
    with the proliferation biomarker in expression profiles of a
    \mbox{time-course} experiment where thyrocytes in culture were treated with
    \smallcaps{tsh} hormone (the \mbox{thyroid-stimulating} hormone,
    \smallcaps{tsh}, induces both the metabolic activity and proliferation of
    thyrocytes).  These observations support the independence of the two
    biomarkers and prove that the differentiation biomarker does capture a
    transcriptome signature particular to the epithelial thyroid cell.
  \item A multidimensional scaling analysis representation of the profiled
    clinical samples exposes a non-overlapping continuum of thyroid tumours of
    increasing malignancy.
  \item The differentiation biomarker can accurately discriminate between
    follicular adenomas and follicular carcinomas; and between follicular
    variants of papillary carcinomas and classical papillary carcinomas---two
    challenging histopathological diagnosis.  Moreover, the accuracy of the
    differentiation biomarker in this supervised classification task was not
    significantly different from the accuracy of two supervised machine learning
    classifiers trained within the whole gene expression space of the tested
    samples.
  \end{enumerate}
\end{itemize}

\subsection{Proliferation}
Uncontrolled proliferation is not just a hallmark of
cancer\cite{hanahan_hallmarks_2011}---but its very own operational definition.
Using a proliferation biomarker consisting of 129 genes derived from healthy
tissues, Venet and Detours\cite{venet_most_2011} showed that most of the
prognostic content found in the reference \smallcaps{nki}-295 dataset was
linked to a pervasive proliferative signal in the neoplastic transcriptome.

We took to a wider collection of 102 distinct \mbox{outcome-related} cancer
cohorts, spanning 22 types of cancer to (\emph{a}) evaluate the extent of
prognostic signals in human cancers transcriptomes; and (\emph{b}) dissect the
potential technical and biological variables linked to prognostic content in
different cancer.  The results of this work are currently under submission and
will be thoroughly detailed in the \hyperref[chap:results]{\textsf{Results}}
section.

\medskip

Contributions made in the context of this work include:
\begin{itemize}
\item Evidence of an extensive correlation structure in cancer transcriptomes as
  assayed by microarrays.  This is concluded on the count that, in 76\% of the
  cancer cohorts analyzed, more than 5\% of random gene expression signatures is
  associated either with patient death or relapse of disease.
\item Demonstration that variables of both technical and biological essence are
  linked with the heterogeneity in prognostic content observed across 33 breast
  cancer cohorts.  This was determined from a thorough statistical multivariate
  analysis of the \mbox{clinico-pathological} variables associated in each study
  with the prognostic fraction of the transcriptome.
\item Corroboration of the significance of proliferation as the biological
  program comprising most of the prognostic content in cancer transcriptomes.
  % Serial deconvolution of distinct biologically motivated biomarkers was
  % undertaken across the studied cohorts to yield this result.
\end{itemize}

% A proliferation metagene captures a significant fraction of the
% pan-transcriptomic pervasive signals associated with outcome; other biological
% programs may also account for prognostic content to various degrees.

%%% Local Variables:
%%% mode: latex
%%% TeX-master: "../../thesis"
%%% End:



\chapter{Methods}
\label{chap:methods}

The contributions of this thesis are based on the analysis of global gene
expression profiles of cancer biopsies with microarray technology.  This chapter
begins by introducing the technology itself.  It then presents the flow of
microarray data analysis, including the preprocessing of raw data.  Next, it
lists and details the analytic methods used to make inferences about gene
expression data. Finally, it describes the public microarray datasets used in
the analyses reported in the \hyperref[chap:results]{\textsf{Results}} section.

% Pre-processing of raw-data includes reporting of internal quality controls, the
% normalization of probe intensities and their summarization.

\section{Microarray technology}
\label{sec:microarray-methods}

Microarray technology relies on the non-covalent, sequence-specific interaction
between complementary strands of nucleic acids\cite{watson_molecular_1953} to
detect and quantify specific populations of m\smallcaps{rna} in a solution.  A
microarray chip consists of a universe of oligonucleotide probes attached to a
substrate through covalent bonds.  Each such probe is synthesized to
specifically match a unique messenger \smallcaps{rna} molecule.  When the chip
is exposed to a solution of fluorecently labeled m\smallcaps{rna}s, only those
that hybridize with their respective probes will be retained upon washing off
non-specific bonding sequences.  This allows for the quantification of the
fluorescent signals emitted when the chip is scanned with a laser beam of a
specific wavelength.  The measured signals relay the relative quantity of each
m\smallcaps{rna} molecule assayed by the microarray, as each spot has a known
position on the chip (Figure~\ref{fig:microarray-economist}).

% Oligonucleotide microarrays often carry control probes designed to hybridize
% with RNA spike-ins. The degree of hybridization between the spike-ins and the
% control probes is used to normalize the hybridization measurements for the
% target probes.

\begin{marginfigure}%
  \begin{center}
    % \includegraphics[width=9cm]{microarrays-economist.jpg}
    \includegraphics{microarrays-economist-17Feb2015.pdf}
    \caption[Schematic representation of how microarrays work]{A schematic
      representation of how microarrays work.  \textbf{1.}~Microarrays rely on a
      fundamental property of nucleic acids, the monomeric units that polymerize
      into \smallcaps{dna} or \smallcaps{rna} strands.  Adenine (\smallcaps{A})
      are complementary to thymine (\smallcaps{t}), and cytosine (\smallcaps{c})
      are complementary to guanine (\smallcaps{g}).  Just one incorrect base can
      prevent two strands from binding.  \textbf{2.}~A microarray typically
      contains thousands of squares, or spots.  Each spot anchors many copies of
      a particular sequence of single-stranded \smallcaps{dna}, corresponding to
      a particular gene.  \textbf{3.}~Messenger \smallcaps{rna} fragments
      extracted from a tissue and labeled with different fluorescent dyes are
      washed over the microarray and hybridize with \smallcaps{dna} strands with
      the complementary sequence.  \textbf{4.}~The dyes are illuminated using
      fluorescent light.  It is then possible to show which \smallcaps{rna}
      fragments were retained in which spots---and hence which genes were being
      expressed in the tissue from which the \smallcaps{rna} was extracted.
      Source: \emph{The Economist;
        Affymetrix}.}\label{fig:microarray-economist}%
  \end{center}
\end{marginfigure}

\section{Microarray data preprocessing}
\label{sec:microarray-methods-data-preprocessing}

By virtue of their design, microarrays allow for the monitoring of expression
levels for thousands of gene transcription products simultaneously.  Microarray
expression data are thus characterized by high dimensionality and noisiness.
This prompts the need for preprocessing methods aiming at removing systematic
biases in expression measurements, introduced during
experimentation.\cite{shakya_comparison_2010}

The goal of microarray data preprocessing is to convert raw imaging data into
meaningful biological data and to enable comparison of results obtained from
different arrays.  It comprises three steps: (\emph{a})~the transformation of
image data into intensity values; (\emph{b})~the assessment of array quality;
and (\emph{c})~the removing of technical biases (through background adjustment,
normalization and feature filtering and summarization).

The digital imaging of fluorescence signals is typically performed by
proprietary software designed by the microarray manufacturer.  These software
packages assign coordinates to each spot in the array, quantify signal intensity
and uniformity of each spot, and compare their signal intensity relative to
background.

Quality control is performed by visual inspection of imaging data from the
scanner or the platform software, with special attention given to washing
artifacts, odd or missing spots, and array uniformity.  Aberrant chips may have
to be discarded at this stage.

Chips meeting quality standards then undergo background adjustment and
normalization.  Normalization methods aim to compensate for procedural biases
that are independent from biological signal.  Early approaches for microarray
normalization were based on the assumption that most genes, and in particular
so-called housekeeping genes,\footnote{\emph{Housekeeping} genes are genes
  defined as participating in basic, thus universal, cellular processes.}
should have similar expression levels across samples.  Housekeeping genes have
since been shown to vary in expression by 30\% or more across healthy samples,
and even more in tumour samples.\cite{lee_control_2002,eisenberg_human_2003}
Data-driven normalization approaches were then developed, such as median
correction,\cite{cho_genome-wide_1998,selinger_rna_2000} variance stabilizing
transformation,\cite{durbin_variance-stabilizing_2002} locally weighted linear
regression\cite{yang_normalization_2002} and spline based
methods.\cite{workman_new_2002} Data-driven presupposes that most of the
observed variation in expression values is due to technical biases rather than
to biological sources.\cite{hicks_when_2014}

Normalization strategies for double-channel microarrays (spotted oligonucleotide
or c\smallcaps{dna} arrays\cite{schena_quantitative_1995}) are different from
those for single-channel microarrays (\emph{in situ} synthesized high density
oligonucleotide arrays,\cite{lockhart_expression_1996} such as \emph{Affymetrix
  GeneChip}).  \emph{Affymetrix GeneChip} arrays use multiple probes per gene
and a single-colour detection system, as one sample is hybridized per chip.
Spotted oligonucleotide or c\smallcaps{dna} arrays use one probe per gene and a
two-colour scheme, where two different samples are hybridized on the same array.
Consequently, single-channel arrays measure the overall abundance of a probe
sequence in a target sample, whereas c\smallcaps{dna} arrays measure the
relative abundance of a probe sequence in two target samples.
% In other words, the expression measures for single-channel arrays are absolute
% (log) intensities, whereas they represent (log) ratios of intensities for
% c\smallcaps{dna} arrays.\footnote{In many cases, one of the samples in a
% c\smallcaps{DNA} array hybridization is a common reference used across
% multiple slides and whose sole purpose is to provide a baseline for direct
% comparison of expression measures between arrays.}
As a result, normalization of single-channel microarrays is performed at the
``between-array'' level, whereas normalization of double-channel microarrays is
conducted at the ``within-array'' level.\cite{do_normalization_2006}

For double-channel arrays chips, normalization methods commonly seek to remove
biases within each array with local regression algorithms.  Terry Speed's lab,
at Berkeley, identified an intensity-dependent dye bias concerning
c\smallcaps{dna} arrays.  In these arrays, the $\log_2$ of the dye intensity
ratios shows a systematic dependence on intensity, characterized by a deviation
from zero for low-intensity spots.  Frequently, under-expressed genes appear
up-regulated in the red channel ($R$), and moderately expressed genes appear
up-regulated in the green channel ($G$).
% This artifact is the result of a ``quenching'' effect, whereby dye molecules
% in close proximity tend to re-absorb light from each other, hence diminishing
% the signal.  The amount of re-absorption is a function of the template
% concentration and differs for the two dyes.
This effect can be visualized by plotting the measured
$\log_2(\frac{R_{i}}{G_{i}})$ for each feature in the array as a function of the
$\log_2(R_{i}G_{i})$ product intensities.  This ratio-intensity plot is termed
\smallcaps{ma} plot.\footnote[][-4.5cm]{The name of the plot comes from
  ``minus'' and ``add'', respectively the ratio and product in the logarithmic
  scale.}  This technical bias may be corrected by fitting a locally weighted
regression, known as \emph{lowess} smoothing
(Figure~\ref{fig:ma-plot}).\cite[-4.2cm]{yang_normalization_2001} More specific
sources of technical bias, including spatially-dependent bias resulting from the
print tips used in the manufacturing process of the array, may also be addressed
by a \emph{lowess}-based, within group normalization.

\begin{marginfigure}[-4.2cm]%
  \begin{center}
    \includegraphics{lowess-normalization-23Feb2013.pdf}
    \caption[\emph{Lowess} normalization]{Example of \emph{lowess}
      normalization.  \textbf{A:}~\smallcaps{ma} plot showing colour dye
      dependent bias.  \textbf{B:}~\smallcaps{ma} plot after correction with
      \emph{lowess} normalization
      (\citealp{yang_normalization_2002}).}\label{fig:ma-plot}%
  \end{center}
\end{marginfigure}

\emph{Affymetrix GeneChip} are the reference arrays in the single-channel class,
and the platform of choice for the development of between-array normalization
methods.  \emph{Affymetrix} chips consist of several tens of thousands
probe-sets.  A probe-set is a collection of probe pairs designed to interrogate
a specific sequence and contains \numrange{11}{20} probe pairs of 25-mer
oligonucleotides each.  Each probe pair consists of a perfect match probe
(\smallcaps{pm}) and a mismatch probe (\smallcaps{mm}).  The \smallcaps{mm}
probe differs from the \smallcaps{pm} probe by a single substitution at the
center base position, conceived to disturb the binding of the target gene
transcript.  This design allows for the quantification of background and
nonspecific hybridization effects.

Different between-array normalization methods have been proposed in the
literature.  The \smallcaps{mas5} algorithm\cite{hubbell_robust_2002} normalizes
each array independently and sequentially and uses the value of \smallcaps{mm}
probes to compute summarized averages with linear scaling.  The \smallcaps{rma}
(robust multi-array average) algorithm\cite{irizarry_exploration_2003}
normalizes the value of each probe by quantile normalization\footnote{Quantile
  normalization is a global adjustment method that assumes the statistical
  distribution of expression values of each sample is the same.  Normalization
  is achieved by imposing the same distribution to all samples, using an average
  distribution as reference.  The average distribution is estimated from the
  average of each quantile across samples.} in multiple arrays, neglecting
information from \smallcaps{mm} probes.  The \smallcaps{gcrma}
algorithm\cite{wu_model-based_2004} applies the same normalization and
summarization methods as \smallcaps{rma}, but uses probe sequence information to
estimate and correct for probe affinity to non-specific binding.  More recently,
the f\smallcaps{rma} (frozen \smallcaps{rma}) algorithm\cite{mccall_frozen_2010}
leverages pre-computed estimates of probe-specific effects and variance from
public microarray databases to, in concert with the information from new arrays,
normalize and summarize the data.

%%%%%%%%%%%%%%%%%%%%%%%%%%%%%%%%%%%%%%%%%%%%%%%%%%%%%%%%%%%%%%%%%%%%%%%%%%%%%%%%
%% http://www.people.vcu.edu/~mreimers/OGMDA/normalize.expression.html
%%%%%%%%%%%%%%%%%%%%%%%%%%%%%%%%%%%%%%%%%%%%%%%%%%%%%%%%%%%%%%%%%%%%%%%%%%%%%%%%

% The first ``\mbox{across-array}'' normalization methods relied on the
% assumption that most genes, and in particular ``housekeeping''
% genes\footnote{\emph{Housekeeping} genes are genes recognized as participating
% in basic, thus universal, cellular processes.}, should not be expected to be
% differentially expressed across samples.  However, housekeeping genes have
% since been shown to vary in expression by 30\% or more across healthy samples,
% and even more in tumour samples.\cite{lee_control_2002} This lead to new
% approaches to identify stable patterns of expression around which to pin
% global levels of expression.  Yet, these approaches have been shown to suffer
% from significant levels of cross-hybridization between \mbox{25-mers} in
% \emph{Affymetrix} chips.

% Terry Speed's lab, at Berkeley, identified an important intensity-dependent
% dye bias regarding double channel microarrays, and introduced a popular method
% for adjusting it.  In these arrays, the $log_{2} (ratio)$ shows a systematic
% dependence on intensity, characterized by a deviation from zero for
% low-intensity spots.  Typically, under-expressed genes appear up-regulated in
% the red channel ($R$) and moderately expressed genes appear up-regulated in
% the green channel ($G$).  This artifact is the result of a ``quenching''
% effect, whereby dye molecules in close proximity tend to re-absorb light from
% each other, hence diminishing the signal.  The amount of re-absorption is a
% function of the template concentration and differs for the two dyes.  This
% effect can be visualized by plotting the measured
% $log_{2}(\frac{R_{i}}{G_{i}})$ for each feature in the array as a function of
% the $log_{2}(R_{i}G_{i})$ product intensities.  This ratio-intensity plot is
% termed \smallcaps{ma} plot.\footnote{The name of the plot comes from ``minus''
% and ``add'', or ratio and product in the logarithmic scale.}

%%%%%%%%%%%%%%%%%%%%%%%%%%%%%%%%%%%%%%%%%%%%%%%%%%%%%%%%%%%%%%%%%%%%%%%%%%%%%%%%

% Among these are methods designed to normalize single chips, known as
% ``within-arrays'' methods, such as quantile normalization, or \smallcaps{mas5}
% for \emph{Affymetrix} chips.

Several reviews on these and other normalization methods were produced in the
specialized
literature,\cite{ploner_correlation_2005,bolstad_comparison_2003,harr_comparison_2006}
and a more detailed technical exposition of the computational implementation of
these algorithms can be found in Gentleman et
al.,\cite{gentleman_bioinformatics_2006} in the context of the
\href{http://www.bioconductor.org/}{\textsf{Bioconductor}} project.\footnote{The
  Bioconductor project is an open source and open development software project
  for the analysis and comprehension of genomic data
  (\citealp{gentleman_bioconductor:_2004}).  It is rooted in the open source
  statistical computing environment \textsf{R}.}

\medskip

The normalized microarray data for $p$ genes and $n$ biological samples are
denoted by $X_{n \times p}$ such that $x_{ij}$ represents the expression of gene
$j$ of sample $i$.

\section{Microarray data analysis}
\label{sec:microarray-methods-data-analysis}

%%%%%%%%%%%%%%%%%%%%%%%%%%%%%%%%%%%%%%%%%%%%%%%%%%%%%%%%%%%%%%%%%%%%%%%%%%%%%%%%
%% http://discover.nci.nih.gov/microarrayAnalysis/Exploratory.Analysis.jsp
%%%%%%%%%%%%%%%%%%%%%%%%%%%%%%%%%%%%%%%%%%%%%%%%%%%%%%%%%%%%%%%%%%%%%%%%%%%%%%%%

In cancer research, analysis of genomic experiments performed with
\smallcaps{dna} microarray data may address a variety of tasks, including
finding gene associations with particular phenotypes, tumour class discovery or
tumour class prediction.  The work contributed in this thesis concerns:
(\emph{a}) the class prediction of tumour types based on gene expression
profiles; and (\emph{b}) the modeling of association of gene expression profiles
with the time until a particular outcome is observed (e.g., death of a patient
or relapse of disease), using survival analysis.  Both these problems are
examples of application of supervised learning techniques to the analysis of
microarray data.

In machine learning,\footnote{Machine learning is a branch of computational
  science whose purpose is to implement algorithms capable to infer models from
  training data.  Models derived from machine learning methods are then expected
  to make predictions or to inform decisions on testing data.} supervised
learning methods seek to infer a function from labeled training data.
Conversely, unsupervised learning methods aim to find intrinsic structure from
unlabeled data.\cite{webb_statistical_2003} Unsupervised learning is thus suited
to uncover coherent genomic signals from microarray data, whereas supervised
learning can be used to find associations between genomic signals and phenotypic
classes of samples.

Due to the high dimensionality of expression profiles, methods for
dimensionality reduction are required for microarray data analysis.  These
include \emph{feature transformation} methods and \emph{feature selection}
methods.\cite{haibe-kains_identification_2009}

Feature transformation is an unsupervised approach that consists in reducing the
feature space of a microarray gene expression matrix, such that new features
retain biological pertinence, maximum information and generalizability to
similar experiments:
\begin{equation}
  \label{eq:feature-transformation}
  X_{n \times p} \to X'_{n \times p'} : p \gg p'.
\end{equation}
Examples of microarray feature transformation techniques include principal
component analysis and clustering methods.

Feature selection techniques, on the other hand, are supervised approaches to
reduce data dimensionality.  In this work, particular attention will be given to
gene expression signatures, which configure an example of microarray feature
selection.  Gene expression signatures, or metagenes, are collections of genes
sharing a combined expression pattern associated with a given phenotype.  The
range of biological phenotypes confining the characterization of expression
signatures include the modulation of signaling pathways,\cite{itadani_can_2008}
the specification of tumour classes,\cite{ramaswamy_multiclass_2001} or the
definition of distinct clinical outcomes.\cite{vant_veer_gene_2002} In the last
two decades, a wealth of microarray studies on perturbed \emph{in vitro}
biological systems have generated an extensive number of gene signatures related
to various cellular mechanisms.\cite{chibon_cancer_2013} Public repositories,
like the Gene Ontology consortium (\smallcaps{go}),\cite{ashburner_gene_2000}
the Kyoto Encyclopedia of Genes and Genomes
(\smallcaps{kegg}),\cite{kanehisa_kegg:_2000}
\mbox{GeneSigDB},\cite{culhane_genesigdb:_2012} or the Molecular Signatures
Database (\mbox{MSigDB}),\cite{subramanian_gene_2005} have sought to curate and
articulate this volume of information in order to guide the interpretation of
genome-wide expression profiles.  Because biologically motivated gene signatures
can act as surrogate markers for the molecular processes they capture, they
provide entry points for hypothesis testing in public microarray data.

The use of a common vocabulary to refer to microarray features is thus essential
to this goal.  Given the wide range of microarray platforms on the market,
preprocessing routines often produce genomic expression data with feature
annotations that are disjoint, inconsistent, or conflicting.  Computational
strategies to interface curated annotation databases in order to update and
standardize feature nomenclatures are generally poorly discussed in the
literature.  A solid foundation for \emph{in silico} solutions to bridge the gap
between the knowledge of transcript sequence and the knowledge of transcript
function is provided in Gentleman et al.\cite{gentleman_bioinformatics_2006}

Accordingly, for all datasets used in this work (described in the
\hyperref[sec:methods-datasets]{\textsf{Microarray datasets}} section),
\href{http://www.bioconductor.org/}{\textsf{Bioconductor}} resources were used
to update feature annotations, and referents for \smallcaps{hugo} Gene
Nomenclature Committee (\smallcaps{hgnc}) gene
symbols\footnote{\href{http://www.genenames.org/}{http://www.genenames.org/}}
were universally retained as feature descriptors.  Microarray gene annotation
may also be used to perform dimension reduction of the expression feature space.
Hence, expression matrices where multiple features addressed the expression of
the same gene product were collapsed using a \texttt{maxSum}
routine.\cite{miller_strategies_2011}

All analyses described in the \hyperref[chap:results]{\textsf{Results}} chapter
were conducted in the \href{http://www.r-project.org/}{\textsf{R}} environment
for statistical computing,\cite{r_core_team_r:_2014} with extensive use of
computational resources from the
\href{http://www.bioconductor.org/}{\textsf{Bioconductor}}
project.\cite{gentleman_bioconductor:_2004}

% metagenes. how are they derived. correlations with gene expression. they can
% behave as surrogate markers of biological processes, states or responses and
% be used as classifiers.  Metagenes can also be also non-biologically
% motivated, and thus be defined randomly.

% a word about meta-data resources and bioconductor tools.

%%%%%%%%%%%%%%%%%%%%%%%%%%%%%%%%%%%%%%%%%%%%%%%%%%%%%%%%%%%%%%%%%%%%%%%%%%%%%%%%
%% abstract of chapter seven of the book
%% Bioinformatics and Computational Biology Solutions Using R and Bioconductor
%% Meta-data Resources and Tools in Bioconductor
%%%%%%%%%%%%%%%%%%%%%%%%%%%%%%%%%%%%%%%%%%%%%%%%%%%%%%%%%%%%%%%%%%%%%%%%%%%%%%%%

% Closing the gap between knowledge of sequence and knowledge of function
% requires aggressive, integrative use of biological research databases of many
% different types.  For greatest effectiveness, analysis processes and
% interpretation of analytic results must be guided using relevant knowledge
% about the systems under investigation.  However, this knowledge is often
% widely scattered and encoded in a variety of formats.  In this section, we
% consider some of the different sources of biological information as well as
% the software tools that can be used to access these data and to integrate them
% into an analysis.  Bioconductor provides tools for creating, distributing, and
% accessing annotation resources in ways that have been found effective in work-
% flows for statistical analysis of microarray and other high-throughput assays.

%%%%%%%%%%%%%%%%%%%%%%%%%%%%%%%%%%%%%%%%%%%%%%%%%%%%%%%%%%%%%%%%%%%%%%%%%%%%%%%%
%%%%%%%%%%%%%%%%%%%%%%%%%%%%%%%%%%%%%%%%%%%%%%%%%%%%%%%%%%%%%%%%%%%%%%%%%%%%%%%%

\medskip

This section will proceed with a brief overview of tools for visualization of
genomic data, namely heatmaps and multidimensional scaling.  It will then
discuss the unsupervised learning tools used in this work, including principal
component analysis, clustering analysis and a summary on machine learning
algorithms.  Next, it will cover tools to evaluate the performance of
classifiers, including \smallcaps{roc} curves (for binary classifiers) and
\smallcaps{gsea} (an algorithm that uses microarray data to determine whether a
gene expression classifier is statistically associated with any of two
phenotypic classes of biological samples).  Finally, it will detail survival
analysis, the branch of supervised learning that seeks to explain the
relationship between a number of measured features (gene expression data) and
the time duration until the occurrence of a particular event (survival outcome).

\subsection{Visualization techniques}
\label{sec:methods-visualization}
% 02Mar2015

\subsection{Principal component analysis}
\label{sec:methods-pc}
% 03Mar2015

\subsection{Clustering analysis}
\label{sec:methods-clustering}
% 04Mar2015

\subsection{Machine learning analysis}
\label{sec:methods-machine-learning}
% 05Mar2015

\subsection{Receiver operating characteristic curves}
\label{sec:methods-roc}
% 06Mar2015

\subsection{Gene set enrichment analysis}
\label{sec:methods-gsea}
% 07Mar2015

\subsection{Survival analysis}
\label{sec:methods-survival-analysis}

Survival analysis is a collection of statistical procedures for data analysis
for which the outcome variable of interest is time until the event
occurs.\cite{kleinbaum_survival_1996} In follow-up studies of cancer patients,
survival analysis is used to model association of the expression of genomic
markers in cancer biopsies with the time until a given clinical outcome is
observed.

Clinical outcomes may include death of the patient (overall survival, or
\smallcaps{os}), death of the patient caused by the cancer (disease-specific
survival, or \smallcaps{dss}), the finding of new metastases in the patient
(distant metastasis free survival, or \smallcaps{dmfs}) or recurrence of the
cancer (disease-free survival, or \smallcaps{dfs}).  \emph{Survival time} refers
to the lapse of time since the beginning of the study up to the moment when an
event is observed, regardless of the clinical outcome considered.  Whenever the
information about an individual's survival time is incomplete, that observation
is said to be censored (Figure~\ref{fig:censorship}).  This may be due because
the event was not observed by the end of the study (in which case the follow-up
time considered for that patient is the entire duration of the study) or because
the patient quit or withdrew from the study before its end (in which case the
follow-up time considered for that patient is the time up to dropping out).

\begin{marginfigure}%
  \includegraphics{censorship-28Feb2015.pdf}
  \caption[Right-censored survival data]{A schematic representation of
    right-censored survival data.  Survival time is said to be
    \emph{right-}censored when the information regarding the right side of the
    follow-up period is incomplete.  Observed events are denoted by (\CIRCLE).
    Censored observations are denoted by (\Circle).  Notice that patient
    \smallcaps{b} is also censored, as no event had been observed by the end
    of the study (see text for details).}\label{fig:censorship}%
\end{marginfigure}

Follow-up studies are not amenable to ordinary regression models because the
time to event is typically not normally distributed and these models cannot
incorporate censoring data.  Instead, survival analysis uses two functions to
estimate survival time, the \emph{survival} function and the \emph{hazard}
function.

The survival function, $S(t)$, describes the probability of an event occurring
later than some specified time $t$:
\begin{equation}
  \label{eq:survival-function}
  S(t) = P"r(T > t),
\end{equation}
where $T$ is a random variable for a patient's survival time.

The hazard function, $h(t)$, describes the instantaneous potential per unit time
for the event to occur, given the patient has survived up to time $t$:
\begin{equation}
  \label{eq:hazard-function}
  h(t) = \lim_{\Delta t \to 0}\frac{P"r\{t \leqslant T \less t + \Delta t
    \mid T \geqslant t \}}{\Delta t}.
\end{equation}

% While the survival function is non-increasing i.e., $S(t)$ heads downwards as
% $t$ increases, (Figure~\ref{fig:survival-function}), the hazard function
% describes a failure rate conditional to the interval of time considered, and
% can therefore be modeled by any number of distributions.

Both functions are related to each other.  However, while the survival function
is non-increasing (Figure~\ref{fig:survival-function}), the hazard function may
be modeled by any number of distributions, as it describes a failure rate
conditional to the interval of time considered.  Building on these functions,
survival analysis stipulates a suite of parametric, non-parametric and
semi-parametric methods to make inferences over survival time.  These methods
can then be used to ascertain the relationship between a variable of interest
and the time to a clinical outcome.

\begin{marginfigure}%
  \includegraphics{survival-function-28Feb2015.pdf}
  \caption[Survival function]{The survival function, \emph{S}\,(\emph{t}),
    describes the likelihood that a patient will have a lifetime exceeding time
    \emph{t}.  \textbf{A:}~The theoretical distribution is non-increasing, and
    characterized by \mbox{\emph{S}\,(0) = 1} and \mbox{\emph{S}\,($\infty$) =
      0}.  \textbf{B:}~In practice, the estimated survival function,
    \emph{\^{S}}\,(\emph{t}), often takes a shape of a step function.  Because
    study periods are never infinite and there may be competing risks for
    failure, it is likely that not all patients will experience a clinical
    outcome by the end of the study.}\label{fig:survival-function}%
\end{marginfigure}

In the biomedical literature, the most recognizable non-parametric method to
estimate the survival function is the Kaplan-Meier
estimator.\cite{kaplan_nonparametric_1958} The Kaplan-Meier estimator is defined
as the probability of surviving in a given length of time while considering time
in many small intervals.\footnote{\citealp[pp. 365--93]{altman_practical_1990}}
It estimates the probability of occurrence of an event at time $t$ by
cumulatively multiplying prior probabilities of survival at preceding $t_i$
intervals.  The Kaplan-Meier, or product limit estimator, is thus formulated as:
\begin{equation}
  \label{eq:kaplan-meier}
\hat{S}(t)=\prod_{t_i<t}\frac{n_i-d_i}{n_i},
\end{equation}
where $n_i$ and $d_i$ are, respectively and for each prior time interval $t_i$,
the number of patients at risk (right-censored observations are removed), and
the number of patients experiencing an event.  The Kaplan-Meier estimator can be
used to obtain univariate descriptive statistics for survival data or to compare
the survival time for two or more groups of subjects.

The logrank test is a non-parametric hypothesis test to compare the survival
distribution of two
samples.\cite{mantel_evaluation_1966,peto_asymptotically_1972} It challenges the
null hypothesis that there is no difference between the populations in the
probability of an event at any time point.  To do so, it compares the estimates
of the hazard functions of the two groups at each observed event time.  The
logrank test is based on the same assumptions as the Kaplan-Meier survival
curve---namely, that censoring is unrelated to prognosis, the survival
probabilities are the same for subjects recruited early and late in the study,
and the events happened at the times specified.\cite{bland_logrank_2004}
Importantly, both the Kaplan-Meier estimator and the logrank test are able to
incorporate right-censored survival data.

When modeling the presence of covariates or explanatory variables to explain
survival time, fully parametric and semi-parametric approaches are available.
Parametric approaches, like the accelerated life class of models, assume that
the effect of covariates is proportional with respect to survival
time.\cite{kalbfleisch_statistical_2011} Under this model,
\begin{equation}
  \label{eq:aft}
  S_1(t) = S_0(t/\gamma).
\end{equation}
Effectively, this means that the probability that a member of group one will be
alive at time $t$ is exactly the same as the probability that a member of group
zero will be alive at time $t/\gamma$.  In parametric models, the estimation of
the covariates is conditional to the prior definition of the hazard
distributions in both groups.

% \footnote{For instance, for $\lambda = 2$, this would be half the age, so the
% probability that a member of group one would be alive at age 40 (or 60) would
% be the same as the probability that a member of group zero would be alive at
% age 20 (or 30). Thus, $\lambda$ can be interpreted as affecting the passage of
% time.  In this example, people in group zero age ``twice as fast''.}

Alternatively, the Cox proportional hazards family of models assumes that the
effects of the covariates is proportional with respect to the
hazard.\cite{cox_regression_1972} This assumption obviates the need of
specifying the underlying hazard functions---the model only seeks to fit the
regression parameters, hence being referred to has semi-parametric.  In its
simplest form, it can be formulated in such a way that the hazard at time $t$
for an individual with covariates $X$ is assumed to be:
\begin{equation}
  \label{eq:proportional-hazards}
  h(t \mid X)=h_0(t) \exp(\beta_1 X_{1}+\ldots+\beta_pX_{p}).
\end{equation}
This model formalizes the hazard of individual $i$ at time $t$ as the product of
a baseline hazard function, $h_0(t)$, and of a linear function of a set of $p$
covariates, which is exponentiated.  Along with the specification of the model,
Sir David Cox developed a maximum partial likelihood method for estimation
of its covariates.\cite{cox_regression_1972} A logrank statistic can be
derived as the score test for the Cox proportional hazards model comparing two
groups.  The term Cox regression refers to the combination of both the model and
the estimation procedure.  It has become the tool of choice to estimate the
association of the expression of genomic metagenes with differential survival
times in cancer follow-up studies.

% An important advantage The modeling of the survival time by the Kaplan-Meier
% can also incorporate right-censored data

% Parametric methods assume that the underlying distribution of survival times
% follow certain known probability distributions.  For instance, the hazard
% function can be assumed to:
% \begin{itemize}
% \item be constant over time, e.g., when considering the likelihood of an healthy
%   individual to become ill;
% \item follow an increasing Weibull distribution, e.g., when considering the
%   likelihood of death for cancer patients not responding to treatment---as it is
%   expected to increase with time;
% \item follow an decreasing Weibull distribution, e.g., when considering the
%   likelihood of death for post-surgical cancer patients with fair prognosis---as
%   it is expected to decrease with time;
% \item follow a lognormal distribution, e.g., when considering the likelihood of
%   death for tuberculosis patients---as it first increases upon diagnosis, only
%   to decrease later.
% \end{itemize}
% Once the parametric function of choice is defined, model parameters
% are usually estimated using an appropriate modification of a maximum likelihood
% procedure.

% be constant over time (e.g., when considering the likelihood of an healthy
% individual of becoming ill); follow an increasing Weibull distribution (e.g.,
% when considering the likelihood of death for cancer patients not responding to
% treatment, as it is expected to increase with time); follow a decreasing
% Weibull distribution (e.g., when considering the likelihood of death for
% post-surgical cancer patients, as it is expected to decrease with time), or
% follow a lognormal distribution (e.g., when considering the likelihood of
% death for tuberculosis patients, as it first increases after diagnosis, then
% decreases).

% In follow-up studies of cancer patients, it is generally of interest to
% describe the relationship of the expression of a biologically motivated
% metagene to the time to event---possibly in the presence of clinical
% covariates, such as age, treatment, or tumour class.  This calls for models
% that describe the relationship between the predictor variable and survival
% time.  Survival analysis offers a range of parametric, non-parametric and
% semi-parametric approaches to model this relationship.

\medskip

When considering the association of genomic markers with survival time, the
dependent variable is composed of two parts: one consisting of the time to the
event (or lack thereof), encoded as a non-negative number; the other registering
the event status, encoded as a binary variable (routinely 1 if the event is
observed, and 0 if not).

Typically, the predictor is also a binary class, the result of discretizing the
patients in groups of good and bad prognosis as a function of the metagene
classifier.  This requires a method to stratify the cohort with an unsupervised
classification procedure.  Venet et al.\cite{venet_most_2011}\,investigated
three methods to this goal: (\emph{a})~the standard spliting of the cohort along
the two main clusters defined by hierarchical clustering of the metagene's
expression data;
% \footnote{Computed with the \texttt{hcluster} function of the \textsf{amap}
% \textsf{R} package, and implemented with average linkage and (1
% -- \emph{correlation}) between pairs of observations as the dissimilarity
% measure.}
(\emph{b})~spliting the cohort along the two main clusters defined
by applying the \emph{k}-means clustering algorithm to the metagene's expression
data;
% \footnote{Computed with the \textsf{R} function \texttt{kmeans}.}
and (\emph{c})~spliting the cohort along the median of the metagene's first
principal component.
% \footnote{Computed with the \textsf{R} function \texttt{prcomp}.}
They confirmed that the methods based on the first principal
component (henceforth referred to as \textsf{PC1} method) yielded significantly
higher hazard ratios and smaller \emph{p}-values.

In this work, we chose to model survival time as a function of a single
\emph{continuous} predictor, defined by the first principal component of the
metagene's expression (or by a single vector of gene expression) throughout the
entire cohort.  This departure from the standard approach has the disadvantage
of making it impossible to calculate a Kaplan-Meier estimator for the
model---therefore lacking a visual representation and making it harder to assess
the validity of the proportional hazards assumption.  However, it has the
advantage of using a predictor variable that faithfully mirrors the patterns of
expression captured by the metagene and of not requiring the artificial
stratification of the cohort in (balanced or not) groups of differential
prognosis.

Whenever applicable, association of metagene expression with outcome was thusly
computed by fitting a proportional hazards regression model between predictor
and dependent variables with the \texttt{coxph} function of the \textsf{R}
\textsf{survival} package.

% We have shown this approach to be more sensitive (data not shown).

\section{Microarray datasets}
\label{sec:methods-datasets}


%%% Local Variables:
%%% TeX-engine: xetex
%%% mode: latex
%%% TeX-master: "../thesis"
%%% End:



\chapter{Results}
\label{chap:results}

This chapter is comprised of three sections.  The first consists of a published
article, entitled ``A general method to derive robust organ-specific gene
expression-based differentiation indices: application to thyroid cancer
diagnostic.''\cite{tomas_general_2012} The results exposed in this manuscript
are developed in the
\hyperref[discussion-differentiation-microarrays]{\textsf{Discussion}} chapter,
under the section \emph{Differentiation and proliferation signatures in cancer
  diagnostic}.

The second section, entitled ``The extent of prognostic signals in the cancer
transcriptomes,'' details the results of a systematic analysis of 114 microarray
datasets of cancers affecting 22 different tissue types, aiming at quantifying
the range and the nature of the signals linked with differential survival in
neoplastic expression profiles.  These results are elaborated upon in the
corresponding section of the
\hyperref[sec:discussion-prognostic-microarrays]{\textsf{Discussion}} chapter.

The remaining section reports other publications to which the author contributed
during this dissertation.

% The results of this analysis are currently under revision.

% \resetcitations

\section{Differentiation and proliferation signatures in cancer diagnostic}
\label{sec:results-differentiation-proliferation}

\emph{See next page.}

% \includepdf[pages=1-9,fitpaper=true]{../pdf/include/onc-tomas-2012.pdf}
\includepdf[pages=1-9,trim=5cm 3cm 5cm 3cm]{../pdf/include/onc-tomas-2012.pdf}
% \includegraphics{../pdf/include/onc-tomas-2012.pdf}

\section{The extent of prognostic signals in the cancer transcriptomes}
\label{sec:results-prognostic-survival}

The vast majority of mechanistic cancer studies are performed in animals and/or
\emph{in vitro} models for ethical reasons.  Proving that a biological mechanism
established in these experimental systems contributes to cancer progression in
human requires clinical trials, which are costly and take years to perform.
Therefore, researchers have relied on weaker correlative evidence to back the
relevance of their findings to human diseases.  One such evidence is the
statistically significant association of a molecular maker of the biological
mechanism under investigation with disease outcome in human.  Such association
can be established with non-interventional clinical studies and has been
extensively used in the past decades.  Moreover, this approach has recently
gained popularity with the availability of web
servers\cite{gyorffy_online_2010,ringner_gobo:_2011,gyorffy_online_2013} that
provide, for free, association statistics between any single- or multi-gene
markers and cancer outcomes using publicly available cancer transcriptome
databases.

To retain a biological bearing, the reported association must however be
specific of the biomarker under analysis and not reflect a global property of
the transcriptome.  Most studies estimate the association with a log-rank test
based on a Cox Proportional Hazards model and use a Kaplan-Meier curve to
visualize it.  None of these tools control for the possibility that a large
fraction of transcriptome could be associated with outcome and that, therefore,
the prognostic signal is non-specific.

Disturbingly, this possibility has been proven true in
bladder\cite{lauss_prediction_2010} and
breast\cite{ein-dor_outcome_2005,mosley_cell_2008,venet_most_2011} cancers.  In
particular, our group has shown\cite{venet_most_2011} that more than half of the
transcriptome is correlated with proliferation in two breast cancer cohorts and
that one out of four single genes and that nine out of ten random signatures
comprised of more than 100 genes were associated with overall survival at
$p<0.05$.  As a consequence, most published signatures were found to be
significantly prognostic, but no more prognostic than random sets of genes.

To investigate whether the pervasive association of the transcriptome with
outcome extends to other cohorts and other types of cancers we compiled 114
published gene-expression datasets with patient follow-up from the
\smallcaps{geo}\cite{edgar_gene_2002}, InSilicoDB\cite{coletta_insilico_2012}
and the \smallcaps{tcga} Research Network
databases.\footnote{\href{http://cancergenome.nih.gov/}{http://cancergenome.nih.gov/}}
These included human cancers from 19 organ systems, and a representative range
of microarray platforms (Table~\ref{tab:datasets}).

\begin{figure}%
  \includegraphics{fraction-significant-tests-base-R-graphics-15Jan2015.pdf}
  \caption[Prognostic content in human cancers]{Prognostic content in human
    cancers.  The fraction of markers significantly associated with outcome at
    logrank $p<0.05$ is shown for single-gene markers, multi-gene markers from
    the \mbox{MSigDB \smallcaps{c2}} database, and randomly generated
    multi-genes markers devoid of biological meaning.  The latter was used to
    order the datasets. Fractions were computed from death-related end-points
    when available (\smallcaps{os} or \smallcaps{dss}), or relapse otherwise
    (\smallcaps{dfs} or \smallcaps{dmfs}). The dotted red line marks the 0.05
    threshold.}
  \label{fig:prognostic-fraction}%
\end{figure}

For each dataset, we computed the fraction of genes associated with outcome at
log-rank $p<0.05$ (Figure~\ref{fig:prognostic-fraction}).  This quantity may be
viewed as the probability of observing a significantly prognostic single gene
marker by chance alone.  We thus refer to these estimates as the baseline
prognostic content\footnote{Prognostic content will henceforth refer to the
  fraction of tested markers found associated with outcome in a given dataset.}
of cancer transcriptomes.  Overall, the median prognostic content across all
studies for single-gene markers was 12\%.  In 100 of the 114 datasets
analyzed (88\%), more than five percent of single-gene markers showed a
significant association with outcome.  % It was higher than 5\% and random %
% multi-gene markers was, respectively, 12\% and 16\%.                     %
% The median fraction was 9%. It was higher than 5% for 95 (83%) datasets.
To take an extreme example, an investigator measuring the association of a gene
with outcome in the \smallcaps{kirc} datasets (kidney cancer) has a 50\% chance
to obtain a positive result---a value far above the canonical 5\% significance
threshold.

To investigate recent multi-gene approaches, we ran a similar calculation for
each of the 4722 curated gene sets of the \mbox{MSigDB \smallcaps{c2}}
database\cite{liberzon_molecular_2011} (Figure~\ref{fig:prognostic-fraction}).
% The prognostic fraction was larger for signatures than for single genes, with
% a median of 12%.
The prognostic content was larger for multi-gene markers than for single-gene
markers, with a median of 19\%.  The prognostic content for multi-gene markers
was larger than 5\% in 76\% of the datasets (87 out of 114), larger than 20\% in
48\% (55/114), and larger than 50\% in 19\% (22/114) of the datasets.  To
control for possible biases of \mbox{MSigDB \smallcaps{c2}} towards
oncology-related signatures, we reran the same computation, but replacing each
signature by a similarly sized set of randomly selected genes.  The overall
qualitative result is unchanged (Figure~\ref{fig:prognostic-fraction}).

Intriguingly, single-gene prognostic content was found to be largely
heterogeneous across datasets related to the same organ system.  For example, it
ranged from 4\% to 59\% among the 33 breast cancer datasets analyzed.  To
investigate the contributions of potential biological and demographic
dataset-specific factors to this effect, we quantified single-marker prognostic
fraction for re-samplings of the \num{1972} transcriptomes of the
\smallcaps{metabric} breast cancer cohort, regarding modulations of four
variables: sample size, duration of follow-up time, fraction of \smallcaps{er}+
patients, and fraction of node positive patients
(Figure~\ref{fig:bootstrap-metabric}).  We chose \smallcaps{metabric} for this
analysis because it is one of the largest cohorts of cancer patients with
follow-up and extensive clinical annotation data available in the public domain.

\begin{figure*}
  \includegraphics{metabric-sims-19Mar2015.pdf}
  \caption[Bootstrapping experiments on the \smallcaps{metabric}
  dataset]{Bootstrapping experiments on the \num{1972} combined breast cancer
    transcriptomes of the \smallcaps{metabric} dataset.
    \textbf{A--}Distribution of the prognostic fraction in 100 samplings of 500
    expression profiles, out of the total 1972 \smallcaps{metabric} profiles.
    \textbf{B--}Effect of sample size.  \num{100} samplings (grey points) were
    assessed for each specified sample size.  \textbf{C--}Effect of follow-up
    times.  For each time $t$, \num{100} samplings were assessed for which
    patients beyond time $t$ were considered censored at time $t$.
    \textbf{D--}Effect of the fraction of \smallcaps{er}+ patients.  \num{100}
    samplings, each of 300 patients, were assessed with the respective fraction
    of \smallcaps{er}+ samples.  \textbf{E--}Effect of the fraction of node
    positive patients.  \num{100} samplings, each of 500 patients, were assessed
    with the respective fraction of node positive patients.}
  \label{fig:bootstrap-metabric}
\end{figure*}

The estimates of prognostic content were found to be markedly sensitive to
sampling variance, as suggested by the dispersal of the distribution of
estimates (confidence interval: 12\% to 23\%), when \num{100} samplings of
\num{500} random transcriptomes were examined for the fraction of genes
associated with outcome (Figure~\ref{fig:bootstrap-metabric}\emph{a}).  This
feature alone is likely to yield a significant contribution to the range of
estimates computed in our meta-analysis, as 108 of the 114 datasets analyzed
included less than \num{500} profiled tumours.  Moreover, the sensitivity of our
estimation procedure is, unsurprisingly, largely dependent on the number of
profiles included in the analysis, as shown by a bootstrapping experiment of
sample sizes towards the assessment of the prognostic fraction
(Figure~\ref{fig:bootstrap-metabric}\emph{b}).  Provocatively, the estimates of
prognostic content do not appear to level off even when the experimental sample
size reached \num{1750}---by far the largest in the field.  A sampling
experiment of sequential truncation of follow-up times was equally shown to
impact estimates of prognostic content
(Figure~\ref{fig:bootstrap-metabric}\emph{c}).  Sharply increasing predictive
fractions were observed up to the fifth year of follow-up, followed by a gradual
decrease of the estimates for higher follow-up times.  This trend suggests that,
in breast cancers, prognostic patterns of expression are optimally correlated
with short-term forms of progression of the disease, and that long-term forms of
progression are less efficiently predicted from primary tumour transcriptomes.
The modulation of the fraction of \smallcaps{er}+ transcriptomes towards
experimental samplings of our estimate has exposed a tendency congruent with the
clinical relevance of this receptor in breast cancer pathology
(Figure~\ref{fig:bootstrap-metabric}\emph{d}).  Thus, an increase of
\smallcaps{er}+ profiles to up to 50\% in our samplings leads to a corresponding
linear rise in estimates of prognostic content, at which point a further
increase in the proportion of \smallcaps{er}+ profiles yields little impact on
fraction estimates.  This observation is in line with the fact that the
predictive power of most signatures in breast cancer is mostly confined to
\smallcaps{er}+ phenotypes.\cite{weigelt_challenges_2012} A last sampling
experiment with increasing fractions of profiles from node positive patients
(Figure~\ref{fig:bootstrap-metabric}\emph{e}) also revealed an increasing
pattern of prognostic fraction estimates.  This trend could be explained by the
fact that nodal status is clinically correlated to \smallcaps{er} status in
breast cancer.

% First, repeated random selection of profiles reveals that sampling effects are   %
% substantial (prognostic fraction CI=6-17%) with 500 patients                     %
% (Fig. 2A), a size in par with the largest studies. Second, the prognostic        %
% fraction, as expected, depends on sample size (Fig. 2B). Interestingly, it shows %
% no sign of inflection as N reaches 1750, by far the largest sample size in the   %
% field. Third, we investigated the impact of follow-up time by artificially       %
% truncating follow-up data (Fig. 2C; Supplementary Methods). Shorter and longer   %
% maximum follow-up times yielded lower prognostic fractions, consistent with the  %
% fact that incomplete data, but also late events12, are not predictable from      %
% primary tumor transcriptomes. Fourth, increasing the fraction of ER+ patients up %
% to 50% increases the prognostic fraction, but has no effect beyond               %
% 50% (Fig. 2D). This is compatible with the notion that                           %
% transcriptional marker are more prognostic within the ER+ group and that the ER+ %
% patients have, on average, a better prognostic than ER- patients.13 Fifth,       %
% increasing the fraction of node positive patients, which is correlated with ER   %
% status, also increases the prognostic fraction (Fig. 2E). Sixth, lower           %
% cellularity is unexpectedly associated with a higher prognostic fraction         %
% (Fig. 2F). This may reflect the role of the microenvironment in breast cancer    %
% progression and the massive impact of cell types proportions on the              %
% transcriptomes of bulk tissues.14                                                %

Finally, dataset-specific processing details may also distort prognostic
estimates.  Consider, for instance, dataset \smallcaps{gse9893}, which exhibits
the highest prognostic fractions measured in our study
(Figure~\ref{fig:prognostic-fraction}).  A thorough reanalysis of this dataset,
detailed in the following section, reveals that its normalization was
performed in two batches and induced massive spurious correlations between
global values of expression and survival outcome.  Accordingly, a proper
single-batch normalization of the raw expression data restores a signal-to-noise
metrics to comparable values with other datasets, and decreases measurements of
prognostic content from 59\% to 19\% for \smallcaps{gse9893}.

We have shown that the fraction of prognostic single- and multi-gene biomarkers
is greater that 5\% in the majority of publicly available transcriptome
datasets.  Furthermore, we have demonstrated that the probability of a
significant single- or multi-gene marker association with outcome depends on
cohorts’ demographics, but also to a large extent on technical factors that
include sampling effects, cohort size, patient follow-up protocols, protocol
randomization and possibly other factors not addressed here.  These findings
call for the reappraisal of conclusions made by previous studies pertaining to
the implication of biological mechanisms to human cancer based on associations
of biomarkers with outcome---including low-throughput \smallcaps{pcr}-based
studies.  They also call for study-specific controls akin to those presented in
Figure~\ref{fig:prognostic-fraction} in future studies.

However, a biomarker does not need to convey relevant biological information
regarding the course of disease in order to be useful in the clinic.  Therefore,
our results have no bearing on the clinical utility of published biomarker
associations.

\subsection{Re-analysis of dataset \smallcaps{gse9893}}
\label{sec:reanalysis-gse9893}
% The following vignette details the re-analysis of the extent of prognostic
% signals in a breast cancer dataset and exposes how a normalization artifact
% can distort estimations of prognostic content.
To illustrate how procedural biases in the preprocessing of expression profiles
may impact estimates of association to outcome, we present here a detailed
re-analysis of dataset \smallcaps{gse}9893 (Table~\ref{tab:datasets}).

\smallcaps{gse}9893 is comprised of 155 samples of tamoxifen-treated primary
breast cancers.  These samples were hybridized on a homemade \mbox{70-mer} chip
containing \SI{22680} probes, mapping to \SI{21329} human specific genes.  The
original experiment was carried out to look for a gene expression signature to
predict the recurrence of tamoxifen-treated primary breast
cancer.\cite{chanrion_gene_2008}

The data-set was downloaded from \smallcaps{geo} with the \textsf{Bioconducor
  GEOquery} package, with original normalization.  The expression matrix was
then feature collapsed using a \textsf{maxMean} routine and median polished.

Among the 114 studies considered in our analysis, \smallcaps{gse}9893
shows the highest fraction of genes associated with outcome at
$p < 0.05$ (59\%).  Interestingly, nearly all \mbox{MSigDB
  \smallcaps{c2}} signatures appear associated with outcome in this
dataset (Table~\ref{tab:top-datasets}).

\begin{table}[ht]
  \begin{center}
    % \footnotesize%
    \begin{tabular}{lcc}
      \toprule
      Dataset                                  & Fraction of significant tests & Event           \\
      \midrule
      \smallcaps{GSE}9893-breast               & 0.958                         & \smallcaps{os}  \\
      \smallcaps{GSE}10846-lymphoma            & 0.856                         & \smallcaps{os}  \\
      \smallcaps{GSE}32894-bladder             & 0.847                         & \smallcaps{dss} \\
      \smallcaps{GSE}31210-lung-adenocarcinoma & 0.838                         & \smallcaps{dfs} \\
      \smallcaps{KIRC}                         & 0.821                         & \smallcaps{os}  \\
      \smallcaps{GSE}41258-colon               & 0.814                         & \smallcaps{dss} \\
      \bottomrule
    \end{tabular}
  \end{center}
  \caption[Top six studies with hightest fraction of \mbox{MSigDB
    \smallcaps{c2}} signatures associated with outcome]{Top six studies with
    highest fraction of \mbox{MSigDB \smallcaps{c2}} signatures associated with
    outcome.  Detailed information regarding each dataset can be found on to Table~\ref{tab:datasets}.}
  \label{tab:top-datasets}
\end{table}

A closer inspection of the metadata associated with the expression profiles
reveals that the arrays were scanned in two discrete time intervals during 2005
and 2006, separated by eight months (Figure~\ref{fig:array-dates}).
Surprisingly, patients whose tumours were hybridized in 2006 show a poorer
prognosis than those hybridized in 2005 (Figure~\ref{fig:km-gse9893-prior}).

\begin{marginfigure}%
  \begin{center}
    \includegraphics{gse9893-array-dates-23Feb2016.pdf}
    \caption[Time distribution of hybridization dates of the samples in
    \smallcaps{gse}9893]{Frequency distribution of hybridization dates of
      \smallcaps{gse}9893 samples relative to the first hybridization date.
      Information regarding date of hybridization of each of the 155 arrays was
      parsed from the \texttt{gpr} files downloaded from \smallcaps{geo}.  The
      dataset is composed by a batch of samples hybridized during May to July
      2005 and a second batch of samples hybridized during February and March
      2006, roughly 200 days apart.}
    \label{fig:array-dates}
  \end{center}
\end{marginfigure}

\begin{marginfigure}%
  \begin{center}
    \includegraphics{gse9893-km-prior-23Feb2016.pdf}
    \caption[Kaplan Meyer of dataset \smallcaps{gse}9893 discretized by time
    batches]{A Kaplan Meier visualization of differential overall survival,
      between patients included in \smallcaps{gse}9893 whose samples were
      hybridized in 2005 (in black) and those whose samples were hybridized in
      2006 (in red). Logrank test: $p = 1.76^{-10}$.}
    \label{fig:km-gse9893-prior}
  \end{center}
\end{marginfigure}

This observation can be explained by two facts.  First, we discovered a
normalization artifact in this dataset related to the 2005 and 2006 batches
of samples, as shown by the distribution of expression values and respective
batch associations with the first and second principal components of the global
expression matrix (Figure~\ref{fig:gse9893-prior-post-norm}, left panels).
Second, there is an enrichment in the 2006 batch of observed death events
compared with the 2005 batch (Figure~\ref{fig:km-gse9893-density-survival}).
Because the majority of observed events are linked with a subset of samples
whose global expression patterns were distorted due to the normalization
artifact, 59\% of genes in original matrix appear artificially associated with
overall survival in this dataset.

In order to correct for this bias, we downloaded the raw data \texttt{gpr} files
and proceeded to re-normalize them with the \textsf{Bioconductor limma} package.
As a result, the distribution of values of expression no longer showed a
correlation between batches of samples
(Figure~\ref{fig:gse9893-prior-post-norm}, right panels).  In addition, a
signal-to-noise quality metric based on gene-gene correlations across expression
profiles\cite{venet_measure_2012} suggests that the re-normalization of the
raw-data has significantly improved the data quality of \smallcaps{gse}9893
(Figure~\ref{fig:gse9893-snr}).  As a result, the fraction of genes associated
with outcome in this study is reduced from 58\% to 19\%, and only 74 out of the
original 4556 (2\%) \mbox{MSigDB \smallcaps{c2}} remain associated with overall
survival.

\begin{figure}%
  \begin{center}
    \includegraphics{gse9893-prior-post-normalization-28Feb2016.pdf}
    \caption[Batch effect in \smallcaps{gse}9893 prior and post
    re-normalization]{\textbf{A}--Gene expression distributions of each of the
      155 samples in \smallcaps{gse}9893. \mbox{\textbf{B}--\smallcaps{gse}9893}
      samples projected in the space of the first two principal components of
      their expression matrix.  \emph{Left,} original normalization;
      \emph{right,} quantile re-normalization on the original \texttt{gpr}
      files.  Samples in black are from the 2005 batch and samples in red are
      from the 2006 batch (See text for details).}
    \label{fig:gse9893-prior-post-norm}
  \end{center}
\end{figure}

\begin{marginfigure}%
  \begin{center}
    \includegraphics{gse9893-density-distribution-survival-events.pdf}
    \caption[Density distribution of overall survival events in
    \smallcaps{gse}9893]{Density distribution of overall survival
      events in \smallcaps{gse}9893. Censored observations are denoted
      in black and observed death events are denoted in red.  Out of
      the 116 patients in the 2005 batch, only 10 died during the
      course of the study; whereas out of the 49 patients whose
      samples where hybridized in 2006, 32 were observed events
      ($\chi^2$ test: $p = 1.42^{-12}$).}
    \label{fig:km-gse9893-density-survival}
  \end{center}
\end{marginfigure}

\begin{figure}%
  \begin{center}
    \includegraphics{gse9893-plot-hist-qual-study-1.pdf}
    \caption[Distribution of signal-to-noise quality metrics across 114 human
    cancer datasets]{Distribution of signal-to-noise quality metrics across the
      114 human cancer datasets included in our study. The
      signal-to-noise-ratios (\smallcaps{snr}) were computed with the
      \textsf{Bioconductor SNAGEE} package.  The black vertical line marks the
      value computed for \smallcaps{gse}9893.  The red vertical line shows the
      value computed for the re-normalized expression matrix of
      \smallcaps{gse}9893.  The \smallcaps{SNR} of a study is based on the
      correlation between its gene-gene correlation matrix and the expected
      matrix, and is thus a number between $-1$ and $1$.  Practically, numbers
      near or below 0 are symptomatic of seriously problematic studies
      (e.g. gene annotation problems, serious normalization issues).  Numbers
      around 20--30\% are average, depending on the platform.}
    \label{fig:gse9893-snr}
  \end{center}
\end{figure}

\clearpage

% \includepdf[pages=1-9,trim=5cm 3cm 5cm
% 3cm]{../pdf/include/supplementary-reanalysis-GSE9893.pdf}
% \includepdf[pages=1-17,trim=5cm 3cm 5cm 3cm]{../pdf/include/supplementary-reanalysis-GSE9893.pdf}

% \subsection{Datasets}
% \label{sec:results-prognostic-survival-datasets}

% A total of 114 public datasets of cancer gene expression profiles spanning 19
% types of cancer were downloaded from public repositories, manually curated and
% pre-processed for downstream analysis.  Sources for datasets include the Gene
% Expression Omnibus,\cite{edgar_gene_2002}
% InSilicoDB,\cite{taminau_insilicodb:_2011} and the \smallcaps{tcga} Research
% Network
% site.\footnote{\href{http://cancergenome.nih.gov/}{http://cancergenome.nih.gov/}}
% Individual datasets are described in the
% \hyperref[sec:methods-datasets]{\textsf{Methods}} chapter.

% \subsection{Gene sets}
% \label{sec:results-prognostic-survival-genesets}

% For each cancer dataset, we evaluated the association with clinical outcome of
% the 4722 curated gene sets from the
% \href{http://www.broadinstitute.org/gsea/msigdb/index.jsp}Molecular Signatures
% Database (\mbox{MSigDB
% \smallcaps{c2}}).\footnote{\href{http://www.broadinstitute.org/gsea/msigdb/index.jsp}{http://www.broadinstitute.org/gsea/msigdb/index.jsp}}
% \mbox{MSigDB \smallcaps{c2}} signatures are manually curated from the literature
% on gene expression and also include gene sets from curated pathways databases
% such as \smallcaps{kegg}.

% Version 4.0 of the MSigDB curated gene sets (updated on May 31, 2013) was
% downloaded and parsed to produce two lists of gene expression signatures: one
% containing the 4722 original signatures with gene symbols as identifiers, and a
% randomized list consisting of 4722 signatures of the same size as the originals,
% but composed of random samplings from the total pool of genes in the collection.

% \subsection{Association with outcome}
% \label{sec:results-prognostic-survival-association}

% Association with outcome was computed as follows.  For each cancer dataset, we
% retained the first of the following clinical outcomes available, in decreasing
% order of importance: disease-specific survival (\smallcaps{dss}); overall
% survival (\smallcaps{os}); disease-free survival (\smallcaps{dfs}); and distant
% metastasis free survival (\smallcaps{dmfs}).  The survival times for the
% selected clinical outcome specified the dependent variables.

% We then modeled the association of each gene set with survival times by fitting
% a univariate Cox proportional hazards model, using the continuous first
% principal component loading vector of the signature in the global expression
% matrix as the predictor variable.  The \emph{p}-value of the associated logrank
% test was used to estimate association of the gene set with survival times.

% A similar procedure was followed to compute the association of each single gene
% in the expression matrix with survival times, by using the respective continuous
% vector of expression as the predictor variable in the Cox model.  Here again, the
% \emph{p}-value of the logrank test was used to estimate association with
% outcome.

% \subsection{Prognostic content in human cancers}
% \label{sec:results-prognostic-survival-prognostic-content}

% Figure~\ref{fig:prognostic-fraction} shows, for each human cancer dataset
% analyzed, the fraction of \mbox{MSigDB \smallcaps{c2}} signatures, randomized
% signatures and single genes found significantly associated with outcome, at
% $p < 0.05$.

% \begin{figure}%
%   \includegraphics{fraction-significant-tests-base-R-graphics-15Jan2015.pdf}
%   \caption[Prognostic content in human cancers]{Prognostic content in human
%   cancers.  The fraction of markers significantly associated with outcome at
%   logrank $p<0.05$ is shown for single-gene markers, multi-gene markers from
%   the \mbox{MSigDB \smallcaps{c2}} database, and randomly generated
%   multi-genes markers devoid of biological meaning.  The latter was used to
%   order the datasets. Fractions were computed from death-related end-points
%   when available (\smallcaps{os} or \smallcaps{dss}), or relapse otherwise
%   (\smallcaps{dfs} or \smallcaps{dmfs}). The dotted red line marks the 0.05
%   threshold.}
%   \label{fig:prognostic-fraction}%
% \end{figure}

% \clearpage

% \subsection{Distribution of survival times}
% \label{sec:results-prognostic-survival-survival-times}

% \begin{marginfigure}%
%   \includegraphics{event-distribution-breast-os.pdf}
%   \caption[Distribution of overall survival events in thirteen breast cancer
%   datasets]{Distribution of overall survival events in 13 breast cancer
%   datasets.  The density distributions for censored observations (black) and
%   death events (red) are shown for \num{4663} breast cancer patients whose
%   tumours were profiled across 13 studies.}
%   \label{fig:distribution-survival-times}%
% \end{marginfigure}

% For each of the 13 breast cancer datasets with overall survival (\smallcaps{os})
% data in our analysis (Figure~\ref{fig:distribution-survival-times}), we computed
% the fraction of genes associated with \smallcaps{os} at logrank $p < 0.05$ for
% \num{100} permutations of the respective survival times constructs
% (Figure~\ref{fig:null-breast-os}).

% \subsection{Re-sampling of \smallcaps{metabric} dataset}
% \label{sec:results-prognostic-survival-metabric}

% The discovery and validation \smallcaps{metabric} datasets comprise altogether
% \num{1972} expression profiles of breast cancers.\cite{curtis_genomic_2012} This
% constitutes one of the largest cohorts of cancer patients with follow-up and
% extensive clinical annotation data available in the public domain.

% We ran a series of bootstrapping experiments in a merged \smallcaps{metabric}
% dataset aiming at assessing the effect of potential confounding factors in the
% quantification of the prognostic signals of cancer transcriptomes.
% Figure~\ref{fig:bootstrap-metabric} shows distributions of single-marker
% prognostic fractions for the following experimental setups: (\emph{a})~\num{100}
% samplings of size \num{500} each, out of the total \mbox{1972}
% \smallcaps{metabric} transcriptomes; (\emph{b})~\num{100} samplings of size $N$,
% for every $N \in \{100, 200, 300, 500, 750, 1000, 1250, 1500, 1750\}$;
% (\emph{c})~\num{100} samplings of size \num{500}, with survival times
% beyond time $t$ months artificially censored at time $t$, for every
% $t \in \{25, 50, 75, 100, 125, 150, 175, 200, \max(\text{follow-up time})\}$;
% (\emph{d})~\num{100} samplings, each comprised of $300 \times E$ \smallcaps{er}+
% samples and $300 \times (1-E)$ \smallcaps{er}-- samples, for every
% $E \in \{0, 0.1, 0.2, \ldots{}, 0.9, 1\}$; and (\emph{e})~\num{100} samplings,
% each comprised of $500 \times L$ lymph node positive samples and
% $500 \times (1-L)$ lymph node negative samples, for every
% $L \in \{0, 0.1, 0.2, \ldots{}, 0.9, 1\}$.

% \vspace{8\baselineskip}

% \begin{figure}
%   \includegraphics{null-breast-os.pdf}
%   %   \setfloatalignment{t}
%   \caption[Null distributions of prognostic content in 13 breast cancers cohorts
%   with randomized survival objects]{Null distributions of prognostic content in
%   13 breast cancers cohorts with randomized survival objects.  Prognostic
%   content, here defined as the fraction of single-gene markers associated with
%   overall survival at logrank \mbox{$p < 0.05$}, was computed for \num{100}
%   iterations of randomized survival times constructs for each dataset.  Each
%   boxplot shows the respective null distribution of prognostic fractions.
%   Actual prognostic fractions for each dataset are also displayed (refer to
%   Figure~\ref{fig:prognostic-fraction}), in red (if the observation is beyond
%   the 95\textsuperscript{th} quantile of the null distribution), or in grey
%   (if not).  The red horizontal dotted line shows the five percent expected
%   median of each null distribution.}
%   \label{fig:null-breast-os}%
% \end{figure}

% \begin{figure*}
%   \includegraphics{metabric-sims-19Mar2015.pdf}
%   \caption[Bootstrapping experiments on the \smallcaps{metabric}
%   dataset]{Bootstrapping experiments on the \num{1972} combined breast cancer
%   transcriptomes of the \smallcaps{metabric} dataset.
%   \textbf{A--}Distribution of the prognostic fraction in 100 samplings of 500
%   expression profiles, out of the total 1972 \smallcaps{metabric} profiles.
%   \textbf{B--}Effect of sample size.  \num{100} samplings (grey points) were
%   assessed for each specified sample size.  \textbf{C--}Effect of follow-up
%   times.  For each time $t$, \num{100} samplings were assessed for which
%   patients beyond time $t$ were considered censored at time $t$.
%   \textbf{D--}Effect of the fraction of \smallcaps{er}+ patients.  \num{100}
%   samplings, each of 300 patients, were assessed with the respective fraction
%   of \smallcaps{er}+ samples.  \textbf{E--}Effect of the fraction of node
%   positive patients.  \num{100} samplings, each of 500 patients, were assessed
%   with the respective fraction of node positive patients.}
%   \label{fig:bootstrap-metabric}
% \end{figure*}

%%%%%%%%%%%%%%%%%%%%%%%%%%%%%%%%%%%%%%%%%%%%%%%%%%%%%%%%%%%%%%%%%%%%%%%%%%%%%%%%%%%%
% example, dataset GSE9893 has one of the highest prognostic fractions (Fig. 1),   %
% but surprisingly its signal-to-noise metrics15 is very low (Supplementary        %
% Fig. S1). Examination of the data reveals that normalization was performed in    %
% two batches and induced massive spurious correlations between expression and     %
% outcome (Supplementary Fig. S1). Accordingly, a proper single-batch              %
% normalization of the raw expression data restores the signal-to-noise metrics to %
% normal and decreases the prognostic fraction from 59% to 19% (Supplementary      %
% Fig. S1).  We have shown the fraction of prognostic single- and multi-gene       %
% markers is greater that                                                          %
% 5% in the vast majority of the datasets. Furthermore, the probability of a       %
%  % significant single- or multi-genes marker association with outcome depends on %
%  % cohort’s demographics, but also to a large extent on technical factors that   %
%  % include sampling effects, cohort size, patient follow-up protocols,           %
%  % cellularity standards, protocol randomization and possibly other factors not  %
%  % addressed here. These findings call for the reappraisal of previous study     %
%  % resting on the associations of markers with outcome to support relevance to   %
%  % human cancer, including low-throughput PCR-based studies. They also call for  %
%  % study-specific controls akin to those presented Fig. 1 in future studies.     %
% A marker does not need to convey interesting biological information              %
% research-wise in order to be useful in the clinic. Therefore, our results have   %
% no bearing on the clinical utility of published marker associations.             %
%%%%%%%%%%%%%%%%%%%%%%%%%%%%%%%%%%%%%%%%%%%%%%%%%%%%%%%%%%%%%%%%%%%%%%%%%%%%%%%%%%%%

% \subsection{Datasets}
% \label{sec:results-prognostic-survival-datasets}

% A total of 114 public datasets of cancer gene expression profiles spanning 19
% types of cancer were downloaded from public repositories, manually curated and
% pre-processed for downstream analysis.  Sources for datasets include the Gene
% Expression Omnibus,\cite{edgar_gene_2002}
% InSilicoDB,\cite{taminau_insilicodb:_2011} and the \smallcaps{tcga} Research
% Network
% site.\footnote{\href{http://cancergenome.nih.gov/}{http://cancergenome.nih.gov/}}
% Individual datasets are described in the
% \hyperref[sec:methods-datasets]{\textsf{Methods}} chapter.

% \subsection{Gene sets}
% \label{sec:results-prognostic-survival-genesets}

% For each cancer dataset, we evaluated the association with clinical outcome of
% the 4722 curated gene sets from the
% \href{http://www.broadinstitute.org/gsea/msigdb/index.jsp}Molecular Signatures
% Database (\mbox{MSigDB
% \smallcaps{c2}}).\footnote{\href{http://www.broadinstitute.org/gsea/msigdb/index.jsp}{http://www.broadinstitute.org/gsea/msigdb/index.jsp}}
% \mbox{MSigDB \smallcaps{c2}} signatures are manually curated from the literature
% on gene expression and also include gene sets from curated pathways databases
% such as \smallcaps{kegg}.

% Version 4.0 of the MSigDB curated gene sets (updated on May 31, 2013) was
% downloaded and parsed to produce two lists of gene expression signatures: one
% containing the 4722 original signatures with gene symbols as identifiers, and a
% randomized list consisting of 4722 signatures of the same size as the originals,
% but composed of random samplings from the total pool of genes in the collection.

% \subsection{Association with outcome}
% \label{sec:results-prognostic-survival-association}

% Association with outcome was computed as follows.  For each cancer dataset, we
% retained the first of the following clinical outcomes available, in decreasing
% order of importance: disease-specific survival (\smallcaps{dss}); overall
% survival (\smallcaps{os}); disease-free survival (\smallcaps{dfs}); and distant
% metastasis free survival (\smallcaps{dmfs}).  The survival times for the
% selected clinical outcome specified the dependent variables.

% We then modeled the association of each gene set with survival times by fitting
% a univariate Cox proportional hazards model, using the continuous first
% principal component loading vector of the signature in the global expression
% matrix as the predictor variable.  The \emph{p}-value of the associated logrank
% test was used to estimate association of the gene set with survival times.

% A similar procedure was followed to compute the association of each single gene
% in the expression matrix with survival times, by using the respective continuous
% vector of expression as the predictor variable in the Cox model.  Here again, the
% \emph{p}-value of the logrank test was used to estimate association with
% outcome.

% \subsection{Prognostic content in human cancers}
% \label{sec:results-prognostic-survival-prognostic-content}

% Figure~\ref{fig:prognostic-fraction} shows, for each human cancer dataset
% analyzed, the fraction of \mbox{MSigDB \smallcaps{c2}} signatures, randomized
% signatures and single genes found significantly associated with outcome, at
% $p < 0.05$.

% \begin{figure}%
%   \includegraphics{fraction-significant-tests-base-R-graphics-15Jan2015.pdf}
%   \caption[Prognostic content in human cancers]{Prognostic content in human
%   cancers.  The fraction of markers significantly associated with outcome at
%   logrank $p<0.05$ is shown for single-gene markers, multi-gene markers from
%   the \mbox{MSigDB \smallcaps{c2}} database, and randomly generated
%   multi-genes markers devoid of biological meaning.  The latter was used to
%   order the datasets. Fractions were computed from death-related end-points
%   when available (\smallcaps{os} or \smallcaps{dss}), or relapse otherwise
%   (\smallcaps{dfs} or \smallcaps{dmfs}). The dotted red line marks the 0.05
%   threshold.}
%   \label{fig:prognostic-fraction}%
% \end{figure}

% \clearpage

% \subsection{Distribution of survival times}
% \label{sec:results-prognostic-survival-survival-times}

% \begin{marginfigure}%
%   \includegraphics{event-distribution-breast-os.pdf}
%   \caption[Distribution of overall survival events in thirteen breast cancer
%   datasets]{Distribution of overall survival events in 13 breast cancer
%   datasets.  The density distributions for censored observations (black) and
%   death events (red) are shown for \num{4663} breast cancer patients whose
%   tumours were profiled across 13 studies.}
%   \label{fig:distribution-survival-times}%
% \end{marginfigure}

% For each of the 13 breast cancer datasets with overall survival (\smallcaps{os})
% data in our analysis (Figure~\ref{fig:distribution-survival-times}), we computed
% the fraction of genes associated with \smallcaps{os} at logrank $p < 0.05$ for
% \num{100} permutations of the respective survival times constructs
% (Figure~\ref{fig:null-breast-os}).

% \subsection{Re-sampling of \smallcaps{metabric} dataset}
% \label{sec:results-prognostic-survival-metabric}

% The discovery and validation \smallcaps{metabric} datasets comprise altogether
% \num{1972} expression profiles of breast cancers.\cite{curtis_genomic_2012} This
% constitutes one of the largest cohorts of cancer patients with follow-up and
% extensive clinical annotation data available in the public domain.

% We ran a series of bootstrapping experiments in a merged \smallcaps{metabric}
% dataset aiming at assessing the effect of potential confounding factors in the
% quantification of the prognostic signals of cancer transcriptomes.
% Figure~\ref{fig:bootstrap-metabric} shows distributions of single-marker
% prognostic fractions for the following experimental setups: (\emph{a})~\num{100}
% samplings of size \num{500} each, out of the total \mbox{1972}
% \smallcaps{metabric} transcriptomes; (\emph{b})~\num{100} samplings of size $N$,
% for every $N \in \{100, 200, 300, 500, 750, 1000, 1250, 1500, 1750\}$;
% (\emph{c})~\num{100} samplings of size \num{500}, with survival times
% beyond time $t$ months artificially censored at time $t$, for every
% $t \in \{25, 50, 75, 100, 125, 150, 175, 200, \max(\text{follow-up time})\}$;
% (\emph{d})~\num{100} samplings, each comprised of $300 \times E$ \smallcaps{er}+
% samples and $300 \times (1-E)$ \smallcaps{er}-- samples, for every
% $E \in \{0, 0.1, 0.2, \ldots{}, 0.9, 1\}$; and (\emph{e})~\num{100} samplings,
% each comprised of $500 \times L$ lymph node positive samples and
% $500 \times (1-L)$ lymph node negative samples, for every
% $L \in \{0, 0.1, 0.2, \ldots{}, 0.9, 1\}$.

% \vspace{8\baselineskip}

% \begin{figure}
%   \includegraphics{null-breast-os.pdf}
%   %   \setfloatalignment{t}
%   \caption[Null distributions of prognostic content in 13 breast cancers cohorts
%   with randomized survival objects]{Null distributions of prognostic content in
%   13 breast cancers cohorts with randomized survival objects.  Prognostic
%   content, here defined as the fraction of single-gene markers associated with
%   overall survival at logrank \mbox{$p < 0.05$}, was computed for \num{100}
%   iterations of randomized survival times constructs for each dataset.  Each
%   boxplot shows the respective null distribution of prognostic fractions.
%   Actual prognostic fractions for each dataset are also displayed (refer to
%   Figure~\ref{fig:prognostic-fraction}), in red (if the observation is beyond
%   the 95\textsuperscript{th} quantile of the null distribution), or in grey
%   (if not).  The red horizontal dotted line shows the five percent expected
%   median of each null distribution.}
%   \label{fig:null-breast-os}%
% \end{figure}

% \begin{figure*}
%   \includegraphics{metabric-sims-19Mar2015.pdf}
%   \caption[Bootstrapping experiments on the \smallcaps{metabric}
%   dataset]{Bootstrapping experiments on the \num{1972} combined breast cancer
%   transcriptomes of the \smallcaps{metabric} dataset.
%   \textbf{A--}Distribution of the prognostic fraction in 100 samplings of 500
%   expression profiles, out of the total 1972 \smallcaps{metabric} profiles.
%   \textbf{B--}Effect of sample size.  \num{100} samplings (grey points) were
%   assessed for each specified sample size.  \textbf{C--}Effect of follow-up
%   times.  For each time $t$, \num{100} samplings were assessed for which
%   patients beyond time $t$ were considered censored at time $t$.
%   \textbf{D--}Effect of the fraction of \smallcaps{er}+ patients.  \num{100}
%   samplings, each of 300 patients, were assessed with the respective fraction
%   of \smallcaps{er}+ samples.  \textbf{E--}Effect of the fraction of node
%   positive patients.  \num{100} samplings, each of 500 patients, were assessed
%   with the respective fraction of node positive patients.}
%   \label{fig:bootstrap-metabric}
% \end{figure*

\section{Other Contributions}

This sections reports published results to which the author contributed at large
during this dissertation.  For each work, a short synopsis of the main findings
as well as the specific contributions of the author will be presented.

\subsection{Role of Epac and protein kinase A in thyrotropin-induced gene
  expression in primary thyrocytes}
\begin{figure}[h]
  \includegraphics{../pdf/include/epac-cover.pdf}
\end{figure}

This work sought to clarify which partners of the c\smallcaps{amp} cascade
regulate the \smallcaps{tsh}-induced gene expression modulation in thyrocytes:
protein kinase \smallcaps{a} and/or the \smallcaps{epac}
proteins.\cite{van_staveren_role_2012} Contingent to this objective was the
characterization of the potential role of the Epac-Rap-RapGAP pathway in thyroid
tumorigenesis.  The author contributed with data analysis, figure generation and
results discussion.

\clearpage

\subsection{5-Aza-2'-Deoxycytidine has minor effects on differentiation in human
  thyroid cancer cell lines, but modulates genes that are involved in adaptation
  \emph{in vitro}}
\begin{figure}[h]
  \includegraphics{../pdf/include/5-aza-cover.pdf}
\end{figure}

This work aimed at investigating the extent to which 5-aza-2'-deoxycytidine, a
\smallcaps{dna} demethylation agent, is able to reactivate the expression of
differentiation markers potentially repressed by epigenetic modifications in
thyroid cancer cell lines.\cite{dom_5-aza-2-deoxycytidine_2013} The author
contributed with data analysis, figure generation and results discussion.

\clearpage

\subsection{Intratumor heterogeneity and clonal evolution in an aggressive
  papillary thyroid cancer and matched metastases}
\begin{figure}[h]
  \includegraphics{../pdf/include/ptc-heterogeneity-cover.pdf}
\end{figure}

This study sought to characterize the intratmoural heterogeneity of a specimen
of aggressive papillary thyroid carcinoma, and the clonal relationships between
the primary tumour and their corresponding lymph node and distant
metastases.\cite{pennec_intratumor_2015} The author contributed with
experimental design input and results discussion.

%%% Local Variables:
%%% mode: latex
%%% TeX-master: "../thesis.tex"
%%% End:



\chapter{Discussion}
\label{chap:discussion}

\section{Microarrays}
\label{microarray-discussion}

This is the \emph{microarrays} section of the discussion.

\clearpage

%%% Local Variables:
%%% TeX-engine: xetex
%%% mode: latex
%%% TeX-master: "../../thesis"
%%% End:

\section{Cancer}
\label{cancer-discussion}

% This is the \emph{cancer} section of the discussion.

\newthought{Cancer research} aims to improve the diagnosis and treatment through
better disease classification and patient stratification, which allows for the
design of therapies that are more targeted to specific cancer sub-types and
potentially improves the effectiveness of existing regimens based on therapeutic
response and adverse events.

Topics to discuss:
\begin{itemize}
\item expectations \emph{vs} reality
\item solid achievements (differentiation sigs)
\item questions raised and issues to address (extent of the prognostic signals)
\item shortcomings of the technology regarding the translation into the clinical realm (nrc-weigelt-2012.pdf)
\end{itemize}

\bigskip

Where is cancer research now?
\begin{itemize}
\item The importance of genomics to seize the biology of cancer.
\item Vanguard treatments and their experimental motivations.
\item is curing cancer a realistic expectation (as it was in the mid-20th
  century), or will we have to settle for improvement of diagnostic/prognostic
  tools?
\item why are genomics failing in producing a clarification of cancer as
  a biological phenomena?
\item paradox: maximum depth of analytical tools, minimal expectations regarding
  clinical management of disease
\end{itemize}

\clearpage

%%% Local Variables:
%%% TeX-engine: xetex
%%% mode: latex
%%% TeX-master: "../../thesis"
%%% End:

% \section{Life}
\label{life-discussion}

% \newthought{Can cancer} be defeated?

% \smallskip{}

\newthought{With projects} like The Cancer Genome Atlas drawing to an end,
discussion is now shifting to where should cancer research go next.  Eric
Lander, at the Broad Institute, has argued that ``completing the genomic
analysis of this disease should be a biomedical
imperative.''\cite{lawrence_discovery_2014} According to Lander, the
identification of novel cancer genes from the mining of the \smallcaps{tcga}
data provides the motivation to extend the search for genetic determinants of
the disease to even larger panels of samples.

Elsewhere, skepticism has started to be voiced regarding the value of pursuing
the entire catalog of cancer genes towards the advancement of new therapies.
Bert Vogelstein, who championed the genetic nature of cancer in the early
2000's, is less sure about whether extending the atlas project would be the best
allocation of research funds: ``There’s no question that it would be
valuable. The question is whether it’s worth it.''\cite{zimmer_catalog_2014}

\medskip

In this dissertation, we discussed applications of molecular biomarkers for
cancer diagnostics and prognosis, based on microarray technology.  We conclude
that the use of gene expression signatures in cancer research has shown great
promise (e.g., with the use of differentiation and proliferation signatures to
assist cancer diagnosis), but may have also promoted unfunded expectations
(e.g., by mis-interpreting the nature of pervasive prognostic signals in cancer
transcriptomes).  The research programs here detailed were designed to exploit
the vast compendia of cancer expression profiles available in the public
domain, and their results must, therefore, first be contextualized within the
technical and analytic frame of the technology that enabled them.

While microarray technology allows for the quantification of m\smallcaps{rna}
products of cancer biospecimens in solution, it only lends itself to the
detection of gene products already mapped and printed in the chip of the
platform of choice.  Furthermore, eventual mutations (either nucleotide changes
or structural variants) in actively transcribed genes may impair the correct
estimation of their transcription levels due to the adulteration of their
m\smallcaps{rna} sequences.  Even assuming correct estimates of the
transcriptomic load of a given biospecimen, tissue cellularity and heterogeneity
may complicate the interpretation of bulk expression profiles.  Finally, as
microarray technology only probes the Central Dogma of Biology\footnote{The
  central dogma of molecular biology, postulated by Francis Crick in 1958 and
  reasserted in 1970 (\citealp{crick_protein_1958,crick_central_1970}), pertains
  to the rules that govern the sequential flow of genetic information between
  \smallcaps{dna}, \smallcaps{rna} and proteins.  It can be summarized as
  ``\smallcaps{dna} makes \smallcaps{rna} makes protein,'' which provides the
  template for the enactment of hereditary information for all living organisms,
  and frames the scope of evolutionary forces on genetic systems.} at the
transcriptional level, it provides no insight on upstream modulatory effects on
gene expression (e.g., epigenetic determinants), as well as on downstream
modulatory effects on gene products (e.g., post-transcriptional modifications).

The interpretation of microarray data is challenged by issues regarding
non-specific probe hybridization, non-consensual probe annotations, poorly
defined gene ontologies and an incomplete understanding of the dynamics of gene
expression on \emph{in vivo} biological systems.  Nevertheless, and in spite of
having fell short of the promise for personalized medicine, microarrays have
notably delivered in a number of ways: by offering a first glimpse at the
previously unsuspected molecular taxonomy of many forms of cancer; by allowing
for a quantitative diagnostic of neoplastic disease; by enabling the molecular
dissection of cancer biospecimens with biologically motivated gene expression
signatures; by unveiling disease-specific molecular prognostic markers; or by
providing a framework to allow for patient stratification based on gene
expression profiles.  Looking forward, we foresee microarray technology to
retain its relevance in the clinical setting as a diagnostic tool, and for the
plethora of publicly available expression profiles to remain a valuable mining
resource, as novel insights on the biology of cancer come to the fore.

\medskip{}

A more insightful discussion of the merits and limits of microarrays and
genomics in cancer research could benefit from a wider historical perspective.
% the first high throughput genomic technology used to map the genetics of
% cancer, could be made from a wider historical perspective.
Current biomedical research is conducted under the paradigms defined by the
emergence of molecular biology in the 1970's.  Following the discovery of the
structure of \smallcaps{dna} and the unlocking of the genetic code, cellular
biology turned its attention to the genome and to the mechanisms by which
hereditary information is enacted throughout the cell.  This surge of interest
in \smallcaps{dna}, complemented by the molecule's stability and amenability to
experimental manipulation, led to the development of technologies aiming for its
extensive isolation and characterization.  These technologies include molecular
cloning, polymerase chain reaction (\smallcaps{pcr}) and Southern blotting---the
precursor of microarrays.

Up to then, medical research was typically carried out by physicians, like
Sydney Farber, with a keen interest in the physiological aspects of disease.
The emergence of biomedical medicine consummated the split between clinical and
basic research, with generations of highly specialized researchers trained under
the central dogma taking over from physician-scientists.

The underlying premise of molecular and genomics approaches to tackle cancer is
the univocal relationship between genotype and a disease phenotype.\footnote{Of
  which this dissertation is an example: statistical associations between
  genetic correlates and, respectively, histo-pathological thyroid cancer types
  and differential survival times of cancer patients are at the core of the
  results here discussed.  Incidentally, this idea is typified by the concept of
  genetic penetrance, or the faction of a given phenotype that is explained by
  its underlying genotype.} This conceptualization does not take into
consideration the possibility of disease phenotypes being the result of dynamic
inter-cellular interplays (e.g., heterotypic signalings in the tumour
ecosystem), does not account for dynamically reversible neoplastic states (e.g.,
\smallcaps{emt}-\smallcaps{met} transitions), and is unconcerned with the
constrictions imposed by evolutionary forces on genetic systems (see bellow).
By seeking to apprehend cancer mainly through the prism of its genetic
alterations, molecular genomics has pushed to the background the
characterization of the many cellular states visited by neoplastic types that,
while enabled by genetic disruptions, are not fully explained by
them.\footnote{Another side-effect of the bias towards genomics in cancer
  research is the large number of biosynthesis products and enzymes remaining
  unannotated in the human genome, resulting in poorly documented ontology
  databases.}

% \footnote{Incidentally, this fact may account for \ldots{}.  These databases,
% mainly populated from results}

To illustrate this point, consider the Warburg effect.  Otto Warburg's
observation in 1924 that cancer cells metabolize glucose in a manner that is
less efficient from that of cells in normal tissues\cite{warburg_ueber_1924} has
remained a matter of contentious speculation throughout the
20\textsuperscript{th} century.  Only recent observations on the metabolism of
neoplastic cells have started to shed light into a possible link between the
uptake and incorporation of nutrients into the biomass during proliferation and
anaerobic glycolysis.\cite{heiden_understanding_2009}

Equally poorly understood are metabolic phenomena driven by nutrient conditions
of the tumour microenvironment and intercellular metabolism---much of our
understanding of metabolism was drawn from work in simple
organisms.\cite{hsu_cancer_2008} If cancer is to be portrayed has a collection
of phenotypic dynamic ranges enabled by a succession of genetic disruptions,
perhaps a thorough characterization of these metabolic states and their
inter-dependencies could provide better clues for targeted treatments than the
exhaustive characterization of all putative cancer
genes.\cite{kroemer_tumor_2008}

\medskip{}

Nevertheless, cancer remains at its core a disease of genetic deregulation.  The
evolutionary history of genetic systems provides another orthogonal perspective
when seeking a more integrated approach to the disease.

The multistep progression of cancer has been largely explained by models of
clonal evolution, driven by the acquisition of genetic variants, then pruned by
Darwinian selection.\cite{greaves_clonal_2012} These models fittingly mirror the
views of Richard Dawkins, who has strongly argued for a gene-centered
interpretation of evolution.\cite{dawkins_selfish_1976} This formulation is
however weakened by the lack of genetic correlates of the hallmarks of cancer
that could be tagged along the evolution of fitter cancer clones---the only
\emph{bona fide} cancer genes reported to date are still classical oncogenes and
tumour suppressor genes.

The paleontologist Stephen J. Gould was a vocal opponent of the gene-centric
perspective of evolution, arguing that most genes may not have a consistent
enough effect on phenotypes' fitness to drive their
evolution.\footnote{Pleiotropy (the capacity of a gene to influence two or more
  seemingly unrelated traits), epistasis (the reliance of a gene on the presence
  of one or more modifier genes to express its phenotype) and polygenic traits
  (traits that are controlled by different genes to different degrees) are given
  as examples of why natural selection lacks reach to shape genetic landscapes.}
One of Gould's main points of contention was what he described as
\emph{adaptationism}, the assertion that natural selection explains the majority
of evolved traits.\cite{gould_spandrels_1979} Instead, Gould quoted functional,
structural and genetic constraints, as well as serendipity itself, as major
evolutionary forces.  For instance, \emph{exaptation}, or the co-option of
structures with different prior functions to a new context, has been proposed to
take part in the evolution of complex traits, such as the human
eye.\cite{gould_exaptation_1982}

% The gene-centered reductionist approach is largely in line with the current genomics
% interpretation of cancer progression.  Gene fitness selection, penetrance.  If
% selection, as Gould sees it, can act only to magnify and sculpt variations
% previously embedded in genetic systems, \ldots{}.

The projection of this debate into cancer evolution has implications at many
conceptual levels.  Consider, for instance, the role of genetic instability in
cancer progression.  Here, orthodoxy posits that: ``instability leads to an
increased mutation rate and can shape the evolution of the cancer genome through
a plethora of mechanisms.''\cite{burrell_causes_2013} This is to say that
genetic instability is an \emph{advantageous} feature, and therefore is
\emph{selected for} by cancers, in order to keep producing more \emph{adaptive}
genetic variants.  This point of view is so widespread in the field that one has
to reach out to the fringes of the literature to find an alternative, more
sensible rationale: ``taking a molecular perspective, nothing can be selected
because it mutates. (\ldots{}) The evolutionary breakdown of a repair strategy
in mutagenic environments is thereby easily explained by the accumulating costs
of repair.''\cite{breivik_evolutionary_2005} In other words, mutagenic
environments in cancer occur simply because it is \emph{too costly} for
defective neoplastic cells to repair their \smallcaps{dna} \emph{and} keep
dividing at the same time.  Similarly, virtually every hallmark of cancer has
been interpreted from an adaptationist stance, without ever making proof of the
dependency of their putative genetic correlates in cancer
progression.\cite{bernards_progression_2002}

% Taken together, it seems evident that a reductionist gene-centric formulation of
% cancer evolution is falling short to explain the progression of the disease.
% Neoplastic evolution cannot just be modeled as the relentless selection of
% increasingly fitter hallmark genetic variants.  Rather, it could be refashioned
% as the by-product of the loosening of the genetic stack evolved to enforce
% social compliance, leaving neoplastic cells to scramble through contextual
% adaptive solutions within the scope of the underlying genetic\ circuitries.  If
% so, a more consequent strategy to curb cancer progression could pass by focusing
% on the universal modular adaptations cancer cells depend on to thrive.  Such
% adaptations are likely to be featured, in some form, in models of present-day
% genetic lineages that never undertook the transition to multicellularity.

% \mediumskip{}

% Would cancer, defined by genomic instability and its phenotypic heterogeneity
% and plasticity be the best model to probe for these rules?  Perhaps such
% undertaking would best be envisaged in biological models where the deployment of
% the genetic information is faithfully mirrored by phenotypes strictly under
% control of natural selection---embryogenesis, connected with If pursuing the complete
% catalog of cancer genes remains an imperative task in cancer research,
% addressing the gap between phenome and genome.

% non-coding \smallcaps{rna}s (nc\smallcaps{rna}s) are an heterogeneous group of
% biomolecules that has been classified in three families concerning their size:
% (\emph{a}) molecules ranging from 18 to 25 nucleotides in length, comprising
% micro\smallcaps{rna}s and small interfering \smallcaps{rna}s; (\emph{b})
% molecules from 30 to 300 nucleotides, involving small nucleolar
% \smallcaps{rna}s, Piwi \smallcaps{rna}s (pi-\smallcaps{rna}s), transfer
% \smallcaps{rna}s, ribosomal \smallcaps{rna}s and small-ncRNAs; and (\emph{c})
% molecules larger than 300 nucleotides, referred to as long non-coding
% \smallcaps{rnao}s (nc\smallcaps{rna}s).

% For example, it has been demonstrated that some mi\samllcaps{rna}s can either
% repress or induce the transcription of a given m\smallcaps{rna}, depending on
% the proliferative status of the cell

% The elucidation of these matters will only further our understanding of the
% nature of cancer; % \footnote{``It's bad bile.  It's bad habits.  It's bad bosses.
%   % It's bad genes.''---\emph{Mel Greaves}}
%  and of that other very peculiar wonder---life.

% Pervasive transcription of the human genome produces thousands of previously
% unidentified long intergenic coding RNAs.  Only 2% of our genome displays
% protein-coding capacity.

% non-coding \smallcaps{rna}s (nc\smallcaps{rna}s) are an heterogeneous group of
% biomolecules that has been classified in three families concerning their size:
% (\emph{a}) molecules ranging from 18 to 25 nucleotides in length, comprising
% micro\smallcaps{rna}s and small interfering \smallcaps{rna}s; (\emph{b})
% molecules from 30 to 300 nucleotides, involving small nucleolar
% \smallcaps{rna}s, Piwi \smallcaps{rna}s (pi-\smallcaps{rna}s), transfer
% \smallcaps{rna}s, ribosomal \smallcaps{rna}s and small-ncRNAs; and (\emph{c})
% molecules larger than 300 nucleotides, referred to as long non-coding
% \smallcaps{rnao}s (nc\smallcaps{rna}s).

% For example, it has been demonstrated that some mi\samllcaps{rna}s can either
% repress or induce the transcription of a given m\smallcaps{rna}, depending on
% the proliferative status of the cell

\bigskip{}

While modern notions of the disease have dramatically departed from its humoural
depictions of a hundred and fifty years ago, many dots remain to be connected to
yield a definitive, cohesive, and actionable picture of cancer.  To this end,
engaging with historical, biological and evolutionary perspectives could provide
an invaluable contribution.  And no one quite like Gould knew how to weave these
themes into a rendering that could so well play as a metaphor for the malignancy
itself:\footnote{\citealp[p. 54]{gould_flamingos_1987}}
% , this ``strange eventful history'':

\begin{quotation}
  Our world is not an optimal place, fine tuned by omnipotent forces of
  selection.  It is a quirky mass of imperfections, working well enough (often
  admirably); a jury-rigged set of adaptations built of curious parts made
  available by past histories in different contexts.  (\ldots) A world optimally
  adapted to current environments is a world without history, and a world
  without history might have been created as we find it.  History matters; it
  confounds perfection and proves that current life transformed its own past.
\end{quotation}

Exacting as it may be, the unraveling of these threads may well be key to
elucidate the nature of cancer; and of that other very peculiar wonder---life.

\clearpage

% The elucidation of these matters will only further our understanding of the
% nature of cancer; % \footnote{``It's bad bile.  It's bad habits.  It's bad bosses.
%   % It's bad genes.''---\emph{Mel Greaves}}
%  and of that other very peculiar wonder---life.

%%% Local Variables:
%%% mode: latex
%%% TeX-master: "../../thesis"
%%% End:



\backmatter

\bibliography{thesis}
\bibliographystyle{plainnat}


\end{document}

%%% Local Variables:
%%% TeX-engine: xetex
%%% mode: latex
%%% TeX-master: t
%%% zotero-collection: #("8" 0 1 (name "thesis"))
%%% End:
